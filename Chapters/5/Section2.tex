\subsection{Subgroups}
%%%%%%%%%%%%%%%%%%%%%%%%%%%%%%%%%%%%%%%%%%%%%%%%%%%%%%%%%%%%%%%%%%%%%%%%%%%%%%%%
%%%%%%%%%%%%%%%%%%%%%%%%%%%%%%%%%%%%%%%%%%%%%%%%%%%%%%%%%%%%%%%%%%%%%%%%%%%%%%%%
%%%%%%%%%%%%%%%%%%%%%%%%%%%%%%%%%%%%%%%%%%%%%%%%%%%%%%%%%%%%%%%%%%%%%%%%%%%%%%%%
%%%%%%%%%%%%%%%%%%%%%%%%%%%%%%%%%%%%%%%%%%%%%%%%%%%%%%%%%%%%%%%%%%%%%%%%%%%%%%%%
\begin{mdframed}[style=darkQuesion]
1. In $\mathrm{GL}_{2}(\mathbf{R}),$ find the order of each of the following elements.
\begin{multicols}{4}
\begin{itemize}
\item[]{
    (a)$\dagger\left[\begin{array}{rr}1 & \m 1 \\
          1 & 0\end{array}\right]$}
\item[]{
    (b) $\left[\begin{array}{rr}0 & 1 \\
          \m 1 & 0\end{array}\right]$}
\item[]{
    $\dagger(\mathrm{c})\left[\begin{array}{ll}1 & 1 \\
          0 & 1\end{array}\right]$}
\item[]{
    (d) $\left[\begin{array}{rr}\m 1 & 1 \\
          0 & 1\end{array}\right]$}
\end{itemize}
\end{multicols}
\vspace{.5cm}
\end{mdframed}

%%%%%%%%%%%%%%%%%%%%%%%%%%%%%%%%%%%%%%%%%%%%%%%%%%%%%%%%%%%%%%%%%%%%%%%%%%%%%%%%
\begin{mdframed}[style=darkAnswer,frametitle={Joe Starr}]
\begin{multicols}{2}
\begin{itemize}
\item[(a)]{
    $$\begin{bmatrix}1 & \m 1 \\
      1 & 0 \\
      \end{bmatrix}^2=\begin{bmatrix}0 & \m 1 \\
      1 & \m 1\\
      \end{bmatrix}$$
    $$\begin{bmatrix}1 & \m 1 \\
      1 & 0 \\
      \end{bmatrix}^3=\begin{bmatrix}\m 1 & 0 \\
      0 & \m 1\\
      \end{bmatrix}$$
    $$\begin{bmatrix}1 & \m 1 \\
      1 & 0 \\
      \end{bmatrix}^6=\begin{bmatrix}1 & 0 \\
      0 & 1\\
      \end{bmatrix}$$
  }
\item[(b)]{
    $$\begin{bmatrix}0 & 1 \\
      \m 1 & 0 \\
      \end{bmatrix}^2=\begin{bmatrix}\m 1 & 0 \\
      0 & \m 1\\
      \end{bmatrix}$$
    $$\begin{bmatrix}0 & 1 \\
      \m 1 & 0 \\
      \end{bmatrix}^4=\begin{bmatrix}1 & 0 \\
      0 & 1\\
      \end{bmatrix}$$}
\end{itemize}
\end{multicols}
\begin{multicols}{2}
\begin{itemize}
\item[(c)]{
    $$\begin{bmatrix}1 & 1 \\
      0 & 1 \\
      \end{bmatrix}^2=\begin{bmatrix}1 & 2 \\
      0 & 1 \\
      \end{bmatrix}$$
    Infinite order.
  }
\item[(d)]{
    $$\begin{bmatrix}\m 1 & 1 \\
      0 & 1 \\
      \end{bmatrix}^2=\begin{bmatrix}1 & 0 \\
      0 & 1 \\
      \end{bmatrix}$$}
\end{itemize}
\end{multicols}
\end{mdframed}
\newpage
%%%%%%%%%%%%%%%%%%%%%%%%%%%%%%%%%%%%%%%%%%%%%%%%%%%%%%%%%%%%%%%%%%%%%%%%%%%%%%%%
%%%%%%%%%%%%%%%%%%%%%%%%%%%%%%%%%%%%%%%%%%%%%%%%%%%%%%%%%%%%%%%%%%%%%%%%%%%%%%%%
%%%%%%%%%%%%%%%%%%%%%%%%%%%%%%%%%%%%%%%%%%%%%%%%%%%%%%%%%%%%%%%%%%%%%%%%%%%%%%%%
%%%%%%%%%%%%%%%%%%%%%%%%%%%%%%%%%%%%%%%%%%%%%%%%%%%%%%%%%%%%%%%%%%%%%%%%%%%%%%%%
\begin{mdframed}[style=darkQuesion]
2. Let $A=\left[\begin{array}{rr}1 & \m 1 \\
      \m 1 & 0\end{array}\right] \in \mathrm{GL}_{2}(\mathrm{R})$.
Show that $A$ has infinite order by proving that
$A^{n}=\left[\begin{array}{cc}F_{n+1} & -F_{n} \\
      -F_{n} & F_{n-1}\end{array}\right],$
for $n \geq 1,$ where $F_{0}=0, F_{1}=1,$ and $F_{n+1}=$
$F_{n}+F_{n-1}$ is the Fibonacci sequence.
\end{mdframed}

%%%%%%%%%%%%%%%%%%%%%%%%%%%%%%%%%%%%%%%%%%%%%%%%%%%%%%%%%%%%%%%%%%%%%%%%%%%%%%%%
\begin{mdframed}[style=darkAnswer,frametitle={Joe Starr}]
We will proceed with induction:
\begin{itemize}[align=left]
\item[Base Case:]{ Consider $1$ for the basecase.
    $\begin{bmatrix}F_{n+1} & \m F_{n} \\
      \m F_{n} & F_{n-1} \\
      \end{bmatrix}=\begin{bmatrix}1 & \m 1 \\
      \m 1 & 0 \\
      \end{bmatrix}$ showing the base case to be true.
  }
\item[Inductive Case:]{ Assume that it's ture for the nth power we will show
    this implies the $n+1$th case to be true.
    $$A^{n+1}=A^nA^1=\begin{bmatrix}F_{n+1} & \m F_{n} \\
      \m F_{n} & F_{n-1} \\
      \end{bmatrix}\begin{bmatrix}1 & \m 1 \\
      \m 1 & 0 \\
      \end{bmatrix}=\begin{bmatrix}F_{n+2} & \m F_{n+1} \\
      \m F_{n+1} & F_{n} \\
      \end{bmatrix}$$ showing the Inductive case to be true, and $A$ of infinite
    order.
  }
\end{itemize}
\end{mdframed}
\newpage
%%%%%%%%%%%%%%%%%%%%%%%%%%%%%%%%%%%%%%%%%%%%%%%%%%%%%%%%%%%%%%%%%%%%%%%%%%%%%%%%
%%%%%%%%%%%%%%%%%%%%%%%%%%%%%%%%%%%%%%%%%%%%%%%%%%%%%%%%%%%%%%%%%%%%%%%%%%%%%%%%
%%%%%%%%%%%%%%%%%%%%%%%%%%%%%%%%%%%%%%%%%%%%%%%%%%%%%%%%%%%%%%%%%%%%%%%%%%%%%%%%
%%%%%%%%%%%%%%%%%%%%%%%%%%%%%%%%%%%%%%%%%%%%%%%%%%%%%%%%%%%%%%%%%%%%%%%%%%%%%%%%
\begin{mdframed}[style=darkQuesion]
3. Prove that the set of all rational numbers of the form $m / n$, where $m, n \in \mathbf{Z}$ and $n$ is square-free, is a subgroup of Q (under addition).

\end{mdframed}

%%%%%%%%%%%%%%%%%%%%%%%%%%%%%%%%%%%%%%%%%%%%%%%%%%%%%%%%%%%%%%%%%%%%%%%%%%%%%%%%
\begin{mdframed}[style=darkAnswer,frametitle={Joe Starr}]
\begin{itemize}[align=left]
\Invs{Let $m,n \in \Z$ with the given properties. Take $\frac{\m m}{n}$
    and consider \\$\frac{m}{n}+\frac{\m m}{n}=\frac{m-m}{n}=0$ as desired.
  }
\Clos{
    Let $m,n,a,b\in \Z$ with the given properties. Take
    $\frac{m}{n}+\frac{a}{b}=\frac{mb+an}{bn}$, since $b$ and $n$ are square
    free $bn$ is also square free.
  }
\end{itemize}
\end{mdframed}
\newpage
%%%%%%%%%%%%%%%%%%%%%%%%%%%%%%%%%%%%%%%%%%%%%%%%%%%%%%%%%%%%%%%%%%%%%%%%%%%%%%%%
%%%%%%%%%%%%%%%%%%%%%%%%%%%%%%%%%%%%%%%%%%%%%%%%%%%%%%%%%%%%%%%%%%%%%%%%%%%%%%%%
%%%%%%%%%%%%%%%%%%%%%%%%%%%%%%%%%%%%%%%%%%%%%%%%%%%%%%%%%%%%%%%%%%%%%%%%%%%%%%%%
%%%%%%%%%%%%%%%%%%%%%%%%%%%%%%%%%%%%%%%%%%%%%%%%%%%%%%%%%%%%%%%%%%%%%%%%%%%%%%%%
\begin{mdframed}[style=darkQuesion]
4. Show that $\{\ (1),\ (1,2)(3,4),\ (1,3)(2,4),\ (1,4)(2,3)\ \}$ is a subgroup
of $S_{4}$
\end{mdframed}

%%%%%%%%%%%%%%%%%%%%%%%%%%%%%%%%%%%%%%%%%%%%%%%%%%%%%%%%%%%%%%%%%%%%%%%%%%%%%%%%
\begin{mdframed}[style=darkAnswer,frametitle={Joe Starr}]
We begin by labeling each permutation
$$A=(1,2)(3,4)=$$$$B=(1,3)(2,4)$$$$C=(1,4)(2,3)$$
\begin{itemize}[align=left]
\Invs{
    $$AA=(1,2)(3,4)(1,2)(3,4)=(1)$$
    $$BB=(1,3)(2,4)(1,3)(2,4)=(1)$$
    $$CC=(1,4)(2,3)(1,4)(2,3)=(1)$$
  }
\Clos{
    $$AB= (1,4)(2,3)$$
    $$BA= (1,4)(2,3)$$
    $$AC= (1,3)(2,4)$$
    $$CA= (1,3)(2,4)$$
    $$CB= (1,2)(3,4)$$
    $$BC= (1,2)(3,4)$$
  }
\end{itemize}
\end{mdframed}
\newpage
%%%%%%%%%%%%%%%%%%%%%%%%%%%%%%%%%%%%%%%%%%%%%%%%%%%%%%%%%%%%%%%%%%%%%%%%%%%%%%%%
%%%%%%%%%%%%%%%%%%%%%%%%%%%%%%%%%%%%%%%%%%%%%%%%%%%%%%%%%%%%%%%%%%%%%%%%%%%%%%%%
%%%%%%%%%%%%%%%%%%%%%%%%%%%%%%%%%%%%%%%%%%%%%%%%%%%%%%%%%%%%%%%%%%%%%%%%%%%%%%%%
%%%%%%%%%%%%%%%%%%%%%%%%%%%%%%%%%%%%%%%%%%%%%%%%%%%%%%%%%%%%%%%%%%%%%%%%%%%%%%%%
\begin{mdframed}[style=darkQuesion]
\begin{itemize}[align=left]
\item [(a)] {
    Show that
    $T=\lrs{\begin{bmatrix}
          a & 0 \\
          c & d\\
          \end{bmatrix}
          \vert a d \neq 0}$ is a subgroup of $G$.
  }
\item [(b)] {
    Show that
    $D=\lrs{\begin{bmatrix}
          a & 0 \\
          0 & d\\
          \end{bmatrix} \vert a d \neq 0}$
    is a subgroup of $G$.
  }
\end{itemize}
\end{mdframed}

%%%%%%%%%%%%%%%%%%%%%%%%%%%%%%%%%%%%%%%%%%%%%%%%%%%%%%%%%%%%%%%%%%%%%%%%%%%%%%%%
\begin{mdframed}[style=darkAnswer,frametitle={Joe Starr}]
\begin{itemize}[align=left]
\item [(a)]{
    \begin{itemize}[align=left]
    \Invs{We know by construction of $G$ there exist an inverse of the form $\begin{bmatrix}
          \frac{1}{a} & 0\\
          \frac{\m c}{ad} & \frac{1}{d}\\
          \end{bmatrix}$, by taking $\frac{1}{a}\frac{1}{d}$ that this is not $0$
        so the inverse is in $T$.
      }
    \Clos{
        If we take $A,B\in T$ $$AB=\begin{bmatrix}
          a & 0\\
          c & d\\
          \end{bmatrix}\begin{bmatrix}
          w & 0\\
          y & z\\
          \end{bmatrix}=\begin{bmatrix}
          aw & 0\\
          cw+dy & zd\\
          \end{bmatrix}$$
        since both $ad\neq0$ and $wz\neq 0$ it holds $adwz\neq 0$.
      }
    \end{itemize}
  }
\item [(b)]{
    \begin{itemize}[align=left]
    \Invs{We know by construction of $G$ there exist an inverse of the form $\begin{bmatrix}
          \frac{1}{a} & 0\\
          0 & \frac{1}{d}\\
          \end{bmatrix}$, by taking $\frac{1}{a}\frac{1}{d}$ that this is not $0$
        so the inverse is in $T$.
      }
    \Clos{
        If we take $A,B\in T$ $$AB=\begin{bmatrix}
          a & 0\\
          0 & d\\
          \end{bmatrix}\begin{bmatrix}
          w & 0\\
          0 & z\\
          \end{bmatrix}=\begin{bmatrix}
          aw & 0\\
          0 & zd\\
          \end{bmatrix}$$
        since both $ad\neq0$ and $wz\neq 0$ it holds $adwz\neq 0$.
      }
    \end{itemize}
  }
\end{itemize}
\end{mdframed}
\newpage
%%%%%%%%%%%%%%%%%%%%%%%%%%%%%%%%%%%%%%%%%%%%%%%%%%%%%%%%%%%%%%%%%%%%%%%%%%%%%%%%
%%%%%%%%%%%%%%%%%%%%%%%%%%%%%%%%%%%%%%%%%%%%%%%%%%%%%%%%%%%%%%%%%%%%%%%%%%%%%%%%
%%%%%%%%%%%%%%%%%%%%%%%%%%%%%%%%%%%%%%%%%%%%%%%%%%%%%%%%%%%%%%%%%%%%%%%%%%%%%%%%
%%%%%%%%%%%%%%%%%%%%%%%%%%%%%%%%%%%%%%%%%%%%%%%%%%%%%%%%%%%%%%%%%%%%%%%%%%%%%%%%
\begin{mdframed}[style=darkQuesion]
7. Let $G=\mathrm{GL}_{2}(\mathrm{R})$.
Show that the subset $S$ of $G$ defined by
$S=\left\{\left[\begin{array}{ll}a & b \\
      c & d\end{array}\right] | b=c\right\}$ of symmetric $2   \times 2$ matrices does not form a subgroup of $G .$

\end{mdframed}

%%%%%%%%%%%%%%%%%%%%%%%%%%%%%%%%%%%%%%%%%%%%%%%%%%%%%%%%%%%%%%%%%%%%%%%%%%%%%%%%
\begin{mdframed}[style=darkAnswer,frametitle={Joe Starr}]
Consider $$\begin{bmatrix}
  1 & 3\\
  3 & 5\\
  \end{bmatrix}\begin{bmatrix}
  1 & 3\\
  3 & 1\\
  \end{bmatrix}=\begin{bmatrix}
  10 & 6\\
  18 & 14\\
  \end{bmatrix}$$ showing this set is not closed.
\end{mdframed}
\newpage
%%%%%%%%%%%%%%%%%%%%%%%%%%%%%%%%%%%%%%%%%%%%%%%%%%%%%%%%%%%%%%%%%%%%%%%%%%%%%%%%
%%%%%%%%%%%%%%%%%%%%%%%%%%%%%%%%%%%%%%%%%%%%%%%%%%%%%%%%%%%%%%%%%%%%%%%%%%%%%%%%
%%%%%%%%%%%%%%%%%%%%%%%%%%%%%%%%%%%%%%%%%%%%%%%%%%%%%%%%%%%%%%%%%%%%%%%%%%%%%%%%
%%%%%%%%%%%%%%%%%%%%%%%%%%%%%%%%%%%%%%%%%%%%%%%%%%%%%%%%%%%%%%%%%%%%%%%%%%%%%%%%
\begin{mdframed}[style=darkQuesion]
8. Let $G=\mathrm{GL}_{2}(\mathrm{R}) .$ For each of the following subsets of $M_{2}(\mathbf{R}),$ determine whether
or not the subset is a subgroup of $G .$
\begin{itemize}[align=left]

\item []{(a) $A=\left\{\left[\begin{array}{ll}a & b \\ 0 & 0\end{array}\right] | a b \neq 0\right\}$}
\item []{(b) $B=\left\{\left[\begin{array}{ll}0 & b \\ c & 0\end{array}\right] | b c \neq 0\right\}$}
\item []{(c) $C=\left\{\left[\begin{array}{ll}1 & 0 \\ 0 & c\end{array}\right] | c \neq 0\right\}$}

\end{itemize}
\end{mdframed}

%%%%%%%%%%%%%%%%%%%%%%%%%%%%%%%%%%%%%%%%%%%%%%%%%%%%%%%%%%%%%%%%%%%%%%%%%%%%%%%%
\begin{mdframed}[style=darkAnswer,frametitle={Joe Starr}]
\begin{itemize}[align=left]
\item [(a)]{This set doesn't contain the identity so can not be a subgroup}
\item [(b)]{This set doesn't contain the identity so can not be a subgroup}
\item [(c)]{
    \begin{itemize}[align=left]
    \Invs{Let $\begin{bmatrix}
          1 & 0\\
          0 & c\\
          \end{bmatrix}$ with the given properties. We consider the inverse of $\begin{bmatrix}
          1 & 0\\
          0 & c\\
          \end{bmatrix}$ which is $\begin{bmatrix}
          1 & 0\\
          0 & \frac{1}{c}\\
          \end{bmatrix}$ which is in $C$;}
    \Clos{Let $\begin{bmatrix}
          1 & 0\\
          0 & c\\
          \end{bmatrix}$ and Let $\begin{bmatrix}
          1 & 0\\
          0 & a\\
          \end{bmatrix}$ with the given properties. Consider
        $$\begin{bmatrix}
          1 & 0\\
          0 & c\\
          \end{bmatrix}\begin{bmatrix}
          1 & 0\\
          0 & a\\
          \end{bmatrix}=\begin{bmatrix}
          1 & 0\\
          0 & ca\\
          \end{bmatrix}$$ We observe that $a\neq0$ and $c\neq 0$, consequently $ac\neq0$.
      }
    \end{itemize}
  }
\end{itemize}
\end{mdframed}
\newpage
%%%%%%%%%%%%%%%%%%%%%%%%%%%%%%%%%%%%%%%%%%%%%%%%%%%%%%%%%%%%%%%%%%%%%%%%%%%%%%%%
%%%%%%%%%%%%%%%%%%%%%%%%%%%%%%%%%%%%%%%%%%%%%%%%%%%%%%%%%%%%%%%%%%%%%%%%%%%%%%%%
%%%%%%%%%%%%%%%%%%%%%%%%%%%%%%%%%%%%%%%%%%%%%%%%%%%%%%%%%%%%%%%%%%%%%%%%%%%%%%%%
%%%%%%%%%%%%%%%%%%%%%%%%%%%%%%%%%%%%%%%%%%%%%%%%%%%%%%%%%%%%%%%%%%%%%%%%%%%%%%%%
\begin{mdframed}[style=darkQuesion]
9. Let $G=\mathrm{GL}_{3}(\mathbf{R}) .$ Show that $H=\left\{\left[\begin{array}{lll}1 & 0 & 0 \\ a & 1 & 0 \\ b & c & 1\end{array}\right]\right\}$ is a subgroup of $G .$

\end{mdframed}

%%%%%%%%%%%%%%%%%%%%%%%%%%%%%%%%%%%%%%%%%%%%%%%%%%%%%%%%%%%%%%%%%%%%%%%%%%%%%%%%
\begin{mdframed}[style=darkAnswer,frametitle={Joe Starr}]
\begin{itemize}[align=left]
\Invs{Let $\begin{bmatrix}
      1 & 0 & 0 \\
      a & 1 & 0 \\
      b & c & 1 \\
      \end{bmatrix}$ with the given properties. We consider the inverse of $\begin{bmatrix}
      1 & 0 & 0 \\
      a & 1 & 0 \\
      b & c & 1 \\
      \end{bmatrix}$ which is $\begin{bmatrix}
      1 & 0 & 0 \\
      \m a & 1 & 0 \\
      ac-b & \m c & 1 \\
      \end{bmatrix}$ which is in $H$;}
\Clos{Let $\begin{bmatrix}
      1 & 0 & 0 \\
      a & 1 & 0 \\
      b & c & 1 \\
      \end{bmatrix}$ and Let $\begin{bmatrix}
      1 & 0 & 0 \\
      x & 1 & 0 \\
      y & z & 1 \\
      \end{bmatrix}$ with the given properties. Consider
    $$\begin{bmatrix}
      1 & 0 & 0 \\
      a & 1 & 0 \\
      b & c & 0 \\
      \end{bmatrix}\begin{bmatrix}
      1 & 0 & 0 \\
      x & 1 & 0 \\
      y & z & 1 \\
      \end{bmatrix}=\begin{bmatrix}
      1 & 0 & 0 \\
      a+x & 1 & 0 \\
      b+cx+y & c+z & 1 \\
      \end{bmatrix}$$
  }
\end{itemize}
\end{mdframed}
\newpage%%%%%%%%%%%%%%%%%%%%%%%%%%%%%%%%%%%%%%%%%%%%%%%%%%%%%%%%%%%%%%%%%%%%%%%%%%%%%%%%
%%%%%%%%%%%%%%%%%%%%%%%%%%%%%%%%%%%%%%%%%%%%%%%%%%%%%%%%%%%%%%%%%%%%%%%%%%%%%%%%
%%%%%%%%%%%%%%%%%%%%%%%%%%%%%%%%%%%%%%%%%%%%%%%%%%%%%%%%%%%%%%%%%%%%%%%%%%%%%%%%
%%%%%%%%%%%%%%%%%%%%%%%%%%%%%%%%%%%%%%%%%%%%%%%%%%%%%%%%%%%%%%%%%%%%%%%%%%%%%%%%
\begin{mdframed}[style=darkQuesion]
10. Let $m$ and $n$ be nonzero integers, with ( $m, n$ ) = $d$. Show that $m$ and $n$ belong to $d \mathbf{Z},$ and that if $H$ is any subgroup of $\mathbf{Z}$ that contains both $m$ and $n,$ then $d \mathbf{Z} \subseteq H$
\end{mdframed}

%%%%%%%%%%%%%%%%%%%%%%%%%%%%%%%%%%%%%%%%%%%%%%%%%%%%%%%%%%%%%%%%%%%%%%%%%%%%%%%%
\begin{mdframed}[style=darkAnswer,frametitle={Joe Starr}]
We will first show that $m$ and $n$ are in $d\Z$. Let $m,n\in \Z$ with the given
properties. Since $\gcd\lrp{m,n}=d$ we observe that $dq_1=m$ and $dq_2=n$
making $m,n\in d\Z$ as desired.

Next let $H$ be a subgroup of $\Z$ with $m,n\in H$. Let $a\in d\Z$, with
$a<m\leq n$ by construction we have $a=dq$ for some $q$.
%TODO need to finish this one. 
\end{mdframed}
\newpage
%%%%%%%%%%%%%%%%%%%%%%%%%%%%%%%%%%%%%%%%%%%%%%%%%%%%%%%%%%%%%%%%%%%%%%%%%%%%%%%%
%%%%%%%%%%%%%%%%%%%%%%%%%%%%%%%%%%%%%%%%%%%%%%%%%%%%%%%%%%%%%%%%%%%%%%%%%%%%%%%%
%%%%%%%%%%%%%%%%%%%%%%%%%%%%%%%%%%%%%%%%%%%%%%%%%%%%%%%%%%%%%%%%%%%%%%%%%%%%%%%%
%%%%%%%%%%%%%%%%%%%%%%%%%%%%%%%%%%%%%%%%%%%%%%%%%%%%%%%%%%%%%%%%%%%%%%%%%%%%%%%%
\begin{mdframed}[style=darkQuesion]
11. Let $S$ be a set, and let $a$ be a fixed element of $S .$ Show that $\{\sigma \in \operatorname{Sym}(S) | \sigma(a)=a\}$ is a subgroup of $\operatorname{Sym}(S)$

\end{mdframed}

%%%%%%%%%%%%%%%%%%%%%%%%%%%%%%%%%%%%%%%%%%%%%%%%%%%%%%%%%%%%%%%%%%%%%%%%%%%%%%%%
\begin{mdframed}[style=darkAnswer,frametitle={Joe Starr}]
Let $A=\{\sigma \in \operatorname{Sym}(S) | \sigma(a)=a\}$
\begin{itemize}[align=left]
\Invs{
    By proposition 2.1.7 in the text we have inverses.
  }
\Clos{
    Let $\sigma,\varphi \in A$, consider the composition of these two functions
    around $a$, $\pof{\sof{a}}=\pof{a}=a$ showing closure, as desired.
  }
\end{itemize}
\end{mdframed}
\newpage
%%%%%%%%%%%%%%%%%%%%%%%%%%%%%%%%%%%%%%%%%%%%%%%%%%%%%%%%%%%%%%%%%%%%%%%%%%%%%%%%
%%%%%%%%%%%%%%%%%%%%%%%%%%%%%%%%%%%%%%%%%%%%%%%%%%%%%%%%%%%%%%%%%%%%%%%%%%%%%%%%
%%%%%%%%%%%%%%%%%%%%%%%%%%%%%%%%%%%%%%%%%%%%%%%%%%%%%%%%%%%%%%%%%%%%%%%%%%%%%%%%
%%%%%%%%%%%%%%%%%%%%%%%%%%%%%%%%%%%%%%%%%%%%%%%%%%%%%%%%%%%%%%%%%%%%%%%%%%%%%%%%
\begin{mdframed}[style=darkQuesion]
12. For each of the following groups, find all elements of finite order.
\begin{itemize}
\item []{
    (a) $\mathbf{R}^{\times}$
  }
\item []{
    (b) $\mathbf{C}^{\times}$
  }
\end{itemize}

\end{mdframed}

%%%%%%%%%%%%%%%%%%%%%%%%%%%%%%%%%%%%%%%%%%%%%%%%%%%%%%%%%%%%%%%%%%%%%%%%%%%%%%%%
\begin{mdframed}[style=darkAnswer,frametitle={Joe Starr}]
\begin{itemize}[align=left]
\item [(a)]{
    1 and $\m 1$ are the only elements of finite order.
  }d
\item [(b)]{
    $1,\m 1 ,\m i$, and $i$ are the only elements of
    finite order.
  }
\end{itemize}
\end{mdframed}
\newpage
%%%%%%%%%%%%%%%%%%%%%%%%%%%%%%%%%%%%%%%%%%%%%%%%%%%%%%%%%%%%%%%%%%%%%%%%%%%%%%%%
%%%%%%%%%%%%%%%%%%%%%%%%%%%%%%%%%%%%%%%%%%%%%%%%%%%%%%%%%%%%%%%%%%%%%%%%%%%%%%%%
%%%%%%%%%%%%%%%%%%%%%%%%%%%%%%%%%%%%%%%%%%%%%%%%%%%%%%%%%%%%%%%%%%%%%%%%%%%%%%%%
%%%%%%%%%%%%%%%%%%%%%%%%%%%%%%%%%%%%%%%%%%%%%%%%%%%%%%%%%%%%%%%%%%%%%%%%%%%%%%%%
\begin{mdframed}[style=darkQuesion]
13. Let $G$ be an abelian group, such that the operation on $G$
is denoted additively.
Show that $\{a \in G | 2 a=0\}$ is a subgroup of $G .$
Compute this subgroup for $G=\mathbf{Z}_{12}$

\end{mdframed}

%%%%%%%%%%%%%%%%%%%%%%%%%%%%%%%%%%%%%%%%%%%%%%%%%%%%%%%%%%%%%%%%%%%%%%%%%%%%%%%%
\begin{mdframed}[style=darkAnswer,frametitle={Joe Starr}]
Let $S=\{a \in G | 2 a=0\}$
\begin{itemize}[align=left]
\Invs{
    Let $a\in S$, by transitive proof $2a=0\rightarrow a+a=0\rightarrow a=-a$
  }
\Clos{
    Let $a,b\in S$, by transitive proof,
    \begin{align*}
    2a=0 &\rightarrow 2a+2b=0+0\\
    &\rightarrow 2(a+b)=0\\
    \end{align*}
    showing closure of $S$.
  }
\end{itemize}
next we let $G=\Z_{12}$ we can calculate $a+a$ for all $a\in \Z_{12}$
\begin{align*}
0+0 &= 0\\
1+1 &= 2\\
2+2 &= 4\\
3+3 &= 6\\
4+4 &= 8\\
5+5 &=10\\
6+6 &= 0\\
7+7 &= 2\\
8+8 &= 4\\
9+9 &= 6\\
10+10&= 8\\
\end{align*}
\end{mdframed}
\newpage
%%%%%%%%%%%%%%%%%%%%%%%%%%%%%%%%%%%%%%%%%%%%%%%%%%%%%%%%%%%%%%%%%%%%%%%%%%%%%%%%
%%%%%%%%%%%%%%%%%%%%%%%%%%%%%%%%%%%%%%%%%%%%%%%%%%%%%%%%%%%%%%%%%%%%%%%%%%%%%%%%
%%%%%%%%%%%%%%%%%%%%%%%%%%%%%%%%%%%%%%%%%%%%%%%%%%%%%%%%%%%%%%%%%%%%%%%%%%%%%%%%
%%%%%%%%%%%%%%%%%%%%%%%%%%%%%%%%%%%%%%%%%%%%%%%%%%%%%%%%%%%%%%%%%%%%%%%%%%%%%%%%
\begin{mdframed}[style=darkQuesion]
14. Let $G$ be an abelian group, and let $H$ be the set of all elements of $G$ of finite order.
\begin{itemize}
\item[]{
    (a) Show that $H$ is a subgroup of $G .$}
\item[]{
    (b) For a fixed positive integer $k$, show that $\{a \in G | o(a) \text{is a divisor of } k\}$ is a subgroup of $H$
  }
\item[]{
    (c) For a fixed positive integer $k$, is $\{a \in G | o(a) \leq k\}$ a subgroup of $H ?$ Either give a proof or give a counterexample.
  }
\end{itemize}

\end{mdframed}

%%%%%%%%%%%%%%%%%%%%%%%%%%%%%%%%%%%%%%%%%%%%%%%%%%%%%%%%%%%%%%%%%%%%%%%%%%%%%%%%
\begin{mdframed}[style=darkAnswer,frametitle={Joe Starr}]
\begin{itemize}[align=left]
\item[(a) ]{
    \begin{itemize}[align=left]
    \Invs{
        Let $a\in H$, since $a$ is of finite order we have $a^n=1$ for some
        $n\in \Z$. We observe that $aa^{n-1}=1$, now considering $a^{n-1}$,
        we take this to the nth power $\lrp{a^{n-1}}^n=\lrp{a^n}^{n-1}=1$
        showing existence of inverses in $H$.
      }
    \Clos{
        Let $a,b\in H $, we consider $ab$ if we know $a^n=1$ and $b^m=1$ for
        some $m,n\in \Z$, if we take $a^{mn}b^{mn}=\lrp{a^n}^m\lrp{b^m}^n=1$.
      }
    \end{itemize}
  }
\item[(b)]{
    Let $A=\{a \in G | o(a) \text{is a divisor of } k\}$
    \begin{itemize}[align=left]
    \Invs{
        Let $a\in A$ we observe that since $k \vert \ord{a}$ we have $a^{kq}=1$
        for some $kq\in \Z$ meaning $a^{kq-1}a=1$ now considering $a^{n-1}$,
        we take this to the nth power $\lrp{a^{kq-1}}^{kq}=\lrp{a^{kq}}^{kq-1}=1$
        showing existence of inverses in $A$.
      }
    \Clos{
        Let $a,b\in A $, we consider $ab$ if we know $a^{kn}=1$ and $b^{km}=1$ for
        some $m,n\in \Z$, if we take $a^{kmn}b^{kmn}=\lrp{a^kn}^m\lrp{b^km}^n=1$.
      }
    \end{itemize}
  }
\item[(c)]{
    Let $G=\Z_{10}^+$, and $k=5$, this makes $A=\lrs{2,\ 4,\ 5,\ 6,\ 8}$ if we
    take $2+5=7$ we can see $A$ is not closed under addition.
  }
\end{itemize}
\end{mdframed}
\newpage
%%%%%%%%%%%%%%%%%%%%%%%%%%%%%%%%%%%%%%%%%%%%%%%%%%%%%%%%%%%%%%%%%%%%%%%%%%%%%%%%
%%%%%%%%%%%%%%%%%%%%%%%%%%%%%%%%%%%%%%%%%%%%%%%%%%%%%%%%%%%%%%%%%%%%%%%%%%%%%%%%
%%%%%%%%%%%%%%%%%%%%%%%%%%%%%%%%%%%%%%%%%%%%%%%%%%%%%%%%%%%%%%%%%%%%%%%%%%%%%%%%
%%%%%%%%%%%%%%%%%%%%%%%%%%%%%%%%%%%%%%%%%%%%%%%%%%%%%%%%%%%%%%%%%%%%%%%%%%%%%%%%
\begin{mdframed}[style=darkQuesion]
15. Prove that any cyclic group is abelian.
\end{mdframed}

%%%%%%%%%%%%%%%%%%%%%%%%%%%%%%%%%%%%%%%%%%%%%%%%%%%%%%%%%%%%%%%%%%%%%%%%%%%%%%%%
\begin{mdframed}[style=darkAnswer,frametitle={Joe Starr}]
Let $G$ be a cyclic group generated by $g$, select $a,b\in G$, we consider
$ab\inv{a}\inv{b}$. We observe $a=g^k$ and $\inv{a}=g^{k-1}$ for some $k$,
similarly for $b$ and some $h$. Now rewriting
$ab\inv{a}\inv{b}=g^kg^hg^{k-1}g^{h-1}=1$ showing $G$ abelian.
\end{mdframed}
\newpage
%%%%%%%%%%%%%%%%%%%%%%%%%%%%%%%%%%%%%%%%%%%%%%%%%%%%%%%%%%%%%%%%%%%%%%%%%%%%%%%%
%%%%%%%%%%%%%%%%%%%%%%%%%%%%%%%%%%%%%%%%%%%%%%%%%%%%%%%%%%%%%%%%%%%%%%%%%%%%%%%%
%%%%%%%%%%%%%%%%%%%%%%%%%%%%%%%%%%%%%%%%%%%%%%%%%%%%%%%%%%%%%%%%%%%%%%%%%%%%%%%%
%%%%%%%%%%%%%%%%%%%%%%%%%%%%%%%%%%%%%%%%%%%%%%%%%%%%%%%%%%%%%%%%%%%%%%%%%%%%%%%%
\begin{mdframed}[style=darkQuesion]
16. Prove or disprove this statement. If $G$ is a group in which every proper subgroup is cyclic, then $G$ is cyclic.

\end{mdframed}

%%%%%%%%%%%%%%%%%%%%%%%%%%%%%%%%%%%%%%%%%%%%%%%%%%%%%%%%%%%%%%%%%%%%%%%%%%%%%%%%
\begin{mdframed}[style=darkAnswer,frametitle={Joe Starr}]
Select $G$ with the given property. Let $H$

\end{mdframed}
\newpage
%%%%%%%%%%%%%%%%%%%%%%%%%%%%%%%%%%%%%%%%%%%%%%%%%%%%%%%%%%%%%%%%%%%%%%%%%%%%%%%%
%%%%%%%%%%%%%%%%%%%%%%%%%%%%%%%%%%%%%%%%%%%%%%%%%%%%%%%%%%%%%%%%%%%%%%%%%%%%%%%%
%%%%%%%%%%%%%%%%%%%%%%%%%%%%%%%%%%%%%%%%%%%%%%%%%%%%%%%%%%%%%%%%%%%%%%%%%%%%%%%%
%%%%%%%%%%%%%%%%%%%%%%%%%%%%%%%%%%%%%%%%%%%%%%%%%%%%%%%%%%%%%%%%%%%%%%%%%%%%%%%%
\begin{mdframed}[style=darkQuesion]
17. Prove that the intersection of any collection of subgroups of a group is again a subgroup.

\end{mdframed}

%%%%%%%%%%%%%%%%%%%%%%%%%%%%%%%%%%%%%%%%%%%%%%%%%%%%%%%%%%%%%%%%%%%%%%%%%%%%%%%%
\begin{mdframed}[style=darkAnswer,frametitle={Joe Starr}]
Let $H$ and $K$ be subgroups of a group $G$, consider $a\in H\cap K$. Since
both $K$ and $H$ are groups $\inv{a} \in H$ and $\inv{a} \in K$ putting it in
the intersection. Next we consider $a,b\in H\cap K$, since
both $K$ and $H$ are groups $ab \in H$ and $ab \in K$ putting it in
the intersection. Showing $ H\cap K$ a subgroup.
\end{mdframed}
\newpage
%%%%%%%%%%%%%%%%%%%%%%%%%%%%%%%%%%%%%%%%%%%%%%%%%%%%%%%%%%%%%%%%%%%%%%%%%%%%%%%%
%%%%%%%%%%%%%%%%%%%%%%%%%%%%%%%%%%%%%%%%%%%%%%%%%%%%%%%%%%%%%%%%%%%%%%%%%%%%%%%%
%%%%%%%%%%%%%%%%%%%%%%%%%%%%%%%%%%%%%%%%%%%%%%%%%%%%%%%%%%%%%%%%%%%%%%%%%%%%%%%%
%%%%%%%%%%%%%%%%%%%%%%%%%%%%%%%%%%%%%%%%%%%%%%%%%%%%%%%%%%%%%%%%%%%%%%%%%%%%%%%%
\begin{mdframed}[style=darkQuesion]
18. Let $G$ be the group of rational numbers, under addition, and let $H, K$ be subgroups
of $G .$ Prove that if $H \neq\{0\}$ and $K \neq\{0\}$, then $H \cap K \neq\{0\}$.

\end{mdframed}

%%%%%%%%%%%%%%%%%%%%%%%%%%%%%%%%%%%%%%%%%%%%%%%%%%%%%%%%%%%%%%%%%%%%%%%%%%%%%%%%
\begin{mdframed}[style=darkAnswer,frametitle={Joe Starr}]
We have previously shown that the intersection of two subgroups is a subgroup.
Since neither $H$ nor $K $ are the trivial subgroup we have $\frac{a}{b}\in H$
and $\frac{m}{n}\in K$, $a,\ b,\ m,\ n$ with the usual properties. We observe
$b\frac{a}{b}\in H$ and $n\frac{m}{n}\in K$, we then can add these $m$ and
$a$ times respectively, yielding $bm\frac{a}{b}=ma$ and $na\frac{m}{n}=ma$
putting $ma\in H\cap K$.
\end{mdframed}
\newpage
%%%%%%%%%%%%%%%%%%%%%%%%%%%%%%%%%%%%%%%%%%%%%%%%%%%%%%%%%%%%%%%%%%%%%%%%%%%%%%%%
%%%%%%%%%%%%%%%%%%%%%%%%%%%%%%%%%%%%%%%%%%%%%%%%%%%%%%%%%%%%%%%%%%%%%%%%%%%%%%%%
%%%%%%%%%%%%%%%%%%%%%%%%%%%%%%%%%%%%%%%%%%%%%%%%%%%%%%%%%%%%%%%%%%%%%%%%%%%%%%%%
%%%%%%%%%%%%%%%%%%%%%%%%%%%%%%%%%%%%%%%%%%%%%%%%%%%%%%%%%%%%%%%%%%%%%%%%%%%%%%%%
\begin{mdframed}[style=darkQuesion]
19. Let $G$ be a group, and let $a \in G .$ The set
$C(a)=\{x \in G | x a=a x\}$ of all elements of $G$ that commute with $a$ is
called the centralizer of $a .$
\begin{itemize}
\item[]{(a) Show that  $C(a)$ is a subgroup of $G$}
\item[]{(b) Show that $\langle a\rangle \subseteq C(a)$}
\item[]{(c) Compute $C(a)$ if $G=S_{3}$ and $a=(1,2,3)$}
\item[]{(d) Compute $C(a)$ if $G=S_{3}$ and $a=(1,2)$}
\end{itemize}

\end{mdframed}

%%%%%%%%%%%%%%%%%%%%%%%%%%%%%%%%%%%%%%%%%%%%%%%%%%%%%%%%%%%%%%%%%%%%%%%%%%%%%%%%
\begin{mdframed}[style=darkAnswer,frametitle={Joe Starr}]
\begin{itemize}[align=left]
\item[(a)]{
    \begin{itemize}[align=left]
    \Invs{
        Let $x\in C\lrp{a}$, by construction we know $xa=ax$, by transitive proof
        \begin{align*}
        xa=ax &\rightarrow \inv{x}xa=\inv{x}ax\\
        &\rightarrow a\inv{x}=\inv{x}ax\inv{x}\\
        &\rightarrow a\inv{x}=\inv{x}a
        \end{align*}
        putting $\inv{x}\in C\lrp{a}$.
      }
    \Clos{
        let $x,\ y\in C\lrp{a}$, by construction we know $xa=ax$ and $ya=ay$.
        if we take $xa=ax$ and multiply by $y$ on the left we get $yxa=yax$
        then by commutativity we have $yxa=ayx$ putting $yx\in C\lrp{a}$
        similarly for $xy$.
      }
    \end{itemize}}
\item[(b)]{
    If we consider $x\in \lra{a}$, observe by construction $x=a^n$ for
    some $n$. If we take $xa=a^na=a^{n+1}=aa^n=ax$ putting $x\in C\lrp{a}$.
  }
\item[(c)]{Since $a$ is the identity element $C\lrp{a}=S_3$}
\item[(d)]{
          $C\lrp{a}=\lrs{\lrp{1,2,3},\ \lrp{1,2}}$
  }
\end{itemize}
\end{mdframed}
\newpage
%%%%%%%%%%%%%%%%%%%%%%%%%%%%%%%%%%%%%%%%%%%%%%%%%%%%%%%%%%%%%%%%%%%%%%%%%%%%%%%%
%%%%%%%%%%%%%%%%%%%%%%%%%%%%%%%%%%%%%%%%%%%%%%%%%%%%%%%%%%%%%%%%%%%%%%%%%%%%%%%%
%%%%%%%%%%%%%%%%%%%%%%%%%%%%%%%%%%%%%%%%%%%%%%%%%%%%%%%%%%%%%%%%%%%%%%%%%%%%%%%%
%%%%%%%%%%%%%%%%%%%%%%%%%%%%%%%%%%%%%%%%%%%%%%%%%%%%%%%%%%%%%%%%%%%%%%%%%%%%%%%%
\begin{mdframed}[style=darkQuesion]
20. Compute the centralizer in $\mathrm{GL}_{2}$ ( $\mathbf{R}$ ) of the matrix $\left[\begin{array}{ll}1 & 1 \\ 0 & 1\end{array}\right]$

\end{mdframed}

%%%%%%%%%%%%%%%%%%%%%%%%%%%%%%%%%%%%%%%%%%%%%%%%%%%%%%%%%%%%%%%%%%%%%%%%%%%%%%%%
\begin{mdframed}[style=darkAnswer,frametitle={Joe Starr}]
We can calculate $\begin{bmatrix}
  a & b\\
  c & d\\
\end{bmatrix}\begin{bmatrix}
  1 & 1\\
  0 & 1\\
\end{bmatrix}=\begin{bmatrix}
  a & a+b\\
  c & c+d\\
\end{bmatrix}$
and $\begin{bmatrix}
  1 & 1\\
  0 & 1\\
\end{bmatrix}\begin{bmatrix}
  a & b\\
  c & d\\
\end{bmatrix}=\begin{bmatrix}
  a+c & b+d\\
  c & d\\
\end{bmatrix}$. From here we observe that for a matrix to commute it must 
satisfy $c=c,\ a+c=a,\ d+b=a+b,\ c+d=d$ making it of the form $\begin{bmatrix}
  a & d-a\\
  0 & d\\
\end{bmatrix}$.
\end{mdframed}
\newpage
%%%%%%%%%%%%%%%%%%%%%%%%%%%%%%%%%%%%%%%%%%%%%%%%%%%%%%%%%%%%%%%%%%%%%%%%%%%%%%%%
%%%%%%%%%%%%%%%%%%%%%%%%%%%%%%%%%%%%%%%%%%%%%%%%%%%%%%%%%%%%%%%%%%%%%%%%%%%%%%%%
%%%%%%%%%%%%%%%%%%%%%%%%%%%%%%%%%%%%%%%%%%%%%%%%%%%%%%%%%%%%%%%%%%%%%%%%%%%%%%%%
%%%%%%%%%%%%%%%%%%%%%%%%%%%%%%%%%%%%%%%%%%%%%%%%%%%%%%%%%%%%%%%%%%%%%%%%%%%%%%%%
\begin{mdframed}[style=darkQuesion]
22. Show that if a group $G$ has a unique element $a$ of order $2,$ then $a \in Z(G)$

\end{mdframed}

%%%%%%%%%%%%%%%%%%%%%%%%%%%%%%%%%%%%%%%%%%%%%%%%%%%%%%%%%%%%%%%%%%%%%%%%%%%%%%%%
\begin{mdframed}[style=darkAnswer,frametitle={Joe Starr}]
 %TODO not done yet
\end{mdframed}
\newpage
%%%%%%%%%%%%%%%%%%%%%%%%%%%%%%%%%%%%%%%%%%%%%%%%%%%%%%%%%%%%%%%%%%%%%%%%%%%%%%%%
%%%%%%%%%%%%%%%%%%%%%%%%%%%%%%%%%%%%%%%%%%%%%%%%%%%%%%%%%%%%%%%%%%%%%%%%%%%%%%%%
%%%%%%%%%%%%%%%%%%%%%%%%%%%%%%%%%%%%%%%%%%%%%%%%%%%%%%%%%%%%%%%%%%%%%%%%%%%%%%%%
%%%%%%%%%%%%%%%%%%%%%%%%%%%%%%%%%%%%%%%%%%%%%%%%%%%%%%%%%%%%%%%%%%%%%%%%%%%%%%%%
\begin{mdframed}[style=darkQuesion]
23. If the group $G$ is not abelian, show that its center $Z(G)$ is a proper subgroup of an abelian subgroup of $G .$

\end{mdframed}

%%%%%%%%%%%%%%%%%%%%%%%%%%%%%%%%%%%%%%%%%%%%%%%%%%%%%%%%%%%%%%%%%%%%%%%%%%%%%%%%
\begin{mdframed}[style=darkAnswer,frametitle={Joe Starr}]
 %TODO not done yet
\end{mdframed}
\newpage
%%%%%%%%%%%%%%%%%%%%%%%%%%%%%%%%%%%%%%%%%%%%%%%%%%%%%%%%%%%%%%%%%%%%%%%%%%%%%%%%
%%%%%%%%%%%%%%%%%%%%%%%%%%%%%%%%%%%%%%%%%%%%%%%%%%%%%%%%%%%%%%%%%%%%%%%%%%%%%%%%
%%%%%%%%%%%%%%%%%%%%%%%%%%%%%%%%%%%%%%%%%%%%%%%%%%%%%%%%%%%%%%%%%%%%%%%%%%%%%%%%
%%%%%%%%%%%%%%%%%%%%%%%%%%%%%%%%%%%%%%%%%%%%%%%%%%%%%%%%%%%%%%%%%%%%%%%%%%%%%%%%
\begin{mdframed}[style=darkQuesion]
26. Let $G$ be a group with $a, b \in G$.
(a) Show that $o\left(a^{-1}\right)=o(a)$
(b) Show that $o(a b)=o(b a)$
(c) Show that $o\left(a b a^{-1}\right)=o(b)$

\end{mdframed}

%%%%%%%%%%%%%%%%%%%%%%%%%%%%%%%%%%%%%%%%%%%%%%%%%%%%%%%%%%%%%%%%%%%%%%%%%%%%%%%%
\begin{mdframed}[style=darkAnswer,frametitle={Joe Starr}]
 %TODO not done yet
\end{mdframed}
\newpage
%%%%%%%%%%%%%%%%%%%%%%%%%%%%%%%%%%%%%%%%%%%%%%%%%%%%%%%%%%%%%%%%%%%%%%%%%%%%%%%%
%%%%%%%%%%%%%%%%%%%%%%%%%%%%%%%%%%%%%%%%%%%%%%%%%%%%%%%%%%%%%%%%%%%%%%%%%%%%%%%%
%%%%%%%%%%%%%%%%%%%%%%%%%%%%%%%%%%%%%%%%%%%%%%%%%%%%%%%%%%%%%%%%%%%%%%%%%%%%%%%%
%%%%%%%%%%%%%%%%%%%%%%%%%%%%%%%%%%%%%%%%%%%%%%%%%%%%%%%%%%%%%%%%%%%%%%%%%%%%%%%%
\begin{mdframed}[style=darkQuesion]
27. Let $G$ be a finite group, let $n>2$ be an integer, and let $S$ be the set of elements of
$G$ that have order $n .$ Show that $S$ has an even number of elements.

\end{mdframed}

%%%%%%%%%%%%%%%%%%%%%%%%%%%%%%%%%%%%%%%%%%%%%%%%%%%%%%%%%%%%%%%%%%%%%%%%%%%%%%%%
\begin{mdframed}[style=darkAnswer,frametitle={Joe Starr}]
 %TODO not done yet
\end{mdframed}
\newpage
%%%%%%%%%%%%%%%%%%%%%%%%%%%%%%%%%%%%%%%%%%%%%%%%%%%%%%%%%%%%%%%%%%%%%%%%%%%%%%%%
%%%%%%%%%%%%%%%%%%%%%%%%%%%%%%%%%%%%%%%%%%%%%%%%%%%%%%%%%%%%%%%%%%%%%%%%%%%%%%%%
%%%%%%%%%%%%%%%%%%%%%%%%%%%%%%%%%%%%%%%%%%%%%%%%%%%%%%%%%%%%%%%%%%%%%%%%%%%%%%%%
%%%%%%%%%%%%%%%%%%%%%%%%%%%%%%%%%%%%%%%%%%%%%%%%%%%%%%%%%%%%%%%%%%%%%%%%%%%%%%%%
\begin{mdframed}[style=darkQuesion]
28. Let $G$ be a group with $a, b \in G$. Assume that $o(a)$ and $o(b)$ are finite and relatively prime, and that $a b=b a$. Show that $o(a b)=o(a) o(b)$

\end{mdframed}

%%%%%%%%%%%%%%%%%%%%%%%%%%%%%%%%%%%%%%%%%%%%%%%%%%%%%%%%%%%%%%%%%%%%%%%%%%%%%%%%
\begin{mdframed}[style=darkAnswer,frametitle={Joe Starr}]
 %TODO not done yet
\end{mdframed}
\newpage
%%%%%%%%%%%%%%%%%%%%%%%%%%%%%%%%%%%%%%%%%%%%%%%%%%%%%%%%%%%%%%%%%%%%%%%%%%%%%%%%
%%%%%%%%%%%%%%%%%%%%%%%%%%%%%%%%%%%%%%%%%%%%%%%%%%%%%%%%%%%%%%%%%%%%%%%%%%%%%%%%
%%%%%%%%%%%%%%%%%%%%%%%%%%%%%%%%%%%%%%%%%%%%%%%%%%%%%%%%%%%%%%%%%%%%%%%%%%%%%%%%
%%%%%%%%%%%%%%%%%%%%%%%%%%%%%%%%%%%%%%%%%%%%%%%%%%%%%%%%%%%%%%%%%%%%%%%%%%%%%%%%
\begin{mdframed}[style=darkQuesion]
29. Find an example of a group $G$ and elements $a, b \in G$ such that $a$ and $b$ each have finite order but $a b$ does not.

\end{mdframed}

%%%%%%%%%%%%%%%%%%%%%%%%%%%%%%%%%%%%%%%%%%%%%%%%%%%%%%%%%%%%%%%%%%%%%%%%%%%%%%%%
\begin{mdframed}[style=darkAnswer,frametitle={Joe Starr}]
 %TODO not done yet
\end{mdframed}
%TODO: Section 3.3
\newpage
%%%%%%%%%%%%%%%%%%%%%%%%%%%%%%%%%%%%%%%%%%%%%%%%%%%%%%%%%%%%%%%%%%%%%%%%%%%%%%%%
%%%%%%%%%%%%%%%%%%%%%%%%%%%%%%%%%%%%%%%%%%%%%%%%%%%%%%%%%%%%%%%%%%%%%%%%%%%%%%%%
%%%%%%%%%%%%%%%%%%%%%%%%%%%%%%%%%%%%%%%%%%%%%%%%%%%%%%%%%%%%%%%%%%%%%%%%%%%%%%%%
%%%%%%%%%%%%%%%%%%%%%%%%%%%%%%%%%%%%%%%%%%%%%%%%%%%%%%%%%%%%%%%%%%%%%%%%%%%%%%%%
\begin{mdframed}[style=darkQuesion]
2.
\end{mdframed}

%%%%%%%%%%%%%%%%%%%%%%%%%%%%%%%%%%%%%%%%%%%%%%%%%%%%%%%%%%%%%%%%%%%%%%%%%%%%%%%%
\begin{mdframed}[style=darkAnswer,frametitle={Joe Starr}]
 %TODO not done yet
\end{mdframed}
\newpage
%%%%%%%%%%%%%%%%%%%%%%%%%%%%%%%%%%%%%%%%%%%%%%%%%%%%%%%%%%%%%%%%%%%%%%%%%%%%%%%%
%%%%%%%%%%%%%%%%%%%%%%%%%%%%%%%%%%%%%%%%%%%%%%%%%%%%%%%%%%%%%%%%%%%%%%%%%%%%%%%%
%%%%%%%%%%%%%%%%%%%%%%%%%%%%%%%%%%%%%%%%%%%%%%%%%%%%%%%%%%%%%%%%%%%%%%%%%%%%%%%%
%%%%%%%%%%%%%%%%%%%%%%%%%%%%%%%%%%%%%%%%%%%%%%%%%%%%%%%%%%%%%%%%%%%%%%%%%%%%%%%%
\begin{mdframed}[style=darkQuesion]
2.
\end{mdframed}

%%%%%%%%%%%%%%%%%%%%%%%%%%%%%%%%%%%%%%%%%%%%%%%%%%%%%%%%%%%%%%%%%%%%%%%%%%%%%%%%
\begin{mdframed}[style=darkAnswer,frametitle={Joe Starr}]
 %TODO not done yet
\end{mdframed}