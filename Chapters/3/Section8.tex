\subsection{Cyclic Groups}
%%%%%%%%%%%%%%%%%%%%%%%%%%%%%%%%%%%%%%%%%%%%%%%%%%%%%%%%%%%%%%%%%%%%%%%%%%%%%%%%
%%%%%%%%%%%%%%%%%%%%%%%%%%%%%%%%%%%%%%%%%%%%%%%%%%%%%%%%%%%%%%%%%%%%%%%%%%%%%%%%
%%%%%%%%%%%%%%%%%%%%%%%%%%%%%%%%%%%%%%%%%%%%%%%%%%%%%%%%%%%%%%%%%%%%%%%%%%%%%%%%
%%%%%%%%%%%%%%%%%%%%%%%%%%%%%%%%%%%%%%%%%%%%%%%%%%%%%%%%%%%%%%%%%%%%%%%%%%%%%%%%
\begin{mdframed}[style=darkQuesion]
  1. Let $G$ be a group and let $a \in G$ be an element of order $12 .$ 
  What is the order of $a^{j}$ for $j=2, \ldots, 11 ?$
\end{mdframed}
%%%%%%%%%%%%%%%%%%%%%%%%%%%%%%%%%%%%%%%%%%%%%%%%%%%%%%%%%%%%%%%%%%%%%%%%%%%%%%%%
\begin{mdframed}[style=darkAnswer,frametitle={Joe Starr}]
  By applying 3.5.3 and 3.5.4 we get
$\lrp{a^{1}}^{12},\lrp{a^{2}}^{6},\lrp{a^{3}}^{4},\lrp{a^{4}}^{3},\lrp{a^{5}}^{12},
\lrp{a^{6}}^{2},\lrp{a^{7}}^{12},\lrp{a^{8}}^{3},\\\lrp{a^{9}}^{4},\lrp{a^{10}}^{6},
\lrp{a^{11}}^{12}$ 
\end{mdframed}
\newpage
%%%%%%%%%%%%%%%%%%%%%%%%%%%%%%%%%%%%%%%%%%%%%%%%%%%%%%%%%%%%%%%%%%%%%%%%%%%%%%%%
%%%%%%%%%%%%%%%%%%%%%%%%%%%%%%%%%%%%%%%%%%%%%%%%%%%%%%%%%%%%%%%%%%%%%%%%%%%%%%%%
%%%%%%%%%%%%%%%%%%%%%%%%%%%%%%%%%%%%%%%%%%%%%%%%%%%%%%%%%%%%%%%%%%%%%%%%%%%%%%%%
%%%%%%%%%%%%%%%%%%%%%%%%%%%%%%%%%%%%%%%%%%%%%%%%%%%%%%%%%%%%%%%%%%%%%%%%%%%%%%%%
\begin{mdframed}[style=darkQuesion]
2. Let $G$ be a group and let $a \in G$ be an element of order $30 .$ 
List the powers of $a$ that have order $2,$ order 3 or order 5
\end{mdframed}
%%%%%%%%%%%%%%%%%%%%%%%%%%%%%%%%%%%%%%%%%%%%%%%%%%%%%%%%%%%%%%%%%%%%%%%%%%%%%%%%
\begin{mdframed}[style=darkAnswer,frametitle={Joe Starr}]
\begin{align*}
  2&: 15\\
  3&: 10, 20\\
  5&: 6,12,18,24\\
\end{align*}
\end{mdframed}
\newpage
%%%%%%%%%%%%%%%%%%%%%%%%%%%%%%%%%%%%%%%%%%%%%%%%%%%%%%%%%%%%%%%%%%%%%%%%%%%%%%%%
%%%%%%%%%%%%%%%%%%%%%%%%%%%%%%%%%%%%%%%%%%%%%%%%%%%%%%%%%%%%%%%%%%%%%%%%%%%%%%%%
%%%%%%%%%%%%%%%%%%%%%%%%%%%%%%%%%%%%%%%%%%%%%%%%%%%%%%%%%%%%%%%%%%%%%%%%%%%%%%%%
%%%%%%%%%%%%%%%%%%%%%%%%%%%%%%%%%%%%%%%%%%%%%%%%%%%%%%%%%%%%%%%%%%%%%%%%%%%%%%%%
\begin{mdframed}[style=darkQuesion]
5. Find the cyclic subgroup of $C^{\times}$ generated by $\frac{\sqrt{2}+\sqrt{2}i}{2}$.
\end{mdframed}
%%%%%%%%%%%%%%%%%%%%%%%%%%%%%%%%%%%%%%%%%%%%%%%%%%%%%%%%%%%%%%%%%%%%%%%%%%%%%%%%
\begin{mdframed}[style=darkAnswer,frametitle={Joe Starr}]
 \begin{align*}
   a^{1}&=\frac{\sqrt{2}+\sqrt{2}i}{2}\\
   a^{2}&=\frac{\sqrt{2}+\sqrt{2}i}{2}\frac{\sqrt{2}+\sqrt{2}i}{2}=i\\
   a^{4}&=i=-1\\
   a^{8}&=i=1\\
 \end{align*}
\end{mdframed}
\newpage
%%%%%%%%%%%%%%%%%%%%%%%%%%%%%%%%%%%%%%%%%%%%%%%%%%%%%%%%%%%%%%%%%%%%%%%%%%%%%%%%
%%%%%%%%%%%%%%%%%%%%%%%%%%%%%%%%%%%%%%%%%%%%%%%%%%%%%%%%%%%%%%%%%%%%%%%%%%%%%%%%
%%%%%%%%%%%%%%%%%%%%%%%%%%%%%%%%%%%%%%%%%%%%%%%%%%%%%%%%%%%%%%%%%%%%%%%%%%%%%%%%
%%%%%%%%%%%%%%%%%%%%%%%%%%%%%%%%%%%%%%%%%%%%%%%%%%%%%%%%%%%%%%%%%%%%%%%%%%%%%%%%
\begin{mdframed}[style=darkQuesion]
6. Find the order of the cyclic subgroup of $\mathbf{C}^{\times}$ generated by $1+i$
\end{mdframed}
%%%%%%%%%%%%%%%%%%%%%%%%%%%%%%%%%%%%%%%%%%%%%%%%%%%%%%%%%%%%%%%%%%%%%%%%%%%%%%%%
\begin{mdframed}[style=darkAnswer,frametitle={Joe Starr}]
 \begin{align*}
  a^{1}&=1+i\\
  a^{2}&=\lrp{1+i}\lrp{1+i}=2i\\
  a^{4}&=-4\\
  a^{8}&=16\\
  a^{2^{3+n}}&={2^{4+n}}\\
 \end{align*}
 we observe $2^{4+n}$ has infinite order so $a$ must have infinite order. 
\end{mdframed}
\newpage
%%%%%%%%%%%%%%%%%%%%%%%%%%%%%%%%%%%%%%%%%%%%%%%%%%%%%%%%%%%%%%%%%%%%%%%%%%%%%%%%
%%%%%%%%%%%%%%%%%%%%%%%%%%%%%%%%%%%%%%%%%%%%%%%%%%%%%%%%%%%%%%%%%%%%%%%%%%%%%%%%
%%%%%%%%%%%%%%%%%%%%%%%%%%%%%%%%%%%%%%%%%%%%%%%%%%%%%%%%%%%%%%%%%%%%%%%%%%%%%%%%
%%%%%%%%%%%%%%%%%%%%%%%%%%%%%%%%%%%%%%%%%%%%%%%%%%%%%%%%%%%%%%%%%%%%%%%%%%%%%%%%
\begin{mdframed}[style=darkQuesion]
7. Which of the multiplicative groups $\mathbf{Z}_{15}^{\times}, \mathbf{Z}_{18}^{\times}, \mathbf{Z}_{20}^{\times}, \mathbf{Z}_{27}^{\times}$ are cyclic?
\end{mdframed}
%%%%%%%%%%%%%%%%%%%%%%%%%%%%%%%%%%%%%%%%%%%%%%%%%%%%%%%%%%%%%%%%%%%%%%%%%%%%%%%%
\begin{mdframed}[style=darkAnswer,frametitle={Joe Starr}]
 \begin{align*}
  \mathbf{Z}_{15}^{\times}&:15=5\cdot3 \text{ The product of two odd primes, not cyclic}\\
  \mathbf{Z}_{18}^{\times}&:18=2\cdot3^2 \text{ The product of two and power of an odd prime, cyclic}\\
  \mathbf{Z}_{20}^{\times}&:20=2^2\cdot5 \text{ The product of four and an odd prime, not cyclic}\\
  \mathbf{Z}_{27}^{\times}&:15=3^3 \text{ The power of an odd prime, cyclic}\\
 \end{align*}
\end{mdframed}
\newpage
%%%%%%%%%%%%%%%%%%%%%%%%%%%%%%%%%%%%%%%%%%%%%%%%%%%%%%%%%%%%%%%%%%%%%%%%%%%%%%%%
%%%%%%%%%%%%%%%%%%%%%%%%%%%%%%%%%%%%%%%%%%%%%%%%%%%%%%%%%%%%%%%%%%%%%%%%%%%%%%%%
%%%%%%%%%%%%%%%%%%%%%%%%%%%%%%%%%%%%%%%%%%%%%%%%%%%%%%%%%%%%%%%%%%%%%%%%%%%%%%%%
%%%%%%%%%%%%%%%%%%%%%%%%%%%%%%%%%%%%%%%%%%%%%%%%%%%%%%%%%%%%%%%%%%%%%%%%%%%%%%%%
\begin{mdframed}[style=darkQuesion]
11. Which of the multiplicative groups  $\Z^{\times}_{7},\ \Z^{\times}_{10},\ \Z^{\times}_{12},\ \Z^{\times}_{14}$ are isomorphic.
\end{mdframed}
%%%%%%%%%%%%%%%%%%%%%%%%%%%%%%%%%%%%%%%%%%%%%%%%%%%%%%%%%%%%%%%%%%%%%%%%%%%%%%%%
\begin{mdframed}[style=darkAnswer,frametitle={Joe Starr}]
  \begin{align*}
    \mathbf{\Z}_{7 }^{\times}&:7 =7 \text{ The power of an odd prime, cyclic}\\
    \mathbf{\Z}_{10}^{\times}&:10=2\cdot5 \text{ The product of two and power of an odd prime, cyclic}\\
    \mathbf{\Z}_{12}^{\times}&:12=2^2\cdot3\cdot5 \text{ The product of four and an odd prime, not cyclic}\\
    \mathbf{\Z}_{14}^{\times}&:14=2\cdot7 \text{ The product of two and power of an odd prime, cyclic}\\
   \hline \\
    \mathbf{\Z}_{7 }^{\times}&=\lrs{1, 2, 3, 4, 5, 6} \\
    \mathbf{\Z}_{10}^{\times}&=\lrs{1, 3, 7, 9} \\
    \mathbf{\Z}_{14}^{\times}&=\lrs{1, 3, 5, 9, 11, 13} \\
   \end{align*}
   Since $\Z_{14}^{\times}$ has order six and and $\Zmx{7}$ has order six, 
   and both are cyclic they must both be isomorphic to $\Zm{6}$.

\end{mdframed}
\newpage
%%%%%%%%%%%%%%%%%%%%%%%%%%%%%%%%%%%%%%%%%%%%%%%%%%%%%%%%%%%%%%%%%%%%%%%%%%%%%%%%
%%%%%%%%%%%%%%%%%%%%%%%%%%%%%%%%%%%%%%%%%%%%%%%%%%%%%%%%%%%%%%%%%%%%%%%%%%%%%%%%
%%%%%%%%%%%%%%%%%%%%%%%%%%%%%%%%%%%%%%%%%%%%%%%%%%%%%%%%%%%%%%%%%%%%%%%%%%%%%%%%
%%%%%%%%%%%%%%%%%%%%%%%%%%%%%%%%%%%%%%%%%%%%%%%%%%%%%%%%%%%%%%%%%%%%%%%%%%%%%%%%
\begin{mdframed}[style=darkQuesion]
12. Let $a,b$ be positive integers, and let $d=\ngcd{a}{b}$ and $m=\nlcm{a}{b}$. 
Use proposition 3.5.5 to prove that $\Z_a\times \Z_b\cong \Z_d\times \Z_m$.
\end{mdframed}
%%%%%%%%%%%%%%%%%%%%%%%%%%%%%%%%%%%%%%%%%%%%%%%%%%%%%%%%%%%%%%%%%%%%%%%%%%%%%%%%
\begin{mdframed}[style=darkAnswer,frametitle={Joe Starr}]
 By $3.5.5$ we have that $\Zm{ab}\cong \Z_a\times \Z_b$, and 
 $\Zm{dm}\cong \Z_d\times \Z_m$, we've previously shown $\ngcd{a}{b}\cdot \nlcm{a}{b}=ab$
 making $\Zm{dm}=\Zm{ab}$ showing $\Z_a\times \Z_b\cong \Z_d\times \Z_m$.
\end{mdframed}
\newpage
%%%%%%%%%%%%%%%%%%%%%%%%%%%%%%%%%%%%%%%%%%%%%%%%%%%%%%%%%%%%%%%%%%%%%%%%%%%%%%%%
%%%%%%%%%%%%%%%%%%%%%%%%%%%%%%%%%%%%%%%%%%%%%%%%%%%%%%%%%%%%%%%%%%%%%%%%%%%%%%%%
%%%%%%%%%%%%%%%%%%%%%%%%%%%%%%%%%%%%%%%%%%%%%%%%%%%%%%%%%%%%%%%%%%%%%%%%%%%%%%%%
%%%%%%%%%%%%%%%%%%%%%%%%%%%%%%%%%%%%%%%%%%%%%%%%%%%%%%%%%%%%%%%%%%%%%%%%%%%%%%%%
\begin{mdframed}[style=darkQuesion]
18. Let $G$ be the set of all 3 $\times 3$ matrices of the form $\left[\begin{array}{lll}1 & 0 & 0 \\ a & 1 & 0 \\ b & c & 1\end{array}\right]$
  \begin{itemize}
    \item []{(a) Show that if $a, b, c \in \mathbf{Z}_{3},$ then $G$ is a group with exponent 3}
    \item []{(b) Show that if $a, b, c \in \mathbf{Z}_{2},$ then $G$ is a group with exponent 4.}
  \end{itemize}
  
  
\end{mdframed}
%%%%%%%%%%%%%%%%%%%%%%%%%%%%%%%%%%%%%%%%%%%%%%%%%%%%%%%%%%%%%%%%%%%%%%%%%%%%%%%%
\begin{mdframed}[style=darkAnswer,frametitle={Joe Starr}]
  \begin{itemize}\item[(a)]{
$$
a= \begin{bmatrix}1&0&0\\0&1&0\\0&0&1\end{bmatrix} \\
$$ $$
a= \begin{bmatrix}1&0&0\\0&1&0\\0&1&1\end{bmatrix} \\
a^{ 2 }=\lrp{ \begin{bmatrix}1&0&0\\0&1&0\\0&2&1\end{bmatrix} }
a^{ 3 }=\lrp{ \begin{bmatrix}1&0&0\\0&1&0\\0&0&1\end{bmatrix} }
$$ $$
a= \begin{bmatrix}1&0&0\\0&1&0\\0&2&1\end{bmatrix} \\
a^{ 2 }=\lrp{ \begin{bmatrix}1&0&0\\0&1&0\\0&1&1\end{bmatrix} }
a^{ 3 }=\lrp{ \begin{bmatrix}1&0&0\\0&1&0\\0&0&1\end{bmatrix} }
$$ $$
a= \begin{bmatrix}1&0&0\\0&1&0\\1&0&1\end{bmatrix} \\
a^{ 2 }=\lrp{ \begin{bmatrix}1&0&0\\0&1&0\\2&0&1\end{bmatrix} }
a^{ 3 }=\lrp{ \begin{bmatrix}1&0&0\\0&1&0\\0&0&1\end{bmatrix} }
$$ $$
a= \begin{bmatrix}1&0&0\\0&1&0\\1&1&1\end{bmatrix} \\
a^{ 2 }=\lrp{ \begin{bmatrix}1&0&0\\0&1&0\\2&2&1\end{bmatrix} }
a^{ 3 }=\lrp{ \begin{bmatrix}1&0&0\\0&1&0\\0&0&1\end{bmatrix} }
$$ $$
a= \begin{bmatrix}1&0&0\\0&1&0\\1&2&1\end{bmatrix} \\
a^{ 2 }=\lrp{ \begin{bmatrix}1&0&0\\0&1&0\\2&1&1\end{bmatrix} }
a^{ 3 }=\lrp{ \begin{bmatrix}1&0&0\\0&1&0\\0&0&1\end{bmatrix} }
$$ $$
a= \begin{bmatrix}1&0&0\\0&1&0\\2&0&1\end{bmatrix} \\
a^{ 2 }=\lrp{ \begin{bmatrix}1&0&0\\0&1&0\\1&0&1\end{bmatrix} }
a^{ 3 }=\lrp{ \begin{bmatrix}1&0&0\\0&1&0\\0&0&1\end{bmatrix} }
$$ $$
a= \begin{bmatrix}1&0&0\\0&1&0\\2&1&1\end{bmatrix} \\
a^{ 2 }=\lrp{ \begin{bmatrix}1&0&0\\0&1&0\\1&2&1\end{bmatrix} }
a^{ 3 }=\lrp{ \begin{bmatrix}1&0&0\\0&1&0\\0&0&1\end{bmatrix} }
$$ $$
a= \begin{bmatrix}1&0&0\\0&1&0\\2&2&1\end{bmatrix} \\
a^{ 2 }=\lrp{ \begin{bmatrix}1&0&0\\0&1&0\\1&1&1\end{bmatrix} }
a^{ 3 }=\lrp{ \begin{bmatrix}1&0&0\\0&1&0\\0&0&1\end{bmatrix} }
$$ $$
a= \begin{bmatrix}1&0&0\\1&1&0\\0&0&1\end{bmatrix} \\
a^{ 2 }=\lrp{ \begin{bmatrix}1&0&0\\2&1&0\\0&0&1\end{bmatrix} }
a^{ 3 }=\lrp{ \begin{bmatrix}1&0&0\\0&1&0\\0&0&1\end{bmatrix} }
$$ $$
a= \begin{bmatrix}1&0&0\\1&1&0\\0&1&1\end{bmatrix} \\
a^{ 2 }=\lrp{ \begin{bmatrix}1&0&0\\2&1&0\\1&2&1\end{bmatrix} }
a^{ 3 }=\lrp{ \begin{bmatrix}1&0&0\\0&1&0\\0&0&1\end{bmatrix} }
$$ $$
a= \begin{bmatrix}1&0&0\\1&1&0\\0&2&1\end{bmatrix} \\
a^{ 2 }=\lrp{ \begin{bmatrix}1&0&0\\2&1&0\\2&1&1\end{bmatrix} }
a^{ 3 }=\lrp{ \begin{bmatrix}1&0&0\\0&1&0\\0&0&1\end{bmatrix} }
$$ $$
a= \begin{bmatrix}1&0&0\\1&1&0\\1&0&1\end{bmatrix} \\
a^{ 2 }=\lrp{ \begin{bmatrix}1&0&0\\2&1&0\\2&0&1\end{bmatrix} }
a^{ 3 }=\lrp{ \begin{bmatrix}1&0&0\\0&1&0\\0&0&1\end{bmatrix} }
$$ $$
a= \begin{bmatrix}1&0&0\\1&1&0\\1&1&1\end{bmatrix} \\
a^{ 2 }=\lrp{ \begin{bmatrix}1&0&0\\2&1&0\\0&2&1\end{bmatrix} }
a^{ 3 }=\lrp{ \begin{bmatrix}1&0&0\\0&1&0\\0&0&1\end{bmatrix} }
$$ $$
a= \begin{bmatrix}1&0&0\\1&1&0\\1&2&1\end{bmatrix} \\
a^{ 2 }=\lrp{ \begin{bmatrix}1&0&0\\2&1&0\\1&1&1\end{bmatrix} }
a^{ 3 }=\lrp{ \begin{bmatrix}1&0&0\\0&1&0\\0&0&1\end{bmatrix} }
$$ $$
a= \begin{bmatrix}1&0&0\\1&1&0\\2&0&1\end{bmatrix} \\
a^{ 2 }=\lrp{ \begin{bmatrix}1&0&0\\2&1&0\\1&0&1\end{bmatrix} }
a^{ 3 }=\lrp{ \begin{bmatrix}1&0&0\\0&1&0\\0&0&1\end{bmatrix} }
$$ $$
a= \begin{bmatrix}1&0&0\\1&1&0\\2&1&1\end{bmatrix} \\
a^{ 2 }=\lrp{ \begin{bmatrix}1&0&0\\2&1&0\\2&2&1\end{bmatrix} }
a^{ 3 }=\lrp{ \begin{bmatrix}1&0&0\\0&1&0\\0&0&1\end{bmatrix} }
$$ $$
a= \begin{bmatrix}1&0&0\\1&1&0\\2&2&1\end{bmatrix} \\
a^{ 2 }=\lrp{ \begin{bmatrix}1&0&0\\2&1&0\\0&1&1\end{bmatrix} }
a^{ 3 }=\lrp{ \begin{bmatrix}1&0&0\\0&1&0\\0&0&1\end{bmatrix} }
$$ $$
a= \begin{bmatrix}1&0&0\\2&1&0\\0&0&1\end{bmatrix} \\
a^{ 2 }=\lrp{ \begin{bmatrix}1&0&0\\1&1&0\\0&0&1\end{bmatrix} }
a^{ 3 }=\lrp{ \begin{bmatrix}1&0&0\\0&1&0\\0&0&1\end{bmatrix} }
$$ $$
a= \begin{bmatrix}1&0&0\\2&1&0\\0&1&1\end{bmatrix} \\
a^{ 2 }=\lrp{ \begin{bmatrix}1&0&0\\1&1&0\\2&2&1\end{bmatrix} }
a^{ 3 }=\lrp{ \begin{bmatrix}1&0&0\\0&1&0\\0&0&1\end{bmatrix} }
$$ $$
a= \begin{bmatrix}1&0&0\\2&1&0\\0&2&1\end{bmatrix} \\
a^{ 2 }=\lrp{ \begin{bmatrix}1&0&0\\1&1&0\\1&1&1\end{bmatrix} }
a^{ 3 }=\lrp{ \begin{bmatrix}1&0&0\\0&1&0\\0&0&1\end{bmatrix} }
$$ $$
a= \begin{bmatrix}1&0&0\\2&1&0\\1&0&1\end{bmatrix} \\
a^{ 2 }=\lrp{ \begin{bmatrix}1&0&0\\1&1&0\\2&0&1\end{bmatrix} }
a^{ 3 }=\lrp{ \begin{bmatrix}1&0&0\\0&1&0\\0&0&1\end{bmatrix} }
$$ $$
a= \begin{bmatrix}1&0&0\\2&1&0\\1&1&1\end{bmatrix} \\
a^{ 2 }=\lrp{ \begin{bmatrix}1&0&0\\1&1&0\\1&2&1\end{bmatrix} }
a^{ 3 }=\lrp{ \begin{bmatrix}1&0&0\\0&1&0\\0&0&1\end{bmatrix} }
$$ $$
a= \begin{bmatrix}1&0&0\\2&1&0\\1&2&1\end{bmatrix} \\
a^{ 2 }=\lrp{ \begin{bmatrix}1&0&0\\1&1&0\\0&1&1\end{bmatrix} }
a^{ 3 }=\lrp{ \begin{bmatrix}1&0&0\\0&1&0\\0&0&1\end{bmatrix} }
$$ $$
a= \begin{bmatrix}1&0&0\\2&1&0\\2&0&1\end{bmatrix} \\
a^{ 2 }=\lrp{ \begin{bmatrix}1&0&0\\1&1&0\\1&0&1\end{bmatrix} }
a^{ 3 }=\lrp{ \begin{bmatrix}1&0&0\\0&1&0\\0&0&1\end{bmatrix} }
$$ $$
a= \begin{bmatrix}1&0&0\\2&1&0\\2&1&1\end{bmatrix} \\
a^{ 2 }=\lrp{ \begin{bmatrix}1&0&0\\1&1&0\\0&2&1\end{bmatrix} }
a^{ 3 }=\lrp{ \begin{bmatrix}1&0&0\\0&1&0\\0&0&1\end{bmatrix} }
$$ $$
a= \begin{bmatrix}1&0&0\\2&1&0\\2&2&1\end{bmatrix} \\
a^{ 2 }=\lrp{ \begin{bmatrix}1&0&0\\1&1&0\\2&1&1\end{bmatrix} }
a^{ 3 }=\lrp{ \begin{bmatrix}1&0&0\\0&1&0\\0&0&1\end{bmatrix} }
$$

We observe that every element is order 3 meaning exponent of $G$ is 3. 
}
  \item[(b)]{
  $$a= \begin{bmatrix}1&0&0\\0&1&0\\0&0&1\end{bmatrix} \\
    $$
    $$a= \begin{bmatrix}1&0&0\\0&1&0\\0&1&1\end{bmatrix} \\
    a^{ 2 }={ \begin{bmatrix}1&0&0\\0&1&0\\0&0&1\end{bmatrix} }
    $$
    $$a= \begin{bmatrix}1&0&0\\0&1&0\\0&0&1\end{bmatrix} \\
    a^{ 2 }={ \begin{bmatrix}1&0&0\\0&1&0\\0&0&1\end{bmatrix} }
    $$
    $$a= \begin{bmatrix}1&0&0\\0&1&0\\1&0&1\end{bmatrix} \\
    a^{ 2 }={ \begin{bmatrix}1&0&0\\0&1&0\\0&0&1\end{bmatrix} }
    $$
    $$a= \begin{bmatrix}1&0&0\\0&1&0\\1&1&1\end{bmatrix} \\
    a^{ 2 }={ \begin{bmatrix}1&0&0\\0&1&0\\0&0&1\end{bmatrix} }
    $$
    $$a= \begin{bmatrix}1&0&0\\0&1&0\\1&0&1\end{bmatrix} \\
    a^{ 2 }={ \begin{bmatrix}1&0&0\\0&1&0\\0&0&1\end{bmatrix} }
    $$
    $$a= \begin{bmatrix}1&0&0\\0&1&0\\0&0&1\end{bmatrix} \\
    a^{ 2 }={ \begin{bmatrix}1&0&0\\0&1&0\\0&0&1\end{bmatrix} }
    $$
    $$a= \begin{bmatrix}1&0&0\\0&1&0\\0&1&1\end{bmatrix} \\
    a^{ 2 }={ \begin{bmatrix}1&0&0\\0&1&0\\0&0&1\end{bmatrix} }
    $$
    $$a= \begin{bmatrix}1&0&0\\0&1&0\\0&0&1\end{bmatrix} \\
    a^{ 2 }={ \begin{bmatrix}1&0&0\\0&1&0\\0&0&1\end{bmatrix} }
    $$
    $$a= \begin{bmatrix}1&0&0\\1&1&0\\0&0&1\end{bmatrix} \\
    a^{ 2 }={ \begin{bmatrix}1&0&0\\0&1&0\\0&0&1\end{bmatrix} }
    $$
    $$a= \begin{bmatrix}1&0&0\\1&1&0\\0&1&1\end{bmatrix} \\
    a^{ 2 }={ \begin{bmatrix}1&0&0\\0&1&0\\1&0&1\end{bmatrix} }
    a^{ 3 }={ \begin{bmatrix}1&0&0\\1&1&0\\1&1&1\end{bmatrix} }
    a^{ 4 }={ \begin{bmatrix}1&0&0\\0&1&0\\0&0&1\end{bmatrix} }
    $$
    $$a= \begin{bmatrix}1&0&0\\1&1&0\\0&0&1\end{bmatrix} \\
    a^{ 2 }={ \begin{bmatrix}1&0&0\\0&1&0\\0&0&1\end{bmatrix} }
    $$
    $$a= \begin{bmatrix}1&0&0\\1&1&0\\1&0&1\end{bmatrix} \\
    a^{ 2 }={ \begin{bmatrix}1&0&0\\0&1&0\\0&0&1\end{bmatrix} }
    $$
    $$a= \begin{bmatrix}1&0&0\\1&1&0\\1&1&1\end{bmatrix} \\
    a^{ 2 }={ \begin{bmatrix}1&0&0\\0&1&0\\1&0&1\end{bmatrix} }
    a^{ 3 }={ \begin{bmatrix}1&0&0\\1&1&0\\0&1&1\end{bmatrix} }
    a^{ 4 }={ \begin{bmatrix}1&0&0\\0&1&0\\0&0&1\end{bmatrix} }
    $$
    $$a= \begin{bmatrix}1&0&0\\1&1&0\\1&0&1\end{bmatrix} \\
    a^{ 2 }={ \begin{bmatrix}1&0&0\\0&1&0\\0&0&1\end{bmatrix} }
    $$
    $$a= \begin{bmatrix}1&0&0\\1&1&0\\0&0&1\end{bmatrix} \\
    a^{ 2 }={ \begin{bmatrix}1&0&0\\0&1&0\\0&0&1\end{bmatrix} }
    $$
    $$a= \begin{bmatrix}1&0&0\\1&1&0\\0&1&1\end{bmatrix} \\
    a^{ 2 }={ \begin{bmatrix}1&0&0\\0&1&0\\1&0&1\end{bmatrix} }
    a^{ 3 }={ \begin{bmatrix}1&0&0\\1&1&0\\1&1&1\end{bmatrix} }
    a^{ 4 }={ \begin{bmatrix}1&0&0\\0&1&0\\0&0&1\end{bmatrix} }
    $$
    $$a= \begin{bmatrix}1&0&0\\1&1&0\\0&0&1\end{bmatrix} \\
    a^{ 2 }={ \begin{bmatrix}1&0&0\\0&1&0\\0&0&1\end{bmatrix} }
    $$
    $$a= \begin{bmatrix}1&0&0\\0&1&0\\0&0&1\end{bmatrix} \\
    a^{ 2 }={ \begin{bmatrix}1&0&0\\0&1&0\\0&0&1\end{bmatrix} }
    $$
    $$a= \begin{bmatrix}1&0&0\\0&1&0\\0&1&1\end{bmatrix} \\
    a^{ 2 }={ \begin{bmatrix}1&0&0\\0&1&0\\0&0&1\end{bmatrix} }
    $$
    $$a= \begin{bmatrix}1&0&0\\0&1&0\\0&0&1\end{bmatrix} \\
    a^{ 2 }={ \begin{bmatrix}1&0&0\\0&1&0\\0&0&1\end{bmatrix} }
    $$
    $$a= \begin{bmatrix}1&0&0\\0&1&0\\1&0&1\end{bmatrix} \\
    a^{ 2 }={ \begin{bmatrix}1&0&0\\0&1&0\\0&0&1\end{bmatrix} }
    $$
    $$a= \begin{bmatrix}1&0&0\\0&1&0\\1&1&1\end{bmatrix} \\
    a^{ 2 }={ \begin{bmatrix}1&0&0\\0&1&0\\0&0&1\end{bmatrix} }
    $$
    $$a= \begin{bmatrix}1&0&0\\0&1&0\\1&0&1\end{bmatrix} \\
    a^{ 2 }={ \begin{bmatrix}1&0&0\\0&1&0\\0&0&1\end{bmatrix} }
    $$
    $$a= \begin{bmatrix}1&0&0\\0&1&0\\0&0&1\end{bmatrix} \\
    a^{ 2 }={ \begin{bmatrix}1&0&0\\0&1&0\\0&0&1\end{bmatrix} }
    $$
    $$a= \begin{bmatrix}1&0&0\\0&1&0\\0&1&1\end{bmatrix} \\
    a^{ 2 }={ \begin{bmatrix}1&0&0\\0&1&0\\0&0&1\end{bmatrix} }
    $$
    $$a= \begin{bmatrix}1&0&0\\0&1&0\\0&0&1\end{bmatrix} \\
    a^{ 2 }={ \begin{bmatrix}1&0&0\\0&1&0\\0&0&1\end{bmatrix} }
    $$
    We observe that every element is either order two or four. 
    This makes exponent of $G$ 4. 
  }
\end{itemize}
\end{mdframed}
\newpage
%%%%%%%%%%%%%%%%%%%%%%%%%%%%%%%%%%%%%%%%%%%%%%%%%%%%%%%%%%%%%%%%%%%%%%%%%%%%%%%%
%%%%%%%%%%%%%%%%%%%%%%%%%%%%%%%%%%%%%%%%%%%%%%%%%%%%%%%%%%%%%%%%%%%%%%%%%%%%%%%%
%%%%%%%%%%%%%%%%%%%%%%%%%%%%%%%%%%%%%%%%%%%%%%%%%%%%%%%%%%%%%%%%%%%%%%%%%%%%%%%%
%%%%%%%%%%%%%%%%%%%%%%%%%%%%%%%%%%%%%%%%%%%%%%%%%%%%%%%%%%%%%%%%%%%%%%%%%%%%%%%%
\begin{mdframed}[style=darkQuesion]
  19. Prove that $\sum_{d | n} \varphi(d)=n$ for any positive integer $n$ 
  Hint: Interpret the equation in the cyclic group $\mathbf{Z}_{n},$ 
  by considering all of its sub-groups.
\end{mdframed}
%%%%%%%%%%%%%%%%%%%%%%%%%%%%%%%%%%%%%%%%%%%%%%%%%%%%%%%%%%%%%%%%%%%%%%%%%%%%%%%%
\begin{mdframed}[style=darkAnswer,frametitle={Joe Starr}]
%TODO this one doesn't make sense. 
\end{mdframed}
\newpage
%%%%%%%%%%%%%%%%%%%%%%%%%%%%%%%%%%%%%%%%%%%%%%%%%%%%%%%%%%%%%%%%%%%%%%%%%%%%%%%%
%%%%%%%%%%%%%%%%%%%%%%%%%%%%%%%%%%%%%%%%%%%%%%%%%%%%%%%%%%%%%%%%%%%%%%%%%%%%%%%%
%%%%%%%%%%%%%%%%%%%%%%%%%%%%%%%%%%%%%%%%%%%%%%%%%%%%%%%%%%%%%%%%%%%%%%%%%%%%%%%%
%%%%%%%%%%%%%%%%%%%%%%%%%%%%%%%%%%%%%%%%%%%%%%%%%%%%%%%%%%%%%%%%%%%%%%%%%%%%%%%%
\begin{mdframed}[style=darkQuesion]
  20. Let $n=2^{k}$ for $k>2 .$ Prove that $\mathbf{Z}_{n}^{\times}$ is not cyclic. 
  Hint: Show that $\pm 1$ and $(n / 2) \pm 1$ satisfy the equation $x^{2}=1,$ and that this is impossible in any cyclic group.
\end{mdframed}
%%%%%%%%%%%%%%%%%%%%%%%%%%%%%%%%%%%%%%%%%%%%%%%%%%%%%%%%%%%%%%%%%%%%%%%%%%%%%%%%
\begin{mdframed}[style=darkAnswer,frametitle={Joe Starr}]
  %TODO not done
\end{mdframed}
\newpage
%%%%%%%%%%%%%%%%%%%%%%%%%%%%%%%%%%%%%%%%%%%%%%%%%%%%%%%%%%%%%%%%%%%%%%%%%%%%%%%%
%%%%%%%%%%%%%%%%%%%%%%%%%%%%%%%%%%%%%%%%%%%%%%%%%%%%%%%%%%%%%%%%%%%%%%%%%%%%%%%%
%%%%%%%%%%%%%%%%%%%%%%%%%%%%%%%%%%%%%%%%%%%%%%%%%%%%%%%%%%%%%%%%%%%%%%%%%%%%%%%%
%%%%%%%%%%%%%%%%%%%%%%%%%%%%%%%%%%%%%%%%%%%%%%%%%%%%%%%%%%%%%%%%%%%%%%%%%%%%%%%%
\begin{mdframed}[style=darkQuesion]
  21.  Prove that if $p$ and $q$ are different odd primes, then $\mathbf{Z}_{p q}^{\times}$ is not a cyclic group.
\end{mdframed}
%%%%%%%%%%%%%%%%%%%%%%%%%%%%%%%%%%%%%%%%%%%%%%%%%%%%%%%%%%%%%%%%%%%%%%%%%%%%%%%%
\begin{mdframed}[style=darkAnswer,frametitle={Joe Starr}]
 %TODO not done
\end{mdframed}
\newpage