\subsection{Homomorphism}
%%%%%%%%%%%%%%%%%%%%%%%%%%%%%%%%%%%%%%%%%%%%%%%%%%%%%%%%%%%%%%%%%%%%%%%%%%%%%%%%
%%%%%%%%%%%%%%%%%%%%%%%%%%%%%%%%%%%%%%%%%%%%%%%%%%%%%%%%%%%%%%%%%%%%%%%%%%%%%%%%
%%%%%%%%%%%%%%%%%%%%%%%%%%%%%%%%%%%%%%%%%%%%%%%%%%%%%%%%%%%%%%%%%%%%%%%%%%%%%%%%
%%%%%%%%%%%%%%%%%%%%%%%%%%%%%%%%%%%%%%%%%%%%%%%%%%%%%%%%%%%%%%%%%%%%%%%%%%%%%%%%
\begin{mdframed}[style=darkQuesion]
  7. Define $\phi: \mathbf{C}^{\times} \rightarrow \mathbf{R}^{\times}$ by $\phi(a+b i)=a^{2}+b^{2},$ for all $a+b i \in \mathbf{C}^{\times} .$ Show that $\phi$
is a homomorphism.
\end{mdframed}
%%%%%%%%%%%%%%%%%%%%%%%%%%%%%%%%%%%%%%%%%%%%%%%%%%%%%%%%%%%%%%%%%%%%%%%%%%%%%%%%
\begin{mdframed}[style=darkAnswer,frametitle={Joe Starr}]
Consider $a+bi, c+di\in \mathbf{C}^{\times}$, if we take 
\begin{align*}
  \pof{\lrp{a+bi} \lrp{c+di}}&=\lrp{a c- b d}^2 +  \lrp{a d +  b c}^2\\
  &=a^2 c^2 + b^2 c^2 + a^2 d^2 + b^2 d^2\\
  &=(a^2 + b^2) (c^2 + d^2)\\
  &=\pof{\lrp{a+bi}} \pof{\lrp{c+di}}
\end{align*}
\end{mdframed}
\newpage
%%%%%%%%%%%%%%%%%%%%%%%%%%%%%%%%%%%%%%%%%%%%%%%%%%%%%%%%%%%%%%%%%%%%%%%%%%%%%%%%
%%%%%%%%%%%%%%%%%%%%%%%%%%%%%%%%%%%%%%%%%%%%%%%%%%%%%%%%%%%%%%%%%%%%%%%%%%%%%%%%
%%%%%%%%%%%%%%%%%%%%%%%%%%%%%%%%%%%%%%%%%%%%%%%%%%%%%%%%%%%%%%%%%%%%%%%%%%%%%%%%
%%%%%%%%%%%%%%%%%%%%%%%%%%%%%%%%%%%%%%%%%%%%%%%%%%%%%%%%%%%%%%%%%%%%%%%%%%%%%%%%
\begin{mdframed}[style=darkQuesion]
  9. Which of the following functions are homomorphisms?
\begin{itemize}
\item[]{(a) $\phi: \mathbf{R}^{\times} \rightarrow \mathrm{GL}_{2}(\mathbf{R})$ defined by $\phi(a)=\left[\begin{array}{ll}a & 0 \\ 0 & 1\end{array}\right]$}
\item[]{(b) $\phi: \mathbf{R} \rightarrow \mathrm{GL}_{2}(\mathbf{R})$ defined by $\phi(a)=\left[\begin{array}{ll}1 & 0 \\ a & 1\end{array}\right]$}
\item[]{(c) $\phi: M_{2}(\mathbf{R}) \rightarrow \mathbf{R} \operatorname{defined} \operatorname{by} \phi\left(\left[\begin{array}{ll}a & b \\ c & d\end{array}\right]\right)=a$}
\item[]{(d) $\phi: \mathrm{GL}_{2}(\mathbf{R}) \rightarrow \mathbf{R}^{\times}$ defined by $\phi\left(\left[\begin{array}{ll}a & b \\ c & d\end{array}\right]\right)=a b$}
\item[]{(e) $\phi: \mathrm{GL}_{2}(\mathbf{R}) \rightarrow \mathbf{R}$ defined by $\phi\left(\left[\begin{array}{ll}a & b \\ c & d\end{array}\right]\right)=a+d$}
\item[]{(f) $\phi: \mathrm{GL}_{2}(\mathbf{R}) \rightarrow \mathbf{R}^{\times}$ defined by $\phi\left(\left[\begin{array}{ll}a & b \\ c & d\end{array}\right]\right)=a d-b c$}
\end{itemize}
\end{mdframed}
%%%%%%%%%%%%%%%%%%%%%%%%%%%%%%%%%%%%%%%%%%%%%%%%%%%%%%%%%%%%%%%%%%%%%%%%%%%%%%%%
\begin{mdframed}[style=darkAnswer,frametitle={Joe Starr}]
  \begin{enumerate}[(a)]
  \item{yes}
  \item{yes}
  \item{yes}
  \item{no}
  \item{no}
  \item{yes}
  \end{enumerate}

\end{mdframed}
\newpage
%%%%%%%%%%%%%%%%%%%%%%%%%%%%%%%%%%%%%%%%%%%%%%%%%%%%%%%%%%%%%%%%%%%%%%%%%%%%%%%%
%%%%%%%%%%%%%%%%%%%%%%%%%%%%%%%%%%%%%%%%%%%%%%%%%%%%%%%%%%%%%%%%%%%%%%%%%%%%%%%%
%%%%%%%%%%%%%%%%%%%%%%%%%%%%%%%%%%%%%%%%%%%%%%%%%%%%%%%%%%%%%%%%%%%%%%%%%%%%%%%%
%%%%%%%%%%%%%%%%%%%%%%%%%%%%%%%%%%%%%%%%%%%%%%%%%%%%%%%%%%%%%%%%%%%%%%%%%%%%%%%%
\begin{mdframed}[style=darkQuesion]
  10. Let $\phi: G_{1} \rightarrow G_{2}$ and $\theta: G_{2} \rightarrow G_{3}$ be group homomorphisms. Prove that
  $\theta \phi: G_{1} \rightarrow G_{3}$ is a homomorphism. Prove that $\operatorname{ker}(\phi) \subseteq \operatorname{ker}(\theta \phi)$
\end{mdframed}
%%%%%%%%%%%%%%%%%%%%%%%%%%%%%%%%%%%%%%%%%%%%%%%%%%%%%%%%%%%%%%%%%%%%%%%%%%%%%%%%
\begin{mdframed}[style=darkAnswer,frametitle={Joe Starr}]
by transitive proof let $a,b\in G_1$:
\begin{align*}
  \theta\lrp{\pof{ab}}&=\theta\lrp{\pof{a}\pof{b}}\\
  &=\theta\lrp{\pof{a}}\theta\lrp{\pof{b}}\\
\end{align*}
making $\theta\phi$ a homomorphism. 

We let $k\in \krn{\phi}$, that means $\pof{k}=e_{G_2}$, so 
$\theta\lrp{\pof{k}}=\theta\lrp{e_{G_2}}=e_{G_3}$ putting 
$k\in \krn{\theta\phi}$.
\end{mdframed}
\newpage
%%%%%%%%%%%%%%%%%%%%%%%%%%%%%%%%%%%%%%%%%%%%%%%%%%%%%%%%%%%%%%%%%%%%%%%%%%%%%%%%
%%%%%%%%%%%%%%%%%%%%%%%%%%%%%%%%%%%%%%%%%%%%%%%%%%%%%%%%%%%%%%%%%%%%%%%%%%%%%%%%
%%%%%%%%%%%%%%%%%%%%%%%%%%%%%%%%%%%%%%%%%%%%%%%%%%%%%%%%%%%%%%%%%%%%%%%%%%%%%%%%
%%%%%%%%%%%%%%%%%%%%%%%%%%%%%%%%%%%%%%%%%%%%%%%%%%%%%%%%%%%%%%%%%%%%%%%%%%%%%%%%
\begin{mdframed}[style=darkQuesion]
  13. Let $G$ be a group, and let $H$ be a normal subgroup of G. Show that for 
  each $g\in G$ and $h\in H$ there exist $h_1$ and $h_2$ in $H$ with $gh=h_1g$
  and $hg=gh_2$.
\end{mdframed}
%%%%%%%%%%%%%%%%%%%%%%%%%%%%%%%%%%%%%%%%%%%%%%%%%%%%%%%%%%%%%%%%%%%%%%%%%%%%%%%%
\begin{mdframed}[style=darkAnswer,frametitle={Joe Starr}]
Since $H$ is normal we have $gh\inv{g}=h_1\in H$ and $\inv{g}hg=h_2\in H$. 
We can then multiply by $g$, $gh=h_1g$ and $hg=gh_2$ as desired. 
\end{mdframed}
\newpage
%%%%%%%%%%%%%%%%%%%%%%%%%%%%%%%%%%%%%%%%%%%%%%%%%%%%%%%%%%%%%%%%%%%%%%%%%%%%%%%%
%%%%%%%%%%%%%%%%%%%%%%%%%%%%%%%%%%%%%%%%%%%%%%%%%%%%%%%%%%%%%%%%%%%%%%%%%%%%%%%%
%%%%%%%%%%%%%%%%%%%%%%%%%%%%%%%%%%%%%%%%%%%%%%%%%%%%%%%%%%%%%%%%%%%%%%%%%%%%%%%%
%%%%%%%%%%%%%%%%%%%%%%%%%%%%%%%%%%%%%%%%%%%%%%%%%%%%%%%%%%%%%%%%%%%%%%%%%%%%%%%%
\begin{mdframed}[style=darkQuesion]
  15. Show that the only proper nontrivial normal subgroup of $S_3$ in the 
  subgroup with three elements. 
\end{mdframed}
%%%%%%%%%%%%%%%%%%%%%%%%%%%%%%%%%%%%%%%%%%%%%%%%%%%%%%%%%%%%%%%%%%%%%%%%%%%%%%%%
\begin{mdframed}[style=darkAnswer,frametitle={Joe Starr}]
  We observe that the multiplication table for $S_3$ is:
  $$\vbox{\tabskip0.5em\offinterlineskip
  \halign{\strut$#$\hfil\ \tabskip1em\vrule&&$#$\hfil\cr
  \cdot   &()&(1, 2)&(2, 3)&(1, 3)&(1, 2, 3)&(1, 3, 2)  \cr
      \noalign{\hrule}\vrule height 12pt width 0pt
      ()&()&(1, 2)&(2, 3)&(1, 3)&(1, 2, 3)&(1, 3, 2)     \cr
      (1, 2)&(1, 2)&()&(1, 2, 3)&(1, 3, 2)&(2, 3)&(1, 3)     \cr
      (2, 3)&(2, 3)&(1, 3, 2)&()&(1, 2, 3)&(1, 3)&(1, 2)     \cr
      (1, 3)&(1, 3)&(1, 2, 3)&(1, 3, 2)&()&(1, 2)&(2, 3)     \cr
      (1, 2, 3)&(1, 2, 3)&(1, 3)&(1, 2)&(2, 3)&(1, 3, 2)&()     \cr
      (1, 3, 2)&(1, 3, 2)&(2, 3)&(1, 3)&(1, 2)&()&(1, 2, 3)     \cr
    }}$$
We know by lagrange that the subgroups of $S_3$ are of order $1,2,3,6$. 
We observe from the table above that the subgroups of order two are the 
subgroups of the identity and a transposition. If we take a transposition $a$ and
another $b$ and calculate $ba\inv{b}=c$ where c is the third transposition. 
Showing the subgroups of order $2$ not normal. However if we take the subgroup 
of order 3, $\lrs{\lrp{\ },\ (1, 2, 3),\ (1, 3, 2)}$, and calculate
\begin{align*}
  (1,2)(1, 2, 3)(1,2)&=(1,3,2)\\
  (1,3)(1, 2, 3)(1,3)&=(1,3,2)\\
  (2,3)(1, 2, 3)(2,3)&=(1,3,2)\\
  (1,2)(1, 3, 2)(1,2)&=(1, 2, 3)\\
  (1,3)(1, 3, 2)(1,3)&=(1, 2, 3)\\
  (2,3)(1, 3, 2)(2,3)&=(1, 2, 3)\\
\end{align*}
showing the subgroup of three elements to be normal. 
\end{mdframed}
\newpage
%%%%%%%%%%%%%%%%%%%%%%%%%%%%%%%%%%%%%%%%%%%%%%%%%%%%%%%%%%%%%%%%%%%%%%%%%%%%%%%%
%%%%%%%%%%%%%%%%%%%%%%%%%%%%%%%%%%%%%%%%%%%%%%%%%%%%%%%%%%%%%%%%%%%%%%%%%%%%%%%%
%%%%%%%%%%%%%%%%%%%%%%%%%%%%%%%%%%%%%%%%%%%%%%%%%%%%%%%%%%%%%%%%%%%%%%%%%%%%%%%%
%%%%%%%%%%%%%%%%%%%%%%%%%%%%%%%%%%%%%%%%%%%%%%%%%%%%%%%%%%%%%%%%%%%%%%%%%%%%%%%%
\begin{mdframed}[style=darkQuesion]
17. Recall that the center of a group $G$ is 
$\lrs{x\in G\vert xg=gx \text{ for all }g\in G}$. 
Prove that the center of any group is a normal subgroup. 
\end{mdframed}
%%%%%%%%%%%%%%%%%%%%%%%%%%%%%%%%%%%%%%%%%%%%%%%%%%%%%%%%%%%%%%%%%%%%%%%%%%%%%%%%
\begin{mdframed}[style=darkAnswer,frametitle={Joe Starr}]
We let $C$ be the center of $G$. We take $g\in G$ and $x\in C$, by construction
we have $xg=gx$ by multiplying on the left by $\inv{g}$ we have $\inv{g}xg=x$ 
showing $C$ a normal subgroup.
\end{mdframed}
\newpage