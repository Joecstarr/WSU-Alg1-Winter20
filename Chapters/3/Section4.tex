\subsection{Isomorphisms}
%%%%%%%%%%%%%%%%%%%%%%%%%%%%%%%%%%%%%%%%%%%%%%%%%%%%%%%%%%%%%%%%%%%%%%%%%%%%%%%%
%%%%%%%%%%%%%%%%%%%%%%%%%%%%%%%%%%%%%%%%%%%%%%%%%%%%%%%%%%%%%%%%%%%%%%%%%%%%%%%%
%%%%%%%%%%%%%%%%%%%%%%%%%%%%%%%%%%%%%%%%%%%%%%%%%%%%%%%%%%%%%%%%%%%%%%%%%%%%%%%%
%%%%%%%%%%%%%%%%%%%%%%%%%%%%%%%%%%%%%%%%%%%%%%%%%%%%%%%%%%%%%%%%%%%%%%%%%%%%%%%%
\begin{mdframed}[style=darkQuesion]
1. Show that the multiplicative group $\mathbf{Z}_{10}^{\times}$ is isomorphic to the additive group $\mathbf{Z}_{4}$. Hint: Find a generator $[a]_{10}$ of $\mathbf{Z}_{10}^{\mathrm{x}}$ and define $\phi: \mathbf{Z}_{4} \rightarrow \mathbf{Z}_{10}^{\times}$ by $\phi\left([n]_{4}\right)=[a]_{10}^{n}$
\end{mdframed}
%%%%%%%%%%%%%%%%%%%%%%%%%%%%%%%%%%%%%%%%%%%%%%%%%%%%%%%%%%%%%%%%%%%%%%%%%%%%%%%%
\begin{mdframed}[style=darkAnswer,frametitle={Joe Starr}]
We begin by considering $\mathbf{\Z}_{10}^{\times}=\lrs{1, 3, 7, 9}$, observe
$3=3,\ 3^2=9,\ 3^3=7,\ 3^4=1$ showing $\mathbf{Z}_{10}^{\times}=\lra{3}$. We
can then let $\varphi:\Z_4\to \Z_{10}^\times$ be $\pof{n}=3^n$ we can construct
$\pofinv{3^n}=n$ and observe $\pof{\pofinv{3^n}}=\pof{n}=3^n$ showing
$\varphi$ a bijection. We will now show $\varphi$ is a homomorphism.
Let $n,k\in \Z_4$, and $\pof{n+k}=3^{n+k}=3^{n}3^{n}=\pof{n}\pof{k}$, showing
$\varphi$ an isomorphism.
\end{mdframed}
\newpage
%%%%%%%%%%%%%%%%%%%%%%%%%%%%%%%%%%%%%%%%%%%%%%%%%%%%%%%%%%%%%%%%%%%%%%%%%%%%%%%%
%%%%%%%%%%%%%%%%%%%%%%%%%%%%%%%%%%%%%%%%%%%%%%%%%%%%%%%%%%%%%%%%%%%%%%%%%%%%%%%%
%%%%%%%%%%%%%%%%%%%%%%%%%%%%%%%%%%%%%%%%%%%%%%%%%%%%%%%%%%%%%%%%%%%%%%%%%%%%%%%%
%%%%%%%%%%%%%%%%%%%%%%%%%%%%%%%%%%%%%%%%%%%%%%%%%%%%%%%%%%%%%%%%%%%%%%%%%%%%%%%%
\begin{mdframed}[style=darkQuesion]
2. Show that the multiplicative group $\mathbf{Z}_{7}^{\times}$ is isomorphic to the additive group $\mathbf{Z}_{6}$.
\end{mdframed}
%%%%%%%%%%%%%%%%%%%%%%%%%%%%%%%%%%%%%%%%%%%%%%%%%%%%%%%%%%%%%%%%%%%%%%%%%%%%%%%%
\begin{mdframed}[style=darkAnswer,frametitle={Joe Starr}]
We begin by considering $\mathbf{\Z}_{7}^{\times}=\lrs{1,2, 3,4,5,6}$, observe
$3=3,\ 3^2=2,\ 3^3=6,\ 3^4=4,\ 3^5=5,\ 3^6=1$ showing $\mathbf{\Z}_{7}^{\times}=\lra{3}$. We
can then let $\varphi:\Z_6\to \Z_{7}^\times$ be $\pof{n}=3^n$ we can construct
$\pofinv{3^n}=n$ and observe $\pof{\pofinv{3^n}}=\pof{n}=3^n$ showing
$\varphi$ a bijection. We will now show $\varphi$ is a homomorphism.
Let $n,k\in \Z_6$, and $\pof{n+k}=3^{n+k}=3^{n}3^{n}=\pof{n}\pof{k}$, showing
$\varphi$ an isomorphism.
\end{mdframed}
\newpage
%%%%%%%%%%%%%%%%%%%%%%%%%%%%%%%%%%%%%%%%%%%%%%%%%%%%%%%%%%%%%%%%%%%%%%%%%%%%%%%%
%%%%%%%%%%%%%%%%%%%%%%%%%%%%%%%%%%%%%%%%%%%%%%%%%%%%%%%%%%%%%%%%%%%%%%%%%%%%%%%%
%%%%%%%%%%%%%%%%%%%%%%%%%%%%%%%%%%%%%%%%%%%%%%%%%%%%%%%%%%%%%%%%%%%%%%%%%%%%%%%%
%%%%%%%%%%%%%%%%%%%%%%%%%%%%%%%%%%%%%%%%%%%%%%%%%%%%%%%%%%%%%%%%%%%%%%%%%%%%%%%%
\begin{mdframed}[style=darkQuesion]
3. Show that the multiplicative group $\mathbf{Z}_{8}^{\times}$ is isomorphic to the group $\mathbf{Z}_{2} \times \mathbf{Z}_{2}$
\end{mdframed}
%%%%%%%%%%%%%%%%%%%%%%%%%%%%%%%%%%%%%%%%%%%%%%%%%%%%%%%%%%%%%%%%%%%%%%%%%%%%%%%%
\begin{mdframed}[style=darkAnswer,frametitle={Joe Starr}]
We let $\varphi: \mathbf{Z}_{8}^{\times} \to \mathbf{Z}_{2} \times \mathbf{Z}_{2}$
 by  
 \begin{equation}
  \pof{a}=
  \begin{cases}
      \lrp{0,0}  & \text{ For } a=1\\
      \lrp{0,1}  & \text{ For } a=3\\
      \lrp{1,0}  & \text{ For } a=5\\
      \lrp{1,1}  & \text{ For } a=7
  \end{cases}
\end{equation}
observe the tables for the two groups: 
$$\vbox{\tabskip0.5em\offinterlineskip
      \halign{\strut$#$\hfil\ \tabskip1em\vrule&&$#$\hfil\cr
          \cdot      & \lrp{0,0}  & \lrp{0,1}   & \lrp{1,0}   & \lrp{1,1}  \cr
          \noalign{\hrule}\vrule height 12pt width 0pt
          \lrp{0,0}  & \lrp{0,0}  & \lrp{0,1}   & \lrp{1,0}   & \lrp{1,1}    \cr
          \lrp{0,1}  & \lrp{0,1}  & \lrp{0,0}   & \lrp{1,1}   & \lrp{1,0}    \cr
          \lrp{1,0}  & \lrp{1,0}  & \lrp{1,1}   & \lrp{0,0}   & \lrp{0,1}    \cr
          \lrp{1,1}  & \lrp{1,1}  & \lrp{1,0}   & \lrp{0,1}   & \lrp{0,0}    \cr
        }}$$
        $$\vbox{\tabskip0.5em\offinterlineskip
      \halign{\strut$#$\hfil\ \tabskip1em\vrule&&$#$\hfil\cr
          \cdot      & 1  & 3   & 5   & 7  \cr
          \noalign{\hrule}\vrule height 12pt width 0pt
          1  & 1  & 3   & 5   & 7    \cr
          3  & 3  & 1   & 7   & 5    \cr
          5  & 5  & 7   & 1   & 3    \cr
          7  & 7  & 5   & 3   & 1    \cr
        }}$$
\end{mdframed}
\newpage
%%%%%%%%%%%%%%%%%%%%%%%%%%%%%%%%%%%%%%%%%%%%%%%%%%%%%%%%%%%%%%%%%%%%%%%%%%%%%%%%
%%%%%%%%%%%%%%%%%%%%%%%%%%%%%%%%%%%%%%%%%%%%%%%%%%%%%%%%%%%%%%%%%%%%%%%%%%%%%%%%
%%%%%%%%%%%%%%%%%%%%%%%%%%%%%%%%%%%%%%%%%%%%%%%%%%%%%%%%%%%%%%%%%%%%%%%%%%%%%%%%
%%%%%%%%%%%%%%%%%%%%%%%%%%%%%%%%%%%%%%%%%%%%%%%%%%%%%%%%%%%%%%%%%%%%%%%%%%%%%%%%
\begin{mdframed}[style=darkQuesion]
4. Show that $\mathbf{Z}_{5}^{\times}$ is not isomorphic to $\mathbf{Z}_{8}^{\times}$ by showing that the first group has an element of order 4 but the second group does not.
\end{mdframed}
%%%%%%%%%%%%%%%%%%%%%%%%%%%%%%%%%%%%%%%%%%%%%%%%%%%%%%%%%%%%%%%%%%%%%%%%%%%%%%%%
\begin{mdframed}[style=darkAnswer,frametitle={Joe Starr}]
  We first consider $2^{1}=2,\ 2^{2}=4,\ 2^{3}=3,\ 2^{4}=1$ showing $\lra{2}$ is of
  order $4$. We can then observe the order of the elements of 
  $\mathbf{Z}_{8}^{\times}$, $3^2=1,\ 5^2=1,\ 7^2=1$ since none of the elements 
  of $\mathbf{Z}_{8}^{\times}$ have order $4$ it cant me that 
  $\mathbf{Z}_{8}^{\times}$ and $\mathbf{Z}_{5}^{\times}$ are isomorphic. 

\end{mdframed}
\newpage
%%%%%%%%%%%%%%%%%%%%%%%%%%%%%%%%%%%%%%%%%%%%%%%%%%%%%%%%%%%%%%%%%%%%%%%%%%%%%%%%
%%%%%%%%%%%%%%%%%%%%%%%%%%%%%%%%%%%%%%%%%%%%%%%%%%%%%%%%%%%%%%%%%%%%%%%%%%%%%%%%
%%%%%%%%%%%%%%%%%%%%%%%%%%%%%%%%%%%%%%%%%%%%%%%%%%%%%%%%%%%%%%%%%%%%%%%%%%%%%%%%
%%%%%%%%%%%%%%%%%%%%%%%%%%%%%%%%%%%%%%%%%%%%%%%%%%%%%%%%%%%%%%%%%%%%%%%%%%%%%%%%
\begin{mdframed}[style=darkQuesion]
6. Is the additive group $\mathbf{C}$ of complex numbers isomorphic to the multiplicative group $\mathbf{C}^{\times}$ of nonzero complex numbers?
\end{mdframed}
%%%%%%%%%%%%%%%%%%%%%%%%%%%%%%%%%%%%%%%%%%%%%%%%%%%%%%%%%%%%%%%%%%%%%%%%%%%%%%%%
\begin{mdframed}[style=darkAnswer,frametitle={Joe Starr}]
%OLD-QESTION not done yet
\end{mdframed}
\newpage
%%%%%%%%%%%%%%%%%%%%%%%%%%%%%%%%%%%%%%%%%%%%%%%%%%%%%%%%%%%%%%%%%%%%%%%%%%%%%%%%
%%%%%%%%%%%%%%%%%%%%%%%%%%%%%%%%%%%%%%%%%%%%%%%%%%%%%%%%%%%%%%%%%%%%%%%%%%%%%%%%
%%%%%%%%%%%%%%%%%%%%%%%%%%%%%%%%%%%%%%%%%%%%%%%%%%%%%%%%%%%%%%%%%%%%%%%%%%%%%%%%
%%%%%%%%%%%%%%%%%%%%%%%%%%%%%%%%%%%%%%%%%%%%%%%%%%%%%%%%%%%%%%%%%%%%%%%%%%%%%%%%
\begin{mdframed}[style=darkQuesion]
7. Let $G_{1}$ and $G_{2}$ be groups. Show that $G_{2} \times G_{1}$ is isomorphic to $G_{1} \times G_{2}$
\end{mdframed}
%%%%%%%%%%%%%%%%%%%%%%%%%%%%%%%%%%%%%%%%%%%%%%%%%%%%%%%%%%%%%%%%%%%%%%%%%%%%%%%%
\begin{mdframed}[style=darkAnswer,frametitle={Joe Starr}]
Let $\varphi:G_{2} \times G_{1} \to G_{1} \times G_{2}$ be 
$\pof{\lrp{a,b}}=\lrp{b,a}$ and $\pofinv{\lrp{b,a}}=\lrp{a,b}$. We first 
establish $\varphi$ as a bijection, 
$\pof{\pofinv{(b,a)}}=\pof{\lrp{a,b}}=\lrp{b,a}$. Now we will establish $\varphi$
as a homomorphism, let $\lrp{a,b},\lrp{x,y}\in G_{2} \times G_{1}$
consider 
\begin{align*}
  \pof{\lrp{a,b}\lrp{x,y}}&=\pof{\lrp{ax,by}}\\
  &=\lrp{by,ax}\\
  &=\lrp{b,a}\lrp{y,x}\\
  &=\pof{\lrp{a,b}}\pof{\lrp{x,y}}\\
\end{align*}
showing the groups to be isomorphic. 
\end{mdframed}
\newpage
%%%%%%%%%%%%%%%%%%%%%%%%%%%%%%%%%%%%%%%%%%%%%%%%%%%%%%%%%%%%%%%%%%%%%%%%%%%%%%%%
%%%%%%%%%%%%%%%%%%%%%%%%%%%%%%%%%%%%%%%%%%%%%%%%%%%%%%%%%%%%%%%%%%%%%%%%%%%%%%%%
%%%%%%%%%%%%%%%%%%%%%%%%%%%%%%%%%%%%%%%%%%%%%%%%%%%%%%%%%%%%%%%%%%%%%%%%%%%%%%%%
%%%%%%%%%%%%%%%%%%%%%%%%%%%%%%%%%%%%%%%%%%%%%%%%%%%%%%%%%%%%%%%%%%%%%%%%%%%%%%%%
\begin{mdframed}[style=darkQuesion]
8. Let $G$ be a group. Show that the group $(G, *)$ defined in Exercise 3 of Section 3.1 is isomorphic to $G .$
\end{mdframed}
%%%%%%%%%%%%%%%%%%%%%%%%%%%%%%%%%%%%%%%%%%%%%%%%%%%%%%%%%%%%%%%%%%%%%%%%%%%%%%%%
\begin{mdframed}[style=darkAnswer,frametitle={Joe Starr}]
  Let $\grp{G}{\cdot}$ be a group. Define a new binary operation $\ast$ on
  $G$ by the formula $a \ast b=b\cdot a$, for all $a,b\in G$. Define 
  $\varphi:(G, \cdot)\to(G, *) $, with $\pof{a}=a$ the trivial map is a bijection. 
  Next we will show $\varphi$ to be a homomorphism let $a,b\in G$, observe 
  $\pof{a\cdot b}= a\cdot b=b\ast a=\pof{b}\ast\pof{a}$.
\end{mdframed}
\newpage
%%%%%%%%%%%%%%%%%%%%%%%%%%%%%%%%%%%%%%%%%%%%%%%%%%%%%%%%%%%%%%%%%%%%%%%%%%%%%%%%
%%%%%%%%%%%%%%%%%%%%%%%%%%%%%%%%%%%%%%%%%%%%%%%%%%%%%%%%%%%%%%%%%%%%%%%%%%%%%%%%
%%%%%%%%%%%%%%%%%%%%%%%%%%%%%%%%%%%%%%%%%%%%%%%%%%%%%%%%%%%%%%%%%%%%%%%%%%%%%%%%
%%%%%%%%%%%%%%%%%%%%%%%%%%%%%%%%%%%%%%%%%%%%%%%%%%%%%%%%%%%%%%%%%%%%%%%%%%%%%%%%
\begin{mdframed}[style=darkQuesion]
9. Prove that any group with three elements must be isomorphic to $\mathbf{Z}_{3}$.
\end{mdframed}
%%%%%%%%%%%%%%%%%%%%%%%%%%%%%%%%%%%%%%%%%%%%%%%%%%%%%%%%%%%%%%%%%%%%%%%%%%%%%%%%
\begin{mdframed}[style=darkAnswer,frametitle={Joe Starr}]
Let $G$ be a group of three elements. Since $G$ is a group there exists an 
identity element $e$ in $G$. This makes $a,b\in G$ with $a\neq b \neq e$ and WOLG
we are left with two options either $aa=e$ or $ab=e$. If $aa=e$ we get the table
$$\vbox{\tabskip0.5em\offinterlineskip
\halign{\strut$#$\hfil\ \tabskip1em\vrule&&$#$\hfil\cr
    \cdot      & e  & a   & b   \cr
    \noalign{\hrule}\vrule height 12pt width 0pt
    e  & e  & a   & b      \cr
    a  & a  & e   & b      \cr
    b  & b  & b   & e      \cr
  }}$$
observe $a\cdot\lrp{b\cdot b}=a$ but $\lrp{a\cdot b}\cdot b=e$ showing this is 
not a group under $\cdot$. 
Leaving us with one option given by the table below, which by comparison is 
isomorphic to $\Z_3$
$$\vbox{\tabskip0.5em\offinterlineskip
\halign{\strut$#$\hfil\ \tabskip1em\vrule&&$#$\hfil\cr
    \cdot      & e  & a   & b   \cr
    \noalign{\hrule}\vrule height 12pt width 0pt
    e  & e  & a   & b      \cr
    a  & a  & b   & e      \cr
    b  & b  & e   & a      \cr
  }}$$

$$\vbox{\tabskip0.5em\offinterlineskip
\halign{\strut$#$\hfil\ \tabskip1em\vrule&&$#$\hfil\cr
    \cdot      & 0  & 1   & 2   \cr
    \noalign{\hrule}\vrule height 12pt width 0pt
    0  & 0  & 1   & 2      \cr
    1  & 1  & 2   & 0      \cr
    2  & 2  & 0   & 1      \cr
  }}$$
\end{mdframed}
\newpage
%%%%%%%%%%%%%%%%%%%%%%%%%%%%%%%%%%%%%%%%%%%%%%%%%%%%%%%%%%%%%%%%%%%%%%%%%%%%%%%%
%%%%%%%%%%%%%%%%%%%%%%%%%%%%%%%%%%%%%%%%%%%%%%%%%%%%%%%%%%%%%%%%%%%%%%%%%%%%%%%%
%%%%%%%%%%%%%%%%%%%%%%%%%%%%%%%%%%%%%%%%%%%%%%%%%%%%%%%%%%%%%%%%%%%%%%%%%%%%%%%%
%%%%%%%%%%%%%%%%%%%%%%%%%%%%%%%%%%%%%%%%%%%%%%%%%%%%%%%%%%%%%%%%%%%%%%%%%%%%%%%%
\begin{mdframed}[style=darkQuesion]
13. Let $G$ be the set of all matrices in $\mathrm{GL}_{2}\left(\mathbf{Z}_{3}\right)$ of the form $\left[\begin{array}{ll}1 & 0 \\ c & d\end{array}\right] .$ That is, $c, d \in \mathbf{Z}_{3}$ and $d \neq[0]_{3} .$ Show that $G$ is isomorphic to $S_{3}$
\end{mdframed}
%%%%%%%%%%%%%%%%%%%%%%%%%%%%%%%%%%%%%%%%%%%%%%%%%%%%%%%%%%%%%%%%%%%%%%%%%%%%%%%%
\begin{mdframed}[style=darkAnswer,frametitle={Joe Starr}]
$$\begin{matrix}
  I=\begin{bmatrix}
    1 & 0 \\ 0 & 1 
  \end{bmatrix} & 
  A=\begin{bmatrix}
    1 & 0 \\ 1 & 1 
  \end{bmatrix} &
  B=\begin{bmatrix}
    1 & 0 \\ 1 & 2 
  \end{bmatrix} \\
  C=\begin{bmatrix}
    1 & 0 \\ 0 & 2 
  \end{bmatrix} & 
  D=\begin{bmatrix}
    1 & 0 \\ 2 & 1 
  \end{bmatrix} &
  E=\begin{bmatrix}
    1 & 0 \\ 2 & 2 
  \end{bmatrix} \\
\end{matrix}$$
\end{mdframed}
$$\vbox{\tabskip0.5em\offinterlineskip
      \halign{\strut$#$\hfil\ \tabskip1em\vrule&&$#$\hfil\cr
          \cdot        & \lrp{}       & \lrp{1,2}   & \lrp{2,3}   & \lrp{1,3}   & \lrp{1,2,3}  & \lrp{1,3,2}  \cr
          \noalign{\hrule}\vrule height 12pt width 0pt
          \lrp{}       & \lrp{}       & \lrp{1,2}   & \lrp{2,3}   & \lrp{1,3}   & \lrp{1,2,3}  & \lrp{1,3,2} \cr
          \lrp{1,2}    & \lrp{1,2}    & \lrp{}      & \lrp{1,2,3} & \lrp{1,3,2} & \lrp{2,3}    & \lrp{1,3} \cr
          \lrp{2,3}    & \lrp{2,3}    & \lrp{1,3,2} & \lrp{}      & \lrp{1,2,3} & \lrp{1,3}    & \lrp{1,2} \cr
          \lrp{1,3}    & \lrp{1,3}    & \lrp{1,2,3} & \lrp{1,3,2} & \lrp{}      & \lrp{1,2}    & \lrp{2,3} \cr
          \lrp{1,2,3}  & \lrp{1,2,3}  & \lrp{1,3}   & \lrp{1,2}   & \lrp{2,3}   & \lrp{1,3,2}  & \lrp{}    \cr
          \lrp{1,3,2}  & \lrp{1,3,2}  & \lrp{2,3}   & \lrp{1,3}   & \lrp{1,2}   & \lrp{}       & \lrp{1,2,3} \cr
        }}$$
$$\begin{matrix}
  I \times I = I &
  I \times A = A &
  I \times B = B &
  I \times C = C \\ \hspace{.5in} \\
  I \times D = D &
  I \times E = E &
  A \times I = A &
  A \times A = D \\ \hspace{.5in} \\
  A \times B = E &
  A \times C = B &
  A \times D = I &
  A \times E = C \\ \hspace{.5in} \\
  B \times I = B &
  B \times A = C &
  B \times B = I &
  B \times C = A \\ \hspace{.5in} \\
  B \times D = E &
  B \times E = D &
  C \times I = C &
  C \times A = E \\ \hspace{.5in} \\
  C \times B = D &
  C \times C = I &
  C \times D = B &
  C \times E = A \\ \hspace{.5in} \\
  D \times I = D &
  D \times A = I &
  D \times B = C &
  D \times C = E \\ \hspace{.5in} \\
  D \times D = A &
  D \times E = B &
  E \times I = E &
  E \times A = B \\ \hspace{.5in} \\
  E \times B = A &
  E \times C = D &
  E \times D = C &
  E \times E = I \\ \hspace{.5in} \\
\end{matrix}$$
%OLD-QESTION finish this
\newpage
%%%%%%%%%%%%%%%%%%%%%%%%%%%%%%%%%%%%%%%%%%%%%%%%%%%%%%%%%%%%%%%%%%%%%%%%%%%%%%%%
%%%%%%%%%%%%%%%%%%%%%%%%%%%%%%%%%%%%%%%%%%%%%%%%%%%%%%%%%%%%%%%%%%%%%%%%%%%%%%%%
%%%%%%%%%%%%%%%%%%%%%%%%%%%%%%%%%%%%%%%%%%%%%%%%%%%%%%%%%%%%%%%%%%%%%%%%%%%%%%%%
%%%%%%%%%%%%%%%%%%%%%%%%%%%%%%%%%%%%%%%%%%%%%%%%%%%%%%%%%%%%%%%%%%%%%%%%%%%%%%%%
\begin{mdframed}[style=darkQuesion]
15. Let $C_{2}$ be the subgroup $\{\pm 1\}$ of the multiplicative group
$\mathbf{R}^{\times} .$ Show that $\mathbf{R}^{\times}$ is isomorphic to
$\mathbf{R}^{+} \times C_{2}$
\end{mdframed}
%%%%%%%%%%%%%%%%%%%%%%%%%%%%%%%%%%%%%%%%%%%%%%%%%%%%%%%%%%%%%%%%%%%%%%%%%%%%%%%%
\begin{mdframed}[style=darkAnswer,frametitle={Joe Starr}]
Let $\pof{\lrp{x,a}}=ae^x$, and $\pofinv{ae^x}=\lrp{x,a}$ observe 
$\pof{\pofinv{ae^x}}=\pof{\lrp{x,a}}=ae^x$.

Next consider $\pof{\lrp{x,a}\lrp{y,b}}= \lrp{x+y,ab}=abe^{x+y}
=ae^xbe^y=\pof{\lrp{x,a}}\pof{\lrp{y,b}}$ as desired. 

\end{mdframed}
\newpage
%%%%%%%%%%%%%%%%%%%%%%%%%%%%%%%%%%%%%%%%%%%%%%%%%%%%%%%%%%%%%%%%%%%%%%%%%%%%%%%%
%%%%%%%%%%%%%%%%%%%%%%%%%%%%%%%%%%%%%%%%%%%%%%%%%%%%%%%%%%%%%%%%%%%%%%%%%%%%%%%%
%%%%%%%%%%%%%%%%%%%%%%%%%%%%%%%%%%%%%%%%%%%%%%%%%%%%%%%%%%%%%%%%%%%%%%%%%%%%%%%%
%%%%%%%%%%%%%%%%%%%%%%%%%%%%%%%%%%%%%%%%%%%%%%%%%%%%%%%%%%%%%%%%%%%%%%%%%%%%%%%%
\begin{mdframed}[style=darkQuesion]
17. Let $G$ be any group, and let $a$ be a fixed element of $G .$ Define a function $\phi_{a}: G \rightarrow G$
by $\phi_{a}(x)=a x a^{-1},$ for all $x \in G .$ Show that $\phi_{a}$ is an isomorphism.
\end{mdframed}
%%%%%%%%%%%%%%%%%%%%%%%%%%%%%%%%%%%%%%%%%%%%%%%%%%%%%%%%%%%%%%%%%%%%%%%%%%%%%%%%
\begin{mdframed}[style=darkAnswer,frametitle={Joe Starr}]
Define $\pofinv{x}=\inv{a}xa$, consider $\pofinv{\pof{x}}=\pofinv{a x a^{-1}}=\inv{a}a x \inv{a}a=x$.

Next consider $\pof{xy}= a xy \inv{a}=ax\inv{a}ay\inv{a}=\pof{x}\pof{y}$.
\end{mdframed}
\newpage
%%%%%%%%%%%%%%%%%%%%%%%%%%%%%%%%%%%%%%%%%%%%%%%%%%%%%%%%%%%%%%%%%%%%%%%%%%%%%%%%
%%%%%%%%%%%%%%%%%%%%%%%%%%%%%%%%%%%%%%%%%%%%%%%%%%%%%%%%%%%%%%%%%%%%%%%%%%%%%%%%
%%%%%%%%%%%%%%%%%%%%%%%%%%%%%%%%%%%%%%%%%%%%%%%%%%%%%%%%%%%%%%%%%%%%%%%%%%%%%%%%
%%%%%%%%%%%%%%%%%%%%%%%%%%%%%%%%%%%%%%%%%%%%%%%%%%%%%%%%%%%%%%%%%%%%%%%%%%%%%%%%
\begin{mdframed}[style=darkQuesion]
18. Let $G$ be any group. Define $\phi: G \rightarrow G$ by $\phi(x)=x^{-1},$ for all $x \in G$
(a) Prove that $\phi$ is one-to-one and onto.
(b) Prove that $\phi$ is an isomorphism if and only if $G$ is abelian.
\end{mdframed}
%%%%%%%%%%%%%%%%%%%%%%%%%%%%%%%%%%%%%%%%%%%%%%%%%%%%%%%%%%%%%%%%%%%%%%%%%%%%%%%%
\begin{mdframed}[style=darkAnswer,frametitle={Joe Starr}]
We've shown uniqueness of inverses, and since $G$ is a group $\varphi$ is a 
bijection. 
$G$ is abelian: \\
$\pof{ab}=\inv{ab}=\inv{a}\inv{b}=\pof{a}\pof{b}$
$G$ is non-abelian:\\
$\pof{ab}=\inv{ab}\neq\inv{a}\inv{b}$ so is not a homomorphism. 

\end{mdframed}
\newpage
%%%%%%%%%%%%%%%%%%%%%%%%%%%%%%%%%%%%%%%%%%%%%%%%%%%%%%%%%%%%%%%%%%%%%%%%%%%%%%%%
%%%%%%%%%%%%%%%%%%%%%%%%%%%%%%%%%%%%%%%%%%%%%%%%%%%%%%%%%%%%%%%%%%%%%%%%%%%%%%%%
%%%%%%%%%%%%%%%%%%%%%%%%%%%%%%%%%%%%%%%%%%%%%%%%%%%%%%%%%%%%%%%%%%%%%%%%%%%%%%%%
%%%%%%%%%%%%%%%%%%%%%%%%%%%%%%%%%%%%%%%%%%%%%%%%%%%%%%%%%%%%%%%%%%%%%%%%%%%%%%%%
\begin{mdframed}[style=darkQuesion]
22. Let $G_{1}$ and $G_{2}$ be groups. Show that $G_{1}$ is isomorphic to the subgroup of the direct product $G_{1} \times G_{2}$ defined by $\left\{\left(x_{1}, x_{2}\right) | x_{2}=e\right\}$
\end{mdframed}
%%%%%%%%%%%%%%%%%%%%%%%%%%%%%%%%%%%%%%%%%%%%%%%%%%%%%%%%%%%%%%%%%%%%%%%%%%%%%%%%
\begin{mdframed}[style=darkAnswer,frametitle={Joe Starr}]
  Define $\pof{a}=\lrp{a,e}$, and $\pofinv{\lrp{a,e}}=a$, consider 
  $\pofinv{\pof{a}}=\pofinv{\lrp{a,e}}=a$. 
  
  Now consider $\pof{ab}=\lrp{ab,e}=\lrp{a,e}\lrp{b,e}=\pof{a}\pof{b}$
\end{mdframed}
\newpage
%%%%%%%%%%%%%%%%%%%%%%%%%%%%%%%%%%%%%%%%%%%%%%%%%%%%%%%%%%%%%%%%%%%%%%%%%%%%%%%%
%%%%%%%%%%%%%%%%%%%%%%%%%%%%%%%%%%%%%%%%%%%%%%%%%%%%%%%%%%%%%%%%%%%%%%%%%%%%%%%%
%%%%%%%%%%%%%%%%%%%%%%%%%%%%%%%%%%%%%%%%%%%%%%%%%%%%%%%%%%%%%%%%%%%%%%%%%%%%%%%%
%%%%%%%%%%%%%%%%%%%%%%%%%%%%%%%%%%%%%%%%%%%%%%%%%%%%%%%%%%%%%%%%%%%%%%%%%%%%%%%%
\begin{mdframed}[style=darkQuesion]
23. Prove that if $m, n$ are positive integers such that $\operatorname{gcd}(m, n)=1,$ then $\mathbf{Z}_{m n}^{\times}$ is isomorphic to $\mathbf{Z}_{m}^{\times} \times \mathbf{Z}_{n}^{\times}$
\end{mdframed}
%%%%%%%%%%%%%%%%%%%%%%%%%%%%%%%%%%%%%%%%%%%%%%%%%%%%%%%%%%%%%%%%%%%%%%%%%%%%%%%%
\begin{mdframed}[style=darkAnswer,frametitle={Joe Starr}]
%OLD-QESTION not done yet
\end{mdframed}
\newpage
%%%%%%%%%%%%%%%%%%%%%%%%%%%%%%%%%%%%%%%%%%%%%%%%%%%%%%%%%%%%%%%%%%%%%%%%%%%%%%%%
%%%%%%%%%%%%%%%%%%%%%%%%%%%%%%%%%%%%%%%%%%%%%%%%%%%%%%%%%%%%%%%%%%%%%%%%%%%%%%%%
%%%%%%%%%%%%%%%%%%%%%%%%%%%%%%%%%%%%%%%%%%%%%%%%%%%%%%%%%%%%%%%%%%%%%%%%%%%%%%%%
%%%%%%%%%%%%%%%%%%%%%%%%%%%%%%%%%%%%%%%%%%%%%%%%%%%%%%%%%%%%%%%%%%%%%%%%%%%%%%%%
\begin{mdframed}[style=darkQuesion]
30. Let $G_{1}$ and $G_{2}$ be groups. A function from $G_{1}$ into $G_{2}$ that preserves products but
is not necessarily a one-to-one correspondence will be called a group homomorphism, from the Greek word homos meaning same. Show that $\phi: \mathrm{GL}_{2}(\mathbf{R}) \rightarrow \mathbf{R}^{\times}$ defined by $\phi(A)=\operatorname{det}(A)$ for all matrices $A \in \mathrm{GL}_{2}(\mathbf{R})$ is a group homomorphism.
\end{mdframed}
%%%%%%%%%%%%%%%%%%%%%%%%%%%%%%%%%%%%%%%%%%%%%%%%%%%%%%%%%%%%%%%%%%%%%%%%%%%%%%%%
\begin{mdframed}[style=darkAnswer,frametitle={Joe Starr}]
%OLD-QESTION not done yet
\end{mdframed}
\newpage
