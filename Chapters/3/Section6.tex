\subsection{Permutation Groups}
%1,5,13,16,17,18
%%%%%%%%%%%%%%%%%%%%%%%%%%%%%%%%%%%%%%%%%%%%%%%%%%%%%%%%%%%%%%%%%%%%%%%%%%%%%%%%
%%%%%%%%%%%%%%%%%%%%%%%%%%%%%%%%%%%%%%%%%%%%%%%%%%%%%%%%%%%%%%%%%%%%%%%%%%%%%%%%
%%%%%%%%%%%%%%%%%%%%%%%%%%%%%%%%%%%%%%%%%%%%%%%%%%%%%%%%%%%%%%%%%%%%%%%%%%%%%%%%
%%%%%%%%%%%%%%%%%%%%%%%%%%%%%%%%%%%%%%%%%%%%%%%%%%%%%%%%%%%%%%%%%%%%%%%%%%%%%%%%
\begin{mdframed}[style=darkQuesion]
  1. Find the orders of each of these permutations.
  \begin{itemize}
    \item []{(a)$(1,2)(2,3)(3,4)$}
    \item []{(b)$(1,2,5)(2,3,4)(5,6)$}
    \item []{(c)$(1,3)(2,6)(1,4,5)$}
    \item []{(d)$(1,2,3)(2,4,3,5)(1,3,2)$}
  \end{itemize}
\end{mdframed}
%%%%%%%%%%%%%%%%%%%%%%%%%%%%%%%%%%%%%%%%%%%%%%%%%%%%%%%%%%%%%%%%%%%%%%%%%%%%%%%%
\begin{mdframed}[style=darkAnswer,frametitle={Joe Starr}]
  \begin{enumerate}[(a)]
    \item{$(1,2)(2,3)(3,4)=\lrp{1,2,3,4}$ order is 4}
    \item{$(1,2,5)(2,3,4)(5,6)=\lrp{1,2,3,4,5,6}$ order is 6}
    \item{$(1,3)(2,6)(1,4,5)=(2,6)(1,3)(1,4,5)=(2,6)(3,1,4,5)$ order is 4}
    \item{$(1,2,3)(2,4,3,5)(1,3,2)=\lrp{1,5,3,4}$ order is 4}
  \end{enumerate} 
\end{mdframed}
\newpage
%%%%%%%%%%%%%%%%%%%%%%%%%%%%%%%%%%%%%%%%%%%%%%%%%%%%%%%%%%%%%%%%%%%%%%%%%%%%%%%%
%%%%%%%%%%%%%%%%%%%%%%%%%%%%%%%%%%%%%%%%%%%%%%%%%%%%%%%%%%%%%%%%%%%%%%%%%%%%%%%%
%%%%%%%%%%%%%%%%%%%%%%%%%%%%%%%%%%%%%%%%%%%%%%%%%%%%%%%%%%%%%%%%%%%%%%%%%%%%%%%%
%%%%%%%%%%%%%%%%%%%%%%%%%%%%%%%%%%%%%%%%%%%%%%%%%%%%%%%%%%%%%%%%%%%%%%%%%%%%%%%%
\begin{mdframed}[style=darkQuesion]
  5. Show that no proper subgroup of $S_4$ contains both $\lrp{1,2,3,4}$ and $\lrp{1,2}$
\end{mdframed}
%%%%%%%%%%%%%%%%%%%%%%%%%%%%%%%%%%%%%%%%%%%%%%%%%%%%%%%%%%%%%%%%%%%%%%%%%%%%%%%%
\begin{mdframed}[style=darkAnswer,frametitle={Joe Starr}]
 We let $H\leq S_4$ with $\lrp{1,2,3,4},\ \lrp{1,2}\in H$,by Lagrange's theorem 
 we know that any subgroups of $S_4$ must be of order 
 $1,\ 2,\ 3,\ 4,\ 6,\ 8,\ 12,\ 24$. We observe the cyclic subgroup of 
 $\lra{\lrp{1,2,3,4}}\in H$ the order of $H$ must 
 be larger then 4. Similarly observe that $\lrp{1,2}\lrp{1,2,3,4}=\lrp{2,3,4}$, meaning 
 $6\geq\abs{H}$. Finely we have by construction, $\lrp{1,2}\in H$, so $7\leq\abs{H}$. 
 Since $\abs{H}\ \vert\  \abs{S_4}$, it must be that $8\geq\abs{H}$. 

 %TODO not done
\end{mdframed}
\newpage
%%%%%%%%%%%%%%%%%%%%%%%%%%%%%%%%%%%%%%%%%%%%%%%%%%%%%%%%%%%%%%%%%%%%%%%%%%%%%%%%
%%%%%%%%%%%%%%%%%%%%%%%%%%%%%%%%%%%%%%%%%%%%%%%%%%%%%%%%%%%%%%%%%%%%%%%%%%%%%%%%
%%%%%%%%%%%%%%%%%%%%%%%%%%%%%%%%%%%%%%%%%%%%%%%%%%%%%%%%%%%%%%%%%%%%%%%%%%%%%%%%
%%%%%%%%%%%%%%%%%%%%%%%%%%%%%%%%%%%%%%%%%%%%%%%%%%%%%%%%%%%%%%%%%%%%%%%%%%%%%%%%
\begin{mdframed}[style=darkQuesion]
  13. List the elements of $A_4$
\end{mdframed}
%%%%%%%%%%%%%%%%%%%%%%%%%%%%%%%%%%%%%%%%%%%%%%%%%%%%%%%%%%%%%%%%%%%%%%%%%%%%%%%%
\begin{mdframed}[style=darkAnswer,frametitle={Joe Starr}]
  $(),\ (123),\ (124),\ (132),\ (134),\ (142),\ (143),\ (234),\ (243),\ (12)(34),\ (13)(24),\ (14)(23)$
\end{mdframed}
\newpage
%%%%%%%%%%%%%%%%%%%%%%%%%%%%%%%%%%%%%%%%%%%%%%%%%%%%%%%%%%%%%%%%%%%%%%%%%%%%%%%%
%%%%%%%%%%%%%%%%%%%%%%%%%%%%%%%%%%%%%%%%%%%%%%%%%%%%%%%%%%%%%%%%%%%%%%%%%%%%%%%%
%%%%%%%%%%%%%%%%%%%%%%%%%%%%%%%%%%%%%%%%%%%%%%%%%%%%%%%%%%%%%%%%%%%%%%%%%%%%%%%%
%%%%%%%%%%%%%%%%%%%%%%%%%%%%%%%%%%%%%%%%%%%%%%%%%%%%%%%%%%%%%%%%%%%%%%%%%%%%%%%%
\begin{mdframed}[style=darkQuesion]
  16. Let $H$ be a subgroup of $S_{n}$
  \begin{itemize}
    \item []{(a) Show that either all permutations in $H$ are even, or else half of the permutations
    in $H$ are even and half are odd.}
    \item []{(b) Show that the set of all even permutations in $H$ forms a subgroup of $H$}
  \end{itemize}
\end{mdframed}
%%%%%%%%%%%%%%%%%%%%%%%%%%%%%%%%%%%%%%%%%%%%%%%%%%%%%%%%%%%%%%%%%%%%%%%%%%%%%%%%
\begin{mdframed}[style=darkAnswer,frametitle={Joe Starr}]
  \begin{enumerate}[(a)]
    \item {If H has only even permutations we are done. Otherwise we know there
    exists a set of odd permutations $H_o\subset H$, and even $H_e\subset H$. Further
    we know that $H=H_o\cup H_e$. We know that parity follows normal parity rules, 
    so $p\in H_o$, $pH_o$ makes all $px\in pH_o$ even parity and $pH_e$ makes all $px\in pH_e$
    odd parity. We observe $H=pH_o\cup pH_e$. Now, since for $x\in {pH_e}$ x is 
    an odd Permutation so it must be that $x\in H_o$. Similarly $x\in pH_o$ means
    $x\in H_e$. From here we have  $\abs{H_o}=\abs{pH_o}=\abs{pH_e}=\abs{H_e}$, 
    showing there are the same number of even and odd elements of $H$. Since 
    $H=H_o\cup H_e$ it must be that $\abs{H_e}=\frac{1}{2}\abs{H}$.
    }
    \item {If $\sigma$ and $\tau$ are even permutations, 
    then each can be expressed as a product of an even number of transpositions.
    It follows that $\tau \sigma$ can be expressed as a product of an even 
    number of transpositions, and so the set of all even permutations of 
    $H$ is closed under multiplication of permutations. 
    Furthermore, the identity permutation is even. 
    since $H$ is a finite set, this is enough to imply that we have a 
    subgroup.}
  \end{enumerate}
\end{mdframed}
\newpage
%%%%%%%%%%%%%%%%%%%%%%%%%%%%%%%%%%%%%%%%%%%%%%%%%%%%%%%%%%%%%%%%%%%%%%%%%%%%%%%%
%%%%%%%%%%%%%%%%%%%%%%%%%%%%%%%%%%%%%%%%%%%%%%%%%%%%%%%%%%%%%%%%%%%%%%%%%%%%%%%%
%%%%%%%%%%%%%%%%%%%%%%%%%%%%%%%%%%%%%%%%%%%%%%%%%%%%%%%%%%%%%%%%%%%%%%%%%%%%%%%%
%%%%%%%%%%%%%%%%%%%%%%%%%%%%%%%%%%%%%%%%%%%%%%%%%%%%%%%%%%%%%%%%%%%%%%%%%%%%%%%%
\begin{mdframed}[style=darkQuesion]
  17. For any elements $\sigma, \tau \in S_{n},$ show that $\sigma \tau \sigma^{-1} \tau^{-1} \in A_{n}$
\end{mdframed}
%%%%%%%%%%%%%%%%%%%%%%%%%%%%%%%%%%%%%%%%%%%%%%%%%%%%%%%%%%%%%%%%%%%%%%%%%%%%%%%%
\begin{mdframed}[style=darkAnswer,frametitle={Joe Starr}]
We have previously shown that parity of a Permutation and its inverse are equal. 
We have also shown that parity of permutations follows normal addition rules for 
parity. So we observe $\sigma \tau \sigma^{-1} \tau^{-1}$ is $2$ times the parity
of sigma plus two times the parity of tau. Making the parity of $\sigma \tau \sigma^{-1} \tau^{-1}$
even as desired. 
\end{mdframed}
\newpage
%%%%%%%%%%%%%%%%%%%%%%%%%%%%%%%%%%%%%%%%%%%%%%%%%%%%%%%%%%%%%%%%%%%%%%%%%%%%%%%%
%%%%%%%%%%%%%%%%%%%%%%%%%%%%%%%%%%%%%%%%%%%%%%%%%%%%%%%%%%%%%%%%%%%%%%%%%%%%%%%%
%%%%%%%%%%%%%%%%%%%%%%%%%%%%%%%%%%%%%%%%%%%%%%%%%%%%%%%%%%%%%%%%%%%%%%%%%%%%%%%%
%%%%%%%%%%%%%%%%%%%%%%%%%%%%%%%%%%%%%%%%%%%%%%%%%%%%%%%%%%%%%%%%%%%%%%%%%%%%%%%%
\begin{mdframed}[style=darkQuesion]
  18. Let $S$ be an infinite set. Let $H$ be the set of all elements $\sigma \in \operatorname{Sym}(S)$ such that $\sigma(x)=x$ for all but finitely many $x \in S .$ Prove that $H$ is a subgroup of $\operatorname{Sym}(S)$
\end{mdframed}
%%%%%%%%%%%%%%%%%%%%%%%%%%%%%%%%%%%%%%%%%%%%%%%%%%%%%%%%%%%%%%%%%%%%%%%%%%%%%%%%
\begin{mdframed}[style=darkAnswer,frametitle={Joe Starr}]
\begin{itemize}[align=left]
  \Invs{We observe that $\sigma$ has $k$ elements such that $\sigma\lrp{x}\neq x$. 
  This means that for some $\sigma\lrp{x_i}=x_j$, we select $\inv{\sigma}$ such 
  that $\inv{\sigma}\lrp{x_j}=x_i$. We see that 
  }
  \Clos{We select $\sigma_1 and \sigma_2$, we observer that they have $k_1$ and
  $k_2$ elemetns with $\sigma\lrp{x}\neq x$. If we assume in the worst case that
  these two mappings are disjoint sets of $x_i$'s we sill have the union of 
  countable sets which we know to be countable. 
  }
\end{itemize}
\end{mdframed}
\newpage