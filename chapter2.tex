
\section{Functions}  
\subsection{Functions}
%%%%%%%%%%%%%%%%%%%%%%%%%%%%%%%%%%%%%%%%%%%%%%%%%%%%%%%%%%%%%%%%%%%%%%%%%%%%%%%%
%%%%%%%%%%%%%%%%%%%%%%%%%%%%%%%%%%%%%%%%%%%%%%%%%%%%%%%%%%%%%%%%%%%%%%%%%%%%%%%%
%%%%%%%%%%%%%%%%%%%%%%%%%%%%%%%%%%%%%%%%%%%%%%%%%%%%%%%%%%%%%%%%%%%%%%%%%%%%%%%%
%%%%%%%%%%%%%%%%%%%%%%%%%%%%%%%%%%%%%%%%%%%%%%%%%%%%%%%%%%%%%%%%%%%%%%%%%%%%%%%%
\begin{mdframed}[style=darkQuesion]
  1.   In each of the following parts, determine whether the given function is 1:1
  and whether it is onto. 
\begin{itemize}
  \item [(a)]{
    $f:\R\to \R; \fof{x}=x+3$
  }
  \item [(b)]{
    $f:\C\to \C; \fof{x}=x^2+2x+1$
  }
  \item [(c]{
    $f:\Z_n\to \Z_n; \fof{\lrb{x}_n}=\lrb{mx+b}_n, \text{where} m,b\in \Z$
  }
  \item [(d)]{
    $f:\R^+\to \R; \fof{x}=\ln x$
  }
\end{itemize}
\end{mdframed}

%%%%%%%%%%%%%%%%%%%%%%%%%%%%%%%%%%%%%%%%%%%%%%%%%%%%%%%%%%%%%%%%%%%%%%%%%%%%%%%%
\begin{mdframed}[style=darkAnswer,frametitle={Joe Starr}]

\end{mdframed}
  \newpage
%%%%%%%%%%%%%%%%%%%%%%%%%%%%%%%%%%%%%%%%%%%%%%%%%%%%%%%%%%%%%%%%%%%%%%%%%%%%%%%%
%%%%%%%%%%%%%%%%%%%%%%%%%%%%%%%%%%%%%%%%%%%%%%%%%%%%%%%%%%%%%%%%%%%%%%%%%%%%%%%%
%%%%%%%%%%%%%%%%%%%%%%%%%%%%%%%%%%%%%%%%%%%%%%%%%%%%%%%%%%%%%%%%%%%%%%%%%%%%%%%%
%%%%%%%%%%%%%%%%%%%%%%%%%%%%%%%%%%%%%%%%%%%%%%%%%%%%%%%%%%%%%%%%%%%%%%%%%%%%%%%%
\begin{mdframed}[style=darkQuesion]
  3.   For each 1:1 and onto function in Exercise 1, find the inverse of the 
  function
\begin{itemize}
  \item [(a)]{
    $f:\R\to \R; \fof{x}=x+3$
  }
  \item [(b)]{
    $f:\C\to \C; \fof{x}=x^2+2x+1$
  }
  \item [(c]{
    $f:\Z_n\to \Z_n; \fof{\lrb{x}_n}=\lrb{mx+b}_n, \text{where} m,b\in \Z$
  }
  \item [(d)]{
    $f:\R^+\to \R; \fof{x}=\ln x$
  }
\end{itemize}
\end{mdframed}

%%%%%%%%%%%%%%%%%%%%%%%%%%%%%%%%%%%%%%%%%%%%%%%%%%%%%%%%%%%%%%%%%%%%%%%%%%%%%%%%
\begin{mdframed}[style=darkAnswer,frametitle={Joe Starr}]

\end{mdframed}
\newpage
%%%%%%%%%%%%%%%%%%%%%%%%%%%%%%%%%%%%%%%%%%%%%%%%%%%%%%%%%%%%%%%%%%%%%%%%%%%%%%%%
%%%%%%%%%%%%%%%%%%%%%%%%%%%%%%%%%%%%%%%%%%%%%%%%%%%%%%%%%%%%%%%%%%%%%%%%%%%%%%%%
%%%%%%%%%%%%%%%%%%%%%%%%%%%%%%%%%%%%%%%%%%%%%%%%%%%%%%%%%%%%%%%%%%%%%%%%%%%%%%%%
%%%%%%%%%%%%%%%%%%%%%%%%%%%%%%%%%%%%%%%%%%%%%%%%%%%%%%%%%%%%%%%%%%%%%%%%%%%%%%%%
\begin{mdframed}[style=darkQuesion]
  4.   For each 1:1 and onto function in Exercise 1, find the inverse of the 
  function
\begin{itemize}
  \item [(a)]{
    $f:\R\to \R; \fof{x}=x^2$
  }
  \item [(b)]{
    $f:\C\to \C; \fof{x}=x^2$
  }
  \item [(c]{
    $f:\R^+\to \R^+; \fof{x}=x^2$
  }
  \item [(d)]{
    $f:\R^+\to \R^+; \fof{x}=
    \begin{cases}
      x &\text{if }x\text{ is rational}\\
      x^2 &\text{if }x\text{ is irrational}\\
    \end{cases}$
  }
\end{itemize}
\end{mdframed}

%%%%%%%%%%%%%%%%%%%%%%%%%%%%%%%%%%%%%%%%%%%%%%%%%%%%%%%%%%%%%%%%%%%%%%%%%%%%%%%%
\begin{mdframed}[style=darkAnswer,frametitle={Joe Starr}]

\end{mdframed}
\newpage
%%%%%%%%%%%%%%%%%%%%%%%%%%%%%%%%%%%%%%%%%%%%%%%%%%%%%%%%%%%%%%%%%%%%%%%%%%%%%%%%
%%%%%%%%%%%%%%%%%%%%%%%%%%%%%%%%%%%%%%%%%%%%%%%%%%%%%%%%%%%%%%%%%%%%%%%%%%%%%%%%
%%%%%%%%%%%%%%%%%%%%%%%%%%%%%%%%%%%%%%%%%%%%%%%%%%%%%%%%%%%%%%%%%%%%%%%%%%%%%%%%
%%%%%%%%%%%%%%%%%%%%%%%%%%%%%%%%%%%%%%%%%%%%%%%%%%%%%%%%%%%%%%%%%%%%%%%%%%%%%%%%
\begin{mdframed}[style=darkQuesion]
  13. Let $f:A\to B$ be a function, and let $\fof{A}=\lrs{\fof{a}\vert a\in A}$
  be the image of $f$. Show that $f$is onto if and only if $\fof{A}=B$.
\end{mdframed}

%%%%%%%%%%%%%%%%%%%%%%%%%%%%%%%%%%%%%%%%%%%%%%%%%%%%%%%%%%%%%%%%%%%%%%%%%%%%%%%%
\begin{mdframed}[style=darkAnswer,frametitle={Joe Starr}]

\end{mdframed}
\newpage
%%%%%%%%%%%%%%%%%%%%%%%%%%%%%%%%%%%%%%%%%%%%%%%%%%%%%%%%%%%%%%%%%%%%%%%%%%%%%%%%
%%%%%%%%%%%%%%%%%%%%%%%%%%%%%%%%%%%%%%%%%%%%%%%%%%%%%%%%%%%%%%%%%%%%%%%%%%%%%%%%
%%%%%%%%%%%%%%%%%%%%%%%%%%%%%%%%%%%%%%%%%%%%%%%%%%%%%%%%%%%%%%%%%%%%%%%%%%%%%%%%
%%%%%%%%%%%%%%%%%%%%%%%%%%%%%%%%%%%%%%%%%%%%%%%%%%%%%%%%%%%%%%%%%%%%%%%%%%%%%%%%
\begin{mdframed}[style=darkQuesion]
  15. Let $f:A\to B$ and $g:B\to C$ be functions. Prove that if $g\circ f$ is 
  1:1, then $f$ is 1:1, and that if $g\circ f$ is onto $g$ is onto.
\end{mdframed}

%%%%%%%%%%%%%%%%%%%%%%%%%%%%%%%%%%%%%%%%%%%%%%%%%%%%%%%%%%%%%%%%%%%%%%%%%%%%%%%%
\begin{mdframed}[style=darkAnswer,frametitle={Joe Starr}]

\end{mdframed}
\newpage
%%%%%%%%%%%%%%%%%%%%%%%%%%%%%%%%%%%%%%%%%%%%%%%%%%%%%%%%%%%%%%%%%%%%%%%%%%%%%%%%
%%%%%%%%%%%%%%%%%%%%%%%%%%%%%%%%%%%%%%%%%%%%%%%%%%%%%%%%%%%%%%%%%%%%%%%%%%%%%%%%
%%%%%%%%%%%%%%%%%%%%%%%%%%%%%%%%%%%%%%%%%%%%%%%%%%%%%%%%%%%%%%%%%%%%%%%%%%%%%%%%
%%%%%%%%%%%%%%%%%%%%%%%%%%%%%%%%%%%%%%%%%%%%%%%%%%%%%%%%%%%%%%%%%%%%%%%%%%%%%%%%
\begin{mdframed}[style=darkQuesion]
  17. Let $f:A\to B$ be a function. Prove that $f$ is onto if and only if 
  $h\circ f = k \circ f$ implies $h=k$, for every set $C$ and all choices of 
  functions $h:B\to C$ and $k:B\to C$.
\end{mdframed}

%%%%%%%%%%%%%%%%%%%%%%%%%%%%%%%%%%%%%%%%%%%%%%%%%%%%%%%%%%%%%%%%%%%%%%%%%%%%%%%%
\begin{mdframed}[style=darkAnswer,frametitle={Joe Starr}]

\end{mdframed}
\newpage
%%%%%%%%%%%%%%%%%%%%%%%%%%%%%%%%%%%%%%%%%%%%%%%%%%%%%%%%%%%%%%%%%%%%%%%%%%%%%%%%
%%%%%%%%%%%%%%%%%%%%%%%%%%%%%%%%%%%%%%%%%%%%%%%%%%%%%%%%%%%%%%%%%%%%%%%%%%%%%%%%
%%%%%%%%%%%%%%%%%%%%%%%%%%%%%%%%%%%%%%%%%%%%%%%%%%%%%%%%%%%%%%%%%%%%%%%%%%%%%%%%
%%%%%%%%%%%%%%%%%%%%%%%%%%%%%%%%%%%%%%%%%%%%%%%%%%%%%%%%%%%%%%%%%%%%%%%%%%%%%%%%
\begin{mdframed}[style=darkQuesion]
  19. Let $f:A\to B$ be a function. Prove that $f$ is 1:1 if and only if 
  $f\circ h=f\circ k$ implies $h=k$, for every set $C$ and all choices of 
  functions $h:C\to A$ and $k:C\to A$.
\end{mdframed}
  
%%%%%%%%%%%%%%%%%%%%%%%%%%%%%%%%%%%%%%%%%%%%%%%%%%%%%%%%%%%%%%%%%%%%%%%%%%%%%%%%
\begin{mdframed}[style=darkAnswer,frametitle={Joe Starr}]

\end{mdframed}
\newpage
%%%%%%%%%%%%%%%%%%%%%%%%%%%%%%%%%%%%%%%%%%%%%%%%%%%%%%%%%%%%%%%%%%%%%%%%%%%%%%%%
%%%%%%%%%%%%%%%%%%%%%%%%%%%%%%%%%%%%%%%%%%%%%%%%%%%%%%%%%%%%%%%%%%%%%%%%%%%%%%%%
%%%%%%%%%%%%%%%%%%%%%%%%%%%%%%%%%%%%%%%%%%%%%%%%%%%%%%%%%%%%%%%%%%%%%%%%%%%%%%%%
%%%%%%%%%%%%%%%%%%%%%%%%%%%%%%%%%%%%%%%%%%%%%%%%%%%%%%%%%%%%%%%%%%%%%%%%%%%%%%%%
\subsection{Equivalence Relations}
%%%%%%%%%%%%%%%%%%%%%%%%%%%%%%%%%%%%%%%%%%%%%%%%%%%%%%%%%%%%%%%%%%%%%%%%%%%%%%%%
%%%%%%%%%%%%%%%%%%%%%%%%%%%%%%%%%%%%%%%%%%%%%%%%%%%%%%%%%%%%%%%%%%%%%%%%%%%%%%%%
%%%%%%%%%%%%%%%%%%%%%%%%%%%%%%%%%%%%%%%%%%%%%%%%%%%%%%%%%%%%%%%%%%%%%%%%%%%%%%%%
%%%%%%%%%%%%%%%%%%%%%%%%%%%%%%%%%%%%%%%%%%%%%%%%%%%%%%%%%%%%%%%%%%%%%%%%%%%%%%%%
\newpage
%%%%%%%%%%%%%%%%%%%%%%%%%%%%%%%%%%%%%%%%%%%%%%%%%%%%%%%%%%%%%%%%%%%%%%%%%%%%%%%%
%%%%%%%%%%%%%%%%%%%%%%%%%%%%%%%%%%%%%%%%%%%%%%%%%%%%%%%%%%%%%%%%%%%%%%%%%%%%%%%%
%%%%%%%%%%%%%%%%%%%%%%%%%%%%%%%%%%%%%%%%%%%%%%%%%%%%%%%%%%%%%%%%%%%%%%%%%%%%%%%%
%%%%%%%%%%%%%%%%%%%%%%%%%%%%%%%%%%%%%%%%%%%%%%%%%%%%%%%%%%%%%%%%%%%%%%%%%%%%%%%%
\subsection{Permutations} %1,2,3,4,5
%%%%%%%%%%%%%%%%%%%%%%%%%%%%%%%%%%%%%%%%%%%%%%%%%%%%%%%%%%%%%%%%%%%%%%%%%%%%%%%%
%%%%%%%%%%%%%%%%%%%%%%%%%%%%%%%%%%%%%%%%%%%%%%%%%%%%%%%%%%%%%%%%%%%%%%%%%%%%%%%%
%%%%%%%%%%%%%%%%%%%%%%%%%%%%%%%%%%%%%%%%%%%%%%%%%%%%%%%%%%%%%%%%%%%%%%%%%%%%%%%%
%%%%%%%%%%%%%%%%%%%%%%%%%%%%%%%%%%%%%%%%%%%%%%%%%%%%%%%%%%%%%%%%%%%%%%%%%%%%%%%%
\begin{mdframed}[style=darkQuesion]
  1. Consider the following Permutations in $S_7$.
  \begin{multicols}{2}
    $\sigma=  
    \begin{pmatrix}
      1 & 2 & 3 & 4 & 5 & 6 & 7\\
      3 & 2 & 5 & 4 & 6 & 1 & 7\\
    \end{pmatrix}$
    $\tau=  
    \begin{pmatrix}
      1 & 2 & 3 & 4 & 5 & 6 & 7\\
      2 & 1 & 5 & 7 & 4 & 6 & 3\\
    \end{pmatrix}$
  \end{multicols}
  \vspace{.25in}
  \begin{itemize}
    \item [(a)]{$\sigma\tau$
      
    }
    \item [(b)]{$\tau\sigma$

    }
    \item [(c)]{$\tau^2\sigma$

    }
    \item [(d)]{$\sigma^{-1}$

    }
    \item [(e)]{$\sigma\tau\sigma^{-1}$

    }
    \item [(f)]{$\tau^{-1}\sigma\tau$

    }
  \end{itemize}
\end{mdframed}
  
%%%%%%%%%%%%%%%%%%%%%%%%%%%%%%%%%%%%%%%%%%%%%%%%%%%%%%%%%%%%%%%%%%%%%%%%%%%%%%%%
\begin{mdframed}[style=darkAnswer,frametitle={Joe Starr}]
  \begin{itemize}
    \item [(a)]{$\sigma\tau$
      
    }
    \item [(b)]{$\tau\sigma$

    }
    \item [(c)]{$\tau^2\sigma$

    }
    \item [(d)]{$\sigma^{-1}$

    }
    \item [(e)]{$\sigma\tau\sigma^{-1}$

    }
    \item [(f)]{$\tau^{-1}\sigma\tau$

    }
  \end{itemize}
\end{mdframed}
\newpage
%%%%%%%%%%%%%%%%%%%%%%%%%%%%%%%%%%%%%%%%%%%%%%%%%%%%%%%%%%%%%%%%%%%%%%%%%%%%%%%%
%%%%%%%%%%%%%%%%%%%%%%%%%%%%%%%%%%%%%%%%%%%%%%%%%%%%%%%%%%%%%%%%%%%%%%%%%%%%%%%%
%%%%%%%%%%%%%%%%%%%%%%%%%%%%%%%%%%%%%%%%%%%%%%%%%%%%%%%%%%%%%%%%%%%%%%%%%%%%%%%%
%%%%%%%%%%%%%%%%%%%%%%%%%%%%%%%%%%%%%%%%%%%%%%%%%%%%%%%%%%%%%%%%%%%%%%%%%%%%%%%%
\begin{mdframed}[style=darkQuesion]
  2. Write each of the permutations $\sigma\tau, \tau\sigma, \tau^2\sigma, 
  \sigma^{-1}, \sigma\tau\sigma^{-1}, \text{ and } \tau^{-1}\sigma\tau$ in 
  Exercise 1 as a product of disjoint cycles. Write $\sigma$ anf $\tau$ as 
  products of transpositions. 
\end{mdframed}
  
%%%%%%%%%%%%%%%%%%%%%%%%%%%%%%%%%%%%%%%%%%%%%%%%%%%%%%%%%%%%%%%%%%%%%%%%%%%%%%%%
\begin{mdframed}[style=darkAnswer,frametitle={Joe Starr}]
  \begin{itemize}
    \item [(a)]{$\sigma\tau$
      
    }
    \item [(b)]{$\tau\sigma$

    }
    \item [(c)]{$\tau^2\sigma$

    }
    \item [(d)]{$\sigma^{-1}$

    }
    \item [(e)]{$\sigma\tau\sigma^{-1}$

    }
    \item [(f)]{$\tau^{-1}\sigma\tau$

    }
  \end{itemize}
\end{mdframed}
\newpage
%%%%%%%%%%%%%%%%%%%%%%%%%%%%%%%%%%%%%%%%%%%%%%%%%%%%%%%%%%%%%%%%%%%%%%%%%%%%%%%%
%%%%%%%%%%%%%%%%%%%%%%%%%%%%%%%%%%%%%%%%%%%%%%%%%%%%%%%%%%%%%%%%%%%%%%%%%%%%%%%%
%%%%%%%%%%%%%%%%%%%%%%%%%%%%%%%%%%%%%%%%%%%%%%%%%%%%%%%%%%%%%%%%%%%%%%%%%%%%%%%%
%%%%%%%%%%%%%%%%%%%%%%%%%%%%%%%%%%%%%%%%%%%%%%%%%%%%%%%%%%%%%%%%%%%%%%%%%%%%%%%%
\begin{mdframed}[style=darkQuesion]
  3. Write 
    $\begin{pmatrix}
      1 & 2 & 3 & 4 & 5 & 6 & 7 & 8 & 9 & 10\\
      3 & 4 & 10 & 5 & 7 & 8 & 2 & 6 & 9 & 1\\
    \end{pmatrix}$ as the product of disjoint cycles and as a product of 
    transpositions. Construct its associated diagram, find its inverse, and find 
    it's order. 
\end{mdframed}
  
%%%%%%%%%%%%%%%%%%%%%%%%%%%%%%%%%%%%%%%%%%%%%%%%%%%%%%%%%%%%%%%%%%%%%%%%%%%%%%%%
\begin{mdframed}[style=darkAnswer,frametitle={Joe Starr}]
\end{mdframed}
\newpage
%%%%%%%%%%%%%%%%%%%%%%%%%%%%%%%%%%%%%%%%%%%%%%%%%%%%%%%%%%%%%%%%%%%%%%%%%%%%%%%%
%%%%%%%%%%%%%%%%%%%%%%%%%%%%%%%%%%%%%%%%%%%%%%%%%%%%%%%%%%%%%%%%%%%%%%%%%%%%%%%%
%%%%%%%%%%%%%%%%%%%%%%%%%%%%%%%%%%%%%%%%%%%%%%%%%%%%%%%%%%%%%%%%%%%%%%%%%%%%%%%%
%%%%%%%%%%%%%%%%%%%%%%%%%%%%%%%%%%%%%%%%%%%%%%%%%%%%%%%%%%%%%%%%%%%%%%%%%%%%%%%%
\begin{mdframed}[style=darkQuesion]
  3. Find the oder of each of the following permutations. 
  \begin{itemize}
    \item [(a)] {
      $\begin{pmatrix}
        1 & 2 & 3 & 4 & 5 & 6\\
        6 & 4 & 5 & 3 & 2 & 1\\
      \end{pmatrix}$
    }
    \item [(b)] {
      $\begin{pmatrix}
        1 & 2 & 3 & 4 & 5 & 6 & 7 & 8\\
        4 & 6 & 7 & 5 & 1 & 8 & 2 & 3\\
      \end{pmatrix}$
    }
    \item [(c)] {
      $\begin{pmatrix}
        1 & 2 & 3 & 4 & 5 & 6 & 7 & 8 & 9\\
        5 & 9 & 8 & 7 & 3 & 4 & 6 & 1 & 2\\
      \end{pmatrix}$
    }
    \item [(d)] {
      $\begin{pmatrix}
        1 & 2 & 3 & 4 & 5 & 6 & 7 & 8 & 9 \\
        8 & 4 & 9 & 6 & 5 & 2 & 3 & 1 & 7\\
      \end{pmatrix}$
    }
  \end{itemize} 
\end{mdframed}
  
%%%%%%%%%%%%%%%%%%%%%%%%%%%%%%%%%%%%%%%%%%%%%%%%%%%%%%%%%%%%%%%%%%%%%%%%%%%%%%%%
\begin{mdframed}[style=darkAnswer,frametitle={Joe Starr}]
  \begin{itemize}
    \item [(a)] {
      $\begin{pmatrix}
        1 & 2 & 3 & 4 & 5 & 6\\
        6 & 4 & 5 & 3 & 2 & 1\\
      \end{pmatrix}$
    }
    \item [(b)] {
      $\begin{pmatrix}
        1 & 2 & 3 & 4 & 5 & 6 & 7 & 8\\
        4 & 6 & 7 & 5 & 1 & 8 & 2 & 3\\
      \end{pmatrix}$
    }
    \item [(c)] {
      $\begin{pmatrix}
        1 & 2 & 3 & 4 & 5 & 6 & 7 & 8 & 9\\
        5 & 9 & 8 & 7 & 3 & 4 & 6 & 1 & 2\\
      \end{pmatrix}$
    }
    \item [(d)] {
      $\begin{pmatrix}
        1 & 2 & 3 & 4 & 5 & 6 & 7 & 8 & 9 \\
        8 & 4 & 9 & 6 & 5 & 2 & 3 & 1 & 7\\
      \end{pmatrix}$
    }
  \end{itemize} 
\end{mdframed}
\newpage
%%%%%%%%%%%%%%%%%%%%%%%%%%%%%%%%%%%%%%%%%%%%%%%%%%%%%%%%%%%%%%%%%%%%%%%%%%%%%%%%
%%%%%%%%%%%%%%%%%%%%%%%%%%%%%%%%%%%%%%%%%%%%%%%%%%%%%%%%%%%%%%%%%%%%%%%%%%%%%%%%
%%%%%%%%%%%%%%%%%%%%%%%%%%%%%%%%%%%%%%%%%%%%%%%%%%%%%%%%%%%%%%%%%%%%%%%%%%%%%%%%
%%%%%%%%%%%%%%%%%%%%%%%%%%%%%%%%%%%%%%%%%%%%%%%%%%%%%%%%%%%%%%%%%%%%%%%%%%%%%%%%
\begin{mdframed}[style=darkQuesion]
  5. Let $3\leq m\leq n$. Calculate $\sigma\tau^{-1}$ for cycles 
  $\sigma= \lrp{1,2,\dots , m-1}$ and \\ $\tau= \lrp{1,2,\dots,m-1,m}$ in $S_n$.
\end{mdframed}
  
%%%%%%%%%%%%%%%%%%%%%%%%%%%%%%%%%%%%%%%%%%%%%%%%%%%%%%%%%%%%%%%%%%%%%%%%%%%%%%%%
\begin{mdframed}[style=darkAnswer,frametitle={Joe Starr}]
\end{mdframed}
\newpage
