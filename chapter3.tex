\section{Groups}
\subsection{Definition of a Group}
%3.2: 1-4 6-20 22 23 26 27 28 29
%%%%%%%%%%%%%%%%%%%%%%%%%%%%%%%%%%%%%%%%%%%%%%%%%%%%%%%%%%%%%%%%%%%%%%%%%%%%%%%%
%%%%%%%%%%%%%%%%%%%%%%%%%%%%%%%%%%%%%%%%%%%%%%%%%%%%%%%%%%%%%%%%%%%%%%%%%%%%%%%%
%%%%%%%%%%%%%%%%%%%%%%%%%%%%%%%%%%%%%%%%%%%%%%%%%%%%%%%%%%%%%%%%%%%%%%%%%%%%%%%%
%%%%%%%%%%%%%%%%%%%%%%%%%%%%%%%%%%%%%%%%%%%%%%%%%%%%%%%%%%%%%%%%%%%%%%%%%%%%%%%%
\begin{mdframed}[style=darkQuesion]
1. Using ordinary addition of integers as the operation, show that the set of
even integers is a group but the set of odd integers is not.
\end{mdframed}
%%%%%%%%%%%%%%%%%%%%%%%%%%%%%%%%%%%%%%%%%%%%%%%%%%%%%%%%%%%%%%%%%%%%%%%%%%%%%%%%
\begin{mdframed}[style=darkAnswer,frametitle={Joe Starr}]
We begin considering the even integers, that is integers of the form $2k$. We
must also include 0 in the even integers. We get the identity element as well
as Associativity and inverses for free from integer addition on $\Z$.
We then consider closure. Let $n$ and $m$ be even integers, if we take $m+n$
we can see we have, $m+n=2k_m+2k_n=2(k_m+k_n)$ an even integer. Making the
even integers a group under addition.

Next we consider the odd integers, take $3+3=2(3)$ an even integer, showing
odds are not closed under addition and not a group.
\end{mdframed}
\newpage
%%%%%%%%%%%%%%%%%%%%%%%%%%%%%%%%%%%%%%%%%%%%%%%%%%%%%%%%%%%%%%%%%%%%%%%%%%%%%%%%
%%%%%%%%%%%%%%%%%%%%%%%%%%%%%%%%%%%%%%%%%%%%%%%%%%%%%%%%%%%%%%%%%%%%%%%%%%%%%%%%
%%%%%%%%%%%%%%%%%%%%%%%%%%%%%%%%%%%%%%%%%%%%%%%%%%%%%%%%%%%%%%%%%%%%%%%%%%%%%%%%
%%%%%%%%%%%%%%%%%%%%%%%%%%%%%%%%%%%%%%%%%%%%%%%%%%%%%%%%%%%%%%%%%%%%%%%%%%%%%%%%
\begin{mdframed}[style=darkQuesion]
2. For each binary operation $\ast$ defined on a set below,
determine whether or not $\ast$ gives a group structure on the set.
If it is not a group, say which axioms fail to hold.
\begin{multicols}{2}
\begin{itemize}
\item[(a)]{Define $\ast$ on $\Z$ by $a\ast b=ab$.}
\item[(b)]{Define $\ast$ on $\Z$ by $a\ast b=\max{a,b}$.}
\item[(c)]{Define $\ast$ on $\Z$ by $a\ast b=a-b$.}
\item[(d)]{Define $\ast$ on $\Z$ by $a\ast b=\abs{ab}$.}
\item[(e)]{Define $\ast$ on $\R^{+}$ by $a\ast b=ab$.}
\item[(f)]{Define $\ast$ on $\Q$ by $a\ast b=ab$.}
\end{itemize}
\end{multicols}
\end{mdframed}

%%%%%%%%%%%%%%%%%%%%%%%%%%%%%%%%%%%%%%%%%%%%%%%%%%%%%%%%%%%%%%%%%%%%%%%%%%%%%%%%
\begin{mdframed}[style=darkAnswer,frametitle={Joe Starr}]
\begin{multicols}{2}
\begin{itemize}[align=left]
\item[(a)]{
    \begin{itemize}[align=left]
    \Invs{Let $a\in \Z$ but $a\neq 1$ and $a\neq 1$,
        $\inv{a}\notin \Z$.}
    \end{itemize}
  }
\item[(b)]{
    \begin{itemize}[align=left]
    \Ident{$\max{a,a-1}=a$ for all $a\in \Z$ this means there is no
        $e$ with $\max{a,e}=a$ for all $a$. }
    \end{itemize}
  }
\item[(c)]{
    \begin{itemize}[align=left]
    \Assoc{\begin{align*}
        \lrp{a\ast b}\ast c & =\lrp{a-b}\ast c     \\
        & =\lrp{a-b}-c         \\
        & =a-\lrp{b-c}         \\
        & =a\ast \lrp{b\ast c}
        \end{align*}}
    \Invs{Select $a\in \Z$, $$a\ast a = a-a = 0$$}
    \Clos{Obvious from closure of $\lrp{\Z,+}$}
    \Ident{0 is the identity, Obvious from $\lrp{\Z,+}$}
    \end{itemize}

  }
\item[(d)]{
    \begin{itemize}[align=left]
    \Invs{Let $a\in \Z$ but $a\neq 1$ and $a\neq 1$,
        $\inv{a}\notin \Z$.}
    \end{itemize}
  }
\item[(e)]{
    \begin{itemize}[align=left]
    \Assoc{\begin{align*}
        \lrp{a\ast b}\ast c & =\lrp{ab}\ast c      \\
        & =\lrp{ab}c           \\
        & =a\lrp{bc}           \\
        & =a\ast \lrp{b\ast c}
        \end{align*}}
    \Invs{Select $a\in \R$, $$a\ast a = a\frac{1}{a} = 1$$}
    \Clos{Obvious from closure of $\lrp{\R,\cdot}$}
    \Ident{1 is the identity, Obvious from $\lrp{\R,\cdot}$}
    \end{itemize}
  }
\item[(f)]{
    \begin{itemize}[align=left]
    \Assoc{\begin{align*}
        \lrp{a\ast b}\ast c & =\lrp{ab}\ast c      \\
        & =\lrp{ab}c           \\
        & =a\lrp{bc}           \\
        & =a\ast \lrp{b\ast c}
        \end{align*}}
    \Invs{Select $a\in \Q$, $$a\ast \inv{a} = a\frac{1}{a} = 1$$}
    \Clos{$a,b\in \Q$, $a=\frac{p_1}{q_1}$ $b=\frac{p_2}{q_2}$,
        ${p_1},{q_1},{p_2},{q_2}\in \Z$, $q_1\neq0\neq q_2$.
        \begin{align*}
        a\ast b & =ab
        & =                                 \\
        & =\frac{p_1{p_2}}{q_1{q_2}} \in \Q
        \end{align*}}
    \Ident{1 is the identity, Obvious from $\lrp{\R,\cdot}$}
    \end{itemize}
  }
\end{itemize}
\end{multicols}
\end{mdframed}
\newpage
%%%%%%%%%%%%%%%%%%%%%%%%%%%%%%%%%%%%%%%%%%%%%%%%%%%%%%%%%%%%%%%%%%%%%%%%%%%%%%%%
%%%%%%%%%%%%%%%%%%%%%%%%%%%%%%%%%%%%%%%%%%%%%%%%%%%%%%%%%%%%%%%%%%%%%%%%%%%%%%%%
%%%%%%%%%%%%%%%%%%%%%%%%%%%%%%%%%%%%%%%%%%%%%%%%%%%%%%%%%%%%%%%%%%%%%%%%%%%%%%%%
%%%%%%%%%%%%%%%%%%%%%%%%%%%%%%%%%%%%%%%%%%%%%%%%%%%%%%%%%%%%%%%%%%%%%%%%%%%%%%%%
\begin{mdframed}[style=darkQuesion]
3. Let $\grp{G}{\cdot}$ be a group. Define a new binary operation $\ast$ on
$G$ by the formula $a \ast b=b\cot a$, for all $a,b\in G$.
\begin{itemize}
\item[(a)]{Show that $\grp{G}{\cdot}$ is a group.}
\item[(b)]{Give examples to show that $\grp{G}{\cdot}$ may or may not be the
    same as $\grp{G}{\ast}$.}
\end{itemize}
\end{mdframed}

%%%%%%%%%%%%%%%%%%%%%%%%%%%%%%%%%%%%%%%%%%%%%%%%%%%%%%%%%%%%%%%%%%%%%%%%%%%%%%%%
\begin{mdframed}[style=darkAnswer,frametitle={Joe Starr}]
\begin{itemize}
\item[(a)]{
    \begin{itemize}[align=left]
    \Assoc{\begin{align*}
        \lrp{a\ast b}\ast c & =\lrp{b\cdot a}\ast c     \\
        & =c \cdot  \lrp{b \cdot a} \\
        & =\lrp{c \cdot  b} \cdot a \\
        & =\lrp{b\ast c} \cdot a    \\
        & =a\ast \lrp{b\ast c}
        \end{align*}}
    \Invs{Let $a\in G$ since $\grp{G}{\cdot}$ is a group we know
        $\inv{a}\in G$. Now $a\ast \inv{a}=\inv{a}\cdot a=1$.
      }
    \Clos{We select $a,b\in G$ consier $a\ast b=b\cdot a$ by closure of
        $\grp{G}{\cdot}$, $a\ast b\in G$.
      }
    \Ident{Let $a\in G$, consider $1\ast a = a \cdot 1 = a$ and
        $a\ast 1 = 1 \cdot a = a$.}
    \end{itemize}
  }
\item[(b)]{
    Let $G=\begin{tabular}{|c|c|c|c|c|c|c|}
      \hline
    $\cdot$ & e & a & b & c & d & f \\
      \hline
      a       & a & e & d & f & b & c \\
      \hline
      b       & b & f & e & d & c & a \\
      \hline
      c       & c & d & f & e & a & b \\
      \hline
      d       & d & c & a & b & f & e \\
      \hline
      f       & f & b & c & a & e & d \\
      \hline
      \end{tabular}$ so
    $a\ast b=b\cdot a = f$ but $a\cdot b=d$. In this case they are not equal.

    If we let $G= \grp{\Z}{+}$, we have $a\ast b = b+a=a+b$. In this case they
    are equal.
  }
\end{itemize}
\end{mdframed}
\newpage
%%%%%%%%%%%%%%%%%%%%%%%%%%%%%%%%%%%%%%%%%%%%%%%%%%%%%%%%%%%%%%%%%%%%%%%%%%%%%%%%
%%%%%%%%%%%%%%%%%%%%%%%%%%%%%%%%%%%%%%%%%%%%%%%%%%%%%%%%%%%%%%%%%%%%%%%%%%%%%%%%
%%%%%%%%%%%%%%%%%%%%%%%%%%%%%%%%%%%%%%%%%%%%%%%%%%%%%%%%%%%%%%%%%%%%%%%%%%%%%%%%
%%%%%%%%%%%%%%%%%%%%%%%%%%%%%%%%%%%%%%%%%%%%%%%%%%%%%%%%%%%%%%%%%%%%%%%%%%%%%%%%
\begin{mdframed}[style=darkQuesion]
5. Is $\GL{n}{\R}$ an Abelian group? Support your answer by either proof or a
counter example.
\end{mdframed}

%%%%%%%%%%%%%%%%%%%%%%%%%%%%%%%%%%%%%%%%%%%%%%%%%%%%%%%%%%%%%%%%%%%%%%%%%%%%%%%%
\begin{mdframed}[style=darkAnswer,frametitle={Joe Starr}]
No, select $$A=\begin{bmatrix}
  2 & 3 \\
  5 & 7 \\
  \end{bmatrix}
  B=\begin{bmatrix}
  11 & 13 \\
  17 & 19 \\
  \end{bmatrix}$$
we calculate
$$AB=\begin{bmatrix}
  73  & 83  \\
  174 & 198 \\
  \end{bmatrix}
  BA=\begin{bmatrix}
  87  & 124 \\
  129 & 184 \\
  \end{bmatrix}$$
\end{mdframed}
\newpage
%%%%%%%%%%%%%%%%%%%%%%%%%%%%%%%%%%%%%%%%%%%%%%%%%%%%%%%%%%%%%%%%%%%%%%%%%%%%%%%%
%%%%%%%%%%%%%%%%%%%%%%%%%%%%%%%%%%%%%%%%%%%%%%%%%%%%%%%%%%%%%%%%%%%%%%%%%%%%%%%%
%%%%%%%%%%%%%%%%%%%%%%%%%%%%%%%%%%%%%%%%%%%%%%%%%%%%%%%%%%%%%%%%%%%%%%%%%%%%%%%%
%%%%%%%%%%%%%%%%%%%%%%%%%%%%%%%%%%%%%%%%%%%%%%%%%%%%%%%%%%%%%%%%%%%%%%%%%%%%%%%%
\begin{mdframed}[style=darkQuesion]
8. Write out the multiplication table for the following set of matrices over
$\Q$:
$$\begin{bmatrix}
  1 & 0 \\
  0 & 1 \\
  \end{bmatrix},
  \begin{bmatrix}
  \m1 & 0 \\
  0   & 1 \\
  \end{bmatrix},
  \begin{bmatrix}
  1 & 0   \\
  0 & \m1 \\
  \end{bmatrix},
  \begin{bmatrix}
  \m1 & 0   \\
  0   & \m1 \\
  \end{bmatrix}$$


\end{mdframed}

%%%%%%%%%%%%%%%%%%%%%%%%%%%%%%%%%%%%%%%%%%%%%%%%%%%%%%%%%%%%%%%%%%%%%%%%%%%%%%%%
\begin{mdframed}[style=darkAnswer,frametitle={Joe Starr}]
Let $$i=\begin{bmatrix}
  1 & 0 \\
  0 & 1 \\
  \end{bmatrix},
  j=\begin{bmatrix}
  \m1 & 0 \\
  0   & 1 \\
  \end{bmatrix},
  k=\begin{bmatrix}
  1 & 0   \\
  0 & \m1 \\
  \end{bmatrix},
  l=\begin{bmatrix}
  \m1 & 0   \\
  0   & \m1 \\
  \end{bmatrix}$$
$$\begin{tabular}{|c|c|c|c|c|}
  \hline
  $\cdot$ & i & j & k & l \\
  \hline
  i       & i & j & k & l \\
  \hline
  j       & j & i & l & k \\
  \hline
  k       & k & l & i & j \\
  \hline
  l       & l & k & j & i \\
  \hline
  \end{tabular}$$
\end{mdframed}
\newpage
%%%%%%%%%%%%%%%%%%%%%%%%%%%%%%%%%%%%%%%%%%%%%%%%%%%%%%%%%%%%%%%%%%%%%%%%%%%%%%%%
%%%%%%%%%%%%%%%%%%%%%%%%%%%%%%%%%%%%%%%%%%%%%%%%%%%%%%%%%%%%%%%%%%%%%%%%%%%%%%%%
%%%%%%%%%%%%%%%%%%%%%%%%%%%%%%%%%%%%%%%%%%%%%%%%%%%%%%%%%%%%%%%%%%%%%%%%%%%%%%%%
%%%%%%%%%%%%%%%%%%%%%%%%%%%%%%%%%%%%%%%%%%%%%%%%%%%%%%%%%%%%%%%%%%%%%%%%%%%%%%%%
\begin{mdframed}[style=darkQuesion]
9. Let $G=\lrs{x\in \R \vert x>0   ext{ and } x\neq 1}$. Define the operation
$\ast$ on $G$ by $a\ast b=a^{\ln{b}}$, for all $a,b\in G$. Prove that $G$ is
an Abelian group under the operation $\ast$.
\end{mdframed}

%%%%%%%%%%%%%%%%%%%%%%%%%%%%%%%%%%%%%%%%%%%%%%%%%%%%%%%%%%%%%%%%%%%%%%%%%%%%%%%%
\begin{mdframed}[style=darkAnswer,frametitle={Joe Starr}]
\begin{itemize}[align=left]
\Assoc{\begin{align*}
    \lrp{a\ast b}\ast c & =\lrp{a^{\ln{b}}}\ast c    \\
    & =\lrp{a^{\ln{b}}}^{\ln{c}} \\
    & ={a^{\ln{b}\ln{c}}}        \\
    & =a^{\ln{b^{\ln{c}}}}       \\
    & =a\ast \lrp{b\ast c}
    \end{align*}}
\Invs{
    Let $a\in G$, consider $\inv{a}=e^{\frac{1}{\ln{a}}}$.
    \begin{multicols}{2}
    \begin{align*}
    a\ast\inv{a} & = a ^{\ln{\inv{a}}}             \\
    & = a^{\ln{e^{\frac{1}{\ln{a}}}}} \\
    & = a^{{{\frac{1}{\ln{a}}}}}      \\
    & = a^{{{\log_{a}{e}}}}           \\
    & = {e}                           \\
    \end{align*}
    \begin{align*}
    \inv{a}\ast a & = \lrp{\inv{a}} ^{\ln{a}}              \\
    & = \lrp{e^{\frac{1}{\ln{a}}}} ^{\ln{a}} \\
    & = \lrp{e^{\log_{a}{e}}} ^{\ln{a}}      \\
    & = \lrp{e^{\ln{a}} } ^{\log_{a}{e}}     \\
    & = a ^{\log_{a}{e}}                     \\
    & = {e}                                  \\
    \end{align*}
    \end{multicols}
  }
\Clos{Let $a,b\in G$, $a\ast b = a^{\ln{b}}$ we know that $b>0$ so
    $\ln{b}$ exists, further since $1\notin G$ we have $\ln{b}\neq 0$.
    We observe that for any $a\in G$ $a^{\ln{b}}>0$ since $a>0$ and since
    $\ln{b}\neq 0$ $a^{\ln{b}}\neq 1$.}
\Ident{Our conjecture is that $e$ is the identity element.
    Let $a\in G$, $e\ast a= e^{\ln{a}}=a$ and $a\ast e= a^{\ln{e}}=a$}
\end{itemize}
\end{mdframed}
\newpage
%%%%%%%%%%%%%%%%%%%%%%%%%%%%%%%%%%%%%%%%%%%%%%%%%%%%%%%%%%%%%%%%%%%%%%%%%%%%%%%%
%%%%%%%%%%%%%%%%%%%%%%%%%%%%%%%%%%%%%%%%%%%%%%%%%%%%%%%%%%%%%%%%%%%%%%%%%%%%%%%%
%%%%%%%%%%%%%%%%%%%%%%%%%%%%%%%%%%%%%%%%%%%%%%%%%%%%%%%%%%%%%%%%%%%%%%%%%%%%%%%%
%%%%%%%%%%%%%%%%%%%%%%%%%%%%%%%%%%%%%%%%%%%%%%%%%%%%%%%%%%%%%%%%%%%%%%%%%%%%%%%%
\begin{mdframed}[style=darkQuesion]
10. Show that the set $A=\lrs{f_{m,b}:\R   o \R\vert m\neq 0   ext{ and } f_{m,b}=mx+b}$.
of affine functions from $\R$ to $\R$ forms a group under function composition.
\end{mdframed}

%%%%%%%%%%%%%%%%%%%%%%%%%%%%%%%%%%%%%%%%%%%%%%%%%%%%%%%%%%%%%%%%%%%%%%%%%%%%%%%%
\begin{mdframed}[style=darkAnswer,frametitle={Joe Starr}]
\begin{itemize}[align=left]
\Assoc{We've proved this previously. }
\Invs{Let $f\in A$, $\fof{x}=mx+b$. Consider $I\lrp{x}=\frac{1}{m}\lrp{x-b}$
    \begin{multicols}{2}
    \begin{align*}
    \fof{I\lrp{x}} & = m\lrp{\frac{1}{m}\lrp{x-b}}+b \\
    & = x-b+b                         \\
    & = x                             \\
    \end{align*}
    \begin{align*}
    I\lrp{\fof{x}} & = \frac{1}{m}\lrp{\lrp{mx+b}-b} \\
    & = \frac{1}{m}\lrp{mx}           \\
    & = x                             \\
    \end{align*}
    \end{multicols}
  }
\Clos{Let $f,g\in A$, so $\fof{x}=m_1x+b_1$ and $\gof{x}=m_2x+b_2$.
    Now composing $f$ and $g$ $\fof{\gof{x}}$.
    \begin{align*}
    \fof{\gof{x}} & = m_1\lrp{m_2x+b_2}+b_1 \\
    & = {m_1m_2x+m_1b_2}+b_1  \\
    & = mx+m_1b_2+b_1         \\
    & = mx+b                  \\
    \end{align*}
  }
\Ident{Let $f\in A$, $\fof{x}=mx+b$. Conjecture $e\lrp{x}=x$
    \begin{multicols}{2}
    \begin{align*}
    \fof{e\lrp{x}} & = m\lrp{x}+b \\
    & = mx+b       \\
    \end{align*}
    \vfill
    \columnbreak
    \begin{align*}
    e\lrp{\fof{x}} & = mx+b \\
    \end{align*}
    \end{multicols}
  }

\end{itemize}
\end{mdframed}
\newpage
%%%%%%%%%%%%%%%%%%%%%%%%%%%%%%%%%%%%%%%%%%%%%%%%%%%%%%%%%%%%%%%%%%%%%%%%%%%%%%%%
%%%%%%%%%%%%%%%%%%%%%%%%%%%%%%%%%%%%%%%%%%%%%%%%%%%%%%%%%%%%%%%%%%%%%%%%%%%%%%%%
%%%%%%%%%%%%%%%%%%%%%%%%%%%%%%%%%%%%%%%%%%%%%%%%%%%%%%%%%%%%%%%%%%%%%%%%%%%%%%%%
%%%%%%%%%%%%%%%%%%%%%%%%%%%%%%%%%%%%%%%%%%%%%%%%%%%%%%%%%%%%%%%%%%%%%%%%%%%%%%%%
\begin{mdframed}[style=darkQuesion]
11. Show that the set of all $2  imes 2$ matrices over $\R$ of the form
$\begin{bmatrix}
  m & b \\
  0 & 1 \\
  \end{bmatrix}$ with $m\neq 0$ forms a group under matrix multiplication.
\end{mdframed}

%%%%%%%%%%%%%%%%%%%%%%%%%%%%%%%%%%%%%%%%%%%%%%%%%%%%%%%%%%%%%%%%%%%%%%%%%%%%%%%%
\begin{mdframed}[style=darkAnswer,frametitle={Joe Starr}]
Let $G$ be the set of all $2  imes 2$ matrices over $\R$ of the form
$\begin{bmatrix}
  m & b \\
  0 & 1 \\
  \end{bmatrix}$ with $m\neq 0$.
\begin{itemize}[align=left]
\Assoc{Free from $\Mn{2}{\R}$}
\Invs{Let $a\in G$, so $a=\begin{bmatrix}
      m & b \\
      0 & 1 \\
      \end{bmatrix}$ we can calculate the determinate of $a$. $m1-b0=m$ and by
    definition of the set $m\neq 0$. So we have inverses.
  }
\Clos{Let $a,b\in G$, so $a=\begin{bmatrix}
      m_1 & b_1 \\
      0   & 1   \\
      \end{bmatrix}$ and $b=\begin{bmatrix}
      m_2 & b_2 \\
      0   & 1   \\
      \end{bmatrix}$
    \begin{align*}
    ab     & = \begin{bmatrix}
    m_1    & b_1               \\
    0      & 1                 \\
    \end{bmatrix} \begin{bmatrix}
    m_2    & b_2               \\
    0      & 1                 \\
    \end{bmatrix} \\
    & = \begin{bmatrix}
    m_2m_1 & b_1+m_1b_2        \\
    0      & 1                 \\
    \end{bmatrix}                            \\
    \end{align*}
  }
\Ident{Free from $\Mn{2}{\R}$
  }

\end{itemize}
\end{mdframed}
\newpage
%%%%%%%%%%%%%%%%%%%%%%%%%%%%%%%%%%%%%%%%%%%%%%%%%%%%%%%%%%%%%%%%%%%%%%%%%%%%%%%%
%%%%%%%%%%%%%%%%%%%%%%%%%%%%%%%%%%%%%%%%%%%%%%%%%%%%%%%%%%%%%%%%%%%%%%%%%%%%%%%%
%%%%%%%%%%%%%%%%%%%%%%%%%%%%%%%%%%%%%%%%%%%%%%%%%%%%%%%%%%%%%%%%%%%%%%%%%%%%%%%%
%%%%%%%%%%%%%%%%%%%%%%%%%%%%%%%%%%%%%%%%%%%%%%%%%%%%%%%%%%%%%%%%%%%%%%%%%%%%%%%%
\begin{mdframed}[style=darkQuesion]
12. In the group defined in question 11 find all elements that commute with
$\begin{bmatrix}
  2 & 0 \\
  0 & 1 \\
  \end{bmatrix}$
\end{mdframed}

%%%%%%%%%%%%%%%%%%%%%%%%%%%%%%%%%%%%%%%%%%%%%%%%%%%%%%%%%%%%%%%%%%%%%%%%%%%%%%%%
\begin{mdframed}[style=darkAnswer,frametitle={Joe Starr}]
We can begin by letting $a\in G$ calculating $a\begin{bmatrix}
  2 & 0 \\
  0 & 1 \\
  \end{bmatrix}$ and $\begin{bmatrix}
  2 & 0 \\
  0 & 1 \\
  \end{bmatrix}a$.
\begin{multicols}{2}
\begin{align*}
a\begin{bmatrix}
2             & 0                 \\
0             & 1                 \\
\end{bmatrix} & = \begin{bmatrix}
m             & b                 \\
0             & 1                 \\
\end{bmatrix} \begin{bmatrix}
2             & 0                 \\
0             & 1                 \\
\end{bmatrix} \\
& = \begin{bmatrix}
2m            & b                 \\
0             & 1                 \\
\end{bmatrix}                            \\
\end{align*}
\begin{align*}
\begin{bmatrix}
2              & 0                  \\
0              & 1                  \\
\end{bmatrix}a & =  \begin{bmatrix}
2              & 0                  \\
0              & 1                  \\
\end{bmatrix} \begin{bmatrix}
m              & b                  \\
0              & 1                  \\
\end{bmatrix} \\
& = \begin{bmatrix}
2m             & 2b                 \\
0              & 1                  \\
\end{bmatrix}                             \\
\end{align*}
\end{multicols}
So for a matrix of to commute with $\begin{bmatrix}
  2 & 0 \\
  0 & 1 \\
  \end{bmatrix}$ it must be of the form $\begin{bmatrix}
  m & 0 \\
  0 & 1 \\
  \end{bmatrix}$.
\end{mdframed}
\newpage
%%%%%%%%%%%%%%%%%%%%%%%%%%%%%%%%%%%%%%%%%%%%%%%%%%%%%%%%%%%%%%%%%%%%%%%%%%%%%%%%
%%%%%%%%%%%%%%%%%%%%%%%%%%%%%%%%%%%%%%%%%%%%%%%%%%%%%%%%%%%%%%%%%%%%%%%%%%%%%%%%
%%%%%%%%%%%%%%%%%%%%%%%%%%%%%%%%%%%%%%%%%%%%%%%%%%%%%%%%%%%%%%%%%%%%%%%%%%%%%%%%
%%%%%%%%%%%%%%%%%%%%%%%%%%%%%%%%%%%%%%%%%%%%%%%%%%%%%%%%%%%%%%%%%%%%%%%%%%%%%%%%
\begin{mdframed}[style=darkQuesion]
13. Define $\ast$ on $\R$ by $a\ast b = a+b-1$, for all $a,b\in \R$. Show that
$\grp{\R}{\ast}$ is an Abelian group.
\end{mdframed}

%%%%%%%%%%%%%%%%%%%%%%%%%%%%%%%%%%%%%%%%%%%%%%%%%%%%%%%%%%%%%%%%%%%%%%%%%%%%%%%%
\begin{mdframed}[style=darkAnswer,frametitle={Joe Starr}]
\begin{itemize}[align=left]
\item[]{
    \begin{multicols}{2}
    \begin{itemize}[align=left]
    \Abel{
        \begin{align*}
          {a\ast b} & =a+b-1   \\
        & = b+a-1  \\
        & =b\ast a
        \end{align*}

      }
    \Assoc{\begin{align*}
        \lrp{a\ast b}\ast c & =\lrp{a+b-1}\ast c   \\
        & =\lrp{a+b-1}+c-1     \\
        & =a+b+c-1-1           \\
        & =a+\lrp{b+c-1}-1     \\
        & =a\ast \lrp{b\ast c}
        \end{align*}}
    \end{itemize}
    \end{multicols}
  }
\Invs{ Let $a\in \grp{\R}{\ast} $, consider $\inv{a}=2-a$
    \begin{multicols}{2}
    \begin{align*}
    a\ast \inv{a} & = a+\lrp{2-a}-1 \\
    & = 1             \\
    \end{align*}
    \vfill
    \columnbreak
    \begin{align*}
    \inv{a}\ast a & = \lrp{2-a}+a-1 \\
    & = 1             \\
    \end{align*}
    \end{multicols}
  }
\Clos{Obvious from closure of $\grp{\R}{+}$.
  }
\Ident{ Conjecture is that $1$ is the identity element of $\grp{\R}{\ast}$.
    \begin{multicols}{2}
    \begin{align*}
    a\ast 1 & = a+1-1 \\
    & = a     \\
    \end{align*}
    \vfill
    \columnbreak
    \begin{align*}
    1\ast a & = a+1-1 \\
    & = a     \\
    \end{align*}
    \end{multicols}
  }
\end{itemize}
\end{mdframed}
\begin{mdframed}[style=darkAnswer,frametitle={Joe Starr}]
Let $\varphi:\grp{\R}{\ast}  o\grp{\R}{+}$, with $\pof{x}=x-1$,
$\pof{a\ast b}= \lrp{a+b-1}-1=a-1+b-1=\pof{a}+\pof{b}$. Further
$\inv{\varphi}\lrp{x}=x+1$, $\pof{\inv{\varphi}\lrp{x}}=\lrp{x+1}-1=x$.
Showing a group structure isomorphic to $\grp{\R}{+}$.
\end{mdframed}
\newpage
%%%%%%%%%%%%%%%%%%%%%%%%%%%%%%%%%%%%%%%%%%%%%%%%%%%%%%%%%%%%%%%%%%%%%%%%%%%%%%%%
%%%%%%%%%%%%%%%%%%%%%%%%%%%%%%%%%%%%%%%%%%%%%%%%%%%%%%%%%%%%%%%%%%%%%%%%%%%%%%%%
%%%%%%%%%%%%%%%%%%%%%%%%%%%%%%%%%%%%%%%%%%%%%%%%%%%%%%%%%%%%%%%%%%%%%%%%%%%%%%%%
%%%%%%%%%%%%%%%%%%%%%%%%%%%%%%%%%%%%%%%%%%%%%%%%%%%%%%%%%%%%%%%%%%%%%%%%%%%%%%%%
\begin{mdframed}[style=darkQuesion]
14. Let $S= \R - \lrs{\m1}$. Define $\ast$ on $S$ by $a\ast b=a+b+ab$ for all
$a,b \in S$. Show that $\grp{S}{\ast}$ is an Abelian group.
\end{mdframed}

%%%%%%%%%%%%%%%%%%%%%%%%%%%%%%%%%%%%%%%%%%%%%%%%%%%%%%%%%%%%%%%%%%%%%%%%%%%%%%%%
\begin{mdframed}[style=darkAnswer,frametitle={Joe Starr}]
\begin{itemize}[align=left]

\item[] {
    \begin{multicols}{2}
    \begin{itemize}[align=left]

    \Abel{
        \begin{align*}
          {a\ast b} & =a+b+ab  \\
        & = b+a+ba \\
        & =b\ast a
        \end{align*}
      }
    \Invs{ Consider $\inv{a}=\frac{\m a}{a+1}$
        \begin{align*}
        a\ast \inv{a} & = a+\frac{\m a}{a+1}+a\frac{\m a}{a+1} \\
        & = a+\frac{\m a\lrp{a+1}}{a+1}          \\
        & = a+\m a                               \\
        & = 0                                    \\
        \end{align*}
        \vfill
        \columnbreak
      }
    \Clos{Let $a,b\in \R$, if we take $a\ast b =a+b+ab$. Assume that
        $a\ast b=\m1$
        \begin{align*}
        \m1=a+b+ab & \Rightarrow  \m1-a=b+ab          \\
        & \Rightarrow  \m1-a=b\lrp{1+a}    \\
        & \Rightarrow  \m\frac{a+1}{1+a}=b \\
        & \Rightarrow  \m1=b               \\
        \end{align*}
        a contradiction.
      }
    \end{itemize}
    \end{multicols}
  }
\Ident{ Conjecture is that $0$ is the identity element of $\grp{S}{\ast}$.
    \begin{align*}
    a\ast 0 & = a+0+a0 \\
    & = a      \\
    \end{align*}
  }
\Assoc{\begin{align*}
    \lrp{a\ast b}\ast c & =\lrp{a+b+ab}\ast c            \\
    & = \lrp{a+b+ab}+c+\lrp{a+b+ab}c \\
    & = \lrp{a+b+ab}+c+\lrp{a+b+ab}c \\
    & = \lrp{a+b+ab}+c+\lrp{a+b+ab}c \\
    & = a+b+ab+c+ca+cb+cab           \\
    & = a+\lrp{b+c+bc}+a\lrp{b+c+bc} \\
    & =a\ast\lrp{b+c+bc}             \\
    & =a\ast \lrp{b\ast c}
    \end{align*}}

\end{itemize}
\end{mdframed}
\newpage
%%%%%%%%%%%%%%%%%%%%%%%%%%%%%%%%%%%%%%%%%%%%%%%%%%%%%%%%%%%%%%%%%%%%%%%%%%%%%%%%
%%%%%%%%%%%%%%%%%%%%%%%%%%%%%%%%%%%%%%%%%%%%%%%%%%%%%%%%%%%%%%%%%%%%%%%%%%%%%%%%
%%%%%%%%%%%%%%%%%%%%%%%%%%%%%%%%%%%%%%%%%%%%%%%%%%%%%%%%%%%%%%%%%%%%%%%%%%%%%%%%
%%%%%%%%%%%%%%%%%%%%%%%%%%%%%%%%%%%%%%%%%%%%%%%%%%%%%%%%%%%%%%%%%%%%%%%%%%%%%%%%
\begin{mdframed}[style=darkQuesion]
15. Let $G=\lrs{x\in \R\vert x>1}$. Define $\ast$ on $G$ by $a\ast b = ab-a-b+2$,
for all $a,b\in G$. Show that $\grp{G}{\ast}$ is an Abelian group.
\end{mdframed}

%%%%%%%%%%%%%%%%%%%%%%%%%%%%%%%%%%%%%%%%%%%%%%%%%%%%%%%%%%%%%%%%%%%%%%%%%%%%%%%%
\begin{mdframed}[style=darkAnswer,frametitle={Joe Starr}]
\begin{itemize}[align=left]

\item[] {
    \begin{multicols}{2}
    \begin{itemize}[align=left]
    \Abel{
        \begin{align*}
          {a\ast b} & =ab-a-b+2  \\
        & = ba-b-a+2 \\
        & =b\ast a
        \end{align*}
      }
    \Invs{ Consider $\inv{a}=\frac{a}{a-1}$
        \begin{align*}
        a\ast \inv{a} & = a\frac{a}{a-1}-a-\frac{a}{a-1}+2 \\
        & = \frac{a}{a-1}\lrp{a-1}-a+2       \\
        & = a-a+2                            \\
        & = 2                                \\
        \end{align*}
        \vfill
        \columnbreak
      }
    \Ident{ Conjecture is that $2$ is the identity element of $\grp{G}{\ast}$.
        \begin{align*}
        a\ast 2 & = a2-a-2+2 \\
        & = a        \\
        \end{align*}
      }
    \end{itemize}
    \end{multicols}
  }
\Clos{We begin by letting $a,b\in G$, we observe $a\geq b>1$.
    We can then multiply through by b yeilding $ab\geq bb>b>1$. Next
    we subtract both $a$ and $b$, $ab-a-b\geq1>\m a$, finaly adding
    two gives $ab-a-b+2\geq3$ showing $a\ast b\in G$.
  }
\Assoc{\begin{align*}
    \lrp{a\ast b}\ast c & =\lrp{a+b+ab}\ast c            \\
    & = \lrp{a+b+ab}+c+\lrp{a+b+ab}c \\
    & = \lrp{a+b+ab}+c+\lrp{a+b+ab}c \\
    & = \lrp{a+b+ab}+c+\lrp{a+b+ab}c \\
    & = a+b+ab+c+ca+cb+cab           \\
    & = a+\lrp{b+c+bc}+a\lrp{b+c+bc} \\
    & =a\ast\lrp{b+c+bc}             \\
    & =a\ast \lrp{b\ast c}
    \end{align*}}

\end{itemize}
\end{mdframed}
\newpage
%%%%%%%%%%%%%%%%%%%%%%%%%%%%%%%%%%%%%%%%%%%%%%%%%%%%%%%%%%%%%%%%%%%%%%%%%%%%%%%%
%%%%%%%%%%%%%%%%%%%%%%%%%%%%%%%%%%%%%%%%%%%%%%%%%%%%%%%%%%%%%%%%%%%%%%%%%%%%%%%%
%%%%%%%%%%%%%%%%%%%%%%%%%%%%%%%%%%%%%%%%%%%%%%%%%%%%%%%%%%%%%%%%%%%%%%%%%%%%%%%%
%%%%%%%%%%%%%%%%%%%%%%%%%%%%%%%%%%%%%%%%%%%%%%%%%%%%%%%%%%%%%%%%%%%%%%%%%%%%%%%%
\begin{mdframed}[style=darkQuesion]
16. Let $G$ be a group. We have shown that $\inv{\lrp{ab}}=\inv{b}\inv{a}$.
Find a similar expression for $\lrp{\inv{abc}}$
\end{mdframed}

%%%%%%%%%%%%%%%%%%%%%%%%%%%%%%%%%%%%%%%%%%%%%%%%%%%%%%%%%%%%%%%%%%%%%%%%%%%%%%%%
\begin{mdframed}[style=darkAnswer,frametitle={Joe Starr}]
We will use a transitive proof:
\begin{align*}
\inv{\lrp{abc}} & = \inv{c}\inv{\lrp{ab}} \\
& = \inv{c}\inv{b}\inv{a}
\end{align*}
\end{mdframed}
\newpage
%%%%%%%%%%%%%%%%%%%%%%%%%%%%%%%%%%%%%%%%%%%%%%%%%%%%%%%%%%%%%%%%%%%%%%%%%%%%%%%%
%%%%%%%%%%%%%%%%%%%%%%%%%%%%%%%%%%%%%%%%%%%%%%%%%%%%%%%%%%%%%%%%%%%%%%%%%%%%%%%%
%%%%%%%%%%%%%%%%%%%%%%%%%%%%%%%%%%%%%%%%%%%%%%%%%%%%%%%%%%%%%%%%%%%%%%%%%%%%%%%%
%%%%%%%%%%%%%%%%%%%%%%%%%%%%%%%%%%%%%%%%%%%%%%%%%%%%%%%%%%%%%%%%%%%%%%%%%%%%%%%%
\begin{mdframed}[style=darkQuesion]
17. Let $G$ be a group. If $g\in G$ and $g^2=g$,then prove that $g=e$.
\end{mdframed}

%%%%%%%%%%%%%%%%%%%%%%%%%%%%%%%%%%%%%%%%%%%%%%%%%%%%%%%%%%%%%%%%%%%%%%%%%%%%%%%%
\begin{mdframed}[style=darkAnswer,frametitle={Joe Starr}]
We begin with letting $g\in G$, such $g^2=g$ we then multiply by $\inv{g}$ on
the left:
\begin{align*}
g^2=g & \rightarrow \inv{g}g^2=\inv{g}g \\
& \rightarrow g=e
\end{align*}
as desired.
\end{mdframed}
\newpage
%%%%%%%%%%%%%%%%%%%%%%%%%%%%%%%%%%%%%%%%%%%%%%%%%%%%%%%%%%%%%%%%%%%%%%%%%%%%%%%%
%%%%%%%%%%%%%%%%%%%%%%%%%%%%%%%%%%%%%%%%%%%%%%%%%%%%%%%%%%%%%%%%%%%%%%%%%%%%%%%%
%%%%%%%%%%%%%%%%%%%%%%%%%%%%%%%%%%%%%%%%%%%%%%%%%%%%%%%%%%%%%%%%%%%%%%%%%%%%%%%%
%%%%%%%%%%%%%%%%%%%%%%%%%%%%%%%%%%%%%%%%%%%%%%%%%%%%%%%%%%%%%%%%%%%%%%%%%%%%%%%%
\begin{mdframed}[style=darkQuesion]
18. Show that a nonabelian group must have at least 5 elements.
\end{mdframed}

%%%%%%%%%%%%%%%%%%%%%%%%%%%%%%%%%%%%%%%%%%%%%%%%%%%%%%%%%%%%%%%%%%%%%%%%%%%%%%%%
\begin{mdframed}[style=darkAnswer,frametitle={Joe Starr}]
Let $G$ be a nonabelian group. Since $G$ a group then $e\in G$ the identity. $G$
can't be the trivial group since the trivial group is Abelian, this puts
$a\in G$ with $a\neq e$ further $\inv{a}\in G$. With the same argument $G$ is
not a group of three elements, so $b,\inv{b}\in G$. This puts
$a,b,\inv{b},\inv{a},e\in G$ showing $G$ with at lest 5 elements.
\end{mdframed}
\newpage
%%%%%%%%%%%%%%%%%%%%%%%%%%%%%%%%%%%%%%%%%%%%%%%%%%%%%%%%%%%%%%%%%%%%%%%%%%%%%%%%
%%%%%%%%%%%%%%%%%%%%%%%%%%%%%%%%%%%%%%%%%%%%%%%%%%%%%%%%%%%%%%%%%%%%%%%%%%%%%%%%
%%%%%%%%%%%%%%%%%%%%%%%%%%%%%%%%%%%%%%%%%%%%%%%%%%%%%%%%%%%%%%%%%%%%%%%%%%%%%%%%
%%%%%%%%%%%%%%%%%%%%%%%%%%%%%%%%%%%%%%%%%%%%%%%%%%%%%%%%%%%%%%%%%%%%%%%%%%%%%%%%
\begin{mdframed}[style=darkQuesion]
22. Let $S$ be a nonempty finite set with a binary operation $\ast$ that
satisfies the associative law. Show that $S$ is a group if $a\ast b=a\ast c$
implies $b=c$ and $a\ast c= b\ast c$ implies $a=b$ for all $a,b,c \in S$.
What can we say if $S$ is infinite?
\end{mdframed}

%%%%%%%%%%%%%%%%%%%%%%%%%%%%%%%%%%%%%%%%%%%%%%%%%%%%%%%%%%%%%%%%%%%%%%%%%%%%%%%%
\begin{mdframed}[style=darkAnswer,frametitle={Joe Starr}]
%TODO: Need to do this one still.
\end{mdframed}
\newpage
%%%%%%%%%%%%%%%%%%%%%%%%%%%%%%%%%%%%%%%%%%%%%%%%%%%%%%%%%%%%%%%%%%%%%%%%%%%%%%%%
%%%%%%%%%%%%%%%%%%%%%%%%%%%%%%%%%%%%%%%%%%%%%%%%%%%%%%%%%%%%%%%%%%%%%%%%%%%%%%%%
%%%%%%%%%%%%%%%%%%%%%%%%%%%%%%%%%%%%%%%%%%%%%%%%%%%%%%%%%%%%%%%%%%%%%%%%%%%%%%%%
%%%%%%%%%%%%%%%%%%%%%%%%%%%%%%%%%%%%%%%%%%%%%%%%%%%%%%%%%%%%%%%%%%%%%%%%%%%%%%%%
\begin{mdframed}[style=darkQuesion]
24. Let $G$ be a group. Prove that $G$ is Abelian if and only if
$\inv{\lrp{ab}}=\inv{a}\inv{b}$ for all $a,b\in G$.
\end{mdframed}

%%%%%%%%%%%%%%%%%%%%%%%%%%%%%%%%%%%%%%%%%%%%%%%%%%%%%%%%%%%%%%%%%%%%%%%%%%%%%%%%
\begin{mdframed}[style=darkAnswer,frametitle={Joe Starr}]
wocase{
    Let $G$ be an abelian group and $a,b\in G$. Consider $\inv{\lrp{ab}}$, we have
    shown $\inv{\lrp{ab}}=\inv{b}\inv{a}$ since $G$ is abelian we have
    $\inv{\lrp{ab}}=\inv{a}\inv{b}$.
  }{
    Let $\inv{\lrp{ab}}=\inv{a}\inv{b}$, we have shown
    $\inv{\lrp{ab}}=\inv{b}\inv{a}$ so $\inv{b}\inv{a}=\inv{a}\inv{b}$ showing $G$
    abelian.
  }
\end{mdframed}
\newpage
%%%%%%%%%%%%%%%%%%%%%%%%%%%%%%%%%%%%%%%%%%%%%%%%%%%%%%%%%%%%%%%%%%%%%%%%%%%%%%%%
%%%%%%%%%%%%%%%%%%%%%%%%%%%%%%%%%%%%%%%%%%%%%%%%%%%%%%%%%%%%%%%%%%%%%%%%%%%%%%%%
%%%%%%%%%%%%%%%%%%%%%%%%%%%%%%%%%%%%%%%%%%%%%%%%%%%%%%%%%%%%%%%%%%%%%%%%%%%%%%%%
%%%%%%%%%%%%%%%%%%%%%%%%%%%%%%%%%%%%%%%%%%%%%%%%%%%%%%%%%%%%%%%%%%%%%%%%%%%%%%%%
\begin{mdframed}[style=darkQuesion]
25. Let $G$ be a group. Prove that if $x^2=e$ for all $x\in G$, then $G$ is
abelian.
\end{mdframed}

%%%%%%%%%%%%%%%%%%%%%%%%%%%%%%%%%%%%%%%%%%%%%%%%%%%%%%%%%%%%%%%%%%%%%%%%%%%%%%%%
\begin{mdframed}[style=darkAnswer,frametitle={Joe Starr}]
Let $G$ be a group with the given property $a,b\in G$. Observe that
$a^2=e \Rightarrow a=\inv{a}$. We have shown that
$\inv{\lrp{ab}}=\inv{b}\inv{a}$.
We proceed with a transitive proof:
\begin{align*}
\inv{\lrp{ab}}=\inv{b}\inv{a} & \rightarrow \lrp{ab}=\inv{b}\inv{a} \\
& \rightarrow \lrp{ab}={b}{a}
\end{align*}
showing $G$ abelian as desired.
\end{mdframed}
\newpage
%%%%%%%%%%%%%%%%%%%%%%%%%%%%%%%%%%%%%%%%%%%%%%%%%%%%%%%%%%%%%%%%%%%%%%%%%%%%%%%%
%%%%%%%%%%%%%%%%%%%%%%%%%%%%%%%%%%%%%%%%%%%%%%%%%%%%%%%%%%%%%%%%%%%%%%%%%%%%%%%%
%%%%%%%%%%%%%%%%%%%%%%%%%%%%%%%%%%%%%%%%%%%%%%%%%%%%%%%%%%%%%%%%%%%%%%%%%%%%%%%%
%%%%%%%%%%%%%%%%%%%%%%%%%%%%%%%%%%%%%%%%%%%%%%%%%%%%%%%%%%%%%%%%%%%%%%%%%%%%%%%%
\begin{mdframed}[style=darkQuesion]
26. Show that if $G$ is a finite group with an even number of elements, then
there must exist an element $a\in G$ with $a\neq e$ such that $a^2=e$.
\end{mdframed}

%%%%%%%%%%%%%%%%%%%%%%%%%%%%%%%%%%%%%%%%%%%%%%%%%%%%%%%%%%%%%%%%%%%%%%%%%%%%%%%%
\begin{mdframed}[style=darkAnswer,frametitle={Joe Starr}]
Let $G$ be a group with the given property. Since $G$ a group $e\in G$.
Observe $G$ is not the trivial group since it has even cardinality.
If we consider the cardinality of $G/e$ it's $\abs{G}-1$ an odd number.
Let $a\in G$ with $a\neq e$, observe that since $G$ a group $\inv{a}\in G$.
We are left with two possibilities $a=\inv{a}$ or $a\neq \inv{a}$.
If $a=\inv{a}$ we are done, otherwise we can delete $a$ and $\inv{a}$ from $G$
and select from the remaining elements of $G$. Since $G/e$ has odd cardinality
we can repeat this process until there is a single element remaining. It must
be that $a=\inv{a}$ as desired.
\end{mdframed}
\newpage
\subsection{Subgroups}
%%%%%%%%%%%%%%%%%%%%%%%%%%%%%%%%%%%%%%%%%%%%%%%%%%%%%%%%%%%%%%%%%%%%%%%%%%%%%%%%
%%%%%%%%%%%%%%%%%%%%%%%%%%%%%%%%%%%%%%%%%%%%%%%%%%%%%%%%%%%%%%%%%%%%%%%%%%%%%%%%
%%%%%%%%%%%%%%%%%%%%%%%%%%%%%%%%%%%%%%%%%%%%%%%%%%%%%%%%%%%%%%%%%%%%%%%%%%%%%%%%
%%%%%%%%%%%%%%%%%%%%%%%%%%%%%%%%%%%%%%%%%%%%%%%%%%%%%%%%%%%%%%%%%%%%%%%%%%%%%%%%
\begin{mdframed}[style=darkQuesion]
1. In $\mathrm{GL}_{2}(\mathbf{R}),$ find the order of each of the following elements.
\begin{multicols}{4}
\begin{itemize}
\item[]{
    (a)$\dagger\left[\begin{array}{rr}1 & \m 1 \\
          1 & 0\end{array}\right]$}
\item[]{
    (b) $\left[\begin{array}{rr}0 & 1 \\
          \m 1 & 0\end{array}\right]$}
\item[]{
    $\dagger(\mathrm{c})\left[\begin{array}{ll}1 & 1 \\
          0 & 1\end{array}\right]$}
\item[]{
    (d) $\left[\begin{array}{rr}\m 1 & 1 \\
          0 & 1\end{array}\right]$}
\end{itemize}
\end{multicols}
\vspace{.5cm}
\end{mdframed}

%%%%%%%%%%%%%%%%%%%%%%%%%%%%%%%%%%%%%%%%%%%%%%%%%%%%%%%%%%%%%%%%%%%%%%%%%%%%%%%%
\begin{mdframed}[style=darkAnswer,frametitle={Joe Starr}]
\begin{multicols}{2}
\begin{itemize}
\item[(a)]{
    $$\begin{bmatrix}1 & \m 1 \\
      1 & 0 \\
      \end{bmatrix}^2=\begin{bmatrix}0 & \m 1 \\
      1 & \m 1\\
      \end{bmatrix}$$
    $$\begin{bmatrix}1 & \m 1 \\
      1 & 0 \\
      \end{bmatrix}^3=\begin{bmatrix}\m 1 & 0 \\
      0 & \m 1\\
      \end{bmatrix}$$
    $$\begin{bmatrix}1 & \m 1 \\
      1 & 0 \\
      \end{bmatrix}^6=\begin{bmatrix}1 & 0 \\
      0 & 1\\
      \end{bmatrix}$$
  }
\item[(b)]{
    $$\begin{bmatrix}0 & 1 \\
      \m 1 & 0 \\
      \end{bmatrix}^2=\begin{bmatrix}\m 1 & 0 \\
      0 & \m 1\\
      \end{bmatrix}$$
    $$\begin{bmatrix}0 & 1 \\
      \m 1 & 0 \\
      \end{bmatrix}^4=\begin{bmatrix}1 & 0 \\
      0 & 1\\
      \end{bmatrix}$$}
\end{itemize}
\end{multicols}
\begin{multicols}{2}
\begin{itemize}
\item[(c)]{
    $$\begin{bmatrix}1 & 1 \\
      0 & 1 \\
      \end{bmatrix}^2=\begin{bmatrix}1 & 2 \\
      0 & 1 \\
      \end{bmatrix}$$
    Infinite order.
  }
\item[(d)]{
    $$\begin{bmatrix}\m 1 & 1 \\
      0 & 1 \\
      \end{bmatrix}^2=\begin{bmatrix}1 & 0 \\
      0 & 1 \\
      \end{bmatrix}$$}
\end{itemize}
\end{multicols}
\end{mdframed}
\newpage
%%%%%%%%%%%%%%%%%%%%%%%%%%%%%%%%%%%%%%%%%%%%%%%%%%%%%%%%%%%%%%%%%%%%%%%%%%%%%%%%
%%%%%%%%%%%%%%%%%%%%%%%%%%%%%%%%%%%%%%%%%%%%%%%%%%%%%%%%%%%%%%%%%%%%%%%%%%%%%%%%
%%%%%%%%%%%%%%%%%%%%%%%%%%%%%%%%%%%%%%%%%%%%%%%%%%%%%%%%%%%%%%%%%%%%%%%%%%%%%%%%
%%%%%%%%%%%%%%%%%%%%%%%%%%%%%%%%%%%%%%%%%%%%%%%%%%%%%%%%%%%%%%%%%%%%%%%%%%%%%%%%
\begin{mdframed}[style=darkQuesion]
2. Let $A=\left[\begin{array}{rr}1 & \m 1 \\
      \m 1 & 0\end{array}\right] \in \mathrm{GL}_{2}(\mathrm{R})$.
Show that $A$ has infinite order by proving that
$A^{n}=\left[\begin{array}{cc}F_{n+1} & -F_{n} \\
      -F_{n} & F_{n-1}\end{array}\right],$
for $n \geq 1,$ where $F_{0}=0, F_{1}=1,$ and $F_{n+1}=$
$F_{n}+F_{n-1}$ is the Fibonacci sequence.
\end{mdframed}

%%%%%%%%%%%%%%%%%%%%%%%%%%%%%%%%%%%%%%%%%%%%%%%%%%%%%%%%%%%%%%%%%%%%%%%%%%%%%%%%
\begin{mdframed}[style=darkAnswer,frametitle={Joe Starr}]
We will proceed with induction:
\begin{itemize}[align=left]
\item[Base Case:]{ Consider $1$ for the basecase.
    $\begin{bmatrix}F_{n+1} & \m F_{n} \\
      \m F_{n} & F_{n-1} \\
      \end{bmatrix}=\begin{bmatrix}1 & \m 1 \\
      \m 1 & 0 \\
      \end{bmatrix}$ showing the base case to be true.
  }
\item[Inductive Case:]{ Assume that it's ture for the nth power we will show
    this implies the $n+1$th case to be true.
    $$A^{n+1}=A^nA^1=\begin{bmatrix}F_{n+1} & \m F_{n} \\
      \m F_{n} & F_{n-1} \\
      \end{bmatrix}\begin{bmatrix}1 & \m 1 \\
      \m 1 & 0 \\
      \end{bmatrix}=\begin{bmatrix}F_{n+2} & \m F_{n+1} \\
      \m F_{n+1} & F_{n} \\
      \end{bmatrix}$$ showing the Inductive case to be true, and $A$ of infinite
    order.
  }
\end{itemize}
\end{mdframed}
\newpage
%%%%%%%%%%%%%%%%%%%%%%%%%%%%%%%%%%%%%%%%%%%%%%%%%%%%%%%%%%%%%%%%%%%%%%%%%%%%%%%%
%%%%%%%%%%%%%%%%%%%%%%%%%%%%%%%%%%%%%%%%%%%%%%%%%%%%%%%%%%%%%%%%%%%%%%%%%%%%%%%%
%%%%%%%%%%%%%%%%%%%%%%%%%%%%%%%%%%%%%%%%%%%%%%%%%%%%%%%%%%%%%%%%%%%%%%%%%%%%%%%%
%%%%%%%%%%%%%%%%%%%%%%%%%%%%%%%%%%%%%%%%%%%%%%%%%%%%%%%%%%%%%%%%%%%%%%%%%%%%%%%%
\begin{mdframed}[style=darkQuesion]
3. Prove that the set of all rational numbers of the form $m / n$, where $m, n \in \mathbf{Z}$ and $n$ is square-free, is a subgroup of Q (under addition).

\end{mdframed}

%%%%%%%%%%%%%%%%%%%%%%%%%%%%%%%%%%%%%%%%%%%%%%%%%%%%%%%%%%%%%%%%%%%%%%%%%%%%%%%%
\begin{mdframed}[style=darkAnswer,frametitle={Joe Starr}]
\begin{itemize}[align=left]
\Invs{Let $m,n \in \Z$ with the given properties. Take $\frac{\m m}{n}$
    and consider \\$\frac{m}{n}+\frac{\m m}{n}=\frac{m-m}{n}=0$ as desired.
  }
\Clos{
    Let $m,n,a,b\in \Z$ with the given properties. Take
    $\frac{m}{n}+\frac{a}{b}=\frac{mb+an}{bn}$, since $b$ and $n$ are square
    free $bn$ is also square free.
  }
\end{itemize}
\end{mdframed}
\newpage
%%%%%%%%%%%%%%%%%%%%%%%%%%%%%%%%%%%%%%%%%%%%%%%%%%%%%%%%%%%%%%%%%%%%%%%%%%%%%%%%
%%%%%%%%%%%%%%%%%%%%%%%%%%%%%%%%%%%%%%%%%%%%%%%%%%%%%%%%%%%%%%%%%%%%%%%%%%%%%%%%
%%%%%%%%%%%%%%%%%%%%%%%%%%%%%%%%%%%%%%%%%%%%%%%%%%%%%%%%%%%%%%%%%%%%%%%%%%%%%%%%
%%%%%%%%%%%%%%%%%%%%%%%%%%%%%%%%%%%%%%%%%%%%%%%%%%%%%%%%%%%%%%%%%%%%%%%%%%%%%%%%
\begin{mdframed}[style=darkQuesion]
4. Show that $\{\ (1),\ (1,2)(3,4),\ (1,3)(2,4),\ (1,4)(2,3)\ \}$ is a subgroup
of $S_{4}$
\end{mdframed}

%%%%%%%%%%%%%%%%%%%%%%%%%%%%%%%%%%%%%%%%%%%%%%%%%%%%%%%%%%%%%%%%%%%%%%%%%%%%%%%%
\begin{mdframed}[style=darkAnswer,frametitle={Joe Starr}]
We begin by labeling each permutation
$$A=(1,2)(3,4)=$$$$B=(1,3)(2,4)$$$$C=(1,4)(2,3)$$
\begin{itemize}[align=left]
\Invs{
    $$AA=(1,2)(3,4)(1,2)(3,4)=(1)$$
    $$BB=(1,3)(2,4)(1,3)(2,4)=(1)$$
    $$CC=(1,4)(2,3)(1,4)(2,3)=(1)$$
  }
\Clos{
    $$AB= (1,4)(2,3)$$
    $$BA= (1,4)(2,3)$$
    $$AC= (1,3)(2,4)$$
    $$CA= (1,3)(2,4)$$
    $$CB= (1,2)(3,4)$$
    $$BC= (1,2)(3,4)$$
  }
\end{itemize}
\end{mdframed}
\newpage
%%%%%%%%%%%%%%%%%%%%%%%%%%%%%%%%%%%%%%%%%%%%%%%%%%%%%%%%%%%%%%%%%%%%%%%%%%%%%%%%
%%%%%%%%%%%%%%%%%%%%%%%%%%%%%%%%%%%%%%%%%%%%%%%%%%%%%%%%%%%%%%%%%%%%%%%%%%%%%%%%
%%%%%%%%%%%%%%%%%%%%%%%%%%%%%%%%%%%%%%%%%%%%%%%%%%%%%%%%%%%%%%%%%%%%%%%%%%%%%%%%
%%%%%%%%%%%%%%%%%%%%%%%%%%%%%%%%%%%%%%%%%%%%%%%%%%%%%%%%%%%%%%%%%%%%%%%%%%%%%%%%
\begin{mdframed}[style=darkQuesion]
\begin{itemize}[align=left]
\item [(a)] {
    Show that
    $T=\lrs{\begin{bmatrix}
          a & 0 \\
          c & d\\
          \end{bmatrix}
          \vert a d \neq 0}$ is a subgroup of $G$.
  }
\item [(b)] {
    Show that
    $D=\lrs{\begin{bmatrix}
          a & 0 \\
          0 & d\\
          \end{bmatrix} \vert a d \neq 0}$
    is a subgroup of $G$.
  }
\end{itemize}
\end{mdframed}

%%%%%%%%%%%%%%%%%%%%%%%%%%%%%%%%%%%%%%%%%%%%%%%%%%%%%%%%%%%%%%%%%%%%%%%%%%%%%%%%
\begin{mdframed}[style=darkAnswer,frametitle={Joe Starr}]
\begin{itemize}[align=left]
\item [(a)]{
    \begin{itemize}[align=left]
    \Invs{We know by construction of $G$ there exist an inverse of the form $\begin{bmatrix}
          \frac{1}{a} & 0\\
          \frac{\m c}{ad} & \frac{1}{d}\\
          \end{bmatrix}$, by taking $\frac{1}{a}\frac{1}{d}$ that this is not $0$
        so the inverse is in $T$.
      }
    \Clos{
        If we take $A,B\in T$ $$AB=\begin{bmatrix}
          a & 0\\
          c & d\\
          \end{bmatrix}\begin{bmatrix}
          w & 0\\
          y & z\\
          \end{bmatrix}=\begin{bmatrix}
          aw & 0\\
          cw+dy & zd\\
          \end{bmatrix}$$
        since both $ad\neq0$ and $wz\neq 0$ it holds $adwz\neq 0$.
      }
    \end{itemize}
  }
\item [(b)]{
    \begin{itemize}[align=left]
    \Invs{We know by construction of $G$ there exist an inverse of the form $\begin{bmatrix}
          \frac{1}{a} & 0\\
          0 & \frac{1}{d}\\
          \end{bmatrix}$, by taking $\frac{1}{a}\frac{1}{d}$ that this is not $0$
        so the inverse is in $T$.
      }
    \Clos{
        If we take $A,B\in T$ $$AB=\begin{bmatrix}
          a & 0\\
          0 & d\\
          \end{bmatrix}\begin{bmatrix}
          w & 0\\
          0 & z\\
          \end{bmatrix}=\begin{bmatrix}
          aw & 0\\
          0 & zd\\
          \end{bmatrix}$$
        since both $ad\neq0$ and $wz\neq 0$ it holds $adwz\neq 0$.
      }
    \end{itemize}
  }
\end{itemize}
\end{mdframed}
\newpage
%%%%%%%%%%%%%%%%%%%%%%%%%%%%%%%%%%%%%%%%%%%%%%%%%%%%%%%%%%%%%%%%%%%%%%%%%%%%%%%%
%%%%%%%%%%%%%%%%%%%%%%%%%%%%%%%%%%%%%%%%%%%%%%%%%%%%%%%%%%%%%%%%%%%%%%%%%%%%%%%%
%%%%%%%%%%%%%%%%%%%%%%%%%%%%%%%%%%%%%%%%%%%%%%%%%%%%%%%%%%%%%%%%%%%%%%%%%%%%%%%%
%%%%%%%%%%%%%%%%%%%%%%%%%%%%%%%%%%%%%%%%%%%%%%%%%%%%%%%%%%%%%%%%%%%%%%%%%%%%%%%%
\begin{mdframed}[style=darkQuesion]
7. Let $G=\mathrm{GL}_{2}(\mathrm{R})$.
Show that the subset $S$ of $G$ defined by
$S=\left\{\left[\begin{array}{ll}a & b \\
      c & d\end{array}\right] | b=c\right\}$ of symmetric $2   \times 2$ matrices does not form a subgroup of $G .$

\end{mdframed}

%%%%%%%%%%%%%%%%%%%%%%%%%%%%%%%%%%%%%%%%%%%%%%%%%%%%%%%%%%%%%%%%%%%%%%%%%%%%%%%%
\begin{mdframed}[style=darkAnswer,frametitle={Joe Starr}]
Consider $$\begin{bmatrix}
  1 & 3\\
  3 & 5\\
  \end{bmatrix}\begin{bmatrix}
  1 & 3\\
  3 & 1\\
  \end{bmatrix}=\begin{bmatrix}
  10 & 6\\
  18 & 14\\
  \end{bmatrix}$$ showing this set is not closed.
\end{mdframed}
\newpage
%%%%%%%%%%%%%%%%%%%%%%%%%%%%%%%%%%%%%%%%%%%%%%%%%%%%%%%%%%%%%%%%%%%%%%%%%%%%%%%%
%%%%%%%%%%%%%%%%%%%%%%%%%%%%%%%%%%%%%%%%%%%%%%%%%%%%%%%%%%%%%%%%%%%%%%%%%%%%%%%%
%%%%%%%%%%%%%%%%%%%%%%%%%%%%%%%%%%%%%%%%%%%%%%%%%%%%%%%%%%%%%%%%%%%%%%%%%%%%%%%%
%%%%%%%%%%%%%%%%%%%%%%%%%%%%%%%%%%%%%%%%%%%%%%%%%%%%%%%%%%%%%%%%%%%%%%%%%%%%%%%%
\begin{mdframed}[style=darkQuesion]
8. Let $G=\mathrm{GL}_{2}(\mathrm{R}) .$ For each of the following subsets of $M_{2}(\mathbf{R}),$ determine whether
or not the subset is a subgroup of $G .$
\begin{itemize}[align=left]

\item []{(a) $A=\left\{\left[\begin{array}{ll}a & b \\ 0 & 0\end{array}\right] | a b \neq 0\right\}$}
\item []{(b) $B=\left\{\left[\begin{array}{ll}0 & b \\ c & 0\end{array}\right] | b c \neq 0\right\}$}
\item []{(c) $C=\left\{\left[\begin{array}{ll}1 & 0 \\ 0 & c\end{array}\right] | c \neq 0\right\}$}

\end{itemize}
\end{mdframed}

%%%%%%%%%%%%%%%%%%%%%%%%%%%%%%%%%%%%%%%%%%%%%%%%%%%%%%%%%%%%%%%%%%%%%%%%%%%%%%%%
\begin{mdframed}[style=darkAnswer,frametitle={Joe Starr}]
\begin{itemize}[align=left]
\item [(a)]{This set doesn't contain the identity so can not be a subgroup}
\item [(b)]{This set doesn't contain the identity so can not be a subgroup}
\item [(c)]{
    \begin{itemize}[align=left]
    \Invs{Let $\begin{bmatrix}
          1 & 0\\
          0 & c\\
          \end{bmatrix}$ with the given properties. We consider the inverse of $\begin{bmatrix}
          1 & 0\\
          0 & c\\
          \end{bmatrix}$ which is $\begin{bmatrix}
          1 & 0\\
          0 & \frac{1}{c}\\
          \end{bmatrix}$ which is in $C$;}
    \Clos{Let $\begin{bmatrix}
          1 & 0\\
          0 & c\\
          \end{bmatrix}$ and Let $\begin{bmatrix}
          1 & 0\\
          0 & a\\
          \end{bmatrix}$ with the given properties. Consider
        $$\begin{bmatrix}
          1 & 0\\
          0 & c\\
          \end{bmatrix}\begin{bmatrix}
          1 & 0\\
          0 & a\\
          \end{bmatrix}=\begin{bmatrix}
          1 & 0\\
          0 & ca\\
          \end{bmatrix}$$ We observe that $a\neq0$ and $c\neq 0$, consequently $ac\neq0$.
      }
    \end{itemize}
  }
\end{itemize}
\end{mdframed}
\newpage
%%%%%%%%%%%%%%%%%%%%%%%%%%%%%%%%%%%%%%%%%%%%%%%%%%%%%%%%%%%%%%%%%%%%%%%%%%%%%%%%
%%%%%%%%%%%%%%%%%%%%%%%%%%%%%%%%%%%%%%%%%%%%%%%%%%%%%%%%%%%%%%%%%%%%%%%%%%%%%%%%
%%%%%%%%%%%%%%%%%%%%%%%%%%%%%%%%%%%%%%%%%%%%%%%%%%%%%%%%%%%%%%%%%%%%%%%%%%%%%%%%
%%%%%%%%%%%%%%%%%%%%%%%%%%%%%%%%%%%%%%%%%%%%%%%%%%%%%%%%%%%%%%%%%%%%%%%%%%%%%%%%
\begin{mdframed}[style=darkQuesion]
9. Let $G=\mathrm{GL}_{3}(\mathbf{R}) .$ Show that $H=\left\{\left[\begin{array}{lll}1 & 0 & 0 \\ a & 1 & 0 \\ b & c & 1\end{array}\right]\right\}$ is a subgroup of $G .$

\end{mdframed}

%%%%%%%%%%%%%%%%%%%%%%%%%%%%%%%%%%%%%%%%%%%%%%%%%%%%%%%%%%%%%%%%%%%%%%%%%%%%%%%%
\begin{mdframed}[style=darkAnswer,frametitle={Joe Starr}]
\begin{itemize}[align=left]
\Invs{Let $\begin{bmatrix}
      1 & 0 & 0 \\
      a & 1 & 0 \\
      b & c & 1 \\
      \end{bmatrix}$ with the given properties. We consider the inverse of $\begin{bmatrix}
      1 & 0 & 0 \\
      a & 1 & 0 \\
      b & c & 1 \\
      \end{bmatrix}$ which is $\begin{bmatrix}
      1 & 0 & 0 \\
      \m a & 1 & 0 \\
      ac-b & \m c & 1 \\
      \end{bmatrix}$ which is in $H$;}
\Clos{Let $\begin{bmatrix}
      1 & 0 & 0 \\
      a & 1 & 0 \\
      b & c & 1 \\
      \end{bmatrix}$ and Let $\begin{bmatrix}
      1 & 0 & 0 \\
      x & 1 & 0 \\
      y & z & 1 \\
      \end{bmatrix}$ with the given properties. Consider
    $$\begin{bmatrix}
      1 & 0 & 0 \\
      a & 1 & 0 \\
      b & c & 0 \\
      \end{bmatrix}\begin{bmatrix}
      1 & 0 & 0 \\
      x & 1 & 0 \\
      y & z & 1 \\
      \end{bmatrix}=\begin{bmatrix}
      1 & 0 & 0 \\
      a+x & 1 & 0 \\
      b+cx+y & c+z & 1 \\
      \end{bmatrix}$$
  }
\end{itemize}
\end{mdframed}
\newpage%%%%%%%%%%%%%%%%%%%%%%%%%%%%%%%%%%%%%%%%%%%%%%%%%%%%%%%%%%%%%%%%%%%%%%%%%%%%%%%%
%%%%%%%%%%%%%%%%%%%%%%%%%%%%%%%%%%%%%%%%%%%%%%%%%%%%%%%%%%%%%%%%%%%%%%%%%%%%%%%%
%%%%%%%%%%%%%%%%%%%%%%%%%%%%%%%%%%%%%%%%%%%%%%%%%%%%%%%%%%%%%%%%%%%%%%%%%%%%%%%%
%%%%%%%%%%%%%%%%%%%%%%%%%%%%%%%%%%%%%%%%%%%%%%%%%%%%%%%%%%%%%%%%%%%%%%%%%%%%%%%%
\begin{mdframed}[style=darkQuesion]
10. Let $m$ and $n$ be nonzero integers, with ( $m, n$ ) = $d$. Show that $m$ and $n$ belong to $d \mathbf{Z},$ and that if $H$ is any subgroup of $\mathbf{Z}$ that contains both $m$ and $n,$ then $d \mathbf{Z} \subseteq H$
\end{mdframed}

%%%%%%%%%%%%%%%%%%%%%%%%%%%%%%%%%%%%%%%%%%%%%%%%%%%%%%%%%%%%%%%%%%%%%%%%%%%%%%%%
\begin{mdframed}[style=darkAnswer,frametitle={Joe Starr}]
We will first show that $m$ and $n$ are in $d\Z$. Let $m,n\in \Z$ with the given
properties. Since $\gcd\lrp{m,n}=d$ we observe that $dq_1=m$ and $dq_2=n$
making $m,n\in d\Z$ as desired.

Next let $H$ be a subgroup of $\Z$ with $m,n\in H$. Let $a\in d\Z$, with
$a<m\leq n$ by construction we have $a=dq$ for some $q$.
%TODO need to finish this one. 
\end{mdframed}
\newpage
%%%%%%%%%%%%%%%%%%%%%%%%%%%%%%%%%%%%%%%%%%%%%%%%%%%%%%%%%%%%%%%%%%%%%%%%%%%%%%%%
%%%%%%%%%%%%%%%%%%%%%%%%%%%%%%%%%%%%%%%%%%%%%%%%%%%%%%%%%%%%%%%%%%%%%%%%%%%%%%%%
%%%%%%%%%%%%%%%%%%%%%%%%%%%%%%%%%%%%%%%%%%%%%%%%%%%%%%%%%%%%%%%%%%%%%%%%%%%%%%%%
%%%%%%%%%%%%%%%%%%%%%%%%%%%%%%%%%%%%%%%%%%%%%%%%%%%%%%%%%%%%%%%%%%%%%%%%%%%%%%%%
\begin{mdframed}[style=darkQuesion]
11. Let $S$ be a set, and let $a$ be a fixed element of $S .$ Show that $\{\sigma \in \operatorname{Sym}(S) | \sigma(a)=a\}$ is a subgroup of $\operatorname{Sym}(S)$

\end{mdframed}

%%%%%%%%%%%%%%%%%%%%%%%%%%%%%%%%%%%%%%%%%%%%%%%%%%%%%%%%%%%%%%%%%%%%%%%%%%%%%%%%
\begin{mdframed}[style=darkAnswer,frametitle={Joe Starr}]
Let $A=\{\sigma \in \operatorname{Sym}(S) | \sigma(a)=a\}$
\begin{itemize}[align=left]
\Invs{
    By proposition 2.1.7 in the text we have inverses.
  }
\Clos{
    Let $\sigma,\varphi \in A$, consider the composition of these two functions
    around $a$, $\pof{\sof{a}}=\pof{a}=a$ showing closure, as desired.
  }
\end{itemize}
\end{mdframed}
\newpage
%%%%%%%%%%%%%%%%%%%%%%%%%%%%%%%%%%%%%%%%%%%%%%%%%%%%%%%%%%%%%%%%%%%%%%%%%%%%%%%%
%%%%%%%%%%%%%%%%%%%%%%%%%%%%%%%%%%%%%%%%%%%%%%%%%%%%%%%%%%%%%%%%%%%%%%%%%%%%%%%%
%%%%%%%%%%%%%%%%%%%%%%%%%%%%%%%%%%%%%%%%%%%%%%%%%%%%%%%%%%%%%%%%%%%%%%%%%%%%%%%%
%%%%%%%%%%%%%%%%%%%%%%%%%%%%%%%%%%%%%%%%%%%%%%%%%%%%%%%%%%%%%%%%%%%%%%%%%%%%%%%%
\begin{mdframed}[style=darkQuesion]
12. For each of the following groups, find all elements of finite order.
\begin{itemize}
\item []{
    (a) $\mathbf{R}^{\times}$
  }
\item []{
    (b) $\mathbf{C}^{\times}$
  }
\end{itemize}

\end{mdframed}

%%%%%%%%%%%%%%%%%%%%%%%%%%%%%%%%%%%%%%%%%%%%%%%%%%%%%%%%%%%%%%%%%%%%%%%%%%%%%%%%
\begin{mdframed}[style=darkAnswer,frametitle={Joe Starr}]
\begin{itemize}[align=left]
\item [(a)]{
    1 and $\m 1$ are the only elements of finite order.
  }d
\item [(b)]{
    $1,\m 1 ,\m i$, and $i$ are the only elements of
    finite order.
  }
\end{itemize}
\end{mdframed}
\newpage
%%%%%%%%%%%%%%%%%%%%%%%%%%%%%%%%%%%%%%%%%%%%%%%%%%%%%%%%%%%%%%%%%%%%%%%%%%%%%%%%
%%%%%%%%%%%%%%%%%%%%%%%%%%%%%%%%%%%%%%%%%%%%%%%%%%%%%%%%%%%%%%%%%%%%%%%%%%%%%%%%
%%%%%%%%%%%%%%%%%%%%%%%%%%%%%%%%%%%%%%%%%%%%%%%%%%%%%%%%%%%%%%%%%%%%%%%%%%%%%%%%
%%%%%%%%%%%%%%%%%%%%%%%%%%%%%%%%%%%%%%%%%%%%%%%%%%%%%%%%%%%%%%%%%%%%%%%%%%%%%%%%
\begin{mdframed}[style=darkQuesion]
13. Let $G$ be an abelian group, such that the operation on $G$
is denoted additively.
Show that $\{a \in G | 2 a=0\}$ is a subgroup of $G .$
Compute this subgroup for $G=\mathbf{Z}_{12}$

\end{mdframed}

%%%%%%%%%%%%%%%%%%%%%%%%%%%%%%%%%%%%%%%%%%%%%%%%%%%%%%%%%%%%%%%%%%%%%%%%%%%%%%%%
\begin{mdframed}[style=darkAnswer,frametitle={Joe Starr}]
Let $S=\{a \in G | 2 a=0\}$
\begin{itemize}[align=left]
\Invs{
    Let $a\in S$, by transitive proof $2a=0\rightarrow a+a=0\rightarrow a=-a$
  }
\Clos{
    Let $a,b\in S$, by transitive proof,
    \begin{align*}
    2a=0 &\rightarrow 2a+2b=0+0\\
    &\rightarrow 2(a+b)=0\\
    \end{align*}
    showing closure of $S$.
  }
\end{itemize}
next we let $G=\Z_{12}$ we can calculate $a+a$ for all $a\in \Z_{12}$
\begin{align*}
0+0 &= 0\\
1+1 &= 2\\
2+2 &= 4\\
3+3 &= 6\\
4+4 &= 8\\
5+5 &=10\\
6+6 &= 0\\
7+7 &= 2\\
8+8 &= 4\\
9+9 &= 6\\
10+10&= 8\\
\end{align*}
\end{mdframed}
\newpage
%%%%%%%%%%%%%%%%%%%%%%%%%%%%%%%%%%%%%%%%%%%%%%%%%%%%%%%%%%%%%%%%%%%%%%%%%%%%%%%%
%%%%%%%%%%%%%%%%%%%%%%%%%%%%%%%%%%%%%%%%%%%%%%%%%%%%%%%%%%%%%%%%%%%%%%%%%%%%%%%%
%%%%%%%%%%%%%%%%%%%%%%%%%%%%%%%%%%%%%%%%%%%%%%%%%%%%%%%%%%%%%%%%%%%%%%%%%%%%%%%%
%%%%%%%%%%%%%%%%%%%%%%%%%%%%%%%%%%%%%%%%%%%%%%%%%%%%%%%%%%%%%%%%%%%%%%%%%%%%%%%%
\begin{mdframed}[style=darkQuesion]
14. Let $G$ be an abelian group, and let $H$ be the set of all elements of $G$ of finite order.
\begin{itemize}
\item[]{
    (a) Show that $H$ is a subgroup of $G .$}
\item[]{
    (b) For a fixed positive integer $k$, show that $\{a \in G | o(a) \text{is a divisor of } k\}$ is a subgroup of $H$
  }
\item[]{
    (c) For a fixed positive integer $k$, is $\{a \in G | o(a) \leq k\}$ a subgroup of $H ?$ Either give a proof or give a counterexample.
  }
\end{itemize}

\end{mdframed}

%%%%%%%%%%%%%%%%%%%%%%%%%%%%%%%%%%%%%%%%%%%%%%%%%%%%%%%%%%%%%%%%%%%%%%%%%%%%%%%%
\begin{mdframed}[style=darkAnswer,frametitle={Joe Starr}]
\begin{itemize}[align=left]
\item[(a) ]{
    \begin{itemize}[align=left]
    \Invs{
        Let $a\in H$, since $a$ is of finite order we have $a^n=1$ for some
        $n\in \Z$. We observe that $aa^{n-1}=1$, now considering $a^{n-1}$, 
        we take this to the nth power $\lrp{a^{n-1}}^n=\lrp{a^n}^{n-1}=1$ 
        showing existence of inverses in $H$.
      }
    \Clos{
        Let $a,b\in H $, we consider $ab$ if we know $a^n=1$ and $b^m=1$ for
        some $m,n\in \Z$, if we take $a^{mn}b^{mn}=\lrp{a^n}^m\lrp{b^m}^n=1$.
      }
    \end{itemize}
  }
\item[(b)]{
    Let $A=\{a \in G | o(a) \text{is a divisor of } k\}$
    \begin{itemize}[align=left]
    \Invs{
        Let $a\in A$ we observe that since $k \vert \ord{a}$ we have $a^{kq}=1$ 
        for some $kq\in \Z$ meaning $a^{kq-1}a=1$ now considering $a^{n-1}$, 
        we take this to the nth power $\lrp{a^{kq-1}}^{kq}=\lrp{a^{kq}}^{kq-1}=1$ 
        showing existence of inverses in $A$.
      }
    \Clos{
        Let $a,b\in A $, we consider $ab$ if we know $a^{kn}=1$ and $b^{km}=1$ for
        some $m,n\in \Z$, if we take $a^{kmn}b^{kmn}=\lrp{a^kn}^m\lrp{b^km}^n=1$.
      }
    \end{itemize}
  }
\item[(c)]{
  Let $G=\Z_{10}^+$, and $k=5$, this makes $A=\lrs{2,\ 4,\ 5,\ 6,\ 8}$ if we 
  take $2+5=7$ we can see $A$ is not closed under addition. 
  }
\end{itemize}
\end{mdframed}
\newpage
%%%%%%%%%%%%%%%%%%%%%%%%%%%%%%%%%%%%%%%%%%%%%%%%%%%%%%%%%%%%%%%%%%%%%%%%%%%%%%%%
%%%%%%%%%%%%%%%%%%%%%%%%%%%%%%%%%%%%%%%%%%%%%%%%%%%%%%%%%%%%%%%%%%%%%%%%%%%%%%%%
%%%%%%%%%%%%%%%%%%%%%%%%%%%%%%%%%%%%%%%%%%%%%%%%%%%%%%%%%%%%%%%%%%%%%%%%%%%%%%%%
%%%%%%%%%%%%%%%%%%%%%%%%%%%%%%%%%%%%%%%%%%%%%%%%%%%%%%%%%%%%%%%%%%%%%%%%%%%%%%%%
\begin{mdframed}[style=darkQuesion]
15. Prove that any cyclic group is abelian.
\end{mdframed}

%%%%%%%%%%%%%%%%%%%%%%%%%%%%%%%%%%%%%%%%%%%%%%%%%%%%%%%%%%%%%%%%%%%%%%%%%%%%%%%%
\begin{mdframed}[style=darkAnswer,frametitle={Joe Starr}]
Let $G$ be a cyclic group generated by $g$, select $a,b\in G$, we consider 
$ab\inv{a}\inv{b}$. We observe $a=g^k$ and $\inv{a}=g^{k-1}$ for some $k$, 
similarly for $b$ and some $h$. Now rewriting 
$ab\inv{a}\inv{b}=g^kg^hg^{k-1}g^{h-1}=1$ showing $G$ abelian. 
\end{mdframed}
\newpage
%%%%%%%%%%%%%%%%%%%%%%%%%%%%%%%%%%%%%%%%%%%%%%%%%%%%%%%%%%%%%%%%%%%%%%%%%%%%%%%%
%%%%%%%%%%%%%%%%%%%%%%%%%%%%%%%%%%%%%%%%%%%%%%%%%%%%%%%%%%%%%%%%%%%%%%%%%%%%%%%%
%%%%%%%%%%%%%%%%%%%%%%%%%%%%%%%%%%%%%%%%%%%%%%%%%%%%%%%%%%%%%%%%%%%%%%%%%%%%%%%%
%%%%%%%%%%%%%%%%%%%%%%%%%%%%%%%%%%%%%%%%%%%%%%%%%%%%%%%%%%%%%%%%%%%%%%%%%%%%%%%%
\begin{mdframed}[style=darkQuesion]
16. Prove or disprove this statement. If $G$ is a group in which every proper subgroup is cyclic, then $G$ is cyclic.

\end{mdframed}

%%%%%%%%%%%%%%%%%%%%%%%%%%%%%%%%%%%%%%%%%%%%%%%%%%%%%%%%%%%%%%%%%%%%%%%%%%%%%%%%
\begin{mdframed}[style=darkAnswer,frametitle={Joe Starr}]
Select $G$ with the given property. Let $H$
%TODO This is the next one. 
\end{mdframed}
\newpage
%%%%%%%%%%%%%%%%%%%%%%%%%%%%%%%%%%%%%%%%%%%%%%%%%%%%%%%%%%%%%%%%%%%%%%%%%%%%%%%%
%%%%%%%%%%%%%%%%%%%%%%%%%%%%%%%%%%%%%%%%%%%%%%%%%%%%%%%%%%%%%%%%%%%%%%%%%%%%%%%%
%%%%%%%%%%%%%%%%%%%%%%%%%%%%%%%%%%%%%%%%%%%%%%%%%%%%%%%%%%%%%%%%%%%%%%%%%%%%%%%%
%%%%%%%%%%%%%%%%%%%%%%%%%%%%%%%%%%%%%%%%%%%%%%%%%%%%%%%%%%%%%%%%%%%%%%%%%%%%%%%%
\begin{mdframed}[style=darkQuesion]
17. $\#$ Prove that the intersection of any collection of subgroups of a group is again a subgroup.

\end{mdframed}

%%%%%%%%%%%%%%%%%%%%%%%%%%%%%%%%%%%%%%%%%%%%%%%%%%%%%%%%%%%%%%%%%%%%%%%%%%%%%%%%
\begin{mdframed}[style=darkAnswer,frametitle={Joe Starr}]

\end{mdframed}
\newpage
%%%%%%%%%%%%%%%%%%%%%%%%%%%%%%%%%%%%%%%%%%%%%%%%%%%%%%%%%%%%%%%%%%%%%%%%%%%%%%%%
%%%%%%%%%%%%%%%%%%%%%%%%%%%%%%%%%%%%%%%%%%%%%%%%%%%%%%%%%%%%%%%%%%%%%%%%%%%%%%%%
%%%%%%%%%%%%%%%%%%%%%%%%%%%%%%%%%%%%%%%%%%%%%%%%%%%%%%%%%%%%%%%%%%%%%%%%%%%%%%%%
%%%%%%%%%%%%%%%%%%%%%%%%%%%%%%%%%%%%%%%%%%%%%%%%%%%%%%%%%%%%%%%%%%%%%%%%%%%%%%%%
\begin{mdframed}[style=darkQuesion]
18. Let $G$ be the group of rational numbers, under addition, and let $H, K$ be subgroups
of $G .$ Prove that if $H \neq\{0\}$ and $K \neq\{0\}$, then $H \cap K \neq\{0\}$.

\end{mdframed}

%%%%%%%%%%%%%%%%%%%%%%%%%%%%%%%%%%%%%%%%%%%%%%%%%%%%%%%%%%%%%%%%%%%%%%%%%%%%%%%%
\begin{mdframed}[style=darkAnswer,frametitle={Joe Starr}]

\end{mdframed}
\newpage
%%%%%%%%%%%%%%%%%%%%%%%%%%%%%%%%%%%%%%%%%%%%%%%%%%%%%%%%%%%%%%%%%%%%%%%%%%%%%%%%
%%%%%%%%%%%%%%%%%%%%%%%%%%%%%%%%%%%%%%%%%%%%%%%%%%%%%%%%%%%%%%%%%%%%%%%%%%%%%%%%
%%%%%%%%%%%%%%%%%%%%%%%%%%%%%%%%%%%%%%%%%%%%%%%%%%%%%%%%%%%%%%%%%%%%%%%%%%%%%%%%
%%%%%%%%%%%%%%%%%%%%%%%%%%%%%%%%%%%%%%%%%%%%%%%%%%%%%%%%%%%%%%%%%%%%%%%%%%%%%%%%
\begin{mdframed}[style=darkQuesion]
19. Let $G$ be a group, and let $a \in G .$ The set $C(a)=\{x \in G | x a=a x\}$ of all elements of $G$ that commute with $a$ is called the centralizer of $a .$
$\#(  ext { a ) Show that } C(a)   ext { is a subgroup of } G$
(b) Show that $\langle a\rangle \subseteq C(a)$
(c) Compute $C(a)$ if $G=S_{3}$ and $a=(1,2,3)$
(d) Compute $C(a)$ if $G=S_{3}$ and $a=(1,2)$

\end{mdframed}

%%%%%%%%%%%%%%%%%%%%%%%%%%%%%%%%%%%%%%%%%%%%%%%%%%%%%%%%%%%%%%%%%%%%%%%%%%%%%%%%
\begin{mdframed}[style=darkAnswer,frametitle={Joe Starr}]

\end{mdframed}
\newpage
%%%%%%%%%%%%%%%%%%%%%%%%%%%%%%%%%%%%%%%%%%%%%%%%%%%%%%%%%%%%%%%%%%%%%%%%%%%%%%%%
%%%%%%%%%%%%%%%%%%%%%%%%%%%%%%%%%%%%%%%%%%%%%%%%%%%%%%%%%%%%%%%%%%%%%%%%%%%%%%%%
%%%%%%%%%%%%%%%%%%%%%%%%%%%%%%%%%%%%%%%%%%%%%%%%%%%%%%%%%%%%%%%%%%%%%%%%%%%%%%%%
%%%%%%%%%%%%%%%%%%%%%%%%%%%%%%%%%%%%%%%%%%%%%%%%%%%%%%%%%%%%%%%%%%%%%%%%%%%%%%%%
\begin{mdframed}[style=darkQuesion]
20. Compute the centralizer in $\mathrm{GL}_{2}$ ( $\mathbf{R}$ ) of the matrix $\left[\begin{array}{ll}1 & 1 \\ 0 & 1\end{array}\right]$

\end{mdframed}

%%%%%%%%%%%%%%%%%%%%%%%%%%%%%%%%%%%%%%%%%%%%%%%%%%%%%%%%%%%%%%%%%%%%%%%%%%%%%%%%
\begin{mdframed}[style=darkAnswer,frametitle={Joe Starr}]

\end{mdframed}
\newpage
%%%%%%%%%%%%%%%%%%%%%%%%%%%%%%%%%%%%%%%%%%%%%%%%%%%%%%%%%%%%%%%%%%%%%%%%%%%%%%%%
%%%%%%%%%%%%%%%%%%%%%%%%%%%%%%%%%%%%%%%%%%%%%%%%%%%%%%%%%%%%%%%%%%%%%%%%%%%%%%%%
%%%%%%%%%%%%%%%%%%%%%%%%%%%%%%%%%%%%%%%%%%%%%%%%%%%%%%%%%%%%%%%%%%%%%%%%%%%%%%%%
%%%%%%%%%%%%%%%%%%%%%%%%%%%%%%%%%%%%%%%%%%%%%%%%%%%%%%%%%%%%%%%%%%%%%%%%%%%%%%%%
\begin{mdframed}[style=darkQuesion]
22. Show that if a group $G$ has a unique element $a$ of order $2,$ then $a \in Z(G)$

\end{mdframed}

%%%%%%%%%%%%%%%%%%%%%%%%%%%%%%%%%%%%%%%%%%%%%%%%%%%%%%%%%%%%%%%%%%%%%%%%%%%%%%%%
\begin{mdframed}[style=darkAnswer,frametitle={Joe Starr}]

\end{mdframed}
\newpage
%%%%%%%%%%%%%%%%%%%%%%%%%%%%%%%%%%%%%%%%%%%%%%%%%%%%%%%%%%%%%%%%%%%%%%%%%%%%%%%%
%%%%%%%%%%%%%%%%%%%%%%%%%%%%%%%%%%%%%%%%%%%%%%%%%%%%%%%%%%%%%%%%%%%%%%%%%%%%%%%%
%%%%%%%%%%%%%%%%%%%%%%%%%%%%%%%%%%%%%%%%%%%%%%%%%%%%%%%%%%%%%%%%%%%%%%%%%%%%%%%%
%%%%%%%%%%%%%%%%%%%%%%%%%%%%%%%%%%%%%%%%%%%%%%%%%%%%%%%%%%%%%%%%%%%%%%%%%%%%%%%%
\begin{mdframed}[style=darkQuesion]
23. If the group $G$ is not abelian, show that its center $Z(G)$ is a proper subgroup of an abelian subgroup of $G .$

\end{mdframed}

%%%%%%%%%%%%%%%%%%%%%%%%%%%%%%%%%%%%%%%%%%%%%%%%%%%%%%%%%%%%%%%%%%%%%%%%%%%%%%%%
\begin{mdframed}[style=darkAnswer,frametitle={Joe Starr}]

\end{mdframed}
\newpage
%%%%%%%%%%%%%%%%%%%%%%%%%%%%%%%%%%%%%%%%%%%%%%%%%%%%%%%%%%%%%%%%%%%%%%%%%%%%%%%%
%%%%%%%%%%%%%%%%%%%%%%%%%%%%%%%%%%%%%%%%%%%%%%%%%%%%%%%%%%%%%%%%%%%%%%%%%%%%%%%%
%%%%%%%%%%%%%%%%%%%%%%%%%%%%%%%%%%%%%%%%%%%%%%%%%%%%%%%%%%%%%%%%%%%%%%%%%%%%%%%%
%%%%%%%%%%%%%%%%%%%%%%%%%%%%%%%%%%%%%%%%%%%%%%%%%%%%%%%%%%%%%%%%%%%%%%%%%%%%%%%%
\begin{mdframed}[style=darkQuesion]
26. Let $G$ be a group with $a, b \in G$.
(a) Show that $o\left(a^{-1}\right)=o(a)$
(b) Show that $o(a b)=o(b a)$
(c) Show that $o\left(a b a^{-1}\right)=o(b)$

\end{mdframed}

%%%%%%%%%%%%%%%%%%%%%%%%%%%%%%%%%%%%%%%%%%%%%%%%%%%%%%%%%%%%%%%%%%%%%%%%%%%%%%%%
\begin{mdframed}[style=darkAnswer,frametitle={Joe Starr}]

\end{mdframed}
\newpage
%%%%%%%%%%%%%%%%%%%%%%%%%%%%%%%%%%%%%%%%%%%%%%%%%%%%%%%%%%%%%%%%%%%%%%%%%%%%%%%%
%%%%%%%%%%%%%%%%%%%%%%%%%%%%%%%%%%%%%%%%%%%%%%%%%%%%%%%%%%%%%%%%%%%%%%%%%%%%%%%%
%%%%%%%%%%%%%%%%%%%%%%%%%%%%%%%%%%%%%%%%%%%%%%%%%%%%%%%%%%%%%%%%%%%%%%%%%%%%%%%%
%%%%%%%%%%%%%%%%%%%%%%%%%%%%%%%%%%%%%%%%%%%%%%%%%%%%%%%%%%%%%%%%%%%%%%%%%%%%%%%%
\begin{mdframed}[style=darkQuesion]
27. Let $G$ be a finite group, let $n>2$ be an integer, and let $S$ be the set of elements of
$G$ that have order $n .$ Show that $S$ has an even number of elements.

\end{mdframed}

%%%%%%%%%%%%%%%%%%%%%%%%%%%%%%%%%%%%%%%%%%%%%%%%%%%%%%%%%%%%%%%%%%%%%%%%%%%%%%%%
\begin{mdframed}[style=darkAnswer,frametitle={Joe Starr}]

\end{mdframed}
\newpage
%%%%%%%%%%%%%%%%%%%%%%%%%%%%%%%%%%%%%%%%%%%%%%%%%%%%%%%%%%%%%%%%%%%%%%%%%%%%%%%%
%%%%%%%%%%%%%%%%%%%%%%%%%%%%%%%%%%%%%%%%%%%%%%%%%%%%%%%%%%%%%%%%%%%%%%%%%%%%%%%%
%%%%%%%%%%%%%%%%%%%%%%%%%%%%%%%%%%%%%%%%%%%%%%%%%%%%%%%%%%%%%%%%%%%%%%%%%%%%%%%%
%%%%%%%%%%%%%%%%%%%%%%%%%%%%%%%%%%%%%%%%%%%%%%%%%%%%%%%%%%%%%%%%%%%%%%%%%%%%%%%%
\begin{mdframed}[style=darkQuesion]
28. Let $G$ be a group with $a, b \in G$. Assume that $o(a)$ and $o(b)$ are finite and relatively prime, and that $a b=b a$. Show that $o(a b)=o(a) o(b)$

\end{mdframed}

%%%%%%%%%%%%%%%%%%%%%%%%%%%%%%%%%%%%%%%%%%%%%%%%%%%%%%%%%%%%%%%%%%%%%%%%%%%%%%%%
\begin{mdframed}[style=darkAnswer,frametitle={Joe Starr}]

\end{mdframed}
\newpage
%%%%%%%%%%%%%%%%%%%%%%%%%%%%%%%%%%%%%%%%%%%%%%%%%%%%%%%%%%%%%%%%%%%%%%%%%%%%%%%%
%%%%%%%%%%%%%%%%%%%%%%%%%%%%%%%%%%%%%%%%%%%%%%%%%%%%%%%%%%%%%%%%%%%%%%%%%%%%%%%%
%%%%%%%%%%%%%%%%%%%%%%%%%%%%%%%%%%%%%%%%%%%%%%%%%%%%%%%%%%%%%%%%%%%%%%%%%%%%%%%%
%%%%%%%%%%%%%%%%%%%%%%%%%%%%%%%%%%%%%%%%%%%%%%%%%%%%%%%%%%%%%%%%%%%%%%%%%%%%%%%%
\begin{mdframed}[style=darkQuesion]
29. Find an example of a group $G$ and elements $a, b \in G$ such that $a$ and $b$ each have finite order but $a b$ does not.

\end{mdframed}

%%%%%%%%%%%%%%%%%%%%%%%%%%%%%%%%%%%%%%%%%%%%%%%%%%%%%%%%%%%%%%%%%%%%%%%%%%%%%%%%
\begin{mdframed}[style=darkAnswer,frametitle={Joe Starr}]

\end{mdframed}
%TODO: Section 3.3
\newpage
%%%%%%%%%%%%%%%%%%%%%%%%%%%%%%%%%%%%%%%%%%%%%%%%%%%%%%%%%%%%%%%%%%%%%%%%%%%%%%%%
%%%%%%%%%%%%%%%%%%%%%%%%%%%%%%%%%%%%%%%%%%%%%%%%%%%%%%%%%%%%%%%%%%%%%%%%%%%%%%%%
%%%%%%%%%%%%%%%%%%%%%%%%%%%%%%%%%%%%%%%%%%%%%%%%%%%%%%%%%%%%%%%%%%%%%%%%%%%%%%%%
%%%%%%%%%%%%%%%%%%%%%%%%%%%%%%%%%%%%%%%%%%%%%%%%%%%%%%%%%%%%%%%%%%%%%%%%%%%%%%%%
\begin{mdframed}[style=darkQuesion]
2.
\end{mdframed}

%%%%%%%%%%%%%%%%%%%%%%%%%%%%%%%%%%%%%%%%%%%%%%%%%%%%%%%%%%%%%%%%%%%%%%%%%%%%%%%%
\begin{mdframed}[style=darkAnswer,frametitle={Joe Starr}]

\end{mdframed}
\newpage
%%%%%%%%%%%%%%%%%%%%%%%%%%%%%%%%%%%%%%%%%%%%%%%%%%%%%%%%%%%%%%%%%%%%%%%%%%%%%%%%
%%%%%%%%%%%%%%%%%%%%%%%%%%%%%%%%%%%%%%%%%%%%%%%%%%%%%%%%%%%%%%%%%%%%%%%%%%%%%%%%
%%%%%%%%%%%%%%%%%%%%%%%%%%%%%%%%%%%%%%%%%%%%%%%%%%%%%%%%%%%%%%%%%%%%%%%%%%%%%%%%
%%%%%%%%%%%%%%%%%%%%%%%%%%%%%%%%%%%%%%%%%%%%%%%%%%%%%%%%%%%%%%%%%%%%%%%%%%%%%%%%
\begin{mdframed}[style=darkQuesion]
2.
\end{mdframed}

%%%%%%%%%%%%%%%%%%%%%%%%%%%%%%%%%%%%%%%%%%%%%%%%%%%%%%%%%%%%%%%%%%%%%%%%%%%%%%%%
\begin{mdframed}[style=darkAnswer,frametitle={Joe Starr}]

\end{mdframed}
\newpage