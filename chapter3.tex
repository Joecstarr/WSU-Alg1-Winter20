\section{Groups}
\subsection{Definition of a Group}
%3.2: 1-4 6-20 22 23 26 27 28 29
%%%%%%%%%%%%%%%%%%%%%%%%%%%%%%%%%%%%%%%%%%%%%%%%%%%%%%%%%%%%%%%%%%%%%%%%%%%%%%%%
%%%%%%%%%%%%%%%%%%%%%%%%%%%%%%%%%%%%%%%%%%%%%%%%%%%%%%%%%%%%%%%%%%%%%%%%%%%%%%%%
%%%%%%%%%%%%%%%%%%%%%%%%%%%%%%%%%%%%%%%%%%%%%%%%%%%%%%%%%%%%%%%%%%%%%%%%%%%%%%%%
%%%%%%%%%%%%%%%%%%%%%%%%%%%%%%%%%%%%%%%%%%%%%%%%%%%%%%%%%%%%%%%%%%%%%%%%%%%%%%%%
\begin{mdframed}[style=darkQuesion]
  1. Using ordinary addition of integers as the operation, show that the set of
  even integers is a group but the set of odd integers is not.
\end{mdframed}
%%%%%%%%%%%%%%%%%%%%%%%%%%%%%%%%%%%%%%%%%%%%%%%%%%%%%%%%%%%%%%%%%%%%%%%%%%%%%%%%
\begin{mdframed}[style=darkAnswer,frametitle={Joe Starr}]
  We begin considering the even integers, that is integers of the form $2k$. We
  must also include 0 in the even integers. We get the identity element as well
  as Associativity and inverses for free from integer addition on $\Z$.
  We then consider closure. Let $n$ and $m$ be even integers, if we take $m+n$
  we can see we have, $m+n=2k_m+2k_n=2(k_m+k_n)$ an even integer. Making the
  even integers a group under addition.

  Next we consider the odd integers, take $3+3=2(3)$ an even integer, showing
  odds are not closed under addition and not a group.
\end{mdframed}
\newpage
%%%%%%%%%%%%%%%%%%%%%%%%%%%%%%%%%%%%%%%%%%%%%%%%%%%%%%%%%%%%%%%%%%%%%%%%%%%%%%%%
%%%%%%%%%%%%%%%%%%%%%%%%%%%%%%%%%%%%%%%%%%%%%%%%%%%%%%%%%%%%%%%%%%%%%%%%%%%%%%%%
%%%%%%%%%%%%%%%%%%%%%%%%%%%%%%%%%%%%%%%%%%%%%%%%%%%%%%%%%%%%%%%%%%%%%%%%%%%%%%%%
%%%%%%%%%%%%%%%%%%%%%%%%%%%%%%%%%%%%%%%%%%%%%%%%%%%%%%%%%%%%%%%%%%%%%%%%%%%%%%%%
\begin{mdframed}[style=darkQuesion]
  2. For each binary operation $\ast$ defined on a set below,
  determine whether or not $\ast$ gives a group structure on the set.
  If it is not a group, say which axioms fail to hold.
  \begin{multicols}{2}
    \begin{itemize}
      \item[(a)]{Define $\ast$ on $\Z$ by $a\ast b=ab$.}
      \item[(b)]{Define $\ast$ on $\Z$ by $a\ast b=\max{a,b}$.}
      \item[(c)]{Define $\ast$ on $\Z$ by $a\ast b=a-b$.}
      \item[(d)]{Define $\ast$ on $\Z$ by $a\ast b=\abs{ab}$.}
      \item[(e)]{Define $\ast$ on $\R^{+}$ by $a\ast b=ab$.}
      \item[(f)]{Define $\ast$ on $\Q$ by $a\ast b=ab$.}
    \end{itemize}
  \end{multicols}
\end{mdframed}

%%%%%%%%%%%%%%%%%%%%%%%%%%%%%%%%%%%%%%%%%%%%%%%%%%%%%%%%%%%%%%%%%%%%%%%%%%%%%%%%
\begin{mdframed}[style=darkAnswer,frametitle={Joe Starr}]
  \begin{multicols}{2}
    \begin{itemize}[align=left]
      \item[(a)]{
            \begin{itemize}[align=left]
              \Invs{Let $a\in \Z$ but $a\neq 1$ and $a\neq 1$,
                $\inv{a}\notin \Z$.}
            \end{itemize}
            }
      \item[(b)]{
            \begin{itemize}[align=left]
              \Ident{$\max{a,a-1}=a$ for all $a\in \Z$ this means there is no
                $e$ with $\max{a,e}=a$ for all $a$. }
            \end{itemize}
            }
      \item[(c)]{
            \begin{itemize}[align=left]
              \Assoc{\begin{align*}
                  \lrp{a\ast b}\ast c & =\lrp{a-b}\ast c     \\
                                      & =\lrp{a-b}-c         \\
                                      & =a-\lrp{b-c}         \\
                                      & =a\ast \lrp{b\ast c}
                \end{align*}}
              \Invs{Select $a\in \Z$, $$a\ast a = a-a = 0$$}
              \Clos{Obvious from closure of $\lrp{\Z,+}$}
              \Ident{0 is the identity, Obvious from $\lrp{\Z,+}$}
            \end{itemize}

            }
      \item[(d)]{
            \begin{itemize}[align=left]
              \Invs{Let $a\in \Z$ but $a\neq 1$ and $a\neq 1$,
                $\inv{a}\notin \Z$.}
            \end{itemize}
            }
      \item[(e)]{
            \begin{itemize}[align=left]
              \Assoc{\begin{align*}
                  \lrp{a\ast b}\ast c & =\lrp{ab}\ast c      \\
                                      & =\lrp{ab}c           \\
                                      & =a\lrp{bc}           \\
                                      & =a\ast \lrp{b\ast c}
                \end{align*}}
              \Invs{Select $a\in \R$, $$a\ast a = a\frac{1}{a} = 1$$}
              \Clos{Obvious from closure of $\lrp{\R,\cdot}$}
              \Ident{1 is the identity, Obvious from $\lrp{\R,\cdot}$}
            \end{itemize}
            }
      \item[(f)]{
            \begin{itemize}[align=left]
              \Assoc{\begin{align*}
                  \lrp{a\ast b}\ast c & =\lrp{ab}\ast c      \\
                                      & =\lrp{ab}c           \\
                                      & =a\lrp{bc}           \\
                                      & =a\ast \lrp{b\ast c}
                \end{align*}}
              \Invs{Select $a\in \Q$, $$a\ast \inv{a} = a\frac{1}{a} = 1$$}
              \Clos{$a,b\in \Q$, $a=\frac{p_1}{q_1}$ $b=\frac{p_2}{q_2}$,
                ${p_1},{q_1},{p_2},{q_2}\in \Z$, $q_1\neq0\neq q_2$.
                \begin{align*}
                  a\ast b & =ab
                          & =                                 \\
                          & =\frac{p_1{p_2}}{q_1{q_2}} \in \Q
                \end{align*}}
              \Ident{1 is the identity, Obvious from $\lrp{\R,\cdot}$}
            \end{itemize}
            }
    \end{itemize}
  \end{multicols}
\end{mdframed}
\newpage
%%%%%%%%%%%%%%%%%%%%%%%%%%%%%%%%%%%%%%%%%%%%%%%%%%%%%%%%%%%%%%%%%%%%%%%%%%%%%%%%
%%%%%%%%%%%%%%%%%%%%%%%%%%%%%%%%%%%%%%%%%%%%%%%%%%%%%%%%%%%%%%%%%%%%%%%%%%%%%%%%
%%%%%%%%%%%%%%%%%%%%%%%%%%%%%%%%%%%%%%%%%%%%%%%%%%%%%%%%%%%%%%%%%%%%%%%%%%%%%%%%
%%%%%%%%%%%%%%%%%%%%%%%%%%%%%%%%%%%%%%%%%%%%%%%%%%%%%%%%%%%%%%%%%%%%%%%%%%%%%%%%
\begin{mdframed}[style=darkQuesion]
  3. Let $\grp{G}{\cdot}$ be a group. Define a new binary operation $\ast$ on
  $G$ by the formula $a \ast b=b\cot a$, for all $a,b\in G$.
  \begin{itemize}
    \item[(a)]{Show that $\grp{G}{\cdot}$ is a group.}
    \item[(b)]{Give examples to show that $\grp{G}{\cdot}$ may or may not be the
          same as $\grp{G}{\ast}$.}
  \end{itemize}
\end{mdframed}

%%%%%%%%%%%%%%%%%%%%%%%%%%%%%%%%%%%%%%%%%%%%%%%%%%%%%%%%%%%%%%%%%%%%%%%%%%%%%%%%
\begin{mdframed}[style=darkAnswer,frametitle={Joe Starr}]
  \begin{itemize}
    \item[(a)]{
          \begin{itemize}[align=left]
            \Assoc{\begin{align*}
                \lrp{a\ast b}\ast c & =\lrp{b\cdot a}\ast c     \\
                                    & =c \cdot  \lrp{b \cdot a} \\
                                    & =\lrp{c \cdot  b} \cdot a \\
                                    & =\lrp{b\ast c} \cdot a    \\
                                    & =a\ast \lrp{b\ast c}
              \end{align*}}
            \Invs{Let $a\in G$ since $\grp{G}{\cdot}$ is a group we know
              $\inv{a}\in G$. Now $a\ast \inv{a}=\inv{a}\cdot a=1$.
            }
            \Clos{We select $a,b\in G$ consier $a\ast b=b\cdot a$ by closure of
              $\grp{G}{\cdot}$, $a\ast b\in G$.
            }
            \Ident{Let $a\in G$, consider $1\ast a = a \cdot 1 = a$ and
              $a\ast 1 = 1 \cdot a = a$.}
          \end{itemize}
          }
    \item[(b)]{
          Let $G=\begin{tabular}{|c|c|c|c|c|c|c|}
              \hline
              $\cdot$ & e & a & b & c & d & f \\
              \hline
              a       & a & e & d & f & b & c \\
              \hline
              b       & b & f & e & d & c & a \\
              \hline
              c       & c & d & f & e & a & b \\
              \hline
              d       & d & c & a & b & f & e \\
              \hline
              f       & f & b & c & a & e & d \\
              \hline
            \end{tabular}$ so
          $a\ast b=b\cdot a = f$ but $a\cdot b=d$. In this case they are not equal.

          If we let $G= \grp{\Z}{+}$, we have $a\ast b = b+a=a+b$. In this case they
          are equal.
          }
  \end{itemize}
\end{mdframed}
\newpage
%%%%%%%%%%%%%%%%%%%%%%%%%%%%%%%%%%%%%%%%%%%%%%%%%%%%%%%%%%%%%%%%%%%%%%%%%%%%%%%%
%%%%%%%%%%%%%%%%%%%%%%%%%%%%%%%%%%%%%%%%%%%%%%%%%%%%%%%%%%%%%%%%%%%%%%%%%%%%%%%%
%%%%%%%%%%%%%%%%%%%%%%%%%%%%%%%%%%%%%%%%%%%%%%%%%%%%%%%%%%%%%%%%%%%%%%%%%%%%%%%%
%%%%%%%%%%%%%%%%%%%%%%%%%%%%%%%%%%%%%%%%%%%%%%%%%%%%%%%%%%%%%%%%%%%%%%%%%%%%%%%%
\begin{mdframed}[style=darkQuesion]
  5. Is $\GL{n}{\R}$ an Abelian group? Support your answer by either proof or a
  counter example.
\end{mdframed}

%%%%%%%%%%%%%%%%%%%%%%%%%%%%%%%%%%%%%%%%%%%%%%%%%%%%%%%%%%%%%%%%%%%%%%%%%%%%%%%%
\begin{mdframed}[style=darkAnswer,frametitle={Joe Starr}]
  No, select $$A=\begin{bmatrix}
      2 & 3 \\
      5 & 7 \\
    \end{bmatrix}
    B=\begin{bmatrix}
      11 & 13 \\
      17 & 19 \\
    \end{bmatrix}$$
  we calculate
  $$AB=\begin{bmatrix}
      73  & 83  \\
      174 & 198 \\
    \end{bmatrix}
    BA=\begin{bmatrix}
      87  & 124 \\
      129 & 184 \\
    \end{bmatrix}$$
\end{mdframed}
\newpage
%%%%%%%%%%%%%%%%%%%%%%%%%%%%%%%%%%%%%%%%%%%%%%%%%%%%%%%%%%%%%%%%%%%%%%%%%%%%%%%%
%%%%%%%%%%%%%%%%%%%%%%%%%%%%%%%%%%%%%%%%%%%%%%%%%%%%%%%%%%%%%%%%%%%%%%%%%%%%%%%%
%%%%%%%%%%%%%%%%%%%%%%%%%%%%%%%%%%%%%%%%%%%%%%%%%%%%%%%%%%%%%%%%%%%%%%%%%%%%%%%%
%%%%%%%%%%%%%%%%%%%%%%%%%%%%%%%%%%%%%%%%%%%%%%%%%%%%%%%%%%%%%%%%%%%%%%%%%%%%%%%%
\begin{mdframed}[style=darkQuesion]
  8. Write out the multiplication table for the following set of matrices over
  $\Q$:
  $$\begin{bmatrix}
      1 & 0 \\
      0 & 1 \\
    \end{bmatrix},
    \begin{bmatrix}
      \m1 & 0 \\
      0   & 1 \\
    \end{bmatrix},
    \begin{bmatrix}
      1 & 0   \\
      0 & \m1 \\
    \end{bmatrix},
    \begin{bmatrix}
      \m1 & 0   \\
      0   & \m1 \\
    \end{bmatrix}$$


\end{mdframed}

%%%%%%%%%%%%%%%%%%%%%%%%%%%%%%%%%%%%%%%%%%%%%%%%%%%%%%%%%%%%%%%%%%%%%%%%%%%%%%%%
\begin{mdframed}[style=darkAnswer,frametitle={Joe Starr}]
  Let $$i=\begin{bmatrix}
      1 & 0 \\
      0 & 1 \\
    \end{bmatrix},
    j=\begin{bmatrix}
      \m1 & 0 \\
      0   & 1 \\
    \end{bmatrix},
    k=\begin{bmatrix}
      1 & 0   \\
      0 & \m1 \\
    \end{bmatrix},
    l=\begin{bmatrix}
      \m1 & 0   \\
      0   & \m1 \\
    \end{bmatrix}$$
  $$\begin{tabular}{|c|c|c|c|c|}
      \hline
      $\cdot$ & i & j & k & l \\
      \hline
      i       & i & j & k & l \\
      \hline
      j       & j & i & l & k \\
      \hline
      k       & k & l & i & j \\
      \hline
      l       & l & k & j & i \\
      \hline
    \end{tabular}$$
\end{mdframed}
\newpage
%%%%%%%%%%%%%%%%%%%%%%%%%%%%%%%%%%%%%%%%%%%%%%%%%%%%%%%%%%%%%%%%%%%%%%%%%%%%%%%%
%%%%%%%%%%%%%%%%%%%%%%%%%%%%%%%%%%%%%%%%%%%%%%%%%%%%%%%%%%%%%%%%%%%%%%%%%%%%%%%%
%%%%%%%%%%%%%%%%%%%%%%%%%%%%%%%%%%%%%%%%%%%%%%%%%%%%%%%%%%%%%%%%%%%%%%%%%%%%%%%%
%%%%%%%%%%%%%%%%%%%%%%%%%%%%%%%%%%%%%%%%%%%%%%%%%%%%%%%%%%%%%%%%%%%%%%%%%%%%%%%%
\begin{mdframed}[style=darkQuesion]
  9. Let $G=\lrs{x\in \R \vert x>0 \text{ and } x\neq 1}$. Define the operation
  $\ast$ on $G$ by $a\ast b=a^{\ln{b}}$, for all $a,b\in G$. Prove that $G$ is
  an Abelian group under the operation $\ast$.
\end{mdframed}

%%%%%%%%%%%%%%%%%%%%%%%%%%%%%%%%%%%%%%%%%%%%%%%%%%%%%%%%%%%%%%%%%%%%%%%%%%%%%%%%
\begin{mdframed}[style=darkAnswer,frametitle={Joe Starr}]
  \begin{itemize}[align=left]
    \Assoc{\begin{align*}
        \lrp{a\ast b}\ast c & =\lrp{a^{\ln{b}}}\ast c    \\
                            & =\lrp{a^{\ln{b}}}^{\ln{c}} \\
                            & ={a^{\ln{b}\ln{c}}}        \\
                            & =a^{\ln{b^{\ln{c}}}}       \\
                            & =a\ast \lrp{b\ast c}
      \end{align*}}
    \Invs{
    Let $a\in G$, consider $\inv{a}=e^{\frac{1}{\ln{a}}}$.
    \begin{multicols}{2}
      \begin{align*}
        a\ast\inv{a} & = a ^{\ln{\inv{a}}}             \\
                     & = a^{\ln{e^{\frac{1}{\ln{a}}}}} \\
                     & = a^{{{\frac{1}{\ln{a}}}}}      \\
                     & = a^{{{\log_{a}{e}}}}           \\
                     & = {e}                           \\
      \end{align*}
      \begin{align*}
        \inv{a}\ast a & = \lrp{\inv{a}} ^{\ln{a}}              \\
                      & = \lrp{e^{\frac{1}{\ln{a}}}} ^{\ln{a}} \\
                      & = \lrp{e^{\log_{a}{e}}} ^{\ln{a}}      \\
                      & = \lrp{e^{\ln{a}} } ^{\log_{a}{e}}     \\
                      & = a ^{\log_{a}{e}}                     \\
                      & = {e}                                  \\
      \end{align*}
    \end{multicols}
    }
    \Clos{Let $a,b\in G$, $a\ast b = a^{\ln{b}}$ we know that $b>0$ so
      $\ln{b}$ exists, further since $1\notin G$ we have $\ln{b}\neq 0$.
      We observe that for any $a\in G$ $a^{\ln{b}}>0$ since $a>0$ and since
      $\ln{b}\neq 0$ $a^{\ln{b}}\neq 1$.}
    \Ident{Our conjecture is that $e$ is the identity element.
      Let $a\in G$, $e\ast a= e^{\ln{a}}=a$ and $a\ast e= a^{\ln{e}}=a$}
  \end{itemize}
\end{mdframed}
\newpage
%%%%%%%%%%%%%%%%%%%%%%%%%%%%%%%%%%%%%%%%%%%%%%%%%%%%%%%%%%%%%%%%%%%%%%%%%%%%%%%%
%%%%%%%%%%%%%%%%%%%%%%%%%%%%%%%%%%%%%%%%%%%%%%%%%%%%%%%%%%%%%%%%%%%%%%%%%%%%%%%%
%%%%%%%%%%%%%%%%%%%%%%%%%%%%%%%%%%%%%%%%%%%%%%%%%%%%%%%%%%%%%%%%%%%%%%%%%%%%%%%%
%%%%%%%%%%%%%%%%%%%%%%%%%%%%%%%%%%%%%%%%%%%%%%%%%%%%%%%%%%%%%%%%%%%%%%%%%%%%%%%%
\begin{mdframed}[style=darkQuesion]
  10. Show that the set $A=\lrs{f_{m,b}:\R \to \R\vert m\neq 0 \text{ and } f_{m,b}=mx+b}$.
  of affine functions from $\R$ to $\R$ forms a group under function composition.
\end{mdframed}

%%%%%%%%%%%%%%%%%%%%%%%%%%%%%%%%%%%%%%%%%%%%%%%%%%%%%%%%%%%%%%%%%%%%%%%%%%%%%%%%
\begin{mdframed}[style=darkAnswer,frametitle={Joe Starr}]
  \begin{itemize}[align=left]
    \Assoc{We've proved this previously. }
    \Invs{Let $f\in A$, $\fof{x}=mx+b$. Consider $I\lrp{x}=\frac{1}{m}\lrp{x-b}$
      \begin{multicols}{2}
        \begin{align*}
          \fof{I\lrp{x}} & = m\lrp{\frac{1}{m}\lrp{x-b}}+b \\
                         & = x-b+b                         \\
                         & = x                             \\
        \end{align*}
        \begin{align*}
          I\lrp{\fof{x}} & = \frac{1}{m}\lrp{\lrp{mx+b}-b} \\
                         & = \frac{1}{m}\lrp{mx}           \\
                         & = x                             \\
        \end{align*}
      \end{multicols}
    }
    \Clos{Let $f,g\in A$, so $\fof{x}=m_1x+b_1$ and $\gof{x}=m_2x+b_2$.
      Now composing $f$ and $g$ $\fof{\gof{x}}$.
      \begin{align*}
        \fof{\gof{x}} & = m_1\lrp{m_2x+b_2}+b_1 \\
                      & = {m_1m_2x+m_1b_2}+b_1  \\
                      & = mx+m_1b_2+b_1         \\
                      & = mx+b                  \\
      \end{align*}
    }
    \Ident{Let $f\in A$, $\fof{x}=mx+b$. Conjecture $e\lrp{x}=x$
      \begin{multicols}{2}
        \begin{align*}
          \fof{e\lrp{x}} & = m\lrp{x}+b \\
                         & = mx+b       \\
        \end{align*}
        \vfill
        \columnbreak
        \begin{align*}
          e\lrp{\fof{x}} & = mx+b \\
        \end{align*}
      \end{multicols}
    }

  \end{itemize}
\end{mdframed}
\newpage
%%%%%%%%%%%%%%%%%%%%%%%%%%%%%%%%%%%%%%%%%%%%%%%%%%%%%%%%%%%%%%%%%%%%%%%%%%%%%%%%
%%%%%%%%%%%%%%%%%%%%%%%%%%%%%%%%%%%%%%%%%%%%%%%%%%%%%%%%%%%%%%%%%%%%%%%%%%%%%%%%
%%%%%%%%%%%%%%%%%%%%%%%%%%%%%%%%%%%%%%%%%%%%%%%%%%%%%%%%%%%%%%%%%%%%%%%%%%%%%%%%
%%%%%%%%%%%%%%%%%%%%%%%%%%%%%%%%%%%%%%%%%%%%%%%%%%%%%%%%%%%%%%%%%%%%%%%%%%%%%%%%
\begin{mdframed}[style=darkQuesion]
  11. Show that the set of all $2\times 2$ matrices over $\R$ of the form
  $\begin{bmatrix}
      m & b \\
      0 & 1 \\
    \end{bmatrix}$ with $m\neq 0$ forms a group under matrix multiplication.
\end{mdframed}

%%%%%%%%%%%%%%%%%%%%%%%%%%%%%%%%%%%%%%%%%%%%%%%%%%%%%%%%%%%%%%%%%%%%%%%%%%%%%%%%
\begin{mdframed}[style=darkAnswer,frametitle={Joe Starr}]
  Let $G$ be the set of all $2\times 2$ matrices over $\R$ of the form
  $\begin{bmatrix}
      m & b \\
      0 & 1 \\
    \end{bmatrix}$ with $m\neq 0$.
  \begin{itemize}[align=left]
    \Assoc{Free from $\Mn{2}{\R}$}
    \Invs{Let $a\in G$, so $a=\begin{bmatrix}
          m & b \\
          0 & 1 \\
        \end{bmatrix}$ we can calculate the determinate of $a$. $m1-b0=m$ and by
      definition of the set $m\neq 0$. So we have inverses.
    }
    \Clos{Let $a,b\in G$, so $a=\begin{bmatrix}
          m_1 & b_1 \\
          0   & 1   \\
        \end{bmatrix}$ and $b=\begin{bmatrix}
          m_2 & b_2 \\
          0   & 1   \\
        \end{bmatrix}$
      \begin{align*}
        ab & = \begin{bmatrix}
          m_1 & b_1 \\
          0   & 1   \\
        \end{bmatrix} \begin{bmatrix}
          m_2 & b_2 \\
          0   & 1   \\
        \end{bmatrix} \\
           & = \begin{bmatrix}
          m_2m_1 & b_1+m_1b_2 \\
          0      & 1          \\
        \end{bmatrix}                            \\
      \end{align*}
    }
    \Ident{Free from $\Mn{2}{\R}$
    }

  \end{itemize}
\end{mdframed}
\newpage
%%%%%%%%%%%%%%%%%%%%%%%%%%%%%%%%%%%%%%%%%%%%%%%%%%%%%%%%%%%%%%%%%%%%%%%%%%%%%%%%
%%%%%%%%%%%%%%%%%%%%%%%%%%%%%%%%%%%%%%%%%%%%%%%%%%%%%%%%%%%%%%%%%%%%%%%%%%%%%%%%
%%%%%%%%%%%%%%%%%%%%%%%%%%%%%%%%%%%%%%%%%%%%%%%%%%%%%%%%%%%%%%%%%%%%%%%%%%%%%%%%
%%%%%%%%%%%%%%%%%%%%%%%%%%%%%%%%%%%%%%%%%%%%%%%%%%%%%%%%%%%%%%%%%%%%%%%%%%%%%%%%
\begin{mdframed}[style=darkQuesion]
  12. In the group defined in question 11 find all elements that commute with
  $\begin{bmatrix}
      2 & 0 \\
      0 & 1 \\
    \end{bmatrix}$
\end{mdframed}

%%%%%%%%%%%%%%%%%%%%%%%%%%%%%%%%%%%%%%%%%%%%%%%%%%%%%%%%%%%%%%%%%%%%%%%%%%%%%%%%
\begin{mdframed}[style=darkAnswer,frametitle={Joe Starr}]
  We can begin by letting $a\in G$ calculating $a\begin{bmatrix}
      2 & 0 \\
      0 & 1 \\
    \end{bmatrix}$ and $\begin{bmatrix}
      2 & 0 \\
      0 & 1 \\
    \end{bmatrix}a$.
  \begin{multicols}{2}
    \begin{align*}
      a\begin{bmatrix}
        2 & 0 \\
        0 & 1 \\
      \end{bmatrix} & = \begin{bmatrix}
        m & b \\
        0 & 1 \\
      \end{bmatrix} \begin{bmatrix}
        2 & 0 \\
        0 & 1 \\
      \end{bmatrix} \\
                                  & = \begin{bmatrix}
        2m & b \\
        0  & 1 \\
      \end{bmatrix}                            \\
    \end{align*}
    \begin{align*}
      \begin{bmatrix}
        2 & 0 \\
        0 & 1 \\
      \end{bmatrix}a & =  \begin{bmatrix}
        2 & 0 \\
        0 & 1 \\
      \end{bmatrix} \begin{bmatrix}
        m & b \\
        0 & 1 \\
      \end{bmatrix} \\
                                  & = \begin{bmatrix}
        2m & 2b \\
        0  & 1  \\
      \end{bmatrix}                             \\
    \end{align*}
  \end{multicols}
  So for a matrix of to commute with $\begin{bmatrix}
      2 & 0 \\
      0 & 1 \\
    \end{bmatrix}$ it must be of the form $\begin{bmatrix}
      m & 0 \\
      0 & 1 \\
    \end{bmatrix}$.
\end{mdframed}
\newpage
%%%%%%%%%%%%%%%%%%%%%%%%%%%%%%%%%%%%%%%%%%%%%%%%%%%%%%%%%%%%%%%%%%%%%%%%%%%%%%%%
%%%%%%%%%%%%%%%%%%%%%%%%%%%%%%%%%%%%%%%%%%%%%%%%%%%%%%%%%%%%%%%%%%%%%%%%%%%%%%%%
%%%%%%%%%%%%%%%%%%%%%%%%%%%%%%%%%%%%%%%%%%%%%%%%%%%%%%%%%%%%%%%%%%%%%%%%%%%%%%%%
%%%%%%%%%%%%%%%%%%%%%%%%%%%%%%%%%%%%%%%%%%%%%%%%%%%%%%%%%%%%%%%%%%%%%%%%%%%%%%%%
\begin{mdframed}[style=darkQuesion]
  13. Define $\ast$ on $\R$ by $a\ast b = a+b-1$, for all $a,b\in \R$. Show that
  $\grp{\R}{\ast}$ is an Abelian group.
\end{mdframed}

%%%%%%%%%%%%%%%%%%%%%%%%%%%%%%%%%%%%%%%%%%%%%%%%%%%%%%%%%%%%%%%%%%%%%%%%%%%%%%%%
\begin{mdframed}[style=darkAnswer,frametitle={Joe Starr}]
  \begin{itemize}[align=left]
    \item[]{
          \begin{multicols}{2}
            \begin{itemize}[align=left]
              \Abel{
                \begin{align*}
                  {a\ast b} & =a+b-1   \\
                            & = b+a-1  \\
                            & =b\ast a
                \end{align*}

              }
              \Assoc{\begin{align*}
                  \lrp{a\ast b}\ast c & =\lrp{a+b-1}\ast c   \\
                                      & =\lrp{a+b-1}+c-1     \\
                                      & =a+b+c-1-1           \\
                                      & =a+\lrp{b+c-1}-1     \\
                                      & =a\ast \lrp{b\ast c}
                \end{align*}}
            \end{itemize}
          \end{multicols}
          }
          \Invs{ Let $a\in \grp{\R}{\ast} $, consider $\inv{a}=2-a$
            \begin{multicols}{2}
              \begin{align*}
                a\ast \inv{a} & = a+\lrp{2-a}-1 \\
                              & = 1             \\
              \end{align*}
              \vfill
              \columnbreak
              \begin{align*}
                \inv{a}\ast a & = \lrp{2-a}+a-1 \\
                              & = 1             \\
              \end{align*}
            \end{multicols}
          }
          \Clos{Obvious from closure of $\grp{\R}{+}$.
          }
          \Ident{ Conjecture is that $1$ is the identity element of $\grp{\R}{\ast}$.
            \begin{multicols}{2}
              \begin{align*}
                a\ast 1 & = a+1-1 \\
                        & = a     \\
              \end{align*}
              \vfill
              \columnbreak
              \begin{align*}
                1\ast a & = a+1-1 \\
                        & = a     \\
              \end{align*}
            \end{multicols}
          }
  \end{itemize}
\end{mdframed}
\begin{mdframed}[style=darkAnswer,frametitle={Joe Starr}]
  Let $\varphi:\grp{\R}{\ast}\to\grp{\R}{+}$, with $\pof{x}=x-1$,
  $\pof{a\ast b}= \lrp{a+b-1}-1=a-1+b-1=\pof{a}+\pof{b}$. Further
  $\inv{\varphi}\lrp{x}=x+1$, $\pof{\inv{\varphi}\lrp{x}}=\lrp{x+1}-1=x$.
  Showing a group structure isomorphic to $\grp{\R}{+}$.
\end{mdframed}
\newpage
%%%%%%%%%%%%%%%%%%%%%%%%%%%%%%%%%%%%%%%%%%%%%%%%%%%%%%%%%%%%%%%%%%%%%%%%%%%%%%%%
%%%%%%%%%%%%%%%%%%%%%%%%%%%%%%%%%%%%%%%%%%%%%%%%%%%%%%%%%%%%%%%%%%%%%%%%%%%%%%%%
%%%%%%%%%%%%%%%%%%%%%%%%%%%%%%%%%%%%%%%%%%%%%%%%%%%%%%%%%%%%%%%%%%%%%%%%%%%%%%%%
%%%%%%%%%%%%%%%%%%%%%%%%%%%%%%%%%%%%%%%%%%%%%%%%%%%%%%%%%%%%%%%%%%%%%%%%%%%%%%%%
\begin{mdframed}[style=darkQuesion]
  14. Let $S= \R - \lrs{\m1}$. Define $\ast$ on $S$ by $a\ast b=a+b+ab$ for all
  $a,b \in S$. Show that $\grp{S}{\ast}$ is an Abelian group.
\end{mdframed}

%%%%%%%%%%%%%%%%%%%%%%%%%%%%%%%%%%%%%%%%%%%%%%%%%%%%%%%%%%%%%%%%%%%%%%%%%%%%%%%%
\begin{mdframed}[style=darkAnswer,frametitle={Joe Starr}]
  \begin{itemize}[align=left]

    \item[] {
          \begin{multicols}{2}
            \begin{itemize}[align=left]

              \Abel{
                \begin{align*}
                  {a\ast b} & =a+b+ab  \\
                            & = b+a+ba \\
                            & =b\ast a
                \end{align*}
              }
              \Invs{ Consider $\inv{a}=\frac{\m a}{a+1}$
                \begin{align*}
                  a\ast \inv{a} & = a+\frac{\m a}{a+1}+a\frac{\m a}{a+1} \\
                                & = a+\frac{\m a\lrp{a+1}}{a+1}          \\
                                & = a+\m a                               \\
                                & = 0                                    \\
                \end{align*}
                \vfill
                \columnbreak
              }
              \Clos{Let $a,b\in \R$, if we take $a\ast b =a+b+ab$. Assume that
                $a\ast b=\m1$
                \begin{align*}
                  \m1=a+b+ab & \Rightarrow  \m1-a=b+ab          \\
                             & \Rightarrow  \m1-a=b\lrp{1+a}    \\
                             & \Rightarrow  \m\frac{a+1}{1+a}=b \\
                             & \Rightarrow  \m1=b               \\
                \end{align*}
                a contradiction.
              }
            \end{itemize}
          \end{multicols}
          }
          \Ident{ Conjecture is that $0$ is the identity element of $\grp{S}{\ast}$.
            \begin{align*}
              a\ast 0 & = a+0+a0 \\
                      & = a      \\
            \end{align*}
          }
          \Assoc{\begin{align*}
              \lrp{a\ast b}\ast c & =\lrp{a+b+ab}\ast c            \\
                                  & = \lrp{a+b+ab}+c+\lrp{a+b+ab}c \\
                                  & = \lrp{a+b+ab}+c+\lrp{a+b+ab}c \\
                                  & = \lrp{a+b+ab}+c+\lrp{a+b+ab}c \\
                                  & = a+b+ab+c+ca+cb+cab           \\
                                  & = a+\lrp{b+c+bc}+a\lrp{b+c+bc} \\
                                  & =a\ast\lrp{b+c+bc}             \\
                                  & =a\ast \lrp{b\ast c}
            \end{align*}}

  \end{itemize}
\end{mdframed}
\newpage
%%%%%%%%%%%%%%%%%%%%%%%%%%%%%%%%%%%%%%%%%%%%%%%%%%%%%%%%%%%%%%%%%%%%%%%%%%%%%%%%
%%%%%%%%%%%%%%%%%%%%%%%%%%%%%%%%%%%%%%%%%%%%%%%%%%%%%%%%%%%%%%%%%%%%%%%%%%%%%%%%
%%%%%%%%%%%%%%%%%%%%%%%%%%%%%%%%%%%%%%%%%%%%%%%%%%%%%%%%%%%%%%%%%%%%%%%%%%%%%%%%
%%%%%%%%%%%%%%%%%%%%%%%%%%%%%%%%%%%%%%%%%%%%%%%%%%%%%%%%%%%%%%%%%%%%%%%%%%%%%%%%
\begin{mdframed}[style=darkQuesion]
  15. Let $G=\lrs{x\in \R\vert x>1}$. Define $\ast$ on $G$ by $a\ast b = ab-a-b+2$,
  for all $a,b\in G$. Show that $\grp{G}{\ast}$ is an Abelian group. 
\end{mdframed}

%%%%%%%%%%%%%%%%%%%%%%%%%%%%%%%%%%%%%%%%%%%%%%%%%%%%%%%%%%%%%%%%%%%%%%%%%%%%%%%%
\begin{mdframed}[style=darkAnswer,frametitle={Joe Starr}]
  \begin{itemize}[align=left]

    \item[] {
          \begin{multicols}{2}
            \begin{itemize}[align=left]
              \Abel{
                \begin{align*}
                  {a\ast b} & =ab-a-b+2  \\
                            & = ba-b-a+2 \\
                            & =b\ast a
                \end{align*}
              }
              \Invs{ Consider $\inv{a}=\frac{a}{a-1}$
                \begin{align*}
                  a\ast \inv{a} & = a\frac{a}{a-1}-a-\frac{a}{a-1}+2 \\
                  & = \frac{a}{a-1}\lrp{a-1}-a+2 \\
                  & = a-a+2 \\
                                & = 2                                    \\
                \end{align*}
                \vfill
                \columnbreak
              }
              \Clos{let $a,b\in G$, consider $a\ast b=ab-a-b+2$ we are left with
              th
              }
            \end{itemize}
          \end{multicols}
          }
          \Ident{ Conjecture is that $2$ is the identity element of $\grp{G}{\ast}$.
            \begin{align*}
              a\ast 2 & = a2-a-2+2 \\
                      & = a      \\
            \end{align*}
          }
          \Assoc{\begin{align*}
              \lrp{a\ast b}\ast c & =\lrp{a+b+ab}\ast c            \\
                                  & = \lrp{a+b+ab}+c+\lrp{a+b+ab}c \\
                                  & = \lrp{a+b+ab}+c+\lrp{a+b+ab}c \\
                                  & = \lrp{a+b+ab}+c+\lrp{a+b+ab}c \\
                                  & = a+b+ab+c+ca+cb+cab           \\
                                  & = a+\lrp{b+c+bc}+a\lrp{b+c+bc} \\
                                  & =a\ast\lrp{b+c+bc}             \\
                                  & =a\ast \lrp{b\ast c}
            \end{align*}}

  \end{itemize}
\end{mdframed}
\newpage
%%%%%%%%%%%%%%%%%%%%%%%%%%%%%%%%%%%%%%%%%%%%%%%%%%%%%%%%%%%%%%%%%%%%%%%%%%%%%%%%
%%%%%%%%%%%%%%%%%%%%%%%%%%%%%%%%%%%%%%%%%%%%%%%%%%%%%%%%%%%%%%%%%%%%%%%%%%%%%%%%
%%%%%%%%%%%%%%%%%%%%%%%%%%%%%%%%%%%%%%%%%%%%%%%%%%%%%%%%%%%%%%%%%%%%%%%%%%%%%%%%
%%%%%%%%%%%%%%%%%%%%%%%%%%%%%%%%%%%%%%%%%%%%%%%%%%%%%%%%%%%%%%%%%%%%%%%%%%%%%%%%
\begin{mdframed}[style=darkQuesion]
  16. Let $G$ be a group. We have shown that $\inv{\lrp{ab}}=\inv{b}\inv{a}$. 
  Find a similar expression for $\lrp{\inv{abc}}$
\end{mdframed}

%%%%%%%%%%%%%%%%%%%%%%%%%%%%%%%%%%%%%%%%%%%%%%%%%%%%%%%%%%%%%%%%%%%%%%%%%%%%%%%%
\begin{mdframed}[style=darkAnswer,frametitle={Joe Starr}]
We will use a transitive proof: 
\begin{align*}
  \inv{\lrp{abc}} &= \inv{c}\inv{\lrp{ab}}\\
  &= \inv{c}\inv{b}\inv{a}
\end{align*}
\end{mdframed}
\newpage
%%%%%%%%%%%%%%%%%%%%%%%%%%%%%%%%%%%%%%%%%%%%%%%%%%%%%%%%%%%%%%%%%%%%%%%%%%%%%%%%
%%%%%%%%%%%%%%%%%%%%%%%%%%%%%%%%%%%%%%%%%%%%%%%%%%%%%%%%%%%%%%%%%%%%%%%%%%%%%%%%
%%%%%%%%%%%%%%%%%%%%%%%%%%%%%%%%%%%%%%%%%%%%%%%%%%%%%%%%%%%%%%%%%%%%%%%%%%%%%%%%
%%%%%%%%%%%%%%%%%%%%%%%%%%%%%%%%%%%%%%%%%%%%%%%%%%%%%%%%%%%%%%%%%%%%%%%%%%%%%%%%
\begin{mdframed}[style=darkQuesion]
  17. Let $G$ be a group. If $g\in G$ and $g^2=g$,then prove that $g=e$.
\end{mdframed}

%%%%%%%%%%%%%%%%%%%%%%%%%%%%%%%%%%%%%%%%%%%%%%%%%%%%%%%%%%%%%%%%%%%%%%%%%%%%%%%%
\begin{mdframed}[style=darkAnswer,frametitle={Joe Starr}]
We begin with letting $g\in G$, such $g^2=g$ we then multiply by $\inv{g}$ on 
the left:
\begin{align*}
  g^2=g &\rightarrow \inv{g}g^2=\inv{g}g\\
  &\rightarrow g=e
\end{align*}
as desired. 
\end{mdframed}
\newpage
%%%%%%%%%%%%%%%%%%%%%%%%%%%%%%%%%%%%%%%%%%%%%%%%%%%%%%%%%%%%%%%%%%%%%%%%%%%%%%%%
%%%%%%%%%%%%%%%%%%%%%%%%%%%%%%%%%%%%%%%%%%%%%%%%%%%%%%%%%%%%%%%%%%%%%%%%%%%%%%%%
%%%%%%%%%%%%%%%%%%%%%%%%%%%%%%%%%%%%%%%%%%%%%%%%%%%%%%%%%%%%%%%%%%%%%%%%%%%%%%%%
%%%%%%%%%%%%%%%%%%%%%%%%%%%%%%%%%%%%%%%%%%%%%%%%%%%%%%%%%%%%%%%%%%%%%%%%%%%%%%%%
\begin{mdframed}[style=darkQuesion]
  18. Show that a nonabelian group must have at least 5 elements. 
\end{mdframed}

%%%%%%%%%%%%%%%%%%%%%%%%%%%%%%%%%%%%%%%%%%%%%%%%%%%%%%%%%%%%%%%%%%%%%%%%%%%%%%%%
\begin{mdframed}[style=darkAnswer,frametitle={Joe Starr}]
Let $G$ be a nonabelian group. Since $G$ a group then $e\in G$ the identity. $G$
can't be the trivial group since the trivial group is Abelian, this puts 
$a\in G$ with $a\neq e$ further $\inv{a}\in G$. With the same argument $G$ is 
not a group of three elements, so $b,\inv{b}\in G$. This puts 
$a,b,\inv{b},\inv{a},e\in G$ showing $G$ with at lest 5 elements.
\end{mdframed}
\newpage
%%%%%%%%%%%%%%%%%%%%%%%%%%%%%%%%%%%%%%%%%%%%%%%%%%%%%%%%%%%%%%%%%%%%%%%%%%%%%%%%
%%%%%%%%%%%%%%%%%%%%%%%%%%%%%%%%%%%%%%%%%%%%%%%%%%%%%%%%%%%%%%%%%%%%%%%%%%%%%%%%
%%%%%%%%%%%%%%%%%%%%%%%%%%%%%%%%%%%%%%%%%%%%%%%%%%%%%%%%%%%%%%%%%%%%%%%%%%%%%%%%
%%%%%%%%%%%%%%%%%%%%%%%%%%%%%%%%%%%%%%%%%%%%%%%%%%%%%%%%%%%%%%%%%%%%%%%%%%%%%%%%
\begin{mdframed}[style=darkQuesion]
  22.
\end{mdframed}

%%%%%%%%%%%%%%%%%%%%%%%%%%%%%%%%%%%%%%%%%%%%%%%%%%%%%%%%%%%%%%%%%%%%%%%%%%%%%%%%
\begin{mdframed}[style=darkAnswer,frametitle={Joe Starr}]

\end{mdframed}
\newpage
%%%%%%%%%%%%%%%%%%%%%%%%%%%%%%%%%%%%%%%%%%%%%%%%%%%%%%%%%%%%%%%%%%%%%%%%%%%%%%%%
%%%%%%%%%%%%%%%%%%%%%%%%%%%%%%%%%%%%%%%%%%%%%%%%%%%%%%%%%%%%%%%%%%%%%%%%%%%%%%%%
%%%%%%%%%%%%%%%%%%%%%%%%%%%%%%%%%%%%%%%%%%%%%%%%%%%%%%%%%%%%%%%%%%%%%%%%%%%%%%%%
%%%%%%%%%%%%%%%%%%%%%%%%%%%%%%%%%%%%%%%%%%%%%%%%%%%%%%%%%%%%%%%%%%%%%%%%%%%%%%%%
\begin{mdframed}[style=darkQuesion]
  24. Let $G$ be a group. Prove that $G$ is Abelian if and only if 
  $\inv{\lrp{ab}}=\inv{a}\inv{b}$ for all $a,b\in G$. 
\end{mdframed}

%%%%%%%%%%%%%%%%%%%%%%%%%%%%%%%%%%%%%%%%%%%%%%%%%%%%%%%%%%%%%%%%%%%%%%%%%%%%%%%%
\begin{mdframed}[style=darkAnswer,frametitle={Joe Starr}]
\twocase{
  Let $G$ be an abelian group and $a,b\in G$. Consider $\inv{\lrp{ab}}$, we have
  shown $\inv{\lrp{ab}}=\inv{b}\inv{a}$ since $G$ is abelian we have 
  $\inv{\lrp{ab}}=\inv{a}\inv{b}$.
}{
  Let $\inv{\lrp{ab}}=\inv{a}\inv{b}$, we have shown 
  $\inv{\lrp{ab}}=\inv{b}\inv{a}$ so $\inv{b}\inv{a}=\inv{a}\inv{b}$ showing $G$
  abelian.
}
\end{mdframed}
\newpage
%%%%%%%%%%%%%%%%%%%%%%%%%%%%%%%%%%%%%%%%%%%%%%%%%%%%%%%%%%%%%%%%%%%%%%%%%%%%%%%%
%%%%%%%%%%%%%%%%%%%%%%%%%%%%%%%%%%%%%%%%%%%%%%%%%%%%%%%%%%%%%%%%%%%%%%%%%%%%%%%%
%%%%%%%%%%%%%%%%%%%%%%%%%%%%%%%%%%%%%%%%%%%%%%%%%%%%%%%%%%%%%%%%%%%%%%%%%%%%%%%%
%%%%%%%%%%%%%%%%%%%%%%%%%%%%%%%%%%%%%%%%%%%%%%%%%%%%%%%%%%%%%%%%%%%%%%%%%%%%%%%%
\begin{mdframed}[style=darkQuesion]
  25. Let $G$ be a group. Prove that if $x^2=e$ for all $x\in G$, then $G$ is 
  abelian. 
\end{mdframed}

%%%%%%%%%%%%%%%%%%%%%%%%%%%%%%%%%%%%%%%%%%%%%%%%%%%%%%%%%%%%%%%%%%%%%%%%%%%%%%%%
\begin{mdframed}[style=darkAnswer,frametitle={Joe Starr}]
Let $G$ be a group with the given property $a,b\in G$. Observe that 
$a^2=e \Rightarrow a=\inv{a}$. We have shown that 
$\inv{\lrp{ab}}=\inv{b}\inv{a}$.
We proceed with a transitive proof: 
\begin{align*}
  \inv{\lrp{ab}}=\inv{b}\inv{a} &\rightarrow \lrp{ab}=\inv{b}\inv{a}\\
  &\rightarrow \lrp{ab}={b}{a}
\end{align*}
showing $G$ abelian as desired.
\end{mdframed}
\newpage
%%%%%%%%%%%%%%%%%%%%%%%%%%%%%%%%%%%%%%%%%%%%%%%%%%%%%%%%%%%%%%%%%%%%%%%%%%%%%%%%
%%%%%%%%%%%%%%%%%%%%%%%%%%%%%%%%%%%%%%%%%%%%%%%%%%%%%%%%%%%%%%%%%%%%%%%%%%%%%%%%
%%%%%%%%%%%%%%%%%%%%%%%%%%%%%%%%%%%%%%%%%%%%%%%%%%%%%%%%%%%%%%%%%%%%%%%%%%%%%%%%
%%%%%%%%%%%%%%%%%%%%%%%%%%%%%%%%%%%%%%%%%%%%%%%%%%%%%%%%%%%%%%%%%%%%%%%%%%%%%%%%
\begin{mdframed}[style=darkQuesion]
  26. Show that if $G$ is a finite group with an even number of elements, then 
  there must exist an element $a\in G$ with $a\neq e$ such that $a^2=e$. 
\end{mdframed}

%%%%%%%%%%%%%%%%%%%%%%%%%%%%%%%%%%%%%%%%%%%%%%%%%%%%%%%%%%%%%%%%%%%%%%%%%%%%%%%%
\begin{mdframed}[style=darkAnswer,frametitle={Joe Starr}]
Let $G$ be a group with the given property. Since $G$ a group $e\in G$. 
Observe $G$ is not the trivial group since it has even cardinality. 
If we consider the cardinality of $G/e$ it's $\abs{G}-1$ an odd number. 
Let $a\in G$ with $a\neq e$, observe that since $G$ a group $\inv{a}\in G$. 
We are left with two possibilities $a=\inv{a}$ or $a\neq \inv{a}$. 
If $a=\inv{a}$ we are done, otherwise we can delete $a$ and $\inv{a}$ from $G$ 
and select from the remaining elements of $G$. Since $G/e$ has odd cardinality 
we can repeat this process until there is a single element remaining. It must 
be that $a=\inv{a}$ as desired. 
\end{mdframed}
\newpage
\subsection{Subgroups}
%%%%%%%%%%%%%%%%%%%%%%%%%%%%%%%%%%%%%%%%%%%%%%%%%%%%%%%%%%%%%%%%%%%%%%%%%%%%%%%%
%%%%%%%%%%%%%%%%%%%%%%%%%%%%%%%%%%%%%%%%%%%%%%%%%%%%%%%%%%%%%%%%%%%%%%%%%%%%%%%%
%%%%%%%%%%%%%%%%%%%%%%%%%%%%%%%%%%%%%%%%%%%%%%%%%%%%%%%%%%%%%%%%%%%%%%%%%%%%%%%%
%%%%%%%%%%%%%%%%%%%%%%%%%%%%%%%%%%%%%%%%%%%%%%%%%%%%%%%%%%%%%%%%%%%%%%%%%%%%%%%%
\begin{mdframed}[style=darkQuesion]
  2.
\end{mdframed}

%%%%%%%%%%%%%%%%%%%%%%%%%%%%%%%%%%%%%%%%%%%%%%%%%%%%%%%%%%%%%%%%%%%%%%%%%%%%%%%%
\begin{mdframed}[style=darkAnswer,frametitle={Joe Starr}]

\end{mdframed}
\newpage
%%%%%%%%%%%%%%%%%%%%%%%%%%%%%%%%%%%%%%%%%%%%%%%%%%%%%%%%%%%%%%%%%%%%%%%%%%%%%%%%
%%%%%%%%%%%%%%%%%%%%%%%%%%%%%%%%%%%%%%%%%%%%%%%%%%%%%%%%%%%%%%%%%%%%%%%%%%%%%%%%
%%%%%%%%%%%%%%%%%%%%%%%%%%%%%%%%%%%%%%%%%%%%%%%%%%%%%%%%%%%%%%%%%%%%%%%%%%%%%%%%
%%%%%%%%%%%%%%%%%%%%%%%%%%%%%%%%%%%%%%%%%%%%%%%%%%%%%%%%%%%%%%%%%%%%%%%%%%%%%%%%
\begin{mdframed}[style=darkQuesion]
  2.
\end{mdframed}

%%%%%%%%%%%%%%%%%%%%%%%%%%%%%%%%%%%%%%%%%%%%%%%%%%%%%%%%%%%%%%%%%%%%%%%%%%%%%%%%
\begin{mdframed}[style=darkAnswer,frametitle={Joe Starr}]

\end{mdframed}
\newpage
%%%%%%%%%%%%%%%%%%%%%%%%%%%%%%%%%%%%%%%%%%%%%%%%%%%%%%%%%%%%%%%%%%%%%%%%%%%%%%%%
%%%%%%%%%%%%%%%%%%%%%%%%%%%%%%%%%%%%%%%%%%%%%%%%%%%%%%%%%%%%%%%%%%%%%%%%%%%%%%%%
%%%%%%%%%%%%%%%%%%%%%%%%%%%%%%%%%%%%%%%%%%%%%%%%%%%%%%%%%%%%%%%%%%%%%%%%%%%%%%%%
%%%%%%%%%%%%%%%%%%%%%%%%%%%%%%%%%%%%%%%%%%%%%%%%%%%%%%%%%%%%%%%%%%%%%%%%%%%%%%%%
\begin{mdframed}[style=darkQuesion]
  2.
\end{mdframed}

%%%%%%%%%%%%%%%%%%%%%%%%%%%%%%%%%%%%%%%%%%%%%%%%%%%%%%%%%%%%%%%%%%%%%%%%%%%%%%%%
\begin{mdframed}[style=darkAnswer,frametitle={Joe Starr}]

\end{mdframed}
\newpage
%%%%%%%%%%%%%%%%%%%%%%%%%%%%%%%%%%%%%%%%%%%%%%%%%%%%%%%%%%%%%%%%%%%%%%%%%%%%%%%%
%%%%%%%%%%%%%%%%%%%%%%%%%%%%%%%%%%%%%%%%%%%%%%%%%%%%%%%%%%%%%%%%%%%%%%%%%%%%%%%%
%%%%%%%%%%%%%%%%%%%%%%%%%%%%%%%%%%%%%%%%%%%%%%%%%%%%%%%%%%%%%%%%%%%%%%%%%%%%%%%%
%%%%%%%%%%%%%%%%%%%%%%%%%%%%%%%%%%%%%%%%%%%%%%%%%%%%%%%%%%%%%%%%%%%%%%%%%%%%%%%%
\begin{mdframed}[style=darkQuesion]
  2.
\end{mdframed}

%%%%%%%%%%%%%%%%%%%%%%%%%%%%%%%%%%%%%%%%%%%%%%%%%%%%%%%%%%%%%%%%%%%%%%%%%%%%%%%%
\begin{mdframed}[style=darkAnswer,frametitle={Joe Starr}]

\end{mdframed}
\newpage
%%%%%%%%%%%%%%%%%%%%%%%%%%%%%%%%%%%%%%%%%%%%%%%%%%%%%%%%%%%%%%%%%%%%%%%%%%%%%%%%
%%%%%%%%%%%%%%%%%%%%%%%%%%%%%%%%%%%%%%%%%%%%%%%%%%%%%%%%%%%%%%%%%%%%%%%%%%%%%%%%
%%%%%%%%%%%%%%%%%%%%%%%%%%%%%%%%%%%%%%%%%%%%%%%%%%%%%%%%%%%%%%%%%%%%%%%%%%%%%%%%
%%%%%%%%%%%%%%%%%%%%%%%%%%%%%%%%%%%%%%%%%%%%%%%%%%%%%%%%%%%%%%%%%%%%%%%%%%%%%%%%
\begin{mdframed}[style=darkQuesion]
  2.
\end{mdframed}

%%%%%%%%%%%%%%%%%%%%%%%%%%%%%%%%%%%%%%%%%%%%%%%%%%%%%%%%%%%%%%%%%%%%%%%%%%%%%%%%
\begin{mdframed}[style=darkAnswer,frametitle={Joe Starr}]

\end{mdframed}
\newpage
%%%%%%%%%%%%%%%%%%%%%%%%%%%%%%%%%%%%%%%%%%%%%%%%%%%%%%%%%%%%%%%%%%%%%%%%%%%%%%%%
%%%%%%%%%%%%%%%%%%%%%%%%%%%%%%%%%%%%%%%%%%%%%%%%%%%%%%%%%%%%%%%%%%%%%%%%%%%%%%%%
%%%%%%%%%%%%%%%%%%%%%%%%%%%%%%%%%%%%%%%%%%%%%%%%%%%%%%%%%%%%%%%%%%%%%%%%%%%%%%%%
%%%%%%%%%%%%%%%%%%%%%%%%%%%%%%%%%%%%%%%%%%%%%%%%%%%%%%%%%%%%%%%%%%%%%%%%%%%%%%%%
\begin{mdframed}[style=darkQuesion]
  2.
\end{mdframed}

%%%%%%%%%%%%%%%%%%%%%%%%%%%%%%%%%%%%%%%%%%%%%%%%%%%%%%%%%%%%%%%%%%%%%%%%%%%%%%%%
\begin{mdframed}[style=darkAnswer,frametitle={Joe Starr}]

\end{mdframed}
\newpage
%%%%%%%%%%%%%%%%%%%%%%%%%%%%%%%%%%%%%%%%%%%%%%%%%%%%%%%%%%%%%%%%%%%%%%%%%%%%%%%%
%%%%%%%%%%%%%%%%%%%%%%%%%%%%%%%%%%%%%%%%%%%%%%%%%%%%%%%%%%%%%%%%%%%%%%%%%%%%%%%%
%%%%%%%%%%%%%%%%%%%%%%%%%%%%%%%%%%%%%%%%%%%%%%%%%%%%%%%%%%%%%%%%%%%%%%%%%%%%%%%%
%%%%%%%%%%%%%%%%%%%%%%%%%%%%%%%%%%%%%%%%%%%%%%%%%%%%%%%%%%%%%%%%%%%%%%%%%%%%%%%%
\begin{mdframed}[style=darkQuesion]
  2.
\end{mdframed}

%%%%%%%%%%%%%%%%%%%%%%%%%%%%%%%%%%%%%%%%%%%%%%%%%%%%%%%%%%%%%%%%%%%%%%%%%%%%%%%%
\begin{mdframed}[style=darkAnswer,frametitle={Joe Starr}]

\end{mdframed}
\newpage
%%%%%%%%%%%%%%%%%%%%%%%%%%%%%%%%%%%%%%%%%%%%%%%%%%%%%%%%%%%%%%%%%%%%%%%%%%%%%%%%
%%%%%%%%%%%%%%%%%%%%%%%%%%%%%%%%%%%%%%%%%%%%%%%%%%%%%%%%%%%%%%%%%%%%%%%%%%%%%%%%
%%%%%%%%%%%%%%%%%%%%%%%%%%%%%%%%%%%%%%%%%%%%%%%%%%%%%%%%%%%%%%%%%%%%%%%%%%%%%%%%
%%%%%%%%%%%%%%%%%%%%%%%%%%%%%%%%%%%%%%%%%%%%%%%%%%%%%%%%%%%%%%%%%%%%%%%%%%%%%%%%
\begin{mdframed}[style=darkQuesion]
  2.
\end{mdframed}

%%%%%%%%%%%%%%%%%%%%%%%%%%%%%%%%%%%%%%%%%%%%%%%%%%%%%%%%%%%%%%%%%%%%%%%%%%%%%%%%
\begin{mdframed}[style=darkAnswer,frametitle={Joe Starr}]

\end{mdframed}
\newpage
%%%%%%%%%%%%%%%%%%%%%%%%%%%%%%%%%%%%%%%%%%%%%%%%%%%%%%%%%%%%%%%%%%%%%%%%%%%%%%%%
%%%%%%%%%%%%%%%%%%%%%%%%%%%%%%%%%%%%%%%%%%%%%%%%%%%%%%%%%%%%%%%%%%%%%%%%%%%%%%%%
%%%%%%%%%%%%%%%%%%%%%%%%%%%%%%%%%%%%%%%%%%%%%%%%%%%%%%%%%%%%%%%%%%%%%%%%%%%%%%%%
%%%%%%%%%%%%%%%%%%%%%%%%%%%%%%%%%%%%%%%%%%%%%%%%%%%%%%%%%%%%%%%%%%%%%%%%%%%%%%%%
\begin{mdframed}[style=darkQuesion]
  2.
\end{mdframed}

%%%%%%%%%%%%%%%%%%%%%%%%%%%%%%%%%%%%%%%%%%%%%%%%%%%%%%%%%%%%%%%%%%%%%%%%%%%%%%%%
\begin{mdframed}[style=darkAnswer,frametitle={Joe Starr}]

\end{mdframed}
\newpage
%%%%%%%%%%%%%%%%%%%%%%%%%%%%%%%%%%%%%%%%%%%%%%%%%%%%%%%%%%%%%%%%%%%%%%%%%%%%%%%%
%%%%%%%%%%%%%%%%%%%%%%%%%%%%%%%%%%%%%%%%%%%%%%%%%%%%%%%%%%%%%%%%%%%%%%%%%%%%%%%%
%%%%%%%%%%%%%%%%%%%%%%%%%%%%%%%%%%%%%%%%%%%%%%%%%%%%%%%%%%%%%%%%%%%%%%%%%%%%%%%%
%%%%%%%%%%%%%%%%%%%%%%%%%%%%%%%%%%%%%%%%%%%%%%%%%%%%%%%%%%%%%%%%%%%%%%%%%%%%%%%%
\begin{mdframed}[style=darkQuesion]
  2.
\end{mdframed}

%%%%%%%%%%%%%%%%%%%%%%%%%%%%%%%%%%%%%%%%%%%%%%%%%%%%%%%%%%%%%%%%%%%%%%%%%%%%%%%%
\begin{mdframed}[style=darkAnswer,frametitle={Joe Starr}]

\end{mdframed}
\newpage
%%%%%%%%%%%%%%%%%%%%%%%%%%%%%%%%%%%%%%%%%%%%%%%%%%%%%%%%%%%%%%%%%%%%%%%%%%%%%%%%
%%%%%%%%%%%%%%%%%%%%%%%%%%%%%%%%%%%%%%%%%%%%%%%%%%%%%%%%%%%%%%%%%%%%%%%%%%%%%%%%
%%%%%%%%%%%%%%%%%%%%%%%%%%%%%%%%%%%%%%%%%%%%%%%%%%%%%%%%%%%%%%%%%%%%%%%%%%%%%%%%
%%%%%%%%%%%%%%%%%%%%%%%%%%%%%%%%%%%%%%%%%%%%%%%%%%%%%%%%%%%%%%%%%%%%%%%%%%%%%%%%
\begin{mdframed}[style=darkQuesion]
  2.
\end{mdframed}

%%%%%%%%%%%%%%%%%%%%%%%%%%%%%%%%%%%%%%%%%%%%%%%%%%%%%%%%%%%%%%%%%%%%%%%%%%%%%%%%
\begin{mdframed}[style=darkAnswer,frametitle={Joe Starr}]

\end{mdframed}
\newpage
%%%%%%%%%%%%%%%%%%%%%%%%%%%%%%%%%%%%%%%%%%%%%%%%%%%%%%%%%%%%%%%%%%%%%%%%%%%%%%%%
%%%%%%%%%%%%%%%%%%%%%%%%%%%%%%%%%%%%%%%%%%%%%%%%%%%%%%%%%%%%%%%%%%%%%%%%%%%%%%%%
%%%%%%%%%%%%%%%%%%%%%%%%%%%%%%%%%%%%%%%%%%%%%%%%%%%%%%%%%%%%%%%%%%%%%%%%%%%%%%%%
%%%%%%%%%%%%%%%%%%%%%%%%%%%%%%%%%%%%%%%%%%%%%%%%%%%%%%%%%%%%%%%%%%%%%%%%%%%%%%%%
\begin{mdframed}[style=darkQuesion]
  2.
\end{mdframed}

%%%%%%%%%%%%%%%%%%%%%%%%%%%%%%%%%%%%%%%%%%%%%%%%%%%%%%%%%%%%%%%%%%%%%%%%%%%%%%%%
\begin{mdframed}[style=darkAnswer,frametitle={Joe Starr}]

\end{mdframed}
\newpage
%%%%%%%%%%%%%%%%%%%%%%%%%%%%%%%%%%%%%%%%%%%%%%%%%%%%%%%%%%%%%%%%%%%%%%%%%%%%%%%%
%%%%%%%%%%%%%%%%%%%%%%%%%%%%%%%%%%%%%%%%%%%%%%%%%%%%%%%%%%%%%%%%%%%%%%%%%%%%%%%%
%%%%%%%%%%%%%%%%%%%%%%%%%%%%%%%%%%%%%%%%%%%%%%%%%%%%%%%%%%%%%%%%%%%%%%%%%%%%%%%%
%%%%%%%%%%%%%%%%%%%%%%%%%%%%%%%%%%%%%%%%%%%%%%%%%%%%%%%%%%%%%%%%%%%%%%%%%%%%%%%%
\begin{mdframed}[style=darkQuesion]
  2.
\end{mdframed}

%%%%%%%%%%%%%%%%%%%%%%%%%%%%%%%%%%%%%%%%%%%%%%%%%%%%%%%%%%%%%%%%%%%%%%%%%%%%%%%%
\begin{mdframed}[style=darkAnswer,frametitle={Joe Starr}]

\end{mdframed}
\newpage
%%%%%%%%%%%%%%%%%%%%%%%%%%%%%%%%%%%%%%%%%%%%%%%%%%%%%%%%%%%%%%%%%%%%%%%%%%%%%%%%
%%%%%%%%%%%%%%%%%%%%%%%%%%%%%%%%%%%%%%%%%%%%%%%%%%%%%%%%%%%%%%%%%%%%%%%%%%%%%%%%
%%%%%%%%%%%%%%%%%%%%%%%%%%%%%%%%%%%%%%%%%%%%%%%%%%%%%%%%%%%%%%%%%%%%%%%%%%%%%%%%
%%%%%%%%%%%%%%%%%%%%%%%%%%%%%%%%%%%%%%%%%%%%%%%%%%%%%%%%%%%%%%%%%%%%%%%%%%%%%%%%
\begin{mdframed}[style=darkQuesion]
  2.
\end{mdframed}

%%%%%%%%%%%%%%%%%%%%%%%%%%%%%%%%%%%%%%%%%%%%%%%%%%%%%%%%%%%%%%%%%%%%%%%%%%%%%%%%
\begin{mdframed}[style=darkAnswer,frametitle={Joe Starr}]

\end{mdframed}
\newpage
%%%%%%%%%%%%%%%%%%%%%%%%%%%%%%%%%%%%%%%%%%%%%%%%%%%%%%%%%%%%%%%%%%%%%%%%%%%%%%%%
%%%%%%%%%%%%%%%%%%%%%%%%%%%%%%%%%%%%%%%%%%%%%%%%%%%%%%%%%%%%%%%%%%%%%%%%%%%%%%%%
%%%%%%%%%%%%%%%%%%%%%%%%%%%%%%%%%%%%%%%%%%%%%%%%%%%%%%%%%%%%%%%%%%%%%%%%%%%%%%%%
%%%%%%%%%%%%%%%%%%%%%%%%%%%%%%%%%%%%%%%%%%%%%%%%%%%%%%%%%%%%%%%%%%%%%%%%%%%%%%%%
\begin{mdframed}[style=darkQuesion]
  2.
\end{mdframed}

%%%%%%%%%%%%%%%%%%%%%%%%%%%%%%%%%%%%%%%%%%%%%%%%%%%%%%%%%%%%%%%%%%%%%%%%%%%%%%%%
\begin{mdframed}[style=darkAnswer,frametitle={Joe Starr}]

\end{mdframed}
\newpage
%%%%%%%%%%%%%%%%%%%%%%%%%%%%%%%%%%%%%%%%%%%%%%%%%%%%%%%%%%%%%%%%%%%%%%%%%%%%%%%%
%%%%%%%%%%%%%%%%%%%%%%%%%%%%%%%%%%%%%%%%%%%%%%%%%%%%%%%%%%%%%%%%%%%%%%%%%%%%%%%%
%%%%%%%%%%%%%%%%%%%%%%%%%%%%%%%%%%%%%%%%%%%%%%%%%%%%%%%%%%%%%%%%%%%%%%%%%%%%%%%%
%%%%%%%%%%%%%%%%%%%%%%%%%%%%%%%%%%%%%%%%%%%%%%%%%%%%%%%%%%%%%%%%%%%%%%%%%%%%%%%%
\begin{mdframed}[style=darkQuesion]
  2.
\end{mdframed}

%%%%%%%%%%%%%%%%%%%%%%%%%%%%%%%%%%%%%%%%%%%%%%%%%%%%%%%%%%%%%%%%%%%%%%%%%%%%%%%%
\begin{mdframed}[style=darkAnswer,frametitle={Joe Starr}]

\end{mdframed}
\newpage
%%%%%%%%%%%%%%%%%%%%%%%%%%%%%%%%%%%%%%%%%%%%%%%%%%%%%%%%%%%%%%%%%%%%%%%%%%%%%%%%
%%%%%%%%%%%%%%%%%%%%%%%%%%%%%%%%%%%%%%%%%%%%%%%%%%%%%%%%%%%%%%%%%%%%%%%%%%%%%%%%
%%%%%%%%%%%%%%%%%%%%%%%%%%%%%%%%%%%%%%%%%%%%%%%%%%%%%%%%%%%%%%%%%%%%%%%%%%%%%%%%
%%%%%%%%%%%%%%%%%%%%%%%%%%%%%%%%%%%%%%%%%%%%%%%%%%%%%%%%%%%%%%%%%%%%%%%%%%%%%%%%
\begin{mdframed}[style=darkQuesion]
  2.
\end{mdframed}

%%%%%%%%%%%%%%%%%%%%%%%%%%%%%%%%%%%%%%%%%%%%%%%%%%%%%%%%%%%%%%%%%%%%%%%%%%%%%%%%
\begin{mdframed}[style=darkAnswer,frametitle={Joe Starr}]

\end{mdframed}
\newpage
%%%%%%%%%%%%%%%%%%%%%%%%%%%%%%%%%%%%%%%%%%%%%%%%%%%%%%%%%%%%%%%%%%%%%%%%%%%%%%%%
%%%%%%%%%%%%%%%%%%%%%%%%%%%%%%%%%%%%%%%%%%%%%%%%%%%%%%%%%%%%%%%%%%%%%%%%%%%%%%%%
%%%%%%%%%%%%%%%%%%%%%%%%%%%%%%%%%%%%%%%%%%%%%%%%%%%%%%%%%%%%%%%%%%%%%%%%%%%%%%%%
%%%%%%%%%%%%%%%%%%%%%%%%%%%%%%%%%%%%%%%%%%%%%%%%%%%%%%%%%%%%%%%%%%%%%%%%%%%%%%%%
\begin{mdframed}[style=darkQuesion]
  2.
\end{mdframed}

%%%%%%%%%%%%%%%%%%%%%%%%%%%%%%%%%%%%%%%%%%%%%%%%%%%%%%%%%%%%%%%%%%%%%%%%%%%%%%%%
\begin{mdframed}[style=darkAnswer,frametitle={Joe Starr}]

\end{mdframed}
\newpage
%%%%%%%%%%%%%%%%%%%%%%%%%%%%%%%%%%%%%%%%%%%%%%%%%%%%%%%%%%%%%%%%%%%%%%%%%%%%%%%%
%%%%%%%%%%%%%%%%%%%%%%%%%%%%%%%%%%%%%%%%%%%%%%%%%%%%%%%%%%%%%%%%%%%%%%%%%%%%%%%%
%%%%%%%%%%%%%%%%%%%%%%%%%%%%%%%%%%%%%%%%%%%%%%%%%%%%%%%%%%%%%%%%%%%%%%%%%%%%%%%%
%%%%%%%%%%%%%%%%%%%%%%%%%%%%%%%%%%%%%%%%%%%%%%%%%%%%%%%%%%%%%%%%%%%%%%%%%%%%%%%%
\begin{mdframed}[style=darkQuesion]
  2.
\end{mdframed}

%%%%%%%%%%%%%%%%%%%%%%%%%%%%%%%%%%%%%%%%%%%%%%%%%%%%%%%%%%%%%%%%%%%%%%%%%%%%%%%%
\begin{mdframed}[style=darkAnswer,frametitle={Joe Starr}]

\end{mdframed}
\newpage
%%%%%%%%%%%%%%%%%%%%%%%%%%%%%%%%%%%%%%%%%%%%%%%%%%%%%%%%%%%%%%%%%%%%%%%%%%%%%%%%
%%%%%%%%%%%%%%%%%%%%%%%%%%%%%%%%%%%%%%%%%%%%%%%%%%%%%%%%%%%%%%%%%%%%%%%%%%%%%%%%
%%%%%%%%%%%%%%%%%%%%%%%%%%%%%%%%%%%%%%%%%%%%%%%%%%%%%%%%%%%%%%%%%%%%%%%%%%%%%%%%
%%%%%%%%%%%%%%%%%%%%%%%%%%%%%%%%%%%%%%%%%%%%%%%%%%%%%%%%%%%%%%%%%%%%%%%%%%%%%%%%
\begin{mdframed}[style=darkQuesion]
  2.
\end{mdframed}

%%%%%%%%%%%%%%%%%%%%%%%%%%%%%%%%%%%%%%%%%%%%%%%%%%%%%%%%%%%%%%%%%%%%%%%%%%%%%%%%
\begin{mdframed}[style=darkAnswer,frametitle={Joe Starr}]

\end{mdframed}
\newpage
%%%%%%%%%%%%%%%%%%%%%%%%%%%%%%%%%%%%%%%%%%%%%%%%%%%%%%%%%%%%%%%%%%%%%%%%%%%%%%%%
%%%%%%%%%%%%%%%%%%%%%%%%%%%%%%%%%%%%%%%%%%%%%%%%%%%%%%%%%%%%%%%%%%%%%%%%%%%%%%%%
%%%%%%%%%%%%%%%%%%%%%%%%%%%%%%%%%%%%%%%%%%%%%%%%%%%%%%%%%%%%%%%%%%%%%%%%%%%%%%%%
%%%%%%%%%%%%%%%%%%%%%%%%%%%%%%%%%%%%%%%%%%%%%%%%%%%%%%%%%%%%%%%%%%%%%%%%%%%%%%%%
\begin{mdframed}[style=darkQuesion]
  2.
\end{mdframed}

%%%%%%%%%%%%%%%%%%%%%%%%%%%%%%%%%%%%%%%%%%%%%%%%%%%%%%%%%%%%%%%%%%%%%%%%%%%%%%%%
\begin{mdframed}[style=darkAnswer,frametitle={Joe Starr}]

\end{mdframed}
\newpage
%%%%%%%%%%%%%%%%%%%%%%%%%%%%%%%%%%%%%%%%%%%%%%%%%%%%%%%%%%%%%%%%%%%%%%%%%%%%%%%%
%%%%%%%%%%%%%%%%%%%%%%%%%%%%%%%%%%%%%%%%%%%%%%%%%%%%%%%%%%%%%%%%%%%%%%%%%%%%%%%%
%%%%%%%%%%%%%%%%%%%%%%%%%%%%%%%%%%%%%%%%%%%%%%%%%%%%%%%%%%%%%%%%%%%%%%%%%%%%%%%%
%%%%%%%%%%%%%%%%%%%%%%%%%%%%%%%%%%%%%%%%%%%%%%%%%%%%%%%%%%%%%%%%%%%%%%%%%%%%%%%%
\begin{mdframed}[style=darkQuesion]
  2.
\end{mdframed}

%%%%%%%%%%%%%%%%%%%%%%%%%%%%%%%%%%%%%%%%%%%%%%%%%%%%%%%%%%%%%%%%%%%%%%%%%%%%%%%%
\begin{mdframed}[style=darkAnswer,frametitle={Joe Starr}]

\end{mdframed}
\newpage
%%%%%%%%%%%%%%%%%%%%%%%%%%%%%%%%%%%%%%%%%%%%%%%%%%%%%%%%%%%%%%%%%%%%%%%%%%%%%%%%
%%%%%%%%%%%%%%%%%%%%%%%%%%%%%%%%%%%%%%%%%%%%%%%%%%%%%%%%%%%%%%%%%%%%%%%%%%%%%%%%
%%%%%%%%%%%%%%%%%%%%%%%%%%%%%%%%%%%%%%%%%%%%%%%%%%%%%%%%%%%%%%%%%%%%%%%%%%%%%%%%
%%%%%%%%%%%%%%%%%%%%%%%%%%%%%%%%%%%%%%%%%%%%%%%%%%%%%%%%%%%%%%%%%%%%%%%%%%%%%%%%
\begin{mdframed}[style=darkQuesion]
  2.
\end{mdframed}

%%%%%%%%%%%%%%%%%%%%%%%%%%%%%%%%%%%%%%%%%%%%%%%%%%%%%%%%%%%%%%%%%%%%%%%%%%%%%%%%
\begin{mdframed}[style=darkAnswer,frametitle={Joe Starr}]

\end{mdframed}
\newpage
%%%%%%%%%%%%%%%%%%%%%%%%%%%%%%%%%%%%%%%%%%%%%%%%%%%%%%%%%%%%%%%%%%%%%%%%%%%%%%%%
%%%%%%%%%%%%%%%%%%%%%%%%%%%%%%%%%%%%%%%%%%%%%%%%%%%%%%%%%%%%%%%%%%%%%%%%%%%%%%%%
%%%%%%%%%%%%%%%%%%%%%%%%%%%%%%%%%%%%%%%%%%%%%%%%%%%%%%%%%%%%%%%%%%%%%%%%%%%%%%%%
%%%%%%%%%%%%%%%%%%%%%%%%%%%%%%%%%%%%%%%%%%%%%%%%%%%%%%%%%%%%%%%%%%%%%%%%%%%%%%%%
\begin{mdframed}[style=darkQuesion]
  2.
\end{mdframed}

%%%%%%%%%%%%%%%%%%%%%%%%%%%%%%%%%%%%%%%%%%%%%%%%%%%%%%%%%%%%%%%%%%%%%%%%%%%%%%%%
\begin{mdframed}[style=darkAnswer,frametitle={Joe Starr}]

\end{mdframed}
\newpage
%%%%%%%%%%%%%%%%%%%%%%%%%%%%%%%%%%%%%%%%%%%%%%%%%%%%%%%%%%%%%%%%%%%%%%%%%%%%%%%%
%%%%%%%%%%%%%%%%%%%%%%%%%%%%%%%%%%%%%%%%%%%%%%%%%%%%%%%%%%%%%%%%%%%%%%%%%%%%%%%%
%%%%%%%%%%%%%%%%%%%%%%%%%%%%%%%%%%%%%%%%%%%%%%%%%%%%%%%%%%%%%%%%%%%%%%%%%%%%%%%%
%%%%%%%%%%%%%%%%%%%%%%%%%%%%%%%%%%%%%%%%%%%%%%%%%%%%%%%%%%%%%%%%%%%%%%%%%%%%%%%%
\begin{mdframed}[style=darkQuesion]
  2.
\end{mdframed}

%%%%%%%%%%%%%%%%%%%%%%%%%%%%%%%%%%%%%%%%%%%%%%%%%%%%%%%%%%%%%%%%%%%%%%%%%%%%%%%%
\begin{mdframed}[style=darkAnswer,frametitle={Joe Starr}]

\end{mdframed}
\newpage