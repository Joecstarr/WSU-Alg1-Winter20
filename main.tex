%Select document class: 12 point, article
\documentclass[12pt]{article}

% This template is a combination of work done by Mike Catanzaro and Gabe Angelini-Knoll, both formerly of the WSU math dept, with some additions and synthesizations by Clayton Hayes (clayton.hayes@wayne.edu) for broader use
% Last updated 2018-03-10 by Aaron Willcock (ez9213@wayne.edu)

%AMS = American Mathematical Society
%math, symbols, theorems, fonts
\usepackage{amsmath, amssymb, amsthm, amsfonts}
\usepackage{times, flexisym, mdframed, xcolor}
\usepackage{ulem,multicol}


%%%%%%%%%%%%%%%%%%%%%%%%%%%%%%%%%%%%%%%%%%%%%%%%%%%%%%%%%%%%%%%%%%%%%%%%%%%%%%%%
%Commands definitions
\definecolor{backgroundcolorDef}{rgb}{0.1,0.1,0.12}
\definecolor{textcolorDef}{rgb}{0.77,0.77,0.77}
\definecolor{waynegreen}{RGB}{0,89,79}
\newcommand{\setbackgroundcolour}{\pagecolor{backgroundcolorDef}}  
\newcommand{\settextcolour}{\color{textcolorDef}}   
\newcommand{\invertbackgroundtext}{\setbackgroundcolour\settextcolour}
\mdfdefinestyle{darkAnswer}{%
    fontcolor=textcolorDef,
    backgroundcolor=backgroundcolorDef,
    linecolor=textcolorDef,
    % ----------------------------------
    frametitlebackgroundcolor=backgroundcolorDef,
    frametitlefontcolor=textcolorDef,
    frametitlerulecolor=waynegreen,
    frametitlerule=true, 
    frametitlerulewidth=2pt
    }
\mdfdefinestyle{darkQuesion}{%
    fontcolor=textcolorDef,
    backgroundcolor=backgroundcolorDef,
    linecolor=textcolorDef,
    linewidth=1pt
    }
    
    %If this line is commented, then the appearance remains as usual.
\invertbackgroundtext
\newcommand{\m}{\scalebox{0.25}[1.0]{$-$}}
\newcommand{\lrb}[1]{\left[#1\right]}
\newcommand{\lrp}[1]{\left(#1\right)}
\newcommand{\lrs}[1]{\left\{#1\right\}}
\newcommand{\lra}[1]{\left<#1\right>}
\newcommand{\gof}[1]{g\lrp{#1}}
\newcommand{\fof}[1]{f\lrp{#1}}
\newcommand{\kof}[1]{k\lrp{#1}}
\newcommand{\hof}[1]{h\lrp{#1}}
\newcommand{\pof}[1]{\varphi\lrp{#1}}
\newcommand{\sof}[1]{\sigma\lrp{#1}}
\newcommand{\Zof}[1]{Z\lrp{#1}}
\newcommand{\gofinv}[1]{g^{\m1}\lrp{#1}}
\newcommand{\fofinv}[1]{f^{\m1}\lrp{#1}}
\newcommand{\Zofinv}[1]{Z^{\m1}\lrp{#1}}
\newcommand{\pofinv}[1]{\varphi^{\m1}\lrp{#1}}
\newcommand{\nmod}[2]{#1\,\left(\text{mod}\,#2\right)}
\newcommand{\ngcd}[2]{\text{gcd}\left(#1\,,\,#2\right)}
\newcommand{\nlcm}[2]{\text{lcm}\left(#1\,,\,#2\right)}
\newcommand{\grp}[2]{\left(#1\,,\,#2\right)}
\newcommand{\GL}[2]{\mathrm{GL}_{#1}\lrp{#2}}
\newcommand{\SL}[2]{\mathrm{SL}_{#1}\lrp{#2}}
\newcommand{\Mn}[2]{M_{#1}\left(#2\right)}
\newcommand{\nord}[2]{\text{ord}_{#1}\left(#2\right)}
\newcommand{\ord}[1]{\text{ord}\left(#1\right)}
\newcommand{\dom}[1]{\text{dom}\left(#1\right)}
\newcommand{\ran}[1]{\text{ran}\left(#1\right)}
\newcommand{\degr}[1]{\text{deg}\left(#1\right)}
\newcommand{\N}{\mathbb{N}}
\newcommand{\Z}{\mathbb{Z}}
\newcommand{\Q}{\mathbb{Q}}
\newcommand{\R}{\mathbb{R}}
\newcommand{\Rp}{\mathbb{R}^{+}}
\newcommand{\C}{\mathbb{C}}
\newcommand{\primeDec}{p_1^{\alpha_1}p_2^{\alpha_2}\cdots p_m^{\alpha_m}p_{m+1}^{\alpha_{m+1}}}
\newcommand{\abs}[1]{\left\vert#1\right\vert}
\newcommand{\Zm}[1]{\mathbb{Z}_{#1}}
\newcommand{\Zmx}[1]{\mathbb{Z}_{#1}^{\times}}
\newcommand{\Zp}{\mathbb{Z}_p}
\newcommand{\numb}[1]{\noindent{\bf #1)}}
\newcommand{\inv}[1]{#1^{\m1}}
\newcommand{\totient}[1]{\phi\left(#1\right)} 
\newcommand{\vhalfpg}{\vspace{5in}}
\newcommand{\vthirdpg}{\vspace{3in}}
\newcommand{\vquartpg}{\vspace{2in}}
\newcommand{\Assoc}[1]{\item[Associativity:]{#1}}
\newcommand{\Invs}[1]{\item[Inverses:]{#1}}
\newcommand{\Clos}[1]{\item[Closure:]{#1}}
\newcommand{\Ident}[1]{\item[Identity:]{#1}}
\newcommand{\Abel}[1]{\item[Abelian:]{#1}}
\newcommand{\twocase}[2]{\begin{enumerate}
        \item[$ \implies$]{
            #1
        }
        \bigskip
        \item[$ \impliedby$]{
            #2
        }
    \end{enumerate}}
\newcommand{\msout}[1]{\text{\sout{\ensuremath{#1}}}}
%%%%%%%%%%%%%%%%%%%%%%%%%%%%%%%%%%%%%%%%%%%%%%%%%%%%%%%%%%%%%%%%%%%%%%%%%%%%%%%%
%Enhanced graphics support (https://ctan.org/pkg/graphicx?lang=en)
\usepackage{graphicx}

%Set default graphics path (replace 'figures/' with whatever directory your images are in)
%\graphicspath{{figures/}}


%Control layout of itemize, enumerate, description (https://ctan.org/pkg/enumitem?lang=en)
\usepackage{enumitem}

% Header and footer formatting options
\usepackage{fancyhdr}

%Control float placement. [section] = "stop floats at section boundaries is to change the definition of “\section” to include “\FloatBarrier”"
\usepackage[section]{placeins}

%Hypertext (links) in LaTeX. [option] = remove color and border on links.
\usepackage[hidelinks]{hyperref}

%\usepackage[all]{xy}
%\usepackage{mathtools}

%Pro­gram­ming fa­cil­i­ties for LaTeX class and pack­age authors
\usepackage{etoolbox}

%Indent first paragraph of every 'chapter' aka section
\usepackage{indentfirst}

%Title formatting option. [explicit] = make titles all caps
\usepackage[explicit]{titlesec}

%Standard package for selecting font encodings. [T1] = Support for accented characters
%(https://texfaq.org/FAQ-why-inp-font) - 
%(https://tex.stackexchange.com/questions/664/why-should-i-use-usepackaget1fontenc)
\usepackage[T1]{fontenc}

%Charter fonts
\usepackage{charter}

%\usepackage[expert]{mathdesign}

%Control table of contents, figures, etc
\usepackage{tocloft}

%Set space between lines. [option] = Double/single spacing necessary for properly formatting ToC/LoFT and titles.
\usepackage[singlespacing]{setspace}

%Create normal/logarithmic plots in two and three dimensions
\usepackage{pgfplots}


% Remove extra space above and below theorems, lemmas, props, etc.
%The important point in the following is: 0pt preskip and 0pt postskip
\makeatletter
\def\thm@space@setup{\thm@preskip=0pt
\thm@postskip=0pt}
\makeatother
\newtheoremstyle{newstyle}      
{} % Aboveskip 
{} % Below skip
{\mdseries} % Body font e.g.\mdseries,\bfseries,\scshape,\itshape
{} % Indent
{\bfseries}  % Head font e.g.\bfseries,\scshape,\itshape
{.} % Punctuation afer theorem header
{ } % Space after theorem header
{} % Heading

%The above does not fix the spacing around proof environments.
%Use the following to fix.
%The crucial point is "topsep0\p@", i.e., topsep = 0 pt.
%The rest is essentially copied from the standard AMS environment.
\makeatletter
\renewenvironment{proof}[1][\proofname]{\par
  \pushQED{\qed}%
  \normalfont \topsep0\p@\relax
  \trivlist
  \item[\hskip\labelsep\itshape
  #1\@addpunct{.}]\ignorespaces 
}{
  \popQED\endtrivlist\@endpefalse
}
\makeatother


%Formatting
%   Margins
\usepackage[left=1in,right=1in, top=1in, bottom=1in]{geometry}
%
\newcommand{\nd}{\noindent}
\def\para#1{\vskip 0.4\baselineskip\noindent{\bf #1}}
\newcommand{\Vspc}{\vspace*{0.1in}}



\usepackage{etoolbox}
\newcommand{\zerodisplayskips}{%
  \setlength{\abovedisplayskip}{0pt}%
  \setlength{\belowdisplayskip}{0pt}%
  \setlength{\abovedisplayshortskip}{0pt}%
  \setlength{\belowdisplayshortskip}{0pt}}
\appto{\footnotesize}{\zerodisplayskips}
\appto{\tiny}{\zerodisplayskips}
\appto{\scriptsize}{\zerodisplayskips}
\appto{\footnotesize}{\zerodisplayskips}
\appto{\small}{\zerodisplayskips}
\appto{\large}{\zerodisplayskips}
\appto{\Large}{\zerodisplayskips}
\appto{\LARGE}{\zerodisplayskips}
\appto{\huge}{\zerodisplayskips}
\appto{\Huge}{\zerodisplayskips}
\setlength{\columnseprule}{0.2pt}
%Enter single spacing environment for toc, lot, and lof (see below)
\begin{document}

%Compile the title and jump to new page
%Create title (centered and bold font)
\vfill
  \begin{figure}[!htbp]
    \centering
    \includegraphics[width=0.25\linewidth]{fig/wsu_primary_stacked_color.pdf}
  \end{figure}
  \centerline{\bf Algebra I}
  \centerline{\bf Winter 2020}
\vfill
\thispagestyle{empty}
\newpage
\tableofcontents
\newpage
\setcounter{page}{3}
%%%%%%%%%%%%%%%%%%%%%%%%%%%%%%%%%%%%%%%%%%%%%%%%%%%%%%%%%%%%%%%%%%%%%%%%%%%%%%%%
%Compile Chapter 1
\section{Integers}
\subsection{Divisors}
%%%%%%%%%%%%%%%%%%%%%%%%%%%%%%%%%%%%%%%%%%%%%%%%%%%%%%%%%%%%%%%%%%%%%%%%%%%%%%%%
%%%%%%%%%%%%%%%%%%%%%%%%%%%%%%%%%%%%%%%%%%%%%%%%%%%%%%%%%%%%%%%%%%%%%%%%%%%%%%%%
%%%%%%%%%%%%%%%%%%%%%%%%%%%%%%%%%%%%%%%%%%%%%%%%%%%%%%%%%%%%%%%%%%%%%%%%%%%%%%%%
%%%%%%%%%%%%%%%%%%%%%%%%%%%%%%%%%%%%%%%%%%%%%%%%%%%%%%%%%%%%%%%%%%%%%%%%%%%%%%%%
\begin{mdframed}[style=darkQuesion]
  1.    Let $m,n.r.s\in \Z$. If $m^2+n^2=r^2+s^2=mr+ns$, prove that $m=r$ and
  $n=s$.
\end{mdframed}

%%%%%%%%%%%%%%%%%%%%%%%%%%%%%%%%%%%%%%%%%%%%%%%%%%%%%%%%%%%%%%%%%%%%%%%%%%%%%%%%
\begin{mdframed}[style=darkAnswer,frametitle={Joe Starr}]
  We select $m,n.r.s\in \Z$, given $m^2+n^2=r^2+s^2=mr+ns$ which can write
  as $m^2+n^2-mr-ns=r^2+s^2-mr-ns$. From here we can simplify:
  \begin{align*}
    m^2+n^2-mr-ns=r^2+s^2-mr-ns & \Rightarrow
    m\lrp{m-r}+n\lrp{n-s}=r\lrp{r-m}+s\lrp{s-n}                                             \\
                                & \Rightarrow m\lrp{m-r}+n\lrp{n-s}-r\lrp{r-m}-s\lrp{s-n}=0 \\
                                & \Rightarrow m\lrp{m-r}+r\lrp{m-r}+n\lrp{n-s}+s\lrp{n-s}=0 \\
                                & \Rightarrow \lrp{m-r}\lrp{m+r}+\lrp{n-s}\lrp{n+s}=0       \\
  \end{align*}
  from here we can see that in order for $\lrp{m-r}\lrp{m+r}+\lrp{n-s}\lrp{n+s}=0$
  to be true $m=r$ and $n=s$.
\end{mdframed}
\newpage
%%%%%%%%%%%%%%%%%%%%%%%%%%%%%%%%%%%%%%%%%%%%%%%%%%%%%%%%%%%%%%%%%%%%%%%%%%%%%%%%
%%%%%%%%%%%%%%%%%%%%%%%%%%%%%%%%%%%%%%%%%%%%%%%%%%%%%%%%%%%%%%%%%%%%%%%%%%%%%%%%
%%%%%%%%%%%%%%%%%%%%%%%%%%%%%%%%%%%%%%%%%%%%%%%%%%%%%%%%%%%%%%%%%%%%%%%%%%%%%%%%
%%%%%%%%%%%%%%%%%%%%%%%%%%%%%%%%%%%%%%%%%%%%%%%%%%%%%%%%%%%%%%%%%%%%%%%%%%%%%%%%
\begin{mdframed}[style=darkQuesion]
  3.    Find the quotient and reminder when $a$ id divided by $b$.
  \begin{itemize}
    \item [a] {$a=99$, $b=17$}
    \item [b] {$a=-99$, $b=17$}
    \item [c] {$a=17$, $b=99$}
    \item [d] {$a=-1017$, $b=99$}
  \end{itemize}

\end{mdframed}

%%%%%%%%%%%%%%%%%%%%%%%%%%%%%%%%%%%%%%%%%%%%%%%%%%%%%%%%%%%%%%%%%%%%%%%%%%%%%%%%
\begin{mdframed}[style=darkAnswer,frametitle={Joe Starr}]
  \begin{itemize}
    \item [a] {$99=17q+r\Rightarrow q=5, r=14$}
    \item [b] {$-99=17q+r\Rightarrow q=-6, r=3$}
    \item [c] {$17=99q+r\Rightarrow q=0, r=17$}
    \item [d] {$-1017=99q+r\Rightarrow q=-11, r=72$}
  \end{itemize}
\end{mdframed}

\newpage
%%%%%%%%%%%%%%%%%%%%%%%%%%%%%%%%%%%%%%%%%%%%%%%%%%%%%%%%%%%%%%%%%%%%%%%%%%%%%%%%
%%%%%%%%%%%%%%%%%%%%%%%%%%%%%%%%%%%%%%%%%%%%%%%%%%%%%%%%%%%%%%%%%%%%%%%%%%%%%%%%
%%%%%%%%%%%%%%%%%%%%%%%%%%%%%%%%%%%%%%%%%%%%%%%%%%%%%%%%%%%%%%%%%%%%%%%%%%%%%%%%
%%%%%%%%%%%%%%%%%%%%%%%%%%%%%%%%%%%%%%%%%%%%%%%%%%%%%%%%%%%%%%%%%%%%%%%%%%%%%%%%
\begin{mdframed}[style=darkQuesion]
  5.    Use the Euclidean algorithm to find the following greatest common divisors
  \begin{multicols}{2}
    \begin{itemize}
      \item [a] {$\lrp{6643,2873}$}
      \item [b] {$\lrp{7684,4148}$}
      \item [c] {$\lrp{26460,12600}$}
      \item [d] {$\lrp{6540,1206}$}
      \item [e] {$\lrp{12091,8439}$}
    \end{itemize}
  \end{multicols}
\end{mdframed}

%%%%%%%%%%%%%%%%%%%%%%%%%%%%%%%%%%%%%%%%%%%%%%%%%%%%%%%%%%%%%%%%%%%%%%%%%%%%%%%%
\begin{mdframed}[style=darkAnswer,frametitle={Joe Starr}]
  \begin{multicols}{2}
    \begin{itemize}
      \item [(a)] {$\lrp{6643,2873}$
            \begin{align*}
              6643 & =2873*2+897 \\
              2873 & =897*3+182  \\
              897  & =182*4+169  \\
              182  & =169*1+13   \\
              169  & =13*13      \\
            \end{align*}
            }
      \item [(b)] {$\lrp{7684,4148}$
            \begin{align*}
              7684 & =4148*1+3536 \\
              4148 & =3536*1+612  \\
              3536 & =612*5+476   \\
              612  & =476*1+136   \\
              476  & =136*3+68    \\
              136  & =68*68       \\
            \end{align*}
            }
      \item [(c)] {$\lrp{26460,12600}$
            \begin{align*}
              26460 & =12600*2+1260 \\
              12600 & =1260*10      \\
            \end{align*}
            }
    \end{itemize}
  \end{multicols}
  \begin{multicols}{2}
    \begin{itemize}
      \item [(d)] {$\lrp{6540,1206}$
            \begin{align*}
              6540 & =1206*5+510 \\
              1206 & =510*2+186  \\
              510  & =186*2+138  \\
              186  & =138*1+48   \\
              138  & =48*2+42    \\
              48   & =42*1+6     \\
              42   & =6*7        \\
            \end{align*}
            }
      \item [(e)] {$\lrp{12091,8439}$
            \begin{align*}
              12091 & =8439*1+3652 \\
              8439  & =3652*2+1135 \\
              3652  & =1135*3+247  \\
              1135  & =247*4+147   \\
              247   & =147*1+100   \\
              147   & =100*1+47    \\
              100   & =47*2+6      \\
              47    & =6*7+5       \\
              6     & =5*1+1       \\
              5     & =1*5         \\
            \end{align*}
            }
    \end{itemize}
  \end{multicols}
\end{mdframed}
\newpage
%%%%%%%%%%%%%%%%%%%%%%%%%%%%%%%%%%%%%%%%%%%%%%%%%%%%%%%%%%%%%%%%%%%%%%%%%%%%%%%%
%%%%%%%%%%%%%%%%%%%%%%%%%%%%%%%%%%%%%%%%%%%%%%%%%%%%%%%%%%%%%%%%%%%%%%%%%%%%%%%%
%%%%%%%%%%%%%%%%%%%%%%%%%%%%%%%%%%%%%%%%%%%%%%%%%%%%%%%%%%%%%%%%%%%%%%%%%%%%%%%%
%%%%%%%%%%%%%%%%%%%%%%%%%%%%%%%%%%%%%%%%%%%%%%%%%%%%%%%%%%%%%%%%%%%%%%%%%%%%%%%%
\begin{mdframed}[style=darkQuesion]
  7.    For each part of Exercise 5, find integers $m$ and $n$ such that
  $\lrp{a,b}$ is expressed in the form $ma+nb$.

\end{mdframed}

%%%%%%%%%%%%%%%%%%%%%%%%%%%%%%%%%%%%%%%%%%%%%%%%%%%%%%%%%%%%%%%%%%%%%%%%%%%%%%%%
\begin{mdframed}[style=darkAnswer,frametitle={Joe Starr}]
  \begin{itemize}
    \item [(a)] {$\lrp{6643,2873}$\\
          $\lrp{6643}-16+\lrp{2873}37=13$
          }
    \item [(b)] {$\lrp{7684,4148}$\\
          $\lrp{7684}27+\lrp{4148}-50=68$
          }
    \item [(c)] {$\lrp{26460,12600}$
          $\lrp{26460}1+\lrp{12600}-2=1260$
          }
    \item [(d)] {$\lrp{6540,1206}$
          $\lrp{6540}-26+\lrp{1206}141=6$
          }
    \item [(e)] {$\lrp{12091,8439}$
          $\lrp{12091}1435+\lrp{8439}-2056=1$
          }
  \end{itemize}
\end{mdframed}
\newpage
%%%%%%%%%%%%%%%%%%%%%%%%%%%%%%%%%%%%%%%%%%%%%%%%%%%%%%%%%%%%%%%%%%%%%%%%%%%%%%%%
%%%%%%%%%%%%%%%%%%%%%%%%%%%%%%%%%%%%%%%%%%%%%%%%%%%%%%%%%%%%%%%%%%%%%%%%%%%%%%%%
%%%%%%%%%%%%%%%%%%%%%%%%%%%%%%%%%%%%%%%%%%%%%%%%%%%%%%%%%%%%%%%%%%%%%%%%%%%%%%%%
%%%%%%%%%%%%%%%%%%%%%%%%%%%%%%%%%%%%%%%%%%%%%%%%%%%%%%%%%%%%%%%%%%%%%%%%%%%%%%%%
\begin{mdframed}[style=darkQuesion]
  9.  let $a,b,c$ be integers such that $a+b+c=0$. Show that if $n$ is an integer
  which is a divisor of two of the three integers, then it is also a divisor of
  the third.

\end{mdframed}

%%%%%%%%%%%%%%%%%%%%%%%%%%%%%%%%%%%%%%%%%%%%%%%%%%%%%%%%%%%%%%%%%%%%%%%%%%%%%%%%
\begin{mdframed}[style=darkAnswer,frametitle={Joe Starr}]
  Select $a,b,c\in \Z$ to satisfy $a+b+c=0$, WLOG let $n\in \Z$ such that
  $n\vert a$ and $n\vert b$. Since $\lrp{a+b}+c=0$ it must be that $\lrp{a+b}=-c$.
  From here we must show $n\vert\lrp{a+b}$, or $a+b=nq$. Since $n\vert a$ and
  $n\vert b$ we may write $a=nq_1$ and $b=nq_2$, yielding,
  $nq_1+nq_2=n\lrp{q_1+q_2}=nq$ thus $n\vert c$, as desired.$\square$
\end{mdframed}
\newpage
%%%%%%%%%%%%%%%%%%%%%%%%%%%%%%%%%%%%%%%%%%%%%%%%%%%%%%%%%%%%%%%%%%%%%%%%%%%%%%%%
%%%%%%%%%%%%%%%%%%%%%%%%%%%%%%%%%%%%%%%%%%%%%%%%%%%%%%%%%%%%%%%%%%%%%%%%%%%%%%%%
%%%%%%%%%%%%%%%%%%%%%%%%%%%%%%%%%%%%%%%%%%%%%%%%%%%%%%%%%%%%%%%%%%%%%%%%%%%%%%%%
%%%%%%%%%%%%%%%%%%%%%%%%%%%%%%%%%%%%%%%%%%%%%%%%%%%%%%%%%%%%%%%%%%%%%%%%%%%%%%%%
\begin{mdframed}[style=darkQuesion]
  13.  Show that if $n$ is any integer, then $\lrp{10n_3,5n+2}=1$
\end{mdframed}

%%%%%%%%%%%%%%%%%%%%%%%%%%%%%%%%%%%%%%%%%%%%%%%%%%%%%%%%%%%%%%%%%%%%%%%%%%%%%%%%
\begin{mdframed}[style=darkAnswer,frametitle={Joe Starr}]
  We begin with the Euclidean algorithm,
  \begin{align*}
    10n+3 & =\lrp{5n+2}1+\lrp{5n+1} \\
    5n+2  & =\lrp{5n+1}1+1          \\
  \end{align*}
  from here we have $\lrp{10n+3,5n+2}=\lrp{5n+2,5n+1}=1$, as desired.
\end{mdframed}
\newpage
%%%%%%%%%%%%%%%%%%%%%%%%%%%%%%%%%%%%%%%%%%%%%%%%%%%%%%%%%%%%%%%%%%%%%%%%%%%%%%%%
%%%%%%%%%%%%%%%%%%%%%%%%%%%%%%%%%%%%%%%%%%%%%%%%%%%%%%%%%%%%%%%%%%%%%%%%%%%%%%%%
%%%%%%%%%%%%%%%%%%%%%%%%%%%%%%%%%%%%%%%%%%%%%%%%%%%%%%%%%%%%%%%%%%%%%%%%%%%%%%%%
%%%%%%%%%%%%%%%%%%%%%%%%%%%%%%%%%%%%%%%%%%%%%%%%%%%%%%%%%%%%%%%%%%%%%%%%%%%%%%%%
\begin{mdframed}[style=darkQuesion]
  15.  For what positive integers $n$ is it true that $\lrp{n,n+2}=2$? Prove your
  claim.
\end{mdframed}

%%%%%%%%%%%%%%%%%%%%%%%%%%%%%%%%%%%%%%%%%%%%%%%%%%%%%%%%%%%%%%%%%%%%%%%%%%%%%%%%
\begin{mdframed}[style=darkAnswer,frametitle={Joe Starr}]
  The conjecture is that the statement is true for even values of $n$.
  We begin with rewriting $n$ in terms of $k$, $n=2k$the Euclidean algorithm,
  \begin{align*}
    \lrp{2k}+2 & =\lrp{2k}1+\lrp{2} \\
    2k         & =\lrp{2}{k}        \\
  \end{align*}
  from here we have $\lrp{n+2,n}=\lrp{2k+2,2k}=2$, as desired.
\end{mdframed}
\newpage
%%%%%%%%%%%%%%%%%%%%%%%%%%%%%%%%%%%%%%%%%%%%%%%%%%%%%%%%%%%%%%%%%%%%%%%%%%%%%%%%
%%%%%%%%%%%%%%%%%%%%%%%%%%%%%%%%%%%%%%%%%%%%%%%%%%%%%%%%%%%%%%%%%%%%%%%%%%%%%%%%
%%%%%%%%%%%%%%%%%%%%%%%%%%%%%%%%%%%%%%%%%%%%%%%%%%%%%%%%%%%%%%%%%%%%%%%%%%%%%%%%
%%%%%%%%%%%%%%%%%%%%%%%%%%%%%%%%%%%%%%%%%%%%%%%%%%%%%%%%%%%%%%%%%%%%%%%%%%%%%%%%
\begin{mdframed}[style=darkQuesion]
  17.  Show that the positive integer $k$ is the difference of two odd squares if
  and only if $k$ is divisible by $8$.
\end{mdframed}

%%%%%%%%%%%%%%%%%%%%%%%%%%%%%%%%%%%%%%%%%%%%%%%%%%%%%%%%%%%%%%%%%%%%%%%%%%%%%%%%
\begin{mdframed}[style=darkAnswer,frametitle={Joe Starr}]

  We begin by writing $k=a^2-b^2$, since $a$ and $b$ are odd we can write,
  \begin{align*}
    a & =2r+1 \\
    b & =2s+1
  \end{align*}
  from here we have $q^2-b^2=4\lrp{r+s+1}\lrp{r-s}$. Since $k>0$ we must consider
  two cases $r-s=2m+1$ and $r-s=2m$.
  \begin{itemize}[align=left]
    \item [$r-s=2m$:]{\hspace{.5in}\newline
          In this case we have $q^2-b^2=4\lrp{r+s+1}2m=8\lrp{r+s+1}m$ and we are
          done.
          }
    \item [$r-s=2m+1$:]{\hspace{.5in}\newline
          In this case we have $r-s=2m+1$ and $r+s=r-s+2s=2m+1+2s$
          \begin{align*}
            q^2-b^2 & =4\lrp{r+s+1}\lrp{2m+1}              \\
                    & =4\lrp{2m\lrp{r+s+1}+\lrp{r+s+1}}    \\
                    & =4\lrp{\lrp{2mr+2ms+2m}+\lrp{r+s+1}} \\
                    & =4\lrp{2mr+2ms+2m+r+s+1}             \\
                    & =4\lrp{2mr+2ms+2m+2m+1+2s+1}         \\
                    & =4\lrp{2mr+2ms+2m+2m+2s+2}           \\
                    & =8\lrp{mr+ms+m+m+s+1}                \\
          \end{align*}
          as desired.
          }
  \end{itemize}

\end{mdframed}
\newpage
%%%%%%%%%%%%%%%%%%%%%%%%%%%%%%%%%%%%%%%%%%%%%%%%%%%%%%%%%%%%%%%%%%%%%%%%%%%%%%%%
%%%%%%%%%%%%%%%%%%%%%%%%%%%%%%%%%%%%%%%%%%%%%%%%%%%%%%%%%%%%%%%%%%%%%%%%%%%%%%%%
%%%%%%%%%%%%%%%%%%%%%%%%%%%%%%%%%%%%%%%%%%%%%%%%%%%%%%%%%%%%%%%%%%%%%%%%%%%%%%%%
%%%%%%%%%%%%%%%%%%%%%%%%%%%%%%%%%%%%%%%%%%%%%%%%%%%%%%%%%%%%%%%%%%%%%%%%%%%%%%%%
\subsection{Primes}
%%%%%%%%%%%%%%%%%%%%%%%%%%%%%%%%%%%%%%%%%%%%%%%%%%%%%%%%%%%%%%%%%%%%%%%%%%%%%%%%
%%%%%%%%%%%%%%%%%%%%%%%%%%%%%%%%%%%%%%%%%%%%%%%%%%%%%%%%%%%%%%%%%%%%%%%%%%%%%%%%
%%%%%%%%%%%%%%%%%%%%%%%%%%%%%%%%%%%%%%%%%%%%%%%%%%%%%%%%%%%%%%%%%%%%%%%%%%%%%%%%
%%%%%%%%%%%%%%%%%%%%%%%%%%%%%%%%%%%%%%%%%%%%%%%%%%%%%%%%%%%%%%%%%%%%%%%%%%%%%%%%
\begin{mdframed}[style=darkQuesion]
  1. Find the prime factorizations of each of the following numbers, and use the
  them to compute the greatest common divisor and least common multiple of the
  given pairs of numbers.
  \begin{multicols}{3}
    \begin{itemize}
      \item [(a)] {$\lrp{35,14}$
            }
      \item [(b)] {$\lrp{15,11}$
            }
      \item [(c)] {$\lrp{252,11}$
            }
      \item [(d)] {$\lrp{7684,4148}$
            }
      \item [(e)] {$\lrp{6643,2873}$
            }
    \end{itemize}
  \end{multicols}
\end{mdframed}

%%%%%%%%%%%%%%%%%%%%%%%%%%%%%%%%%%%%%%%%%%%%%%%%%%%%%%%%%%%%%%%%%%%%%%%%%%%%%%%%
\begin{mdframed}[style=darkAnswer,frametitle={Joe Starr}]
  \begin{multicols}{2}
    \begin{itemize}
      \item [(a)] {
            \begin{multicols}{2}
              $\lrp{35,14}$\\
              $35: 5,7$ \\
              $14: 2,7$ \\
              gcd: $7$ \\
              lcm: $70$
            \end{multicols}
            }
      \item [(b)] {
            \begin{multicols}{2}
              $\lrp{15,11}$ \\
              $15: 3,5$ \\
              $11: 11$ \\
              gcd: $1$ \\
              lcm: $165$
            \end{multicols}
            }
      \item [(c)] {
            \begin{multicols}{2}
              $\lrp{252,180}$ \\
              $252: 2,2,3,3,7$ \\
              $180: 2,2,3,3,5$ \\
              gcd: $36$ \\
              lcm: $1260$
            \end{multicols}
            }
      \item [(d)] {
            \begin{multicols}{2}
              $\lrp{7684,4148}$ \\
              $7684: 2,2,17,113$ \\
              $4148: 2,2,17,61$ \\
              gcd: $68$ \\
              lcm: $468724$
            \end{multicols}
            }
      \item [(e)] {
            \begin{multicols}{2}
              $\lrp{6643,2873}$ \\
              $6643: 7,13,73$ \\
              $2873: 13,13,17$ \\
              gcd: $13$ \\
              lcm: $1468103$
            \end{multicols}
            }
    \end{itemize}
  \end{multicols}
\end{mdframed}
\newpage
%%%%%%%%%%%%%%%%%%%%%%%%%%%%%%%%%%%%%%%%%%%%%%%%%%%%%%%%%%%%%%%%%%%%%%%%%%%%%%%%
%%%%%%%%%%%%%%%%%%%%%%%%%%%%%%%%%%%%%%%%%%%%%%%%%%%%%%%%%%%%%%%%%%%%%%%%%%%%%%%%
%%%%%%%%%%%%%%%%%%%%%%%%%%%%%%%%%%%%%%%%%%%%%%%%%%%%%%%%%%%%%%%%%%%%%%%%%%%%%%%%
%%%%%%%%%%%%%%%%%%%%%%%%%%%%%%%%%%%%%%%%%%%%%%%%%%%%%%%%%%%%%%%%%%%%%%%%%%%%%%%%
\begin{mdframed}[style=darkQuesion]
  2. US the sieve of Eratosthenes to find all prime numbers less than 200.
\end{mdframed}

%%%%%%%%%%%%%%%%%%%%%%%%%%%%%%%%%%%%%%%%%%%%%%%%%%%%%%%%%%%%%%%%%%%%%%%%%%%%%%%%
\begin{mdframed}[style=darkAnswer,frametitle={Joe Starr}]
  \begin{center}
    \begin{tabular}{| c | c | c | c | c | c | c | c | c | c |}
      \hline
      $ $           & $2$           & $3$           & $\msout{4}$   & $5$           & $\msout{6}$   & $7$ & $\msout{8}$ &
      $\msout{9}$   & $\msout{10}$                                                                                        \\
      \hline
      $11$          & $\msout{12}$  & $13$          & $\msout{14}$  & $\msout{15}$  & $\msout{16}$
                    & $17$          & $\msout{18}$  & $19$          & $\msout{20}$                                        \\
      \hline
      $\msout{21}$  & $\msout{22}$  & $23$          & $\msout{24}$  & $\msout{25}$  &
      $\msout{26}$  & $\msout{27}$  & $\msout{28}$  & $29$          & $\msout{30}$                                        \\
      \hline
      $31$          & $\msout{32}$  & $\msout{33}$  & $\msout{34}$  & $\msout{35}$  &
      $\msout{36}$  & $37$          & $\msout{38}$  & $\msout{39}$  & $\msout{40}$                                        \\
      \hline
      $41$          & $\msout{42}$  & $43$          & $\msout{44}$  & $\msout{45}$  & $\msout{46}$
                    & $47$          & $\msout{48}$  & $\msout{49}$  & $\msout{50}$                                        \\
      \hline
      $\msout{51}$  & $\msout{52}$  & $53$          & $\msout{54}$  & $\msout{55}$  &
      $\msout{56}$  & $\msout{57}$  & $\msout{58}$  & $59$          & $\msout{60}$                                        \\
      \hline
      $61$          & $\msout{62}$  & $\msout{63}$  & $\msout{64}$  & $\msout{65}$  &
      $\msout{66}$  & $67$          & $\msout{68}$  & $\msout{69}$  & $\msout{70}$                                        \\
      \hline
      $71$          & $\msout{72}$  & $73$          & $\msout{74}$  & $\msout{75}$  & $\msout{76}$
                    & $\msout{77}$  & $\msout{78}$  & $79$          & $\msout{80}$                                        \\
      \hline
      $\msout{81}$  & $\msout{82}$  & $83$          & $\msout{84}$  & $\msout{85}$  &
      $\msout{86}$  & $\msout{87}$  & $\msout{88}$  & $89$          & $\msout{90}$                                        \\
      \hline
      $\msout{91}$  & $\msout{92}$  & $\msout{93}$  & $\msout{94}$  & $\msout{95}$
                    & $\msout{96}$  & $97$          & $\msout{98}$  & $\msout{99}$  & $\msout{100}$                       \\
      \hline
      $101$         & $\msout{102}$ & $103$         & $\msout{104}$ & $\msout{105}$ &
      $\msout{106}$ & $107$         & $\msout{108}$ & $109$         & $\msout{110}$                                       \\
      \hline
      $\msout{111}$ & $\msout{112}$ & $113$         & $\msout{114}$ & $\msout{115}$ &
      $\msout{116}$ & $\msout{117}$ & $\msout{118}$ & $\msout{119}$ &
      $\msout{120}$                                                                                                       \\
      \hline
      $\msout{121}$ & $\msout{122}$ & $\msout{123}$ & $\msout{124}$ &
      $\msout{125}$ & $\msout{126}$ & $127$         & $\msout{128}$ & $\msout{129}$ &
      $\msout{130}$                                                                                                       \\
      \hline
      $131$         & $\msout{132}$ & $\msout{133}$ & $\msout{134}$ & $\msout{135}$ &
      $\msout{136}$ & $137$         & $\msout{138}$ & $139$         & $\msout{140}$                                       \\
      \hline
      $\msout{141}$ & $\msout{142}$ & $\msout{143}$ & $\msout{144}$ &
      $\msout{145}$ & $\msout{146}$ & $\msout{147}$ & $\msout{148}$ & $149$         &
      $\msout{150}$                                                                                                       \\
      \hline
      $151$         & $\msout{152}$ & $\msout{153}$ & $\msout{154}$ & $\msout{155}$ &
      $\msout{156}$ & $157$         & $\msout{158}$ & $\msout{159}$ & $\msout{160}$                                       \\
      \hline
      $\msout{161}$ & $\msout{162}$ & $163$         & $\msout{164}$ & $\msout{165}$ &
      $\msout{166}$ & $167$         & $\msout{168}$ & $\msout{169}$ & $\msout{170}$                                       \\
      \hline
      $\msout{171}$ & $\msout{172}$ & $173$         & $\msout{174}$ & $\msout{175}$ &
      $\msout{176}$ & $\msout{177}$ & $\msout{178}$ & $179$         & $\msout{180}$                                       \\
      \hline
      $181$         & $\msout{182}$ & $\msout{183}$ & $\msout{184}$ & $\msout{185}$ &
      $\msout{186}$ & $\msout{187}$ & $\msout{188}$ & $\msout{189}$ &
      $\msout{190}$                                                                                                       \\
      \hline
      $191$         & $\msout{192}$ & $193$         & $\msout{194}$ & $\msout{195}$ &
      $\msout{196}$ & $197$         & $\msout{198}$ & $199$         & $\msout{200}$                                       \\
      \hline
    \end{tabular}
  \end{center}
\end{mdframed}
\newpage
%%%%%%%%%%%%%%%%%%%%%%%%%%%%%%%%%%%%%%%%%%%%%%%%%%%%%%%%%%%%%%%%%%%%%%%%%%%%%%%%
%%%%%%%%%%%%%%%%%%%%%%%%%%%%%%%%%%%%%%%%%%%%%%%%%%%%%%%%%%%%%%%%%%%%%%%%%%%%%%%%
%%%%%%%%%%%%%%%%%%%%%%%%%%%%%%%%%%%%%%%%%%%%%%%%%%%%%%%%%%%%%%%%%%%%%%%%%%%%%%%%
%%%%%%%%%%%%%%%%%%%%%%%%%%%%%%%%%%%%%%%%%%%%%%%%%%%%%%%%%%%%%%%%%%%%%%%%%%%%%%%%
\begin{mdframed}[style=darkQuesion]
  3. For each composite number $a$. with $4\leq a\leq 20$, find all positive
  numbers less than $a$ that are relatively prime to $a$.
\end{mdframed}

%%%%%%%%%%%%%%%%%%%%%%%%%%%%%%%%%%%%%%%%%%%%%%%%%%%%%%%%%%%%%%%%%%%%%%%%%%%%%%%%
\begin{mdframed}[style=darkAnswer,frametitle={Joe Starr}]
  \begin{multicols}{2}
    \begin{itemize}
      \item[$4:$]  {$2, 3$}
      \item[$6:$]  {$2, 3, 5$}
      \item[$8:$]  {$2, 3, 5, 7$}
      \item[$9:$]  {$2, 3, 4, 5, 7, 8$}
      \item[$10:$] {$2, 3, 5, 7, 9$}
      \item[$12:$] {$2, 3, 5, 7, 11$}
      \item[$14:$] {$2, 3, 5, 7, 9, 11, 13$}
      \item[$15:$] {$2, 3, 4, 5, 7, 8, 11, 13, 14$}
      \item[$16:$] {$2, 3, 5, 7, 9, 11, 13, 15$}
      \item[$18:$] {$2, 3, 5, 7, 11, 13, 17$}
      \item[$20:$] {$2, 3, 5, 7, 9, 11, 13, 17, 19$}
    \end{itemize}
  \end{multicols}
\end{mdframed}
\newpage
%%%%%%%%%%%%%%%%%%%%%%%%%%%%%%%%%%%%%%%%%%%%%%%%%%%%%%%%%%%%%%%%%%%%%%%%%%%%%%%%
%%%%%%%%%%%%%%%%%%%%%%%%%%%%%%%%%%%%%%%%%%%%%%%%%%%%%%%%%%%%%%%%%%%%%%%%%%%%%%%%
%%%%%%%%%%%%%%%%%%%%%%%%%%%%%%%%%%%%%%%%%%%%%%%%%%%%%%%%%%%%%%%%%%%%%%%%%%%%%%%%
%%%%%%%%%%%%%%%%%%%%%%%%%%%%%%%%%%%%%%%%%%%%%%%%%%%%%%%%%%%%%%%%%%%%%%%%%%%%%%%%
\begin{mdframed}[style=darkQuesion]
  4. Find all positive integers less than 60 and relatively prime to $60$.
\end{mdframed}

%%%%%%%%%%%%%%%%%%%%%%%%%%%%%%%%%%%%%%%%%%%%%%%%%%%%%%%%%%%%%%%%%%%%%%%%%%%%%%%%
\begin{mdframed}[style=darkAnswer,frametitle={Joe Starr}]
  \begin{itemize}
    \item[$60:$] {$2, 3, 5, 7, 11, 13, 17, 19, 23, 29, 31,
            37, 41, 43, 47, 49, 53, 59 $}
  \end{itemize}
\end{mdframed}
\newpage
%%%%%%%%%%%%%%%%%%%%%%%%%%%%%%%%%%%%%%%%%%%%%%%%%%%%%%%%%%%%%%%%%%%%%%%%%%%%%%%%
%%%%%%%%%%%%%%%%%%%%%%%%%%%%%%%%%%%%%%%%%%%%%%%%%%%%%%%%%%%%%%%%%%%%%%%%%%%%%%%%
%%%%%%%%%%%%%%%%%%%%%%%%%%%%%%%%%%%%%%%%%%%%%%%%%%%%%%%%%%%%%%%%%%%%%%%%%%%%%%%%
%%%%%%%%%%%%%%%%%%%%%%%%%%%%%%%%%%%%%%%%%%%%%%%%%%%%%%%%%%%%%%%%%%%%%%%%%%%%%%%%
\begin{mdframed}[style=darkQuesion]
  \begin{itemize}
    \item [9. (a)] {For which $n\in \Z^{\text{+}}$ is $n^{3}-1$ a prime number?}
    \item [(b)] {For which $n\in \Z^{\text{+}}$ is $n^{3}+1$ a prime number?}
    \item [(c)] {For which $n\in \Z^{\text{+}}$ is $n^{2}-1$ a prime number?}
    \item [(d)] {For which $n\in \Z^{\text{+}}$ is $n^{2}+1$ a prime number?}
  \end{itemize}
\end{mdframed}

%%%%%%%%%%%%%%%%%%%%%%%%%%%%%%%%%%%%%%%%%%%%%%%%%%%%%%%%%%%%%%%%%%%%%%%%%%%%%%%%
\begin{mdframed}[style=darkAnswer,frametitle={Joe Starr}]
  \begin{itemize}
    \item [(a)] {We can factor $n^{3}-1$ into $(n - 1) (n^2 + n + 1)$. We
          have then $n-1\vert n^{3}-1$, for $n^{3}-1$ to be prime $n-1$ must be
          $1$. This happens only for $n=2$.
          }
    \item [(b)] {We can factor $n^{3}+1$ into $(n + 1) (n^2 - n + 1)$. We
          have then $(n^2 - n + 1)\vert n^{3}+1$,
          for $n^{3}+1$ to be prime $(n^2 - n + 1)$ must be $1$.
          This happens only for $n=1$. }
    \item [(c)] {We can factor $n^{2}-1$ into $(n - 1) (n + 1)$. We
          have then $(n 1 1)\vert n^{2}-1$,
          for $n^{2}-1$ to be prime $(n - 1)$ must be $1$.
          This happens only for $n=2$.
          For which $n\in \Z^{\text{+}}$ is $n^{2}-1$ a prime number?}
    \item [(d)] {????
          }
  \end{itemize}
\end{mdframed}

\newpage
%%%%%%%%%%%%%%%%%%%%%%%%%%%%%%%%%%%%%%%%%%%%%%%%%%%%%%%%%%%%%%%%%%%%%%%%%%%%%%%%
%%%%%%%%%%%%%%%%%%%%%%%%%%%%%%%%%%%%%%%%%%%%%%%%%%%%%%%%%%%%%%%%%%%%%%%%%%%%%%%%
%%%%%%%%%%%%%%%%%%%%%%%%%%%%%%%%%%%%%%%%%%%%%%%%%%%%%%%%%%%%%%%%%%%%%%%%%%%%%%%%
%%%%%%%%%%%%%%%%%%%%%%%%%%%%%%%%%%%%%%%%%%%%%%%%%%%%%%%%%%%%%%%%%%%%%%%%%%%%%%%%
\begin{mdframed}[style=darkQuesion]
  11. Prove that $n^4+4^n$ is composite if $n>1$.
\end{mdframed}

%%%%%%%%%%%%%%%%%%%%%%%%%%%%%%%%%%%%%%%%%%%%%%%%%%%%%%%%%%%%%%%%%%%%%%%%%%%%%%%%
\begin{mdframed}[style=darkAnswer,frametitle={Joe Starr}]
  We are presented with two potability's, $n$ is even or $n$ is odd.
  \begin{itemize}[align=left]
    \item [$n$ even]{\hspace{.5in}\newline
          It's obvious that $n^4+4^n$ is an even not $2$ and can't be prime.
          }
    \item [$n$ odd]{\hspace{.5in}\newline
          We begin by completing the square
          \begin{align*}
            n^4+4^n & = n^4+4^n                 \\
                    & =\lrp{n^2}^2+\lrp{2^n}^2  \\
                    & =\lrp{n^2+2^n}^2-2n^2 2^n
          \end{align*}
          We from here we observe that $2^n2=2^{n+1}$, since $n$ is odd $n+1$ is even
          yielding $2^{n+1}=2^{2k}$. We can see we have a difference of squares
          \begin{align*}
            \lrp{n^2+2^n}^2-2n^2 2^n & =\lrp{n^2+2^n}^2-\lrp{2^nn}^2         \\
                                     & =\lrp{n^2+2^n+2^nn}\lrp{n^2+2^n-2^nn}
          \end{align*}
          since we are restricted to $n>1$ we can see that both $\lrp{n^2+2^n+2^nn}>1$
          and $\lrp{n^2+2^n-2^nn}>1$ for all $n$. Making $n^4+4^n$ composite as
          desired.
          }
  \end{itemize}
\end{mdframed}
\newpage
%%%%%%%%%%%%%%%%%%%%%%%%%%%%%%%%%%%%%%%%%%%%%%%%%%%%%%%%%%%%%%%%%%%%%%%%%%%%%%%%
%%%%%%%%%%%%%%%%%%%%%%%%%%%%%%%%%%%%%%%%%%%%%%%%%%%%%%%%%%%%%%%%%%%%%%%%%%%%%%%%
%%%%%%%%%%%%%%%%%%%%%%%%%%%%%%%%%%%%%%%%%%%%%%%%%%%%%%%%%%%%%%%%%%%%%%%%%%%%%%%%
%%%%%%%%%%%%%%%%%%%%%%%%%%%%%%%%%%%%%%%%%%%%%%%%%%%%%%%%%%%%%%%%%%%%%%%%%%%%%%%%
\begin{mdframed}[style=darkQuesion]
  13. Let $a,b,c$  be positive integers, and let $d=\lrp{a,b}$. Since $d\vert a$,
  there exists an integer $h$ with $a=dh$. Show that $a\vert bc$, then $h\vert c$.
\end{mdframed}

%%%%%%%%%%%%%%%%%%%%%%%%%%%%%%%%%%%%%%%%%%%%%%%%%%%%%%%%%%%%%%%%%%%%%%%%%%%%%%%%
\begin{mdframed}[style=darkAnswer,frametitle={Joe Starr}]
  We will proceed with a transitive proof:
  \begin{align*}
    a\vert abc & \rightarrow a\vert \lrp{a,b}c \\
               & \rightarrow a\vert dc         \\
               & \rightarrow dh\vert dc        \\
               & \rightarrow h\vert c
  \end{align*}
\end{mdframed}
\newpage
%%%%%%%%%%%%%%%%%%%%%%%%%%%%%%%%%%%%%%%%%%%%%%%%%%%%%%%%%%%%%%%%%%%%%%%%%%%%%%%%
%%%%%%%%%%%%%%%%%%%%%%%%%%%%%%%%%%%%%%%%%%%%%%%%%%%%%%%%%%%%%%%%%%%%%%%%%%%%%%%%
%%%%%%%%%%%%%%%%%%%%%%%%%%%%%%%%%%%%%%%%%%%%%%%%%%%%%%%%%%%%%%%%%%%%%%%%%%%%%%%%
%%%%%%%%%%%%%%%%%%%%%%%%%%%%%%%%%%%%%%%%%%%%%%%%%%%%%%%%%%%%%%%%%%%%%%%%%%%%%%%%
\begin{mdframed}[style=darkQuesion]
  14. Show that $a\Z \cap b\Z=\lrb{a,b}$.
\end{mdframed}

%%%%%%%%%%%%%%%%%%%%%%%%%%%%%%%%%%%%%%%%%%%%%%%%%%%%%%%%%%%%%%%%%%%%%%%%%%%%%%%%
\begin{mdframed}[style=darkAnswer,frametitle={Joe Starr}]
  Let $x\in \lrp{a\Z \cap b\Z}$, since $x\in a\Z$ we have $x=aq_1$, similarly
  since $x\in b\Z$ we have$x=bq_2$. We can see that $x=abq$, this means $x$
  is a multiple of $\lrb{a,b}$ putting $x\in \lrb{a,b}$. Next, we let
  $x\in \lrb{a,b}\Z$, this means $x$ is of the form $x=\lrb{a,b}q$. We can see
  that $a\vert x$ and $b\vert x$ since $a\vert \lrb{a,b}$, This makes
  $x\in a\Z$ and $x\in b\Z$, as desired.
\end{mdframed}
\newpage
%%%%%%%%%%%%%%%%%%%%%%%%%%%%%%%%%%%%%%%%%%%%%%%%%%%%%%%%%%%%%%%%%%%%%%%%%%%%%%%%
%%%%%%%%%%%%%%%%%%%%%%%%%%%%%%%%%%%%%%%%%%%%%%%%%%%%%%%%%%%%%%%%%%%%%%%%%%%%%%%%
%%%%%%%%%%%%%%%%%%%%%%%%%%%%%%%%%%%%%%%%%%%%%%%%%%%%%%%%%%%%%%%%%%%%%%%%%%%%%%%%
%%%%%%%%%%%%%%%%%%%%%%%%%%%%%%%%%%%%%%%%%%%%%%%%%%%%%%%%%%%%%%%%%%%%%%%%%%%%%%%%
\begin{mdframed}[style=darkQuesion]
  17. Let $a,b$ be nonzero integers. Prove $\lrp{a,b}=1$ if and only if
  $\lrp{a+b,ab}=1$.
\end{mdframed}

%%%%%%%%%%%%%%%%%%%%%%%%%%%%%%%%%%%%%%%%%%%%%%%%%%%%%%%%%%%%%%%%%%%%%%%%%%%%%%%%
\begin{mdframed}[style=darkAnswer,frametitle={Joe Starr}]
  \begin{itemize}[align=left]
    \item [$\Rightarrow$]{
          We let $\lrp{a,b}=1$, then consider the $\lrp{a+b,ab}$. We assume
          $\lrp{a+b,ab}=d$, with $d>1$. Since $d>1$ there must exist $p$ a prime such that
          $p\vert d$. This means that $p\vert a+b$ and $p \vert ab$. Consequently, either
          $p\vert a$ or $p\vert b$. WOLG we have $p\vert a$, and since $p\vert a+b$ it
          must be that $p\vert b$. Finally, since $p\vert a $ and $p\vert b$,
          $p\vert \lrp{a,b}$ a contradiction. So $\lrp{a+b,ab}=1$.
          }
    \item [$\Leftarrow$]{
          We let $\lrp{a+b,ab}=1$, then consider the $\lrp{a,b}$. We assume
          $\lrp{a,b}=d$, with $d>1$. Since $d>1$ there must exist $p$ a prime such that
          $p\vert d$. This means that $p\vert a$ and $p\vert b$, further $p \vert ab$.
          Since $p$ divides $a$ and $b$, we have $p\vert a+b$. Finally, since $p\vert ab$,
          and $p\vert a+b$, $p\vert \lrp{a+b,ab}$, a contradiction so $\lrp{a,b}=1$.
          }
  \end{itemize}
\end{mdframed}
\newpage
%%%%%%%%%%%%%%%%%%%%%%%%%%%%%%%%%%%%%%%%%%%%%%%%%%%%%%%%%%%%%%%%%%%%%%%%%%%%%%%%
%%%%%%%%%%%%%%%%%%%%%%%%%%%%%%%%%%%%%%%%%%%%%%%%%%%%%%%%%%%%%%%%%%%%%%%%%%%%%%%%
%%%%%%%%%%%%%%%%%%%%%%%%%%%%%%%%%%%%%%%%%%%%%%%%%%%%%%%%%%%%%%%%%%%%%%%%%%%%%%%%
%%%%%%%%%%%%%%%%%%%%%%%%%%%%%%%%%%%%%%%%%%%%%%%%%%%%%%%%%%%%%%%%%%%%%%%%%%%%%%%%
\begin{mdframed}[style=darkQuesion]
  18. Let $a,b$ be nonzero integers with $\lrp{a,b}=1$. Compute $\lrp{a+b,a-b}$.
\end{mdframed}

%%%%%%%%%%%%%%%%%%%%%%%%%%%%%%%%%%%%%%%%%%%%%%%%%%%%%%%%%%%%%%%%%%%%%%%%%%%%%%%%
\begin{mdframed}[style=darkAnswer,frametitle={Joe Starr}]
  We know that $d=\lrp{a+b,a-b}$, this means that $d\vert a+b$ and $d\vert a-b$.
  From here we have that $d\vert \lrp{a+b}+\lrp{a-b} \rightarrow d\vert 2a$ and
  $d\vert \lrp{a+b}-\lrp{a-b} \rightarrow d\vert 2b$. Since $d$ divides both $2a$
  and $2b$, $d$ must also divide $2\lrp{a,b}$. Since $\lrp{a,b}=1$ we have
  $\lrp{a+b,a-b}=2$.
\end{mdframed}
\newpage
%%%%%%%%%%%%%%%%%%%%%%%%%%%%%%%%%%%%%%%%%%%%%%%%%%%%%%%%%%%%%%%%%%%%%%%%%%%%%%%%
%%%%%%%%%%%%%%%%%%%%%%%%%%%%%%%%%%%%%%%%%%%%%%%%%%%%%%%%%%%%%%%%%%%%%%%%%%%%%%%%
%%%%%%%%%%%%%%%%%%%%%%%%%%%%%%%%%%%%%%%%%%%%%%%%%%%%%%%%%%%%%%%%%%%%%%%%%%%%%%%%
%%%%%%%%%%%%%%%%%%%%%%%%%%%%%%%%%%%%%%%%%%%%%%%%%%%%%%%%%%%%%%%%%%%%%%%%%%%%%%%%
\begin{mdframed}[style=darkQuesion]
  19. Let $a$ and $b$ be positive integers, and let $m$ be an integer such that
  $ab=m\lrp{a,b}$. Without using the prime factorization theorem, prove that
  $\lrp{a,b}\lrb{a,b}=ab$ by verifying that $m$ satisfies the necessary properties
  of $\lrb{a,b}$.
\end{mdframed}

%%%%%%%%%%%%%%%%%%%%%%%%%%%%%%%%%%%%%%%%%%%%%%%%%%%%%%%%%%%%%%%%%%%%%%%%%%%%%%%%
\begin{mdframed}[style=darkAnswer,frametitle={Joe Starr}]
  We let $d=\lrp{a,b}$, this means that $ab=md$. We first show $a|m$ and $b|m$,
  \begin{align*}
    ab=md & \rightarrow a\lrp{dq}=md                \\
          & \rightarrow adq-md=0                    \\
          & \rightarrow d(aq-m)=0    & (by def d>0) \\
          & \rightarrow aq=m                        \\
          & \rightarrow a|m                         \\
  \end{align*}
  similarly for  $b$. \\
  Next we will show that if $a|c$ and $b|c$ then $m|c$. We have that $c=aq_1=bq_2$
  or $c^2=abq$. We can multiply $ab=md$ by $q$ giving $abq=mdq$, this means we
  have $c^2=mdq$, which is true only if $c=mdq$, $m|c$ as desired.
\end{mdframed}
\newpage
%%%%%%%%%%%%%%%%%%%%%%%%%%%%%%%%%%%%%%%%%%%%%%%%%%%%%%%%%%%%%%%%%%%%%%%%%%%%%%%%
%%%%%%%%%%%%%%%%%%%%%%%%%%%%%%%%%%%%%%%%%%%%%%%%%%%%%%%%%%%%%%%%%%%%%%%%%%%%%%%%
%%%%%%%%%%%%%%%%%%%%%%%%%%%%%%%%%%%%%%%%%%%%%%%%%%%%%%%%%%%%%%%%%%%%%%%%%%%%%%%%
%%%%%%%%%%%%%%%%%%%%%%%%%%%%%%%%%%%%%%%%%%%%%%%%%%%%%%%%%%%%%%%%%%%%%%%%%%%%%%%%
\begin{mdframed}[style=darkQuesion]
  20. A positive integer $a$ is called a square if $a=n^2$ for some $n\in \Z$.
  Show that the integer $a>1$ is a square if and only if every exponent in its
  prime factorization is even.
\end{mdframed}

%%%%%%%%%%%%%%%%%%%%%%%%%%%%%%%%%%%%%%%%%%%%%%%%%%%%%%%%%%%%%%%%%%%%%%%%%%%%%%%%
\begin{mdframed}[style=darkAnswer,frametitle={Joe Starr}]
  Let $a\in \Z$ be a square. Since $a$ is a square by definiton there exists
  a $n$ such that $nn=a$. Now by the fundamental theorem of arithmetic we know $n$
  has a prime factorization,written $p_1^{n_1}\cdots p_k^{n_k}$. If we consider
  $nn$, we have $nn = \lrp{p_1^{n_1}\cdots p_k^{n_k}}
    \lrp{p_1^{n_1}\cdots p_k^{n_k}}$
  , by combining terms we can see that $nn= \lrp{p_1^{2n_1}\cdots p_k^{2n_k}}$, as
  desired.
\end{mdframed}
\newpage
%%%%%%%%%%%%%%%%%%%%%%%%%%%%%%%%%%%%%%%%%%%%%%%%%%%%%%%%%%%%%%%%%%%%%%%%%%%%%%%%
%%%%%%%%%%%%%%%%%%%%%%%%%%%%%%%%%%%%%%%%%%%%%%%%%%%%%%%%%%%%%%%%%%%%%%%%%%%%%%%%
%%%%%%%%%%%%%%%%%%%%%%%%%%%%%%%%%%%%%%%%%%%%%%%%%%%%%%%%%%%%%%%%%%%%%%%%%%%%%%%%
%%%%%%%%%%%%%%%%%%%%%%%%%%%%%%%%%%%%%%%%%%%%%%%%%%%%%%%%%%%%%%%%%%%%%%%%%%%%%%%%
\begin{mdframed}[style=darkQuesion]
  23. Let $p$ and $q$ be prime numbers. Prove that $pq+1$ is a square if and only
  if $p$ and $q$ are twin primes.
\end{mdframed}

%%%%%%%%%%%%%%%%%%%%%%%%%%%%%%%%%%%%%%%%%%%%%%%%%%%%%%%%%%%%%%%%%%%%%%%%%%%%%%%%
\begin{mdframed}[style=darkAnswer,frametitle={Joe Starr}]
  We begin by letting selecting $p$ a prime and $q$ a prime such that $q=p+2$.
  Now we consider $pq$,
  \begin{align*}
    pq & \rightarrow p\lrp{p+2} \\
       & \rightarrow p^2+p2     \\
  \end{align*}
  We now consider $p+1$, if we take $\lrp{p+1}^2$, we get $p^2+2p+1$. It's
  obvious that $pq+1=p^2+p2+1=\lrp{p+1}^2$, so $pq+1$ is a square when $p$ and $q$ are
  twin primes.
  We can now consider $p$ a prime, and $q$ a prime such that $q=p+n$ with
  $n>2$. If we calculate $pq$ we see that,
  \begin{align*}
    pq & \rightarrow p\lrp{p+n} \\
       & \rightarrow p^2+pn     \\
  \end{align*}
  we then have that $pq+1=p^2+pn+1$ with $n>2$, this is not a square, showing
  when $p$ and $q$ aren't twin primes $pq+1$ is not a square.
\end{mdframed}
\newpage
%%%%%%%%%%%%%%%%%%%%%%%%%%%%%%%%%%%%%%%%%%%%%%%%%%%%%%%%%%%%%%%%%%%%%%%%%%%%%%%%
%%%%%%%%%%%%%%%%%%%%%%%%%%%%%%%%%%%%%%%%%%%%%%%%%%%%%%%%%%%%%%%%%%%%%%%%%%%%%%%%
%%%%%%%%%%%%%%%%%%%%%%%%%%%%%%%%%%%%%%%%%%%%%%%%%%%%%%%%%%%%%%%%%%%%%%%%%%%%%%%%
%%%%%%%%%%%%%%%%%%%%%%%%%%%%%%%%%%%%%%%%%%%%%%%%%%%%%%%%%%%%%%%%%%%%%%%%%%%%%%%%
\begin{mdframed}[style=darkQuesion]
  26. Prove that if $a>1$, then there is a prime $p$ with $a<p\leq a!+1$.
\end{mdframed}

%%%%%%%%%%%%%%%%%%%%%%%%%%%%%%%%%%%%%%%%%%%%%%%%%%%%%%%%%%%%%%%%%%%%%%%%%%%%%%%%
\begin{mdframed}[style=darkAnswer,frametitle={Joe Starr}]
  We observe that $a!+1$ is either prime or composite, if $ a!+1$ is prime
  we are done, if $ a!+1$ is composite we know by the fundamental theorem
  of arithmetic that $ a!+1$ has prime factors. Now if all prime factors $p$
  are such that $p\leq a$ since $p\vert a!$ we see that if we divide $a!+1$ by
  any of these we get a remainder of $1$, a contradiction so there must be
  a prime factor $p$ with $a<p$.

  \textit{Note: this is basically the same argument as euclid's proof of
    infinite primes}
\end{mdframed}
\newpage
%%%%%%%%%%%%%%%%%%%%%%%%%%%%%%%%%%%%%%%%%%%%%%%%%%%%%%%%%%%%%%%%%%%%%%%%%%%%%%%%
%%%%%%%%%%%%%%%%%%%%%%%%%%%%%%%%%%%%%%%%%%%%%%%%%%%%%%%%%%%%%%%%%%%%%%%%%%%%%%%%
%%%%%%%%%%%%%%%%%%%%%%%%%%%%%%%%%%%%%%%%%%%%%%%%%%%%%%%%%%%%%%%%%%%%%%%%%%%%%%%%
%%%%%%%%%%%%%%%%%%%%%%%%%%%%%%%%%%%%%%%%%%%%%%%%%%%%%%%%%%%%%%%%%%%%%%%%%%%%%%%%
\begin{mdframed}[style=darkQuesion]
  29. Show that $\log{2}/\log{3}$ is not a rational number.
\end{mdframed}

%%%%%%%%%%%%%%%%%%%%%%%%%%%%%%%%%%%%%%%%%%%%%%%%%%%%%%%%%%%%%%%%%%%%%%%%%%%%%%%%
\begin{mdframed}[style=darkAnswer,frametitle={Joe Starr}]
  We observe this is an application of the change of base formula, making
  $\frac{\log{2}}{\log{3}}=\log_{3}2$. From here we have
  $x=\log_{3}2 \rightarrow 3^x=2$, if $x$ is rational then there exist $m$ and
  $n$ such that $\frac{m}{n}=x$. We now have
  $3^{\frac{m}{n}}=2\rightarrow 3^m=2^n $, a contradiction since there is no
  $m$ and $n$ that satisfy this equivalence, making $\log{2}/\log{3}$
  irrational as desired.
\end{mdframed}



\clearpage
%%%%%%%%%%%%%%%%%%%%%%%%%%%%%%%%%%%%%%%%%%%%%%%%%%%%%%%%%%%%%%%%%%%%%%%%%%%%%%%%
%Compile Chapter 2

\section{Functions}
\subsection{Functions}
%%%%%%%%%%%%%%%%%%%%%%%%%%%%%%%%%%%%%%%%%%%%%%%%%%%%%%%%%%%%%%%%%%%%%%%%%%%%%%%%
%%%%%%%%%%%%%%%%%%%%%%%%%%%%%%%%%%%%%%%%%%%%%%%%%%%%%%%%%%%%%%%%%%%%%%%%%%%%%%%%
%%%%%%%%%%%%%%%%%%%%%%%%%%%%%%%%%%%%%%%%%%%%%%%%%%%%%%%%%%%%%%%%%%%%%%%%%%%%%%%%
%%%%%%%%%%%%%%%%%%%%%%%%%%%%%%%%%%%%%%%%%%%%%%%%%%%%%%%%%%%%%%%%%%%%%%%%%%%%%%%%
\begin{mdframed}[style=darkQuesion]
  1.   In each of the following parts, determine whether the given function is
  1:1 and whether it is onto.
  \begin{itemize}
    \item [(a)]{
          $f:\R\to \R; \fof{x}=x+3$
          }
    \item [(b)]{
          $f:\C\to \C; \fof{x}=x^2+2x+1$
          }
    \item [(c]{
          $f:\Z_n\to \Z_n; \fof{\lrb{x}_n}=\lrb{mx+b}_n, \text{where} m,b\in \Z$
          }
    \item [(d)]{
          $f:\R^+\to \R; \fof{x}=\ln x$
          }
  \end{itemize}
\end{mdframed}

%%%%%%%%%%%%%%%%%%%%%%%%%%%%%%%%%%%%%%%%%%%%%%%%%%%%%%%%%%%%%%%%%%%%%%%%%%%%%%%%
\begin{mdframed}[style=darkAnswer,frametitle={Joe Starr}]
  \begin{itemize}
    \item [(a)]{ We can see that $\fof{x}=x+3$ then $\fofinv{x}=x-3$,
          $\fof{\fofinv{x}}=\lrp{x-3}+3=x$. Showing $f$ is a bijection.
          }
    \item [(b)]{
          \begin{itemize}
            \item[1:1:]{\hspace{.5in}\newline
                  Assume $\fof{x}=25=\fof{y}$, we can see that if $x=4$, $\fof{x}=25$,
                  and $y=-6$, $\fof{y}=25$, showing $f$ not injective.
                  }
            \item[onto:]{\hspace{.5in}\newline
                  Let $y\in \C$ we must now show there exists a $x \in \C$ such that
                  $\fof{x}=y$. Consider $x=\sqrt{y}-1$, we can then take:
                  \begin{align*}
                    \fof{x} & = x^2+2x+1                               \\
                            & = \lrp{\sqrt{y}-1}^2+2\lrp{\sqrt{y}-1}+1 \\
                            & = \lrp{\sqrt{y}-1}^2+2\sqrt{y}-2+1       \\
                            & = 1 - 2 \sqrt{y} + y+2\sqrt{y}-1         \\
                            & =  y                                     \\
                  \end{align*}
                  showing $f$ surjective.
                  }
          \end{itemize}
          }
    \item [(c)]{
          Consider $\fofinv{x}=\lrb{\lrp{y-b}m^{-1}}_n$, now taking
          $\fof{\fofinv{x}}$
          \begin{align*}
            \fof{\fofinv{x}} & = \lrb{m\lrp{x-b}m^{-1}+b}_n \\
                             & = \lrb{\lrp{x-b}+b}_n        \\
                             & = \lrb{x}_n                  \\
          \end{align*}
          showing $f$ a bijection.
          }
    \item [(d)]{
          $f:\R^+\to \R; \fof{x}=\ln x$
          If we take $\fofinv{x}=e^x$, $\fof{\fofinv{x}}=\ln e^x = x$, showing $f$
          a bijection.
          }
  \end{itemize}
\end{mdframed}
\newpage
%%%%%%%%%%%%%%%%%%%%%%%%%%%%%%%%%%%%%%%%%%%%%%%%%%%%%%%%%%%%%%%%%%%%%%%%%%%%%%%%
%%%%%%%%%%%%%%%%%%%%%%%%%%%%%%%%%%%%%%%%%%%%%%%%%%%%%%%%%%%%%%%%%%%%%%%%%%%%%%%%
%%%%%%%%%%%%%%%%%%%%%%%%%%%%%%%%%%%%%%%%%%%%%%%%%%%%%%%%%%%%%%%%%%%%%%%%%%%%%%%%
%%%%%%%%%%%%%%%%%%%%%%%%%%%%%%%%%%%%%%%%%%%%%%%%%%%%%%%%%%%%%%%%%%%%%%%%%%%%%%%%
\begin{mdframed}[style=darkQuesion]
  3.   For each 1:1 and onto function in Exercise 1, find the inverse of the
  function
  \begin{itemize}
    \item [(a)]{
          $f:\R\to \R; \fof{x}=x+3$
          }
    \item [(b)]{
          $f:\C\to \C; \fof{x}=x^2+2x+1$
          }
    \item [(c]{
          $f:\Z_n\to \Z_n; \fof{\lrb{x}_n}=\lrb{mx+b}_n, \text{where} m,b\in \Z$
          }
    \item [(d)]{
          $f:\R^+\to \R; \fof{x}=\ln x$
          }
  \end{itemize}
\end{mdframed}

%%%%%%%%%%%%%%%%%%%%%%%%%%%%%%%%%%%%%%%%%%%%%%%%%%%%%%%%%%%%%%%%%%%%%%%%%%%%%%%%
\begin{mdframed}[style=darkAnswer,frametitle={Joe Starr}]
  \begin{multicols}{2}
    \begin{itemize}
      \item [(a)]{
            see question 1
            }
      \item [(b)]{
            not a bijection
            }
      \item [(c]{
            see question 1
            }
      \item [(d)]{
            see question 1
            }
    \end{itemize}
  \end{multicols}
\end{mdframed}
\newpage
%%%%%%%%%%%%%%%%%%%%%%%%%%%%%%%%%%%%%%%%%%%%%%%%%%%%%%%%%%%%%%%%%%%%%%%%%%%%%%%%
%%%%%%%%%%%%%%%%%%%%%%%%%%%%%%%%%%%%%%%%%%%%%%%%%%%%%%%%%%%%%%%%%%%%%%%%%%%%%%%%
%%%%%%%%%%%%%%%%%%%%%%%%%%%%%%%%%%%%%%%%%%%%%%%%%%%%%%%%%%%%%%%%%%%%%%%%%%%%%%%%
%%%%%%%%%%%%%%%%%%%%%%%%%%%%%%%%%%%%%%%%%%%%%%%%%%%%%%%%%%%%%%%%%%%%%%%%%%%%%%%%
\begin{mdframed}[style=darkQuesion]
  4.   For each 1:1 and onto function in Exercise 2, find the inverse of the
  function
  \begin{itemize}
    \item [(a)]{
          $f:\R\to \R; \fof{x}=x^2$
          }
    \item [(b)]{
          $f:\C\to \C; \fof{x}=x^2$
          }
    \item [(c]{
          $f:\R^+\to \R^+; \fof{x}=x^2$
          }
    \item [(d)]{
          $f:\R^+\to \R^+; \fof{x}=
            \begin{cases}
              x   & \text{if }x\text{ is rational}   \\
              x^2 & \text{if }x\text{ is irrational} \\
            \end{cases}$
          }
  \end{itemize}
\end{mdframed}

%%%%%%%%%%%%%%%%%%%%%%%%%%%%%%%%%%%%%%%%%%%%%%%%%%%%%%%%%%%%%%%%%%%%%%%%%%%%%%%%
\begin{mdframed}[style=darkAnswer,frametitle={Joe Starr}]
  \begin{itemize}
    \item [(a)]{
          Not a bijection
          }
    \item [(b)]{
          Not a bijection
          }
    \item [(c]{
          $\fofinv{x}=+\sqrt{x}$
          }
    \item [(d)]{
          $f:\R^+\to \R^+; \fofinv{x}=
            \begin{cases}
              x         & \text{if }x\text{ is rational}   \\
              +\sqrt{x} & \text{if }x\text{ is irrational} \\
            \end{cases}$
          }
  \end{itemize}
\end{mdframed}
\newpage
%%%%%%%%%%%%%%%%%%%%%%%%%%%%%%%%%%%%%%%%%%%%%%%%%%%%%%%%%%%%%%%%%%%%%%%%%%%%%%%%
%%%%%%%%%%%%%%%%%%%%%%%%%%%%%%%%%%%%%%%%%%%%%%%%%%%%%%%%%%%%%%%%%%%%%%%%%%%%%%%%
%%%%%%%%%%%%%%%%%%%%%%%%%%%%%%%%%%%%%%%%%%%%%%%%%%%%%%%%%%%%%%%%%%%%%%%%%%%%%%%%
%%%%%%%%%%%%%%%%%%%%%%%%%%%%%%%%%%%%%%%%%%%%%%%%%%%%%%%%%%%%%%%%%%%%%%%%%%%%%%%%
\begin{mdframed}[style=darkQuesion]
  13. Let $f:A\to B$ be a function, and let $\fof{A}=\lrs{\fof{a}\vert a\in A}$
  be the image of $f$. Show that $f$is onto if and only if $\fof{A}=B$.
\end{mdframed}

%%%%%%%%%%%%%%%%%%%%%%%%%%%%%%%%%%%%%%%%%%%%%%%%%%%%%%%%%%%%%%%%%%%%%%%%%%%%%%%%
\begin{mdframed}[style=darkAnswer,frametitle={Joe Starr}]
  Let $\fof{A}=B$, select $y\in B$, since $y\in B$ we have $y\in \fof{A}$, that
  means there exists $a\in A$ such that $\fof{a}=y$. Showing $f$ surjective.
  Let $\fof{A}\neq B$, let $y\in B$, such that $y\notin B \cap \fof{A}$. Since
  we
  have $y\notin \fof{A}$ we have no $a\in A$ that maps to $y$ showing $f$ not
  surjective, as desired.
\end{mdframed}
\newpage
%%%%%%%%%%%%%%%%%%%%%%%%%%%%%%%%%%%%%%%%%%%%%%%%%%%%%%%%%%%%%%%%%%%%%%%%%%%%%%%%
%%%%%%%%%%%%%%%%%%%%%%%%%%%%%%%%%%%%%%%%%%%%%%%%%%%%%%%%%%%%%%%%%%%%%%%%%%%%%%%%
%%%%%%%%%%%%%%%%%%%%%%%%%%%%%%%%%%%%%%%%%%%%%%%%%%%%%%%%%%%%%%%%%%%%%%%%%%%%%%%%
%%%%%%%%%%%%%%%%%%%%%%%%%%%%%%%%%%%%%%%%%%%%%%%%%%%%%%%%%%%%%%%%%%%%%%%%%%%%%%%%
\begin{mdframed}[style=darkQuesion]
  15. Let $f:A\to B$ and $g:B\to C$ be functions. Prove that if $g\circ f$ is
  1:1, then $f$ is 1:1, and that if $g\circ f$ is onto $g$ is onto.
\end{mdframed}

%%%%%%%%%%%%%%%%%%%%%%%%%%%%%%%%%%%%%%%%%%%%%%%%%%%%%%%%%%%%%%%%%%%%%%%%%%%%%%%%
\begin{mdframed}[style=darkAnswer,frametitle={Joe Starr}]
  Let $g\circ f$ be injective, but $f$ not injective. Since $f$ is not injective
  $\exists a,x\in A$ such tha $a\neq x$ but $\fof{x}=\fof{a}$. We consider
  $g\circ f\lrp{a}$ and $g\circ f\lrp{x}$, since $\fof{x}=\fof{a}$ it must be
  that $\gof{\fof{x}}=\gof{\fof{a}}$. This means that with $a\neq b$,
  $\gof{\fof{x}}=\gof{\fof{a}}$, making $g\circ f$ not injective a contradiction
  so $f$ injective.

  Let $g\circ f$ be surjective, but $g$ not surjective. Since $g$ not surjective
  there exists some $c\in C$ such that $\gof{b}\neq c$ for all $b\in B$. However
  since $g\circ f$ surjective there exists $g\circ f\lrp{a}=c$ a contradiction,
  making $g$ surjective.
\end{mdframed}
\newpage
%%%%%%%%%%%%%%%%%%%%%%%%%%%%%%%%%%%%%%%%%%%%%%%%%%%%%%%%%%%%%%%%%%%%%%%%%%%%%%%%
%%%%%%%%%%%%%%%%%%%%%%%%%%%%%%%%%%%%%%%%%%%%%%%%%%%%%%%%%%%%%%%%%%%%%%%%%%%%%%%%
%%%%%%%%%%%%%%%%%%%%%%%%%%%%%%%%%%%%%%%%%%%%%%%%%%%%%%%%%%%%%%%%%%%%%%%%%%%%%%%%
%%%%%%%%%%%%%%%%%%%%%%%%%%%%%%%%%%%%%%%%%%%%%%%%%%%%%%%%%%%%%%%%%%%%%%%%%%%%%%%%
\begin{mdframed}[style=darkQuesion]
  17. Let $f:A\to B$ be a function. Prove that $f$ is onto if and only if
  $h\circ f = k \circ f$ implies $h=k$, for every set $C$ and all choices of
  functions $h:B\to C$ and $k:B\to C$.
\end{mdframed}

%%%%%%%%%%%%%%%%%%%%%%%%%%%%%%%%%%%%%%%%%%%%%%%%%%%%%%%%%%%%%%%%%%%%%%%%%%%%%%%%
\begin{mdframed}[style=darkAnswer,frametitle={Joe Starr}]
  Assume $f$ is surjective, that is for all $y\in B$ there exists $x\in A$ such
  that $\fof{x}=y$. We know $\hof{\fof{x}}=\kof{\fof{x}}$, so
  $\hof{y}=\kof{y}$ for all $y$ in the domain of $f$ as desired.

  Next assume $f$ is not surjective, then for some $y\in B$ there exists no $x$
  such that $\fof{x}=y$. We can select $C=\lrs{a,b}$. We say that $\hof{c}=a$
  now we construct $k$,
  $$\kof{x}=\begin{cases}
      b & \text{if }x=y     \\
      a & \text{if }x\neq y \\
    \end{cases}$$
  from here we have that when the input of $h$ and $k$ are in the domain of $f$
  $h\neq k$, as desired.
\end{mdframed}
\newpage
%%%%%%%%%%%%%%%%%%%%%%%%%%%%%%%%%%%%%%%%%%%%%%%%%%%%%%%%%%%%%%%%%%%%%%%%%%%%%%%%
%%%%%%%%%%%%%%%%%%%%%%%%%%%%%%%%%%%%%%%%%%%%%%%%%%%%%%%%%%%%%%%%%%%%%%%%%%%%%%%%
%%%%%%%%%%%%%%%%%%%%%%%%%%%%%%%%%%%%%%%%%%%%%%%%%%%%%%%%%%%%%%%%%%%%%%%%%%%%%%%%
%%%%%%%%%%%%%%%%%%%%%%%%%%%%%%%%%%%%%%%%%%%%%%%%%%%%%%%%%%%%%%%%%%%%%%%%%%%%%%%%
\begin{mdframed}[style=darkQuesion]
  19. Let $f:A\to B$ be a function. Prove that $f$ is 1:1 if and only if
  $f\circ h=f\circ k$ implies $h=k$, for every set $C$ and all choices of
  functions $h:C\to A$ and $k:C\to A$.
\end{mdframed}

%%%%%%%%%%%%%%%%%%%%%%%%%%%%%%%%%%%%%%%%%%%%%%%%%%%%%%%%%%%%%%%%%%%%%%%%%%%%%%%%
\begin{mdframed}[style=darkAnswer,frametitle={Joe Starr}]
  Assume that $f$ is injective, this means that $\fof{x}=\fof{y}$ implies $x=y$.
  Also assume $f\circ h=f\circ k$ for some $h,k$. We assume $h\neq k$ for some
  $c\in C$. Since we know $f$ is injective we have
  $\fof{\hof{c}}\neq \fof{\kof{c}}$, a contradiction from our assumption that
  $f\circ h=f\circ k$ meaning $h=k$ as desired.

  Assume $f\circ h=f\circ k$ implies $h=k$. suppose $f$ be not injective, this means
  that there exists some $x,y$ such that $\fof{x}=\fof{y}$ but $x\neq y$.
  We can select $C=\lrs{a,b}$, and define $h$, $k$:
  \begin{align*}
    \kof{a}=x & \hspace{.5in}  \hof{a}=y \\
    \kof{b}=y & \hspace{.5in} \hof{b}=x
  \end{align*}
  We can see from here we have that for $\fof{\hof{a}}=\fof{\kof{a}}$ but
  $h\neq k$ a contradiction making $f$ injective as desired.
\end{mdframed}
\newpage
%%%%%%%%%%%%%%%%%%%%%%%%%%%%%%%%%%%%%%%%%%%%%%%%%%%%%%%%%%%%%%%%%%%%%%%%%%%%%%%%
%%%%%%%%%%%%%%%%%%%%%%%%%%%%%%%%%%%%%%%%%%%%%%%%%%%%%%%%%%%%%%%%%%%%%%%%%%%%%%%%
%%%%%%%%%%%%%%%%%%%%%%%%%%%%%%%%%%%%%%%%%%%%%%%%%%%%%%%%%%%%%%%%%%%%%%%%%%%%%%%%
%%%%%%%%%%%%%%%%%%%%%%%%%%%%%%%%%%%%%%%%%%%%%%%%%%%%%%%%%%%%%%%%%%%%%%%%%%%%%%%%
\subsection{Equivalence Relations}
%%%%%%%%%%%%%%%%%%%%%%%%%%%%%%%%%%%%%%%%%%%%%%%%%%%%%%%%%%%%%%%%%%%%%%%%%%%%%%%%
%%%%%%%%%%%%%%%%%%%%%%%%%%%%%%%%%%%%%%%%%%%%%%%%%%%%%%%%%%%%%%%%%%%%%%%%%%%%%%%%
%%%%%%%%%%%%%%%%%%%%%%%%%%%%%%%%%%%%%%%%%%%%%%%%%%%%%%%%%%%%%%%%%%%%%%%%%%%%%%%%
%%%%%%%%%%%%%%%%%%%%%%%%%%%%%%%%%%%%%%%%%%%%%%%%%%%%%%%%%%%%%%%%%%%%%%%%%%%%%%%%
\newpage
%%%%%%%%%%%%%%%%%%%%%%%%%%%%%%%%%%%%%%%%%%%%%%%%%%%%%%%%%%%%%%%%%%%%%%%%%%%%%%%%
%%%%%%%%%%%%%%%%%%%%%%%%%%%%%%%%%%%%%%%%%%%%%%%%%%%%%%%%%%%%%%%%%%%%%%%%%%%%%%%%
%%%%%%%%%%%%%%%%%%%%%%%%%%%%%%%%%%%%%%%%%%%%%%%%%%%%%%%%%%%%%%%%%%%%%%%%%%%%%%%%
%%%%%%%%%%%%%%%%%%%%%%%%%%%%%%%%%%%%%%%%%%%%%%%%%%%%%%%%%%%%%%%%%%%%%%%%%%%%%%%%
\subsection{Permutations} %1,2,3,4,5
%%%%%%%%%%%%%%%%%%%%%%%%%%%%%%%%%%%%%%%%%%%%%%%%%%%%%%%%%%%%%%%%%%%%%%%%%%%%%%%%
%%%%%%%%%%%%%%%%%%%%%%%%%%%%%%%%%%%%%%%%%%%%%%%%%%%%%%%%%%%%%%%%%%%%%%%%%%%%%%%%
%%%%%%%%%%%%%%%%%%%%%%%%%%%%%%%%%%%%%%%%%%%%%%%%%%%%%%%%%%%%%%%%%%%%%%%%%%%%%%%%
%%%%%%%%%%%%%%%%%%%%%%%%%%%%%%%%%%%%%%%%%%%%%%%%%%%%%%%%%%%%%%%%%%%%%%%%%%%%%%%%
\begin{mdframed}[style=darkQuesion]
  1. Consider the following Permutations in $S_7$.
  \begin{multicols}{2}
    $\sigma=
      \begin{pmatrix}
        1 & 2 & 3 & 4 & 5 & 6 & 7 \\
        3 & 2 & 5 & 4 & 6 & 1 & 7 \\
      \end{pmatrix}$
    $\tau=
      \begin{pmatrix}
        1 & 2 & 3 & 4 & 5 & 6 & 7 \\
        2 & 1 & 5 & 7 & 4 & 6 & 3 \\
      \end{pmatrix}$
  \end{multicols}
  \vspace{.25in}
  \begin{multicols}{4}
    \begin{itemize}
      \item [(a)]{$\sigma\tau$

            }
      \item [(b)]{$\tau\sigma$

            }
      \item [(c)]{$\tau^2\sigma$

            }
      \item [(d)]{$\sigma^{-1}$

            }
      \item [(e)]{$\sigma\tau\sigma^{-1}$

            }
      \item [(f)]{$\tau^{-1}\sigma\tau$

            }
    \end{itemize}
  \end{multicols}
\end{mdframed}

%%%%%%%%%%%%%%%%%%%%%%%%%%%%%%%%%%%%%%%%%%%%%%%%%%%%%%%%%%%%%%%%%%%%%%%%%%%%%%%%
\begin{mdframed}[style=darkAnswer,frametitle={Joe Starr}]
  \begin{multicols}{2}
    \begin{itemize}
      \item [(a)]{
            $\sigma\tau=
              \begin{pmatrix}
                1 & 2 & 3 & 4 & 5 & 6 & 7 \\
                2 & 3 & 6 & 7 & 4 & 1 & 5 \\
              \end{pmatrix}$
            }
      \item [(b)]{
            $\tau\sigma=
              \begin{pmatrix}
                1 & 2 & 3 & 4 & 5 & 6 & 7 \\
                5 & 1 & 4 & 7 & 6 & 2 & 3 \\
              \end{pmatrix}$
            }
      \item [(c)]{
            $\tau^2\sigma=
              \begin{pmatrix}
                1 & 2 & 3 & 4 & 5 & 6 & 7 \\
                4 & 2 & 7 & 3 & 6 & 1 & 5 \\
              \end{pmatrix}$
            }
      \item [(d)]{
            $\sigma^{-1}=
              \begin{pmatrix}
                1 & 2 & 3 & 4 & 5 & 6 & 7 \\
                6 & 2 & 1 & 4 & 3 & 5 & 7 \\
              \end{pmatrix}$
            }
      \item [(e)]{
            $\sigma\tau\sigma^{-1}=
              \begin{pmatrix}
                1 & 2 & 3 & 4 & 5 & 6 & 7 \\
                1 & 3 & 2 & 7 & 6 & 4 & 5 \\
              \end{pmatrix}$
            }
      \item [(f)]{
            $\tau^{-1}\sigma\tau=
              \begin{pmatrix}
                1 & 2 & 3 & 4 & 5 & 6 & 7 \\
                3 & 2 & 5 & 4 & 6 & 1 & 7 \\

                %tau inv 1 & 2 & 3 & 4 & 5 & 6 & 7 \\
                %        2 & 1 & 7 & 5 & 3 & 6 & 4 \\
              \end{pmatrix}$
            }
    \end{itemize}
  \end{multicols}
\end{mdframed}
\newpage
%%%%%%%%%%%%%%%%%%%%%%%%%%%%%%%%%%%%%%%%%%%%%%%%%%%%%%%%%%%%%%%%%%%%%%%%%%%%%%%%
%%%%%%%%%%%%%%%%%%%%%%%%%%%%%%%%%%%%%%%%%%%%%%%%%%%%%%%%%%%%%%%%%%%%%%%%%%%%%%%%
%%%%%%%%%%%%%%%%%%%%%%%%%%%%%%%%%%%%%%%%%%%%%%%%%%%%%%%%%%%%%%%%%%%%%%%%%%%%%%%%
%%%%%%%%%%%%%%%%%%%%%%%%%%%%%%%%%%%%%%%%%%%%%%%%%%%%%%%%%%%%%%%%%%%%%%%%%%%%%%%%
\begin{mdframed}[style=darkQuesion]
  2. Write each of the permutations $\sigma\tau, \tau\sigma, \tau^2\sigma,
    \sigma^{-1}, \sigma\tau\sigma^{-1}, \text{ and } \tau^{-1}\sigma\tau$ in
  Exercise 1 as a product of disjoint cycles. Write $\sigma$ and $\tau$ as
  products of transpositions.
\end{mdframed}

%%%%%%%%%%%%%%%%%%%%%%%%%%%%%%%%%%%%%%%%%%%%%%%%%%%%%%%%%%%%%%%%%%%%%%%%%%%%%%%%
\begin{mdframed}[style=darkAnswer,frametitle={Joe Starr}]
  \begin{multicols}{2}
    \begin{itemize}
      \item [(a)]{
            $\sigma\tau=\lrp{1236}\lrp{475}$
            }
      \item [(b)]{
            $\tau\sigma=\lrp{1562}\lrp{347}$

            }
      \item [(c)]{
            $\tau^2\sigma=\lrp{143756}$

            }
      \item [(d)]{
            $\sigma^{-1}=\lrp{1653}$

            }
      \item [(e)]{
            $\sigma\tau\sigma^{-1}=\lrp{23}\lrp{4756}$

            }
      \item [(f)]{
            $\tau^{-1}\sigma\tau=\lrp{1356}$
            }
      \item [($\sigma$)]{
            $\sigma=\lrp{13}\lrp{35}\lrp{56}$
            }
      \item [($\tau$)]{
            $\tau=\lrp{12}\lrp{35}\lrp{54}\lrp{47}\lrp{73}$
            }
    \end{itemize}
  \end{multicols}
\end{mdframed}
\newpage
%%%%%%%%%%%%%%%%%%%%%%%%%%%%%%%%%%%%%%%%%%%%%%%%%%%%%%%%%%%%%%%%%%%%%%%%%%%%%%%%
%%%%%%%%%%%%%%%%%%%%%%%%%%%%%%%%%%%%%%%%%%%%%%%%%%%%%%%%%%%%%%%%%%%%%%%%%%%%%%%%
%%%%%%%%%%%%%%%%%%%%%%%%%%%%%%%%%%%%%%%%%%%%%%%%%%%%%%%%%%%%%%%%%%%%%%%%%%%%%%%%
%%%%%%%%%%%%%%%%%%%%%%%%%%%%%%%%%%%%%%%%%%%%%%%%%%%%%%%%%%%%%%%%%%%%%%%%%%%%%%%%
\begin{mdframed}[style=darkQuesion]
  3. Write
  $\begin{pmatrix}
      1 & 2 & 3  & 4 & 5 & 6 & 7 & 8 & 9 & 10 \\
      3 & 4 & 10 & 5 & 7 & 8 & 2 & 6 & 9 & 1  \\
    \end{pmatrix}$ as the product of disjoint cycles and as a product of
  transpositions. Construct its associated diagram, find its inverse, and find
  it's order.
\end{mdframed}

%%%%%%%%%%%%%%%%%%%%%%%%%%%%%%%%%%%%%%%%%%%%%%%%%%%%%%%%%%%%%%%%%%%%%%%%%%%%%%%%
\begin{mdframed}[style=darkAnswer,frametitle={Joe Starr}]
  \begin{itemize}[align=left]
    \item [Disjoint cycles:]{\hspace{.5in}\newline
          $\lrp{1,3,10}\lrp{2,4,5,7}\lrp{6,8}$
          }
    \item [Transpositions:]{\hspace{.5in}\newline
          $\lrp{1,3}\lrp{3,10}\lrp{2,4}\lrp{4,5}\lrp{5,7}\lrp{6,8}$
          }
    \item [Diagrams:]{\hspace{.5in}\newline
          \begin{tikzpicture}[node distance=2cm]
            % nodes
            \node (A1) at (0, 0) {1};
            \node (B1) at (-1, -2) {3};
            \node (C1) at (1, -2) {10};

            \node (A2) at (3, 0) {2};
            \node (B2) at (5, 0) {4};
            \node (C2) at (5, -2) {5};
            \node (D2) at (3, -2) {7};

            \node (A3) at (7, 0) {6};
            \node (B3) at (9, -2) {8};


            \draw[->]
            (A1) edge (B1) (B1) edge (C1) (C1) edge (A1);
            \draw[->]
            (A2) edge (B2) (B2) edge (C2) (C2) edge (D2) (D2) edge (A2);
            \draw[->, to path={-| (\tikztotarget)}]
            (A3) edge (B3) (B3) edge (A3);
          \end{tikzpicture}
          }
    \item [Inverse:]{\hspace{.5in}\newline\hspace{.5in}\newline
          $\begin{pmatrix}
              1  & 2 & 3 & 4 & 5 & 6 & 7 & 8 & 9 & 10 \\
              10 & 7 & 1 & 2 & 4 & 8 & 5 & 6 & 9 & 3  \\
            \end{pmatrix}$
          }
    \item [Order:]{\hspace{.5in}\newline
          $\lrp{1,3,10}=3, \lrp{2,4,5,7}=4 \lrp{6,8}=2$
          order is $12$
          }
  \end{itemize}
  \begin{multicols}{3}
  \end{multicols}
\end{mdframed}
\newpage
%%%%%%%%%%%%%%%%%%%%%%%%%%%%%%%%%%%%%%%%%%%%%%%%%%%%%%%%%%%%%%%%%%%%%%%%%%%%%%%%
%%%%%%%%%%%%%%%%%%%%%%%%%%%%%%%%%%%%%%%%%%%%%%%%%%%%%%%%%%%%%%%%%%%%%%%%%%%%%%%%
%%%%%%%%%%%%%%%%%%%%%%%%%%%%%%%%%%%%%%%%%%%%%%%%%%%%%%%%%%%%%%%%%%%%%%%%%%%%%%%%
%%%%%%%%%%%%%%%%%%%%%%%%%%%%%%%%%%%%%%%%%%%%%%%%%%%%%%%%%%%%%%%%%%%%%%%%%%%%%%%%
\begin{mdframed}[style=darkQuesion]
  4. Find the oder of each of the following permutations.
  \begin{itemize}
    \item [(a)] {
          $\begin{pmatrix}
              1 & 2 & 3 & 4 & 5 & 6 \\
              6 & 4 & 5 & 3 & 2 & 1 \\
            \end{pmatrix}$
          }
    \item [(b)] {
          $\begin{pmatrix}
              1 & 2 & 3 & 4 & 5 & 6 & 7 & 8 \\
              4 & 6 & 7 & 5 & 1 & 8 & 2 & 3 \\
            \end{pmatrix}$
          }
    \item [(c)] {
          $\begin{pmatrix}
              1 & 2 & 3 & 4 & 5 & 6 & 7 & 8 & 9 \\
              5 & 9 & 8 & 7 & 3 & 4 & 6 & 1 & 2 \\
            \end{pmatrix}$
          }
    \item [(d)] {
          $\begin{pmatrix}
              1 & 2 & 3 & 4 & 5 & 6 & 7 & 8 & 9 \\
              8 & 4 & 9 & 6 & 5 & 2 & 3 & 1 & 7 \\
            \end{pmatrix}$
          }
  \end{itemize}
\end{mdframed}

%%%%%%%%%%%%%%%%%%%%%%%%%%%%%%%%%%%%%%%%%%%%%%%%%%%%%%%%%%%%%%%%%%%%%%%%%%%%%%%%
\begin{mdframed}[style=darkAnswer,frametitle={Joe Starr}]
  \begin{itemize}
    \item [(a)] {
          $\lrp{1,6}\lrp{2,4,2,5}$
          order is 4
          }
    \item [(b)] {
          $\lrp{1,4,5}\lrp{2,6,8,3,7}$
          order 15
          }
    \item [(c)] {
          $\lrp{1,5,3,8}\lrp{2,9}\lrp{4,7,6}$
          order 12
          }
    \item [(d)] {
          $\lrp{1,8}\lrp{2,4,6}\lrp{3,9,7}$
          order 6
          }
  \end{itemize}
\end{mdframed}
\newpage
%%%%%%%%%%%%%%%%%%%%%%%%%%%%%%%%%%%%%%%%%%%%%%%%%%%%%%%%%%%%%%%%%%%%%%%%%%%%%%%%
%%%%%%%%%%%%%%%%%%%%%%%%%%%%%%%%%%%%%%%%%%%%%%%%%%%%%%%%%%%%%%%%%%%%%%%%%%%%%%%%
%%%%%%%%%%%%%%%%%%%%%%%%%%%%%%%%%%%%%%%%%%%%%%%%%%%%%%%%%%%%%%%%%%%%%%%%%%%%%%%%
%%%%%%%%%%%%%%%%%%%%%%%%%%%%%%%%%%%%%%%%%%%%%%%%%%%%%%%%%%%%%%%%%%%%%%%%%%%%%%%%
\begin{mdframed}[style=darkQuesion]
  5. Let $3\leq m\leq n$. Calculate $\sigma\tau^{-1}$ for cycles
  $\sigma= \lrp{1,2,\dots , m-1}$ and \\ $\tau= \lrp{1,2,\dots,m-1,m}$ in $S_n$.
\end{mdframed}

%%%%%%%%%%%%%%%%%%%%%%%%%%%%%%%%%%%%%%%%%%%%%%%%%%%%%%%%%%%%%%%%%%%%%%%%%%%%%%%%
\begin{mdframed}[style=darkAnswer,frametitle={Joe Starr}]
  We begin with finding $\tau^{-1}$. We take $\tau$ of, $\lrb{1,2,\dots,m-1,m}$,
  we get $\lrb{2,3,\dots,m,1}$. If we now apply
  $\tau^{-1}=\lrp{m,1,2,\dots, m-1}$, we get $\lrb{1,2,\dots,m-1,m}$.

  We can now compose $\tau^{-1}$ and $\sigma$ yielding
  $\lrp{m,m-1,1,2,\dots,m-3,m-2}$.
\end{mdframed}
\newpage

\clearpage
%%%%%%%%%%%%%%%%%%%%%%%%%%%%%%%%%%%%%%%%%%%%%%%%%%%%%%%%%%%%%%%%%%%%%%%%%%%%%%%%
%Compile Chapter 3
\section{Groups}
\subsection{Definition of a Group}
%3.2: 1-4 6-20 22 23 26 27 28 29
%%%%%%%%%%%%%%%%%%%%%%%%%%%%%%%%%%%%%%%%%%%%%%%%%%%%%%%%%%%%%%%%%%%%%%%%%%%%%%%%
%%%%%%%%%%%%%%%%%%%%%%%%%%%%%%%%%%%%%%%%%%%%%%%%%%%%%%%%%%%%%%%%%%%%%%%%%%%%%%%%
%%%%%%%%%%%%%%%%%%%%%%%%%%%%%%%%%%%%%%%%%%%%%%%%%%%%%%%%%%%%%%%%%%%%%%%%%%%%%%%%
%%%%%%%%%%%%%%%%%%%%%%%%%%%%%%%%%%%%%%%%%%%%%%%%%%%%%%%%%%%%%%%%%%%%%%%%%%%%%%%%
\begin{mdframed}[style=darkQuesion]
1. Using ordinary addition of integers as the operation, show that the set of
even integers is a group but the set of odd integers is not.
\end{mdframed}
%%%%%%%%%%%%%%%%%%%%%%%%%%%%%%%%%%%%%%%%%%%%%%%%%%%%%%%%%%%%%%%%%%%%%%%%%%%%%%%%
\begin{mdframed}[style=darkAnswer,frametitle={Joe Starr}]
We begin considering the even integers, that is integers of the form $2k$. We
must also include 0 in the even integers. We get the identity element as well
as Associativity and inverses for free from integer addition on $\Z$.
We then consider closure. Let $n$ and $m$ be even integers, if we take $m+n$
we can see we have, $m+n=2k_m+2k_n=2(k_m+k_n)$ an even integer. Making the
even integers a group under addition.

Next we consider the odd integers, take $3+3=2(3)$ an even integer, showing
odds are not closed under addition and not a group.
\end{mdframed}
\newpage
%%%%%%%%%%%%%%%%%%%%%%%%%%%%%%%%%%%%%%%%%%%%%%%%%%%%%%%%%%%%%%%%%%%%%%%%%%%%%%%%
%%%%%%%%%%%%%%%%%%%%%%%%%%%%%%%%%%%%%%%%%%%%%%%%%%%%%%%%%%%%%%%%%%%%%%%%%%%%%%%%
%%%%%%%%%%%%%%%%%%%%%%%%%%%%%%%%%%%%%%%%%%%%%%%%%%%%%%%%%%%%%%%%%%%%%%%%%%%%%%%%
%%%%%%%%%%%%%%%%%%%%%%%%%%%%%%%%%%%%%%%%%%%%%%%%%%%%%%%%%%%%%%%%%%%%%%%%%%%%%%%%
\begin{mdframed}[style=darkQuesion]
2. For each binary operation $\ast$ defined on a set below,
determine whether or not $\ast$ gives a group structure on the set.
If it is not a group, say which axioms fail to hold.
\begin{multicols}{2}
\begin{itemize}
\item[(a)]{Define $\ast$ on $\Z$ by $a\ast b=ab$.}
\item[(b)]{Define $\ast$ on $\Z$ by $a\ast b=\max{a,b}$.}
\item[(c)]{Define $\ast$ on $\Z$ by $a\ast b=a-b$.}
\item[(d)]{Define $\ast$ on $\Z$ by $a\ast b=\abs{ab}$.}
\item[(e)]{Define $\ast$ on $\R^{+}$ by $a\ast b=ab$.}
\item[(f)]{Define $\ast$ on $\Q$ by $a\ast b=ab$.}
\end{itemize}
\end{multicols}
\end{mdframed}

%%%%%%%%%%%%%%%%%%%%%%%%%%%%%%%%%%%%%%%%%%%%%%%%%%%%%%%%%%%%%%%%%%%%%%%%%%%%%%%%
\begin{mdframed}[style=darkAnswer,frametitle={Joe Starr}]
\begin{multicols}{2}
\begin{itemize}[align=left]
\item[(a)]{
    \begin{itemize}[align=left]
    \Invs{Let $a\in \Z$ but $a\neq 1$ and $a\neq 1$,
        $\inv{a}\notin \Z$.}
    \end{itemize}
  }
\item[(b)]{
    \begin{itemize}[align=left]
    \Ident{$\max{a,a-1}=a$ for all $a\in \Z$ this means there is no
        $e$ with $\max{a,e}=a$ for all $a$. }
    \end{itemize}
  }
\item[(c)]{
    \begin{itemize}[align=left]
    \Assoc{\begin{align*}
        \lrp{a\ast b}\ast c & =\lrp{a-b}\ast c     \\
        & =\lrp{a-b}-c         \\
        & =a-\lrp{b-c}         \\
        & =a\ast \lrp{b\ast c}
        \end{align*}}
    \Invs{Select $a\in \Z$, $$a\ast a = a-a = 0$$}
    \Clos{Obvious from closure of $\lrp{\Z,+}$}
    \Ident{0 is the identity, Obvious from $\lrp{\Z,+}$}
    \end{itemize}

  }
\item[(d)]{
    \begin{itemize}[align=left]
    \Invs{Let $a\in \Z$ but $a\neq 1$ and $a\neq 1$,
        $\inv{a}\notin \Z$.}
    \end{itemize}
  }
\item[(e)]{
    \begin{itemize}[align=left]
    \Assoc{\begin{align*}
        \lrp{a\ast b}\ast c & =\lrp{ab}\ast c      \\
        & =\lrp{ab}c           \\
        & =a\lrp{bc}           \\
        & =a\ast \lrp{b\ast c}
        \end{align*}}
    \Invs{Select $a\in \R$, $$a\ast a = a\frac{1}{a} = 1$$}
    \Clos{Obvious from closure of $\lrp{\R,\cdot}$}
    \Ident{1 is the identity, Obvious from $\lrp{\R,\cdot}$}
    \end{itemize}
  }
\item[(f)]{
    \begin{itemize}[align=left]
    \Assoc{\begin{align*}
        \lrp{a\ast b}\ast c & =\lrp{ab}\ast c      \\
        & =\lrp{ab}c           \\
        & =a\lrp{bc}           \\
        & =a\ast \lrp{b\ast c}
        \end{align*}}
    \Invs{Select $a\in \Q$, $$a\ast \inv{a} = a\frac{1}{a} = 1$$}
    \Clos{$a,b\in \Q$, $a=\frac{p_1}{q_1}$ $b=\frac{p_2}{q_2}$,
        ${p_1},{q_1},{p_2},{q_2}\in \Z$, $q_1\neq0\neq q_2$.
        \begin{align*}
        a\ast b & =ab
        & =                                 \\
        & =\frac{p_1{p_2}}{q_1{q_2}} \in \Q
        \end{align*}}
    \Ident{1 is the identity, Obvious from $\lrp{\R,\cdot}$}
    \end{itemize}
  }
\end{itemize}
\end{multicols}
\end{mdframed}
\newpage
%%%%%%%%%%%%%%%%%%%%%%%%%%%%%%%%%%%%%%%%%%%%%%%%%%%%%%%%%%%%%%%%%%%%%%%%%%%%%%%%
%%%%%%%%%%%%%%%%%%%%%%%%%%%%%%%%%%%%%%%%%%%%%%%%%%%%%%%%%%%%%%%%%%%%%%%%%%%%%%%%
%%%%%%%%%%%%%%%%%%%%%%%%%%%%%%%%%%%%%%%%%%%%%%%%%%%%%%%%%%%%%%%%%%%%%%%%%%%%%%%%
%%%%%%%%%%%%%%%%%%%%%%%%%%%%%%%%%%%%%%%%%%%%%%%%%%%%%%%%%%%%%%%%%%%%%%%%%%%%%%%%
\begin{mdframed}[style=darkQuesion]
3. Let $\grp{G}{\cdot}$ be a group. Define a new binary operation $\ast$ on
$G$ by the formula $a \ast b=b\cot a$, for all $a,b\in G$.
\begin{itemize}
\item[(a)]{Show that $\grp{G}{\cdot}$ is a group.}
\item[(b)]{Give examples to show that $\grp{G}{\cdot}$ may or may not be the
    same as $\grp{G}{\ast}$.}
\end{itemize}
\end{mdframed}

%%%%%%%%%%%%%%%%%%%%%%%%%%%%%%%%%%%%%%%%%%%%%%%%%%%%%%%%%%%%%%%%%%%%%%%%%%%%%%%%
\begin{mdframed}[style=darkAnswer,frametitle={Joe Starr}]
\begin{itemize}
\item[(a)]{
    \begin{itemize}[align=left]
    \Assoc{\begin{align*}
        \lrp{a\ast b}\ast c & =\lrp{b\cdot a}\ast c     \\
        & =c \cdot  \lrp{b \cdot a} \\
        & =\lrp{c \cdot  b} \cdot a \\
        & =\lrp{b\ast c} \cdot a    \\
        & =a\ast \lrp{b\ast c}
        \end{align*}}
    \Invs{Let $a\in G$ since $\grp{G}{\cdot}$ is a group we know
        $\inv{a}\in G$. Now $a\ast \inv{a}=\inv{a}\cdot a=1$.
      }
    \Clos{We select $a,b\in G$ consier $a\ast b=b\cdot a$ by closure of
        $\grp{G}{\cdot}$, $a\ast b\in G$.
      }
    \Ident{Let $a\in G$, consider $1\ast a = a \cdot 1 = a$ and
        $a\ast 1 = 1 \cdot a = a$.}
    \end{itemize}
  }
\item[(b)]{
    Let $G=\begin{tabular}{|c|c|c|c|c|c|c|}
      \hline
    $\cdot$ & e & a & b & c & d & f \\
      \hline
      a       & a & e & d & f & b & c \\
      \hline
      b       & b & f & e & d & c & a \\
      \hline
      c       & c & d & f & e & a & b \\
      \hline
      d       & d & c & a & b & f & e \\
      \hline
      f       & f & b & c & a & e & d \\
      \hline
      \end{tabular}$ so
    $a\ast b=b\cdot a = f$ but $a\cdot b=d$. In this case they are not equal.

    If we let $G= \grp{\Z}{+}$, we have $a\ast b = b+a=a+b$. In this case they
    are equal.
  }
\end{itemize}
\end{mdframed}
\newpage
%%%%%%%%%%%%%%%%%%%%%%%%%%%%%%%%%%%%%%%%%%%%%%%%%%%%%%%%%%%%%%%%%%%%%%%%%%%%%%%%
%%%%%%%%%%%%%%%%%%%%%%%%%%%%%%%%%%%%%%%%%%%%%%%%%%%%%%%%%%%%%%%%%%%%%%%%%%%%%%%%
%%%%%%%%%%%%%%%%%%%%%%%%%%%%%%%%%%%%%%%%%%%%%%%%%%%%%%%%%%%%%%%%%%%%%%%%%%%%%%%%
%%%%%%%%%%%%%%%%%%%%%%%%%%%%%%%%%%%%%%%%%%%%%%%%%%%%%%%%%%%%%%%%%%%%%%%%%%%%%%%%
\begin{mdframed}[style=darkQuesion]
5. Is $\GL{n}{\R}$ an Abelian group? Support your answer by either proof or a
counter example.
\end{mdframed}

%%%%%%%%%%%%%%%%%%%%%%%%%%%%%%%%%%%%%%%%%%%%%%%%%%%%%%%%%%%%%%%%%%%%%%%%%%%%%%%%
\begin{mdframed}[style=darkAnswer,frametitle={Joe Starr}]
No, select $$A=\begin{bmatrix}
  2 & 3 \\
  5 & 7 \\
  \end{bmatrix}
  B=\begin{bmatrix}
  11 & 13 \\
  17 & 19 \\
  \end{bmatrix}$$
we calculate
$$AB=\begin{bmatrix}
  73  & 83  \\
  174 & 198 \\
  \end{bmatrix}
  BA=\begin{bmatrix}
  87  & 124 \\
  129 & 184 \\
  \end{bmatrix}$$
\end{mdframed}
\newpage
%%%%%%%%%%%%%%%%%%%%%%%%%%%%%%%%%%%%%%%%%%%%%%%%%%%%%%%%%%%%%%%%%%%%%%%%%%%%%%%%
%%%%%%%%%%%%%%%%%%%%%%%%%%%%%%%%%%%%%%%%%%%%%%%%%%%%%%%%%%%%%%%%%%%%%%%%%%%%%%%%
%%%%%%%%%%%%%%%%%%%%%%%%%%%%%%%%%%%%%%%%%%%%%%%%%%%%%%%%%%%%%%%%%%%%%%%%%%%%%%%%
%%%%%%%%%%%%%%%%%%%%%%%%%%%%%%%%%%%%%%%%%%%%%%%%%%%%%%%%%%%%%%%%%%%%%%%%%%%%%%%%
\begin{mdframed}[style=darkQuesion]
8. Write out the multiplication table for the following set of matrices over
$\Q$:
$$\begin{bmatrix}
  1 & 0 \\
  0 & 1 \\
  \end{bmatrix},
  \begin{bmatrix}
  \m1 & 0 \\
  0   & 1 \\
  \end{bmatrix},
  \begin{bmatrix}
  1 & 0   \\
  0 & \m1 \\
  \end{bmatrix},
  \begin{bmatrix}
  \m1 & 0   \\
  0   & \m1 \\
  \end{bmatrix}$$


\end{mdframed}

%%%%%%%%%%%%%%%%%%%%%%%%%%%%%%%%%%%%%%%%%%%%%%%%%%%%%%%%%%%%%%%%%%%%%%%%%%%%%%%%
\begin{mdframed}[style=darkAnswer,frametitle={Joe Starr}]
Let $$i=\begin{bmatrix}
  1 & 0 \\
  0 & 1 \\
  \end{bmatrix},
  j=\begin{bmatrix}
  \m1 & 0 \\
  0   & 1 \\
  \end{bmatrix},
  k=\begin{bmatrix}
  1 & 0   \\
  0 & \m1 \\
  \end{bmatrix},
  l=\begin{bmatrix}
  \m1 & 0   \\
  0   & \m1 \\
  \end{bmatrix}$$
$$\begin{tabular}{|c|c|c|c|c|}
  \hline
  $\cdot$ & i & j & k & l \\
  \hline
  i       & i & j & k & l \\
  \hline
  j       & j & i & l & k \\
  \hline
  k       & k & l & i & j \\
  \hline
  l       & l & k & j & i \\
  \hline
  \end{tabular}$$
\end{mdframed}
\newpage
%%%%%%%%%%%%%%%%%%%%%%%%%%%%%%%%%%%%%%%%%%%%%%%%%%%%%%%%%%%%%%%%%%%%%%%%%%%%%%%%
%%%%%%%%%%%%%%%%%%%%%%%%%%%%%%%%%%%%%%%%%%%%%%%%%%%%%%%%%%%%%%%%%%%%%%%%%%%%%%%%
%%%%%%%%%%%%%%%%%%%%%%%%%%%%%%%%%%%%%%%%%%%%%%%%%%%%%%%%%%%%%%%%%%%%%%%%%%%%%%%%
%%%%%%%%%%%%%%%%%%%%%%%%%%%%%%%%%%%%%%%%%%%%%%%%%%%%%%%%%%%%%%%%%%%%%%%%%%%%%%%%
\begin{mdframed}[style=darkQuesion]
9. Let $G=\lrs{x\in \R \vert x>0   ext{ and } x\neq 1}$. Define the operation
$\ast$ on $G$ by $a\ast b=a^{\ln{b}}$, for all $a,b\in G$. Prove that $G$ is
an Abelian group under the operation $\ast$.
\end{mdframed}

%%%%%%%%%%%%%%%%%%%%%%%%%%%%%%%%%%%%%%%%%%%%%%%%%%%%%%%%%%%%%%%%%%%%%%%%%%%%%%%%
\begin{mdframed}[style=darkAnswer,frametitle={Joe Starr}]
\begin{itemize}[align=left]
\Assoc{\begin{align*}
    \lrp{a\ast b}\ast c & =\lrp{a^{\ln{b}}}\ast c    \\
    & =\lrp{a^{\ln{b}}}^{\ln{c}} \\
    & ={a^{\ln{b}\ln{c}}}        \\
    & =a^{\ln{b^{\ln{c}}}}       \\
    & =a\ast \lrp{b\ast c}
    \end{align*}}
\Invs{
    Let $a\in G$, consider $\inv{a}=e^{\frac{1}{\ln{a}}}$.
    \begin{multicols}{2}
    \begin{align*}
    a\ast\inv{a} & = a ^{\ln{\inv{a}}}             \\
    & = a^{\ln{e^{\frac{1}{\ln{a}}}}} \\
    & = a^{{{\frac{1}{\ln{a}}}}}      \\
    & = a^{{{\log_{a}{e}}}}           \\
    & = {e}                           \\
    \end{align*}
    \begin{align*}
    \inv{a}\ast a & = \lrp{\inv{a}} ^{\ln{a}}              \\
    & = \lrp{e^{\frac{1}{\ln{a}}}} ^{\ln{a}} \\
    & = \lrp{e^{\log_{a}{e}}} ^{\ln{a}}      \\
    & = \lrp{e^{\ln{a}} } ^{\log_{a}{e}}     \\
    & = a ^{\log_{a}{e}}                     \\
    & = {e}                                  \\
    \end{align*}
    \end{multicols}
  }
\Clos{Let $a,b\in G$, $a\ast b = a^{\ln{b}}$ we know that $b>0$ so
    $\ln{b}$ exists, further since $1\notin G$ we have $\ln{b}\neq 0$.
    We observe that for any $a\in G$ $a^{\ln{b}}>0$ since $a>0$ and since
    $\ln{b}\neq 0$ $a^{\ln{b}}\neq 1$.}
\Ident{Our conjecture is that $e$ is the identity element.
    Let $a\in G$, $e\ast a= e^{\ln{a}}=a$ and $a\ast e= a^{\ln{e}}=a$}
\end{itemize}
\end{mdframed}
\newpage
%%%%%%%%%%%%%%%%%%%%%%%%%%%%%%%%%%%%%%%%%%%%%%%%%%%%%%%%%%%%%%%%%%%%%%%%%%%%%%%%
%%%%%%%%%%%%%%%%%%%%%%%%%%%%%%%%%%%%%%%%%%%%%%%%%%%%%%%%%%%%%%%%%%%%%%%%%%%%%%%%
%%%%%%%%%%%%%%%%%%%%%%%%%%%%%%%%%%%%%%%%%%%%%%%%%%%%%%%%%%%%%%%%%%%%%%%%%%%%%%%%
%%%%%%%%%%%%%%%%%%%%%%%%%%%%%%%%%%%%%%%%%%%%%%%%%%%%%%%%%%%%%%%%%%%%%%%%%%%%%%%%
\begin{mdframed}[style=darkQuesion]
10. Show that the set $A=\lrs{f_{m,b}:\R   o \R\vert m\neq 0   ext{ and } f_{m,b}=mx+b}$.
of affine functions from $\R$ to $\R$ forms a group under function composition.
\end{mdframed}

%%%%%%%%%%%%%%%%%%%%%%%%%%%%%%%%%%%%%%%%%%%%%%%%%%%%%%%%%%%%%%%%%%%%%%%%%%%%%%%%
\begin{mdframed}[style=darkAnswer,frametitle={Joe Starr}]
\begin{itemize}[align=left]
\Assoc{We've proved this previously. }
\Invs{Let $f\in A$, $\fof{x}=mx+b$. Consider $I\lrp{x}=\frac{1}{m}\lrp{x-b}$
    \begin{multicols}{2}
    \begin{align*}
    \fof{I\lrp{x}} & = m\lrp{\frac{1}{m}\lrp{x-b}}+b \\
    & = x-b+b                         \\
    & = x                             \\
    \end{align*}
    \begin{align*}
    I\lrp{\fof{x}} & = \frac{1}{m}\lrp{\lrp{mx+b}-b} \\
    & = \frac{1}{m}\lrp{mx}           \\
    & = x                             \\
    \end{align*}
    \end{multicols}
  }
\Clos{Let $f,g\in A$, so $\fof{x}=m_1x+b_1$ and $\gof{x}=m_2x+b_2$.
    Now composing $f$ and $g$ $\fof{\gof{x}}$.
    \begin{align*}
    \fof{\gof{x}} & = m_1\lrp{m_2x+b_2}+b_1 \\
    & = {m_1m_2x+m_1b_2}+b_1  \\
    & = mx+m_1b_2+b_1         \\
    & = mx+b                  \\
    \end{align*}
  }
\Ident{Let $f\in A$, $\fof{x}=mx+b$. Conjecture $e\lrp{x}=x$
    \begin{multicols}{2}
    \begin{align*}
    \fof{e\lrp{x}} & = m\lrp{x}+b \\
    & = mx+b       \\
    \end{align*}
    \vfill
    \columnbreak
    \begin{align*}
    e\lrp{\fof{x}} & = mx+b \\
    \end{align*}
    \end{multicols}
  }

\end{itemize}
\end{mdframed}
\newpage
%%%%%%%%%%%%%%%%%%%%%%%%%%%%%%%%%%%%%%%%%%%%%%%%%%%%%%%%%%%%%%%%%%%%%%%%%%%%%%%%
%%%%%%%%%%%%%%%%%%%%%%%%%%%%%%%%%%%%%%%%%%%%%%%%%%%%%%%%%%%%%%%%%%%%%%%%%%%%%%%%
%%%%%%%%%%%%%%%%%%%%%%%%%%%%%%%%%%%%%%%%%%%%%%%%%%%%%%%%%%%%%%%%%%%%%%%%%%%%%%%%
%%%%%%%%%%%%%%%%%%%%%%%%%%%%%%%%%%%%%%%%%%%%%%%%%%%%%%%%%%%%%%%%%%%%%%%%%%%%%%%%
\begin{mdframed}[style=darkQuesion]
11. Show that the set of all $2  imes 2$ matrices over $\R$ of the form
$\begin{bmatrix}
  m & b \\
  0 & 1 \\
  \end{bmatrix}$ with $m\neq 0$ forms a group under matrix multiplication.
\end{mdframed}

%%%%%%%%%%%%%%%%%%%%%%%%%%%%%%%%%%%%%%%%%%%%%%%%%%%%%%%%%%%%%%%%%%%%%%%%%%%%%%%%
\begin{mdframed}[style=darkAnswer,frametitle={Joe Starr}]
Let $G$ be the set of all $2  imes 2$ matrices over $\R$ of the form
$\begin{bmatrix}
  m & b \\
  0 & 1 \\
  \end{bmatrix}$ with $m\neq 0$.
\begin{itemize}[align=left]
\Assoc{Free from $\Mn{2}{\R}$}
\Invs{Let $a\in G$, so $a=\begin{bmatrix}
      m & b \\
      0 & 1 \\
      \end{bmatrix}$ we can calculate the determinate of $a$. $m1-b0=m$ and by
    definition of the set $m\neq 0$. So we have inverses.
  }
\Clos{Let $a,b\in G$, so $a=\begin{bmatrix}
      m_1 & b_1 \\
      0   & 1   \\
      \end{bmatrix}$ and $b=\begin{bmatrix}
      m_2 & b_2 \\
      0   & 1   \\
      \end{bmatrix}$
    \begin{align*}
    ab     & = \begin{bmatrix}
    m_1    & b_1               \\
    0      & 1                 \\
    \end{bmatrix} \begin{bmatrix}
    m_2    & b_2               \\
    0      & 1                 \\
    \end{bmatrix} \\
    & = \begin{bmatrix}
    m_2m_1 & b_1+m_1b_2        \\
    0      & 1                 \\
    \end{bmatrix}                            \\
    \end{align*}
  }
\Ident{Free from $\Mn{2}{\R}$
  }

\end{itemize}
\end{mdframed}
\newpage
%%%%%%%%%%%%%%%%%%%%%%%%%%%%%%%%%%%%%%%%%%%%%%%%%%%%%%%%%%%%%%%%%%%%%%%%%%%%%%%%
%%%%%%%%%%%%%%%%%%%%%%%%%%%%%%%%%%%%%%%%%%%%%%%%%%%%%%%%%%%%%%%%%%%%%%%%%%%%%%%%
%%%%%%%%%%%%%%%%%%%%%%%%%%%%%%%%%%%%%%%%%%%%%%%%%%%%%%%%%%%%%%%%%%%%%%%%%%%%%%%%
%%%%%%%%%%%%%%%%%%%%%%%%%%%%%%%%%%%%%%%%%%%%%%%%%%%%%%%%%%%%%%%%%%%%%%%%%%%%%%%%
\begin{mdframed}[style=darkQuesion]
12. In the group defined in question 11 find all elements that commute with
$\begin{bmatrix}
  2 & 0 \\
  0 & 1 \\
  \end{bmatrix}$
\end{mdframed}

%%%%%%%%%%%%%%%%%%%%%%%%%%%%%%%%%%%%%%%%%%%%%%%%%%%%%%%%%%%%%%%%%%%%%%%%%%%%%%%%
\begin{mdframed}[style=darkAnswer,frametitle={Joe Starr}]
We can begin by letting $a\in G$ calculating $a\begin{bmatrix}
  2 & 0 \\
  0 & 1 \\
  \end{bmatrix}$ and $\begin{bmatrix}
  2 & 0 \\
  0 & 1 \\
  \end{bmatrix}a$.
\begin{multicols}{2}
\begin{align*}
a\begin{bmatrix}
2             & 0                 \\
0             & 1                 \\
\end{bmatrix} & = \begin{bmatrix}
m             & b                 \\
0             & 1                 \\
\end{bmatrix} \begin{bmatrix}
2             & 0                 \\
0             & 1                 \\
\end{bmatrix} \\
& = \begin{bmatrix}
2m            & b                 \\
0             & 1                 \\
\end{bmatrix}                            \\
\end{align*}
\begin{align*}
\begin{bmatrix}
2              & 0                  \\
0              & 1                  \\
\end{bmatrix}a & =  \begin{bmatrix}
2              & 0                  \\
0              & 1                  \\
\end{bmatrix} \begin{bmatrix}
m              & b                  \\
0              & 1                  \\
\end{bmatrix} \\
& = \begin{bmatrix}
2m             & 2b                 \\
0              & 1                  \\
\end{bmatrix}                             \\
\end{align*}
\end{multicols}
So for a matrix of to commute with $\begin{bmatrix}
  2 & 0 \\
  0 & 1 \\
  \end{bmatrix}$ it must be of the form $\begin{bmatrix}
  m & 0 \\
  0 & 1 \\
  \end{bmatrix}$.
\end{mdframed}
\newpage
%%%%%%%%%%%%%%%%%%%%%%%%%%%%%%%%%%%%%%%%%%%%%%%%%%%%%%%%%%%%%%%%%%%%%%%%%%%%%%%%
%%%%%%%%%%%%%%%%%%%%%%%%%%%%%%%%%%%%%%%%%%%%%%%%%%%%%%%%%%%%%%%%%%%%%%%%%%%%%%%%
%%%%%%%%%%%%%%%%%%%%%%%%%%%%%%%%%%%%%%%%%%%%%%%%%%%%%%%%%%%%%%%%%%%%%%%%%%%%%%%%
%%%%%%%%%%%%%%%%%%%%%%%%%%%%%%%%%%%%%%%%%%%%%%%%%%%%%%%%%%%%%%%%%%%%%%%%%%%%%%%%
\begin{mdframed}[style=darkQuesion]
13. Define $\ast$ on $\R$ by $a\ast b = a+b-1$, for all $a,b\in \R$. Show that
$\grp{\R}{\ast}$ is an Abelian group.
\end{mdframed}

%%%%%%%%%%%%%%%%%%%%%%%%%%%%%%%%%%%%%%%%%%%%%%%%%%%%%%%%%%%%%%%%%%%%%%%%%%%%%%%%
\begin{mdframed}[style=darkAnswer,frametitle={Joe Starr}]
\begin{itemize}[align=left]
\item[]{
    \begin{multicols}{2}
    \begin{itemize}[align=left]
    \Abel{
        \begin{align*}
          {a\ast b} & =a+b-1   \\
        & = b+a-1  \\
        & =b\ast a
        \end{align*}

      }
    \Assoc{\begin{align*}
        \lrp{a\ast b}\ast c & =\lrp{a+b-1}\ast c   \\
        & =\lrp{a+b-1}+c-1     \\
        & =a+b+c-1-1           \\
        & =a+\lrp{b+c-1}-1     \\
        & =a\ast \lrp{b\ast c}
        \end{align*}}
    \end{itemize}
    \end{multicols}
  }
\Invs{ Let $a\in \grp{\R}{\ast} $, consider $\inv{a}=2-a$
    \begin{multicols}{2}
    \begin{align*}
    a\ast \inv{a} & = a+\lrp{2-a}-1 \\
    & = 1             \\
    \end{align*}
    \vfill
    \columnbreak
    \begin{align*}
    \inv{a}\ast a & = \lrp{2-a}+a-1 \\
    & = 1             \\
    \end{align*}
    \end{multicols}
  }
\Clos{Obvious from closure of $\grp{\R}{+}$.
  }
\Ident{ Conjecture is that $1$ is the identity element of $\grp{\R}{\ast}$.
    \begin{multicols}{2}
    \begin{align*}
    a\ast 1 & = a+1-1 \\
    & = a     \\
    \end{align*}
    \vfill
    \columnbreak
    \begin{align*}
    1\ast a & = a+1-1 \\
    & = a     \\
    \end{align*}
    \end{multicols}
  }
\end{itemize}
\end{mdframed}
\begin{mdframed}[style=darkAnswer,frametitle={Joe Starr}]
Let $\varphi:\grp{\R}{\ast}  o\grp{\R}{+}$, with $\pof{x}=x-1$,
$\pof{a\ast b}= \lrp{a+b-1}-1=a-1+b-1=\pof{a}+\pof{b}$. Further
$\inv{\varphi}\lrp{x}=x+1$, $\pof{\inv{\varphi}\lrp{x}}=\lrp{x+1}-1=x$.
Showing a group structure isomorphic to $\grp{\R}{+}$.
\end{mdframed}
\newpage
%%%%%%%%%%%%%%%%%%%%%%%%%%%%%%%%%%%%%%%%%%%%%%%%%%%%%%%%%%%%%%%%%%%%%%%%%%%%%%%%
%%%%%%%%%%%%%%%%%%%%%%%%%%%%%%%%%%%%%%%%%%%%%%%%%%%%%%%%%%%%%%%%%%%%%%%%%%%%%%%%
%%%%%%%%%%%%%%%%%%%%%%%%%%%%%%%%%%%%%%%%%%%%%%%%%%%%%%%%%%%%%%%%%%%%%%%%%%%%%%%%
%%%%%%%%%%%%%%%%%%%%%%%%%%%%%%%%%%%%%%%%%%%%%%%%%%%%%%%%%%%%%%%%%%%%%%%%%%%%%%%%
\begin{mdframed}[style=darkQuesion]
14. Let $S= \R - \lrs{\m1}$. Define $\ast$ on $S$ by $a\ast b=a+b+ab$ for all
$a,b \in S$. Show that $\grp{S}{\ast}$ is an Abelian group.
\end{mdframed}

%%%%%%%%%%%%%%%%%%%%%%%%%%%%%%%%%%%%%%%%%%%%%%%%%%%%%%%%%%%%%%%%%%%%%%%%%%%%%%%%
\begin{mdframed}[style=darkAnswer,frametitle={Joe Starr}]
\begin{itemize}[align=left]

\item[] {
    \begin{multicols}{2}
    \begin{itemize}[align=left]

    \Abel{
        \begin{align*}
          {a\ast b} & =a+b+ab  \\
        & = b+a+ba \\
        & =b\ast a
        \end{align*}
      }
    \Invs{ Consider $\inv{a}=\frac{\m a}{a+1}$
        \begin{align*}
        a\ast \inv{a} & = a+\frac{\m a}{a+1}+a\frac{\m a}{a+1} \\
        & = a+\frac{\m a\lrp{a+1}}{a+1}          \\
        & = a+\m a                               \\
        & = 0                                    \\
        \end{align*}
        \vfill
        \columnbreak
      }
    \Clos{Let $a,b\in \R$, if we take $a\ast b =a+b+ab$. Assume that
        $a\ast b=\m1$
        \begin{align*}
        \m1=a+b+ab & \Rightarrow  \m1-a=b+ab          \\
        & \Rightarrow  \m1-a=b\lrp{1+a}    \\
        & \Rightarrow  \m\frac{a+1}{1+a}=b \\
        & \Rightarrow  \m1=b               \\
        \end{align*}
        a contradiction.
      }
    \end{itemize}
    \end{multicols}
  }
\Ident{ Conjecture is that $0$ is the identity element of $\grp{S}{\ast}$.
    \begin{align*}
    a\ast 0 & = a+0+a0 \\
    & = a      \\
    \end{align*}
  }
\Assoc{\begin{align*}
    \lrp{a\ast b}\ast c & =\lrp{a+b+ab}\ast c            \\
    & = \lrp{a+b+ab}+c+\lrp{a+b+ab}c \\
    & = \lrp{a+b+ab}+c+\lrp{a+b+ab}c \\
    & = \lrp{a+b+ab}+c+\lrp{a+b+ab}c \\
    & = a+b+ab+c+ca+cb+cab           \\
    & = a+\lrp{b+c+bc}+a\lrp{b+c+bc} \\
    & =a\ast\lrp{b+c+bc}             \\
    & =a\ast \lrp{b\ast c}
    \end{align*}}

\end{itemize}
\end{mdframed}
\newpage
%%%%%%%%%%%%%%%%%%%%%%%%%%%%%%%%%%%%%%%%%%%%%%%%%%%%%%%%%%%%%%%%%%%%%%%%%%%%%%%%
%%%%%%%%%%%%%%%%%%%%%%%%%%%%%%%%%%%%%%%%%%%%%%%%%%%%%%%%%%%%%%%%%%%%%%%%%%%%%%%%
%%%%%%%%%%%%%%%%%%%%%%%%%%%%%%%%%%%%%%%%%%%%%%%%%%%%%%%%%%%%%%%%%%%%%%%%%%%%%%%%
%%%%%%%%%%%%%%%%%%%%%%%%%%%%%%%%%%%%%%%%%%%%%%%%%%%%%%%%%%%%%%%%%%%%%%%%%%%%%%%%
\begin{mdframed}[style=darkQuesion]
15. Let $G=\lrs{x\in \R\vert x>1}$. Define $\ast$ on $G$ by $a\ast b = ab-a-b+2$,
for all $a,b\in G$. Show that $\grp{G}{\ast}$ is an Abelian group.
\end{mdframed}

%%%%%%%%%%%%%%%%%%%%%%%%%%%%%%%%%%%%%%%%%%%%%%%%%%%%%%%%%%%%%%%%%%%%%%%%%%%%%%%%
\begin{mdframed}[style=darkAnswer,frametitle={Joe Starr}]
\begin{itemize}[align=left]

\item[] {
    \begin{multicols}{2}
    \begin{itemize}[align=left]
    \Abel{
        \begin{align*}
          {a\ast b} & =ab-a-b+2  \\
        & = ba-b-a+2 \\
        & =b\ast a
        \end{align*}
      }
    \Invs{ Consider $\inv{a}=\frac{a}{a-1}$
        \begin{align*}
        a\ast \inv{a} & = a\frac{a}{a-1}-a-\frac{a}{a-1}+2 \\
        & = \frac{a}{a-1}\lrp{a-1}-a+2       \\
        & = a-a+2                            \\
        & = 2                                \\
        \end{align*}
        \vfill
        \columnbreak
      }
    \Ident{ Conjecture is that $2$ is the identity element of $\grp{G}{\ast}$.
        \begin{align*}
        a\ast 2 & = a2-a-2+2 \\
        & = a        \\
        \end{align*}
      }
    \end{itemize}
    \end{multicols}
  }
\Clos{We begin by letting $a,b\in G$, we observe $a\geq b>1$.
    We can then multiply through by b yeilding $ab\geq bb>b>1$. Next
    we subtract both $a$ and $b$, $ab-a-b\geq1>\m a$, finaly adding
    two gives $ab-a-b+2\geq3$ showing $a\ast b\in G$.
  }
\Assoc{\begin{align*}
    \lrp{a\ast b}\ast c & =\lrp{a+b+ab}\ast c            \\
    & = \lrp{a+b+ab}+c+\lrp{a+b+ab}c \\
    & = \lrp{a+b+ab}+c+\lrp{a+b+ab}c \\
    & = \lrp{a+b+ab}+c+\lrp{a+b+ab}c \\
    & = a+b+ab+c+ca+cb+cab           \\
    & = a+\lrp{b+c+bc}+a\lrp{b+c+bc} \\
    & =a\ast\lrp{b+c+bc}             \\
    & =a\ast \lrp{b\ast c}
    \end{align*}}

\end{itemize}
\end{mdframed}
\newpage
%%%%%%%%%%%%%%%%%%%%%%%%%%%%%%%%%%%%%%%%%%%%%%%%%%%%%%%%%%%%%%%%%%%%%%%%%%%%%%%%
%%%%%%%%%%%%%%%%%%%%%%%%%%%%%%%%%%%%%%%%%%%%%%%%%%%%%%%%%%%%%%%%%%%%%%%%%%%%%%%%
%%%%%%%%%%%%%%%%%%%%%%%%%%%%%%%%%%%%%%%%%%%%%%%%%%%%%%%%%%%%%%%%%%%%%%%%%%%%%%%%
%%%%%%%%%%%%%%%%%%%%%%%%%%%%%%%%%%%%%%%%%%%%%%%%%%%%%%%%%%%%%%%%%%%%%%%%%%%%%%%%
\begin{mdframed}[style=darkQuesion]
16. Let $G$ be a group. We have shown that $\inv{\lrp{ab}}=\inv{b}\inv{a}$.
Find a similar expression for $\lrp{\inv{abc}}$
\end{mdframed}

%%%%%%%%%%%%%%%%%%%%%%%%%%%%%%%%%%%%%%%%%%%%%%%%%%%%%%%%%%%%%%%%%%%%%%%%%%%%%%%%
\begin{mdframed}[style=darkAnswer,frametitle={Joe Starr}]
We will use a transitive proof:
\begin{align*}
\inv{\lrp{abc}} & = \inv{c}\inv{\lrp{ab}} \\
& = \inv{c}\inv{b}\inv{a}
\end{align*}
\end{mdframed}
\newpage
%%%%%%%%%%%%%%%%%%%%%%%%%%%%%%%%%%%%%%%%%%%%%%%%%%%%%%%%%%%%%%%%%%%%%%%%%%%%%%%%
%%%%%%%%%%%%%%%%%%%%%%%%%%%%%%%%%%%%%%%%%%%%%%%%%%%%%%%%%%%%%%%%%%%%%%%%%%%%%%%%
%%%%%%%%%%%%%%%%%%%%%%%%%%%%%%%%%%%%%%%%%%%%%%%%%%%%%%%%%%%%%%%%%%%%%%%%%%%%%%%%
%%%%%%%%%%%%%%%%%%%%%%%%%%%%%%%%%%%%%%%%%%%%%%%%%%%%%%%%%%%%%%%%%%%%%%%%%%%%%%%%
\begin{mdframed}[style=darkQuesion]
17. Let $G$ be a group. If $g\in G$ and $g^2=g$,then prove that $g=e$.
\end{mdframed}

%%%%%%%%%%%%%%%%%%%%%%%%%%%%%%%%%%%%%%%%%%%%%%%%%%%%%%%%%%%%%%%%%%%%%%%%%%%%%%%%
\begin{mdframed}[style=darkAnswer,frametitle={Joe Starr}]
We begin with letting $g\in G$, such $g^2=g$ we then multiply by $\inv{g}$ on
the left:
\begin{align*}
g^2=g & \rightarrow \inv{g}g^2=\inv{g}g \\
& \rightarrow g=e
\end{align*}
as desired.
\end{mdframed}
\newpage
%%%%%%%%%%%%%%%%%%%%%%%%%%%%%%%%%%%%%%%%%%%%%%%%%%%%%%%%%%%%%%%%%%%%%%%%%%%%%%%%
%%%%%%%%%%%%%%%%%%%%%%%%%%%%%%%%%%%%%%%%%%%%%%%%%%%%%%%%%%%%%%%%%%%%%%%%%%%%%%%%
%%%%%%%%%%%%%%%%%%%%%%%%%%%%%%%%%%%%%%%%%%%%%%%%%%%%%%%%%%%%%%%%%%%%%%%%%%%%%%%%
%%%%%%%%%%%%%%%%%%%%%%%%%%%%%%%%%%%%%%%%%%%%%%%%%%%%%%%%%%%%%%%%%%%%%%%%%%%%%%%%
\begin{mdframed}[style=darkQuesion]
18. Show that a nonabelian group must have at least 5 elements.
\end{mdframed}

%%%%%%%%%%%%%%%%%%%%%%%%%%%%%%%%%%%%%%%%%%%%%%%%%%%%%%%%%%%%%%%%%%%%%%%%%%%%%%%%
\begin{mdframed}[style=darkAnswer,frametitle={Joe Starr}]
Let $G$ be a nonabelian group. Since $G$ a group then $e\in G$ the identity. $G$
can't be the trivial group since the trivial group is Abelian, this puts
$a\in G$ with $a\neq e$ further $\inv{a}\in G$. With the same argument $G$ is
not a group of three elements, so $b,\inv{b}\in G$. This puts
$a,b,\inv{b},\inv{a},e\in G$ showing $G$ with at lest 5 elements.
\end{mdframed}
\newpage
%%%%%%%%%%%%%%%%%%%%%%%%%%%%%%%%%%%%%%%%%%%%%%%%%%%%%%%%%%%%%%%%%%%%%%%%%%%%%%%%
%%%%%%%%%%%%%%%%%%%%%%%%%%%%%%%%%%%%%%%%%%%%%%%%%%%%%%%%%%%%%%%%%%%%%%%%%%%%%%%%
%%%%%%%%%%%%%%%%%%%%%%%%%%%%%%%%%%%%%%%%%%%%%%%%%%%%%%%%%%%%%%%%%%%%%%%%%%%%%%%%
%%%%%%%%%%%%%%%%%%%%%%%%%%%%%%%%%%%%%%%%%%%%%%%%%%%%%%%%%%%%%%%%%%%%%%%%%%%%%%%%
\begin{mdframed}[style=darkQuesion]
22. Let $S$ be a nonempty finite set with a binary operation $\ast$ that
satisfies the associative law. Show that $S$ is a group if $a\ast b=a\ast c$
implies $b=c$ and $a\ast c= b\ast c$ implies $a=b$ for all $a,b,c \in S$.
What can we say if $S$ is infinite?
\end{mdframed}

%%%%%%%%%%%%%%%%%%%%%%%%%%%%%%%%%%%%%%%%%%%%%%%%%%%%%%%%%%%%%%%%%%%%%%%%%%%%%%%%
\begin{mdframed}[style=darkAnswer,frametitle={Joe Starr}]
%OLD-QESTION: Need to do this one still.
\end{mdframed}
\newpage
%%%%%%%%%%%%%%%%%%%%%%%%%%%%%%%%%%%%%%%%%%%%%%%%%%%%%%%%%%%%%%%%%%%%%%%%%%%%%%%%
%%%%%%%%%%%%%%%%%%%%%%%%%%%%%%%%%%%%%%%%%%%%%%%%%%%%%%%%%%%%%%%%%%%%%%%%%%%%%%%%
%%%%%%%%%%%%%%%%%%%%%%%%%%%%%%%%%%%%%%%%%%%%%%%%%%%%%%%%%%%%%%%%%%%%%%%%%%%%%%%%
%%%%%%%%%%%%%%%%%%%%%%%%%%%%%%%%%%%%%%%%%%%%%%%%%%%%%%%%%%%%%%%%%%%%%%%%%%%%%%%%
\begin{mdframed}[style=darkQuesion]
24. Let $G$ be a group. Prove that $G$ is Abelian if and only if
$\inv{\lrp{ab}}=\inv{a}\inv{b}$ for all $a,b\in G$.
\end{mdframed}

%%%%%%%%%%%%%%%%%%%%%%%%%%%%%%%%%%%%%%%%%%%%%%%%%%%%%%%%%%%%%%%%%%%%%%%%%%%%%%%%
\begin{mdframed}[style=darkAnswer,frametitle={Joe Starr}]
wocase{
    Let $G$ be an abelian group and $a,b\in G$. Consider $\inv{\lrp{ab}}$, we have
    shown $\inv{\lrp{ab}}=\inv{b}\inv{a}$ since $G$ is abelian we have
    $\inv{\lrp{ab}}=\inv{a}\inv{b}$.
  }{
    Let $\inv{\lrp{ab}}=\inv{a}\inv{b}$, we have shown
    $\inv{\lrp{ab}}=\inv{b}\inv{a}$ so $\inv{b}\inv{a}=\inv{a}\inv{b}$ showing $G$
    abelian.
  }
\end{mdframed}
\newpage
%%%%%%%%%%%%%%%%%%%%%%%%%%%%%%%%%%%%%%%%%%%%%%%%%%%%%%%%%%%%%%%%%%%%%%%%%%%%%%%%
%%%%%%%%%%%%%%%%%%%%%%%%%%%%%%%%%%%%%%%%%%%%%%%%%%%%%%%%%%%%%%%%%%%%%%%%%%%%%%%%
%%%%%%%%%%%%%%%%%%%%%%%%%%%%%%%%%%%%%%%%%%%%%%%%%%%%%%%%%%%%%%%%%%%%%%%%%%%%%%%%
%%%%%%%%%%%%%%%%%%%%%%%%%%%%%%%%%%%%%%%%%%%%%%%%%%%%%%%%%%%%%%%%%%%%%%%%%%%%%%%%
\begin{mdframed}[style=darkQuesion]
25. Let $G$ be a group. Prove that if $x^2=e$ for all $x\in G$, then $G$ is
abelian.
\end{mdframed}

%%%%%%%%%%%%%%%%%%%%%%%%%%%%%%%%%%%%%%%%%%%%%%%%%%%%%%%%%%%%%%%%%%%%%%%%%%%%%%%%
\begin{mdframed}[style=darkAnswer,frametitle={Joe Starr}]
Let $G$ be a group with the given property $a,b\in G$. Observe that
$a^2=e \Rightarrow a=\inv{a}$. We have shown that
$\inv{\lrp{ab}}=\inv{b}\inv{a}$.
We proceed with a transitive proof:
\begin{align*}
\inv{\lrp{ab}}=\inv{b}\inv{a} & \rightarrow \lrp{ab}=\inv{b}\inv{a} \\
& \rightarrow \lrp{ab}={b}{a}
\end{align*}
showing $G$ abelian as desired.
\end{mdframed}
\newpage
%%%%%%%%%%%%%%%%%%%%%%%%%%%%%%%%%%%%%%%%%%%%%%%%%%%%%%%%%%%%%%%%%%%%%%%%%%%%%%%%
%%%%%%%%%%%%%%%%%%%%%%%%%%%%%%%%%%%%%%%%%%%%%%%%%%%%%%%%%%%%%%%%%%%%%%%%%%%%%%%%
%%%%%%%%%%%%%%%%%%%%%%%%%%%%%%%%%%%%%%%%%%%%%%%%%%%%%%%%%%%%%%%%%%%%%%%%%%%%%%%%
%%%%%%%%%%%%%%%%%%%%%%%%%%%%%%%%%%%%%%%%%%%%%%%%%%%%%%%%%%%%%%%%%%%%%%%%%%%%%%%%
\begin{mdframed}[style=darkQuesion]
26. Show that if $G$ is a finite group with an even number of elements, then
there must exist an element $a\in G$ with $a\neq e$ such that $a^2=e$.
\end{mdframed}

%%%%%%%%%%%%%%%%%%%%%%%%%%%%%%%%%%%%%%%%%%%%%%%%%%%%%%%%%%%%%%%%%%%%%%%%%%%%%%%%
\begin{mdframed}[style=darkAnswer,frametitle={Joe Starr}]
Let $G$ be a group with the given property. Since $G$ a group $e\in G$.
Observe $G$ is not the trivial group since it has even cardinality.
If we consider the cardinality of $G/e$ it's $\abs{G}-1$ an odd number.
Let $a\in G$ with $a\neq e$, observe that since $G$ a group $\inv{a}\in G$.
We are left with two possibilities $a=\inv{a}$ or $a\neq \inv{a}$.
If $a=\inv{a}$ we are done, otherwise we can delete $a$ and $\inv{a}$ from $G$
and select from the remaining elements of $G$. Since $G/e$ has odd cardinality
we can repeat this process until there is a single element remaining. It must
be that $a=\inv{a}$ as desired.
\end{mdframed}

\clearpage
\subsection{Subgroups}
%%%%%%%%%%%%%%%%%%%%%%%%%%%%%%%%%%%%%%%%%%%%%%%%%%%%%%%%%%%%%%%%%%%%%%%%%%%%%%%%
%%%%%%%%%%%%%%%%%%%%%%%%%%%%%%%%%%%%%%%%%%%%%%%%%%%%%%%%%%%%%%%%%%%%%%%%%%%%%%%%
%%%%%%%%%%%%%%%%%%%%%%%%%%%%%%%%%%%%%%%%%%%%%%%%%%%%%%%%%%%%%%%%%%%%%%%%%%%%%%%%
%%%%%%%%%%%%%%%%%%%%%%%%%%%%%%%%%%%%%%%%%%%%%%%%%%%%%%%%%%%%%%%%%%%%%%%%%%%%%%%%
\begin{mdframed}[style=darkQuesion]
1. In $\mathrm{GL}_{2}(\mathbf{R}),$ find the order of each of the following elements.
\begin{multicols}{4}
\begin{itemize}
\item[]{
    (a)$\dagger\left[\begin{array}{rr}1 & \m 1 \\
          1 & 0\end{array}\right]$}
\item[]{
    (b) $\left[\begin{array}{rr}0 & 1 \\
          \m 1 & 0\end{array}\right]$}
\item[]{
    $\dagger(\mathrm{c})\left[\begin{array}{ll}1 & 1 \\
          0 & 1\end{array}\right]$}
\item[]{
    (d) $\left[\begin{array}{rr}\m 1 & 1 \\
          0 & 1\end{array}\right]$}
\end{itemize}
\end{multicols}
\vspace{.5cm}
\end{mdframed}

%%%%%%%%%%%%%%%%%%%%%%%%%%%%%%%%%%%%%%%%%%%%%%%%%%%%%%%%%%%%%%%%%%%%%%%%%%%%%%%%
\begin{mdframed}[style=darkAnswer,frametitle={Joe Starr}]
\begin{multicols}{2}
\begin{itemize}
\item[(a)]{
    $$\begin{bmatrix}1 & \m 1 \\
      1 & 0 \\
      \end{bmatrix}^2=\begin{bmatrix}0 & \m 1 \\
      1 & \m 1\\
      \end{bmatrix}$$
    $$\begin{bmatrix}1 & \m 1 \\
      1 & 0 \\
      \end{bmatrix}^3=\begin{bmatrix}\m 1 & 0 \\
      0 & \m 1\\
      \end{bmatrix}$$
    $$\begin{bmatrix}1 & \m 1 \\
      1 & 0 \\
      \end{bmatrix}^6=\begin{bmatrix}1 & 0 \\
      0 & 1\\
      \end{bmatrix}$$
  }
\item[(b)]{
    $$\begin{bmatrix}0 & 1 \\
      \m 1 & 0 \\
      \end{bmatrix}^2=\begin{bmatrix}\m 1 & 0 \\
      0 & \m 1\\
      \end{bmatrix}$$
    $$\begin{bmatrix}0 & 1 \\
      \m 1 & 0 \\
      \end{bmatrix}^4=\begin{bmatrix}1 & 0 \\
      0 & 1\\
      \end{bmatrix}$$}
\end{itemize}
\end{multicols}
\begin{multicols}{2}
\begin{itemize}
\item[(c)]{
    $$\begin{bmatrix}1 & 1 \\
      0 & 1 \\
      \end{bmatrix}^2=\begin{bmatrix}1 & 2 \\
      0 & 1 \\
      \end{bmatrix}$$
    Infinite order.
  }
\item[(d)]{
    $$\begin{bmatrix}\m 1 & 1 \\
      0 & 1 \\
      \end{bmatrix}^2=\begin{bmatrix}1 & 0 \\
      0 & 1 \\
      \end{bmatrix}$$}
\end{itemize}
\end{multicols}
\end{mdframed}
\newpage
%%%%%%%%%%%%%%%%%%%%%%%%%%%%%%%%%%%%%%%%%%%%%%%%%%%%%%%%%%%%%%%%%%%%%%%%%%%%%%%%
%%%%%%%%%%%%%%%%%%%%%%%%%%%%%%%%%%%%%%%%%%%%%%%%%%%%%%%%%%%%%%%%%%%%%%%%%%%%%%%%
%%%%%%%%%%%%%%%%%%%%%%%%%%%%%%%%%%%%%%%%%%%%%%%%%%%%%%%%%%%%%%%%%%%%%%%%%%%%%%%%
%%%%%%%%%%%%%%%%%%%%%%%%%%%%%%%%%%%%%%%%%%%%%%%%%%%%%%%%%%%%%%%%%%%%%%%%%%%%%%%%
\begin{mdframed}[style=darkQuesion]
2. Let $A=\left[\begin{array}{rr}1 & \m 1 \\
      \m 1 & 0\end{array}\right] \in \mathrm{GL}_{2}(\mathrm{R})$.
Show that $A$ has infinite order by proving that
$A^{n}=\left[\begin{array}{cc}F_{n+1} & -F_{n} \\
      -F_{n} & F_{n-1}\end{array}\right],$
for $n \geq 1,$ where $F_{0}=0, F_{1}=1,$ and $F_{n+1}=$
$F_{n}+F_{n-1}$ is the Fibonacci sequence.
\end{mdframed}

%%%%%%%%%%%%%%%%%%%%%%%%%%%%%%%%%%%%%%%%%%%%%%%%%%%%%%%%%%%%%%%%%%%%%%%%%%%%%%%%
\begin{mdframed}[style=darkAnswer,frametitle={Joe Starr}]
We will proceed with induction:
\begin{itemize}[align=left]
\item[Base Case:]{ Consider $1$ for the basecase.
    $\begin{bmatrix}F_{n+1} & \m F_{n} \\
      \m F_{n} & F_{n-1} \\
      \end{bmatrix}=\begin{bmatrix}1 & \m 1 \\
      \m 1 & 0 \\
      \end{bmatrix}$ showing the base case to be true.
  }
\item[Inductive Case:]{ Assume that it's ture for the nth power we will show
    this implies the $n+1$th case to be true.
    $$A^{n+1}=A^nA^1=\begin{bmatrix}F_{n+1} & \m F_{n} \\
      \m F_{n} & F_{n-1} \\
      \end{bmatrix}\begin{bmatrix}1 & \m 1 \\
      \m 1 & 0 \\
      \end{bmatrix}=\begin{bmatrix}F_{n+2} & \m F_{n+1} \\
      \m F_{n+1} & F_{n} \\
      \end{bmatrix}$$ showing the Inductive case to be true, and $A$ of infinite
    order.
  }
\end{itemize}
\end{mdframed}
\newpage
%%%%%%%%%%%%%%%%%%%%%%%%%%%%%%%%%%%%%%%%%%%%%%%%%%%%%%%%%%%%%%%%%%%%%%%%%%%%%%%%
%%%%%%%%%%%%%%%%%%%%%%%%%%%%%%%%%%%%%%%%%%%%%%%%%%%%%%%%%%%%%%%%%%%%%%%%%%%%%%%%
%%%%%%%%%%%%%%%%%%%%%%%%%%%%%%%%%%%%%%%%%%%%%%%%%%%%%%%%%%%%%%%%%%%%%%%%%%%%%%%%
%%%%%%%%%%%%%%%%%%%%%%%%%%%%%%%%%%%%%%%%%%%%%%%%%%%%%%%%%%%%%%%%%%%%%%%%%%%%%%%%
\begin{mdframed}[style=darkQuesion]
3. Prove that the set of all rational numbers of the form $m / n$, where $m, n \in \mathbf{Z}$ and $n$ is square-free, is a subgroup of Q (under addition).

\end{mdframed}

%%%%%%%%%%%%%%%%%%%%%%%%%%%%%%%%%%%%%%%%%%%%%%%%%%%%%%%%%%%%%%%%%%%%%%%%%%%%%%%%
\begin{mdframed}[style=darkAnswer,frametitle={Joe Starr}]
\begin{itemize}[align=left]
\Invs{Let $m,n \in \Z$ with the given properties. Take $\frac{\m m}{n}$
    and consider \\$\frac{m}{n}+\frac{\m m}{n}=\frac{m-m}{n}=0$ as desired.
  }
\Clos{
    Let $m,n,a,b\in \Z$ with the given properties. Take
    $\frac{m}{n}+\frac{a}{b}=\frac{mb+an}{bn}$, since $b$ and $n$ are square
    free $bn$ is also square free.
  }
\end{itemize}
\end{mdframed}
\newpage
%%%%%%%%%%%%%%%%%%%%%%%%%%%%%%%%%%%%%%%%%%%%%%%%%%%%%%%%%%%%%%%%%%%%%%%%%%%%%%%%
%%%%%%%%%%%%%%%%%%%%%%%%%%%%%%%%%%%%%%%%%%%%%%%%%%%%%%%%%%%%%%%%%%%%%%%%%%%%%%%%
%%%%%%%%%%%%%%%%%%%%%%%%%%%%%%%%%%%%%%%%%%%%%%%%%%%%%%%%%%%%%%%%%%%%%%%%%%%%%%%%
%%%%%%%%%%%%%%%%%%%%%%%%%%%%%%%%%%%%%%%%%%%%%%%%%%%%%%%%%%%%%%%%%%%%%%%%%%%%%%%%
\begin{mdframed}[style=darkQuesion]
4. Show that $\{\ (1),\ (1,2)(3,4),\ (1,3)(2,4),\ (1,4)(2,3)\ \}$ is a subgroup
of $S_{4}$
\end{mdframed}

%%%%%%%%%%%%%%%%%%%%%%%%%%%%%%%%%%%%%%%%%%%%%%%%%%%%%%%%%%%%%%%%%%%%%%%%%%%%%%%%
\begin{mdframed}[style=darkAnswer,frametitle={Joe Starr}]
We begin by labeling each permutation
$$A=(1,2)(3,4)=$$$$B=(1,3)(2,4)$$$$C=(1,4)(2,3)$$
\begin{itemize}[align=left]
\Invs{
    $$AA=(1,2)(3,4)(1,2)(3,4)=(1)$$
    $$BB=(1,3)(2,4)(1,3)(2,4)=(1)$$
    $$CC=(1,4)(2,3)(1,4)(2,3)=(1)$$
  }
\Clos{
    $$AB= (1,4)(2,3)$$
    $$BA= (1,4)(2,3)$$
    $$AC= (1,3)(2,4)$$
    $$CA= (1,3)(2,4)$$
    $$CB= (1,2)(3,4)$$
    $$BC= (1,2)(3,4)$$
  }
\end{itemize}
\end{mdframed}
\newpage
%%%%%%%%%%%%%%%%%%%%%%%%%%%%%%%%%%%%%%%%%%%%%%%%%%%%%%%%%%%%%%%%%%%%%%%%%%%%%%%%
%%%%%%%%%%%%%%%%%%%%%%%%%%%%%%%%%%%%%%%%%%%%%%%%%%%%%%%%%%%%%%%%%%%%%%%%%%%%%%%%
%%%%%%%%%%%%%%%%%%%%%%%%%%%%%%%%%%%%%%%%%%%%%%%%%%%%%%%%%%%%%%%%%%%%%%%%%%%%%%%%
%%%%%%%%%%%%%%%%%%%%%%%%%%%%%%%%%%%%%%%%%%%%%%%%%%%%%%%%%%%%%%%%%%%%%%%%%%%%%%%%
\begin{mdframed}[style=darkQuesion]
\begin{itemize}[align=left]
\item [(a)] {
    Show that
    $T=\lrs{\begin{bmatrix}
          a & 0 \\
          c & d\\
          \end{bmatrix}
          \vert a d \neq 0}$ is a subgroup of $G$.
  }
\item [(b)] {
    Show that
    $D=\lrs{\begin{bmatrix}
          a & 0 \\
          0 & d\\
          \end{bmatrix} \vert a d \neq 0}$
    is a subgroup of $G$.
  }
\end{itemize}
\end{mdframed}

%%%%%%%%%%%%%%%%%%%%%%%%%%%%%%%%%%%%%%%%%%%%%%%%%%%%%%%%%%%%%%%%%%%%%%%%%%%%%%%%
\begin{mdframed}[style=darkAnswer,frametitle={Joe Starr}]
\begin{itemize}[align=left]
\item [(a)]{
    \begin{itemize}[align=left]
    \Invs{We know by construction of $G$ there exist an inverse of the form $\begin{bmatrix}
          \frac{1}{a} & 0\\
          \frac{\m c}{ad} & \frac{1}{d}\\
          \end{bmatrix}$, by taking $\frac{1}{a}\frac{1}{d}$ that this is not $0$
        so the inverse is in $T$.
      }
    \Clos{
        If we take $A,B\in T$ $$AB=\begin{bmatrix}
          a & 0\\
          c & d\\
          \end{bmatrix}\begin{bmatrix}
          w & 0\\
          y & z\\
          \end{bmatrix}=\begin{bmatrix}
          aw & 0\\
          cw+dy & zd\\
          \end{bmatrix}$$
        since both $ad\neq0$ and $wz\neq 0$ it holds $adwz\neq 0$.
      }
    \end{itemize}
  }
\item [(b)]{
    \begin{itemize}[align=left]
    \Invs{We know by construction of $G$ there exist an inverse of the form $\begin{bmatrix}
          \frac{1}{a} & 0\\
          0 & \frac{1}{d}\\
          \end{bmatrix}$, by taking $\frac{1}{a}\frac{1}{d}$ that this is not $0$
        so the inverse is in $T$.
      }
    \Clos{
        If we take $A,B\in T$ $$AB=\begin{bmatrix}
          a & 0\\
          0 & d\\
          \end{bmatrix}\begin{bmatrix}
          w & 0\\
          0 & z\\
          \end{bmatrix}=\begin{bmatrix}
          aw & 0\\
          0 & zd\\
          \end{bmatrix}$$
        since both $ad\neq0$ and $wz\neq 0$ it holds $adwz\neq 0$.
      }
    \end{itemize}
  }
\end{itemize}
\end{mdframed}
\newpage
%%%%%%%%%%%%%%%%%%%%%%%%%%%%%%%%%%%%%%%%%%%%%%%%%%%%%%%%%%%%%%%%%%%%%%%%%%%%%%%%
%%%%%%%%%%%%%%%%%%%%%%%%%%%%%%%%%%%%%%%%%%%%%%%%%%%%%%%%%%%%%%%%%%%%%%%%%%%%%%%%
%%%%%%%%%%%%%%%%%%%%%%%%%%%%%%%%%%%%%%%%%%%%%%%%%%%%%%%%%%%%%%%%%%%%%%%%%%%%%%%%
%%%%%%%%%%%%%%%%%%%%%%%%%%%%%%%%%%%%%%%%%%%%%%%%%%%%%%%%%%%%%%%%%%%%%%%%%%%%%%%%
\begin{mdframed}[style=darkQuesion]
7. Let $G=\mathrm{GL}_{2}(\mathrm{R})$.
Show that the subset $S$ of $G$ defined by
$S=\left\{\left[\begin{array}{ll}a & b \\
      c & d\end{array}\right] | b=c\right\}$ of symmetric $2   \times 2$ matrices does not form a subgroup of $G .$

\end{mdframed}

%%%%%%%%%%%%%%%%%%%%%%%%%%%%%%%%%%%%%%%%%%%%%%%%%%%%%%%%%%%%%%%%%%%%%%%%%%%%%%%%
\begin{mdframed}[style=darkAnswer,frametitle={Joe Starr}]
Consider $$\begin{bmatrix}
  1 & 3\\
  3 & 5\\
  \end{bmatrix}\begin{bmatrix}
  1 & 3\\
  3 & 1\\
  \end{bmatrix}=\begin{bmatrix}
  10 & 6\\
  18 & 14\\
  \end{bmatrix}$$ showing this set is not closed.
\end{mdframed}
\newpage
%%%%%%%%%%%%%%%%%%%%%%%%%%%%%%%%%%%%%%%%%%%%%%%%%%%%%%%%%%%%%%%%%%%%%%%%%%%%%%%%
%%%%%%%%%%%%%%%%%%%%%%%%%%%%%%%%%%%%%%%%%%%%%%%%%%%%%%%%%%%%%%%%%%%%%%%%%%%%%%%%
%%%%%%%%%%%%%%%%%%%%%%%%%%%%%%%%%%%%%%%%%%%%%%%%%%%%%%%%%%%%%%%%%%%%%%%%%%%%%%%%
%%%%%%%%%%%%%%%%%%%%%%%%%%%%%%%%%%%%%%%%%%%%%%%%%%%%%%%%%%%%%%%%%%%%%%%%%%%%%%%%
\begin{mdframed}[style=darkQuesion]
8. Let $G=\mathrm{GL}_{2}(\mathrm{R}) .$ For each of the following subsets of $M_{2}(\mathbf{R}),$ determine whether
or not the subset is a subgroup of $G .$
\begin{itemize}[align=left]

\item []{(a) $A=\left\{\left[\begin{array}{ll}a & b \\ 0 & 0\end{array}\right] | a b \neq 0\right\}$}
\item []{(b) $B=\left\{\left[\begin{array}{ll}0 & b \\ c & 0\end{array}\right] | b c \neq 0\right\}$}
\item []{(c) $C=\left\{\left[\begin{array}{ll}1 & 0 \\ 0 & c\end{array}\right] | c \neq 0\right\}$}

\end{itemize}
\end{mdframed}

%%%%%%%%%%%%%%%%%%%%%%%%%%%%%%%%%%%%%%%%%%%%%%%%%%%%%%%%%%%%%%%%%%%%%%%%%%%%%%%%
\begin{mdframed}[style=darkAnswer,frametitle={Joe Starr}]
\begin{itemize}[align=left]
\item [(a)]{This set doesn't contain the identity so can not be a subgroup}
\item [(b)]{This set doesn't contain the identity so can not be a subgroup}
\item [(c)]{
    \begin{itemize}[align=left]
    \Invs{Let $\begin{bmatrix}
          1 & 0\\
          0 & c\\
          \end{bmatrix}$ with the given properties. We consider the inverse of $\begin{bmatrix}
          1 & 0\\
          0 & c\\
          \end{bmatrix}$ which is $\begin{bmatrix}
          1 & 0\\
          0 & \frac{1}{c}\\
          \end{bmatrix}$ which is in $C$;}
    \Clos{Let $\begin{bmatrix}
          1 & 0\\
          0 & c\\
          \end{bmatrix}$ and Let $\begin{bmatrix}
          1 & 0\\
          0 & a\\
          \end{bmatrix}$ with the given properties. Consider
        $$\begin{bmatrix}
          1 & 0\\
          0 & c\\
          \end{bmatrix}\begin{bmatrix}
          1 & 0\\
          0 & a\\
          \end{bmatrix}=\begin{bmatrix}
          1 & 0\\
          0 & ca\\
          \end{bmatrix}$$ We observe that $a\neq0$ and $c\neq 0$, consequently $ac\neq0$.
      }
    \end{itemize}
  }
\end{itemize}
\end{mdframed}
\newpage
%%%%%%%%%%%%%%%%%%%%%%%%%%%%%%%%%%%%%%%%%%%%%%%%%%%%%%%%%%%%%%%%%%%%%%%%%%%%%%%%
%%%%%%%%%%%%%%%%%%%%%%%%%%%%%%%%%%%%%%%%%%%%%%%%%%%%%%%%%%%%%%%%%%%%%%%%%%%%%%%%
%%%%%%%%%%%%%%%%%%%%%%%%%%%%%%%%%%%%%%%%%%%%%%%%%%%%%%%%%%%%%%%%%%%%%%%%%%%%%%%%
%%%%%%%%%%%%%%%%%%%%%%%%%%%%%%%%%%%%%%%%%%%%%%%%%%%%%%%%%%%%%%%%%%%%%%%%%%%%%%%%
\begin{mdframed}[style=darkQuesion]
9. Let $G=\mathrm{GL}_{3}(\mathbf{R}) .$ Show that $H=\left\{\left[\begin{array}{lll}1 & 0 & 0 \\ a & 1 & 0 \\ b & c & 1\end{array}\right]\right\}$ is a subgroup of $G .$

\end{mdframed}

%%%%%%%%%%%%%%%%%%%%%%%%%%%%%%%%%%%%%%%%%%%%%%%%%%%%%%%%%%%%%%%%%%%%%%%%%%%%%%%%
\begin{mdframed}[style=darkAnswer,frametitle={Joe Starr}]
\begin{itemize}[align=left]
\Invs{Let $\begin{bmatrix}
      1 & 0 & 0 \\
      a & 1 & 0 \\
      b & c & 1 \\
      \end{bmatrix}$ with the given properties. We consider the inverse of $\begin{bmatrix}
      1 & 0 & 0 \\
      a & 1 & 0 \\
      b & c & 1 \\
      \end{bmatrix}$ which is $\begin{bmatrix}
      1 & 0 & 0 \\
      \m a & 1 & 0 \\
      ac-b & \m c & 1 \\
      \end{bmatrix}$ which is in $H$;}
\Clos{Let $\begin{bmatrix}
      1 & 0 & 0 \\
      a & 1 & 0 \\
      b & c & 1 \\
      \end{bmatrix}$ and Let $\begin{bmatrix}
      1 & 0 & 0 \\
      x & 1 & 0 \\
      y & z & 1 \\
      \end{bmatrix}$ with the given properties. Consider
    $$\begin{bmatrix}
      1 & 0 & 0 \\
      a & 1 & 0 \\
      b & c & 0 \\
      \end{bmatrix}\begin{bmatrix}
      1 & 0 & 0 \\
      x & 1 & 0 \\
      y & z & 1 \\
      \end{bmatrix}=\begin{bmatrix}
      1 & 0 & 0 \\
      a+x & 1 & 0 \\
      b+cx+y & c+z & 1 \\
      \end{bmatrix}$$
  }
\end{itemize}
\end{mdframed}
\newpage%%%%%%%%%%%%%%%%%%%%%%%%%%%%%%%%%%%%%%%%%%%%%%%%%%%%%%%%%%%%%%%%%%%%%%%%%%%%%%%%
%%%%%%%%%%%%%%%%%%%%%%%%%%%%%%%%%%%%%%%%%%%%%%%%%%%%%%%%%%%%%%%%%%%%%%%%%%%%%%%%
%%%%%%%%%%%%%%%%%%%%%%%%%%%%%%%%%%%%%%%%%%%%%%%%%%%%%%%%%%%%%%%%%%%%%%%%%%%%%%%%
%%%%%%%%%%%%%%%%%%%%%%%%%%%%%%%%%%%%%%%%%%%%%%%%%%%%%%%%%%%%%%%%%%%%%%%%%%%%%%%%
\begin{mdframed}[style=darkQuesion]
10. Let $m$ and $n$ be nonzero integers, with ( $m, n$ ) = $d$. Show that $m$ and $n$ belong to $d \mathbf{Z},$ and that if $H$ is any subgroup of $\mathbf{Z}$ that contains both $m$ and $n,$ then $d \mathbf{Z} \subseteq H$
\end{mdframed}

%%%%%%%%%%%%%%%%%%%%%%%%%%%%%%%%%%%%%%%%%%%%%%%%%%%%%%%%%%%%%%%%%%%%%%%%%%%%%%%%
\begin{mdframed}[style=darkAnswer,frametitle={Joe Starr}]
We will first show that $m$ and $n$ are in $d\Z$. Let $m,n\in \Z$ with the given
properties. Since $\gcd\lrp{m,n}=d$ we observe that $dq_1=m$ and $dq_2=n$
making $m,n\in d\Z$ as desired.

Next let $H$ be a subgroup of $\Z$ with $m,n\in H$. Let $a\in d\Z$, with
$a<m\leq n$ by construction we have $a=dq$ for some $q$.
%TODO need to finish this one. 
\end{mdframed}
\newpage
%%%%%%%%%%%%%%%%%%%%%%%%%%%%%%%%%%%%%%%%%%%%%%%%%%%%%%%%%%%%%%%%%%%%%%%%%%%%%%%%
%%%%%%%%%%%%%%%%%%%%%%%%%%%%%%%%%%%%%%%%%%%%%%%%%%%%%%%%%%%%%%%%%%%%%%%%%%%%%%%%
%%%%%%%%%%%%%%%%%%%%%%%%%%%%%%%%%%%%%%%%%%%%%%%%%%%%%%%%%%%%%%%%%%%%%%%%%%%%%%%%
%%%%%%%%%%%%%%%%%%%%%%%%%%%%%%%%%%%%%%%%%%%%%%%%%%%%%%%%%%%%%%%%%%%%%%%%%%%%%%%%
\begin{mdframed}[style=darkQuesion]
11. Let $S$ be a set, and let $a$ be a fixed element of $S .$ Show that $\{\sigma \in \operatorname{Sym}(S) | \sigma(a)=a\}$ is a subgroup of $\operatorname{Sym}(S)$

\end{mdframed}

%%%%%%%%%%%%%%%%%%%%%%%%%%%%%%%%%%%%%%%%%%%%%%%%%%%%%%%%%%%%%%%%%%%%%%%%%%%%%%%%
\begin{mdframed}[style=darkAnswer,frametitle={Joe Starr}]
Let $A=\{\sigma \in \operatorname{Sym}(S) | \sigma(a)=a\}$
\begin{itemize}[align=left]
\Invs{
    By proposition 2.1.7 in the text we have inverses.
  }
\Clos{
    Let $\sigma,\varphi \in A$, consider the composition of these two functions
    around $a$, $\pof{\sof{a}}=\pof{a}=a$ showing closure, as desired.
  }
\end{itemize}
\end{mdframed}
\newpage
%%%%%%%%%%%%%%%%%%%%%%%%%%%%%%%%%%%%%%%%%%%%%%%%%%%%%%%%%%%%%%%%%%%%%%%%%%%%%%%%
%%%%%%%%%%%%%%%%%%%%%%%%%%%%%%%%%%%%%%%%%%%%%%%%%%%%%%%%%%%%%%%%%%%%%%%%%%%%%%%%
%%%%%%%%%%%%%%%%%%%%%%%%%%%%%%%%%%%%%%%%%%%%%%%%%%%%%%%%%%%%%%%%%%%%%%%%%%%%%%%%
%%%%%%%%%%%%%%%%%%%%%%%%%%%%%%%%%%%%%%%%%%%%%%%%%%%%%%%%%%%%%%%%%%%%%%%%%%%%%%%%
\begin{mdframed}[style=darkQuesion]
12. For each of the following groups, find all elements of finite order.
\begin{itemize}
\item []{
    (a) $\mathbf{R}^{\times}$
  }
\item []{
    (b) $\mathbf{C}^{\times}$
  }
\end{itemize}

\end{mdframed}

%%%%%%%%%%%%%%%%%%%%%%%%%%%%%%%%%%%%%%%%%%%%%%%%%%%%%%%%%%%%%%%%%%%%%%%%%%%%%%%%
\begin{mdframed}[style=darkAnswer,frametitle={Joe Starr}]
\begin{itemize}[align=left]
\item [(a)]{
    1 and $\m 1$ are the only elements of finite order.
  }d
\item [(b)]{
    $1,\m 1 ,\m i$, and $i$ are the only elements of
    finite order.
  }
\end{itemize}
\end{mdframed}
\newpage
%%%%%%%%%%%%%%%%%%%%%%%%%%%%%%%%%%%%%%%%%%%%%%%%%%%%%%%%%%%%%%%%%%%%%%%%%%%%%%%%
%%%%%%%%%%%%%%%%%%%%%%%%%%%%%%%%%%%%%%%%%%%%%%%%%%%%%%%%%%%%%%%%%%%%%%%%%%%%%%%%
%%%%%%%%%%%%%%%%%%%%%%%%%%%%%%%%%%%%%%%%%%%%%%%%%%%%%%%%%%%%%%%%%%%%%%%%%%%%%%%%
%%%%%%%%%%%%%%%%%%%%%%%%%%%%%%%%%%%%%%%%%%%%%%%%%%%%%%%%%%%%%%%%%%%%%%%%%%%%%%%%
\begin{mdframed}[style=darkQuesion]
13. Let $G$ be an abelian group, such that the operation on $G$
is denoted additively.
Show that $\{a \in G | 2 a=0\}$ is a subgroup of $G .$
Compute this subgroup for $G=\mathbf{Z}_{12}$

\end{mdframed}

%%%%%%%%%%%%%%%%%%%%%%%%%%%%%%%%%%%%%%%%%%%%%%%%%%%%%%%%%%%%%%%%%%%%%%%%%%%%%%%%
\begin{mdframed}[style=darkAnswer,frametitle={Joe Starr}]
Let $S=\{a \in G | 2 a=0\}$
\begin{itemize}[align=left]
\Invs{
    Let $a\in S$, by transitive proof $2a=0\rightarrow a+a=0\rightarrow a=-a$
  }
\Clos{
    Let $a,b\in S$, by transitive proof,
    \begin{align*}
    2a=0 &\rightarrow 2a+2b=0+0\\
    &\rightarrow 2(a+b)=0\\
    \end{align*}
    showing closure of $S$.
  }
\end{itemize}
next we let $G=\Z_{12}$ we can calculate $a+a$ for all $a\in \Z_{12}$
\begin{align*}
0+0 &= 0\\
1+1 &= 2\\
2+2 &= 4\\
3+3 &= 6\\
4+4 &= 8\\
5+5 &=10\\
6+6 &= 0\\
7+7 &= 2\\
8+8 &= 4\\
9+9 &= 6\\
10+10&= 8\\
\end{align*}
\end{mdframed}
\newpage
%%%%%%%%%%%%%%%%%%%%%%%%%%%%%%%%%%%%%%%%%%%%%%%%%%%%%%%%%%%%%%%%%%%%%%%%%%%%%%%%
%%%%%%%%%%%%%%%%%%%%%%%%%%%%%%%%%%%%%%%%%%%%%%%%%%%%%%%%%%%%%%%%%%%%%%%%%%%%%%%%
%%%%%%%%%%%%%%%%%%%%%%%%%%%%%%%%%%%%%%%%%%%%%%%%%%%%%%%%%%%%%%%%%%%%%%%%%%%%%%%%
%%%%%%%%%%%%%%%%%%%%%%%%%%%%%%%%%%%%%%%%%%%%%%%%%%%%%%%%%%%%%%%%%%%%%%%%%%%%%%%%
\begin{mdframed}[style=darkQuesion]
14. Let $G$ be an abelian group, and let $H$ be the set of all elements of $G$ of finite order.
\begin{itemize}
\item[]{
    (a) Show that $H$ is a subgroup of $G .$}
\item[]{
    (b) For a fixed positive integer $k$, show that $\{a \in G | o(a) \text{is a divisor of } k\}$ is a subgroup of $H$
  }
\item[]{
    (c) For a fixed positive integer $k$, is $\{a \in G | o(a) \leq k\}$ a subgroup of $H ?$ Either give a proof or give a counterexample.
  }
\end{itemize}

\end{mdframed}

%%%%%%%%%%%%%%%%%%%%%%%%%%%%%%%%%%%%%%%%%%%%%%%%%%%%%%%%%%%%%%%%%%%%%%%%%%%%%%%%
\begin{mdframed}[style=darkAnswer,frametitle={Joe Starr}]
\begin{itemize}[align=left]
\item[(a) ]{
    \begin{itemize}[align=left]
    \Invs{
        Let $a\in H$, since $a$ is of finite order we have $a^n=1$ for some
        $n\in \Z$. We observe that $aa^{n-1}=1$, now considering $a^{n-1}$,
        we take this to the nth power $\lrp{a^{n-1}}^n=\lrp{a^n}^{n-1}=1$
        showing existence of inverses in $H$.
      }
    \Clos{
        Let $a,b\in H $, we consider $ab$ if we know $a^n=1$ and $b^m=1$ for
        some $m,n\in \Z$, if we take $a^{mn}b^{mn}=\lrp{a^n}^m\lrp{b^m}^n=1$.
      }
    \end{itemize}
  }
\item[(b)]{
    Let $A=\{a \in G | o(a) \text{is a divisor of } k\}$
    \begin{itemize}[align=left]
    \Invs{
        Let $a\in A$ we observe that since $k \vert \ord{a}$ we have $a^{kq}=1$
        for some $kq\in \Z$ meaning $a^{kq-1}a=1$ now considering $a^{n-1}$,
        we take this to the nth power $\lrp{a^{kq-1}}^{kq}=\lrp{a^{kq}}^{kq-1}=1$
        showing existence of inverses in $A$.
      }
    \Clos{
        Let $a,b\in A $, we consider $ab$ if we know $a^{kn}=1$ and $b^{km}=1$ for
        some $m,n\in \Z$, if we take $a^{kmn}b^{kmn}=\lrp{a^kn}^m\lrp{b^km}^n=1$.
      }
    \end{itemize}
  }
\item[(c)]{
    Let $G=\Z_{10}^+$, and $k=5$, this makes $A=\lrs{2,\ 4,\ 5,\ 6,\ 8}$ if we
    take $2+5=7$ we can see $A$ is not closed under addition.
  }
\end{itemize}
\end{mdframed}
\newpage
%%%%%%%%%%%%%%%%%%%%%%%%%%%%%%%%%%%%%%%%%%%%%%%%%%%%%%%%%%%%%%%%%%%%%%%%%%%%%%%%
%%%%%%%%%%%%%%%%%%%%%%%%%%%%%%%%%%%%%%%%%%%%%%%%%%%%%%%%%%%%%%%%%%%%%%%%%%%%%%%%
%%%%%%%%%%%%%%%%%%%%%%%%%%%%%%%%%%%%%%%%%%%%%%%%%%%%%%%%%%%%%%%%%%%%%%%%%%%%%%%%
%%%%%%%%%%%%%%%%%%%%%%%%%%%%%%%%%%%%%%%%%%%%%%%%%%%%%%%%%%%%%%%%%%%%%%%%%%%%%%%%
\begin{mdframed}[style=darkQuesion]
15. Prove that any cyclic group is abelian.
\end{mdframed}

%%%%%%%%%%%%%%%%%%%%%%%%%%%%%%%%%%%%%%%%%%%%%%%%%%%%%%%%%%%%%%%%%%%%%%%%%%%%%%%%
\begin{mdframed}[style=darkAnswer,frametitle={Joe Starr}]
Let $G$ be a cyclic group generated by $g$, select $a,b\in G$, we consider
$ab\inv{a}\inv{b}$. We observe $a=g^k$ and $\inv{a}=g^{k-1}$ for some $k$,
similarly for $b$ and some $h$. Now rewriting
$ab\inv{a}\inv{b}=g^kg^hg^{k-1}g^{h-1}=1$ showing $G$ abelian.
\end{mdframed}
\newpage
%%%%%%%%%%%%%%%%%%%%%%%%%%%%%%%%%%%%%%%%%%%%%%%%%%%%%%%%%%%%%%%%%%%%%%%%%%%%%%%%
%%%%%%%%%%%%%%%%%%%%%%%%%%%%%%%%%%%%%%%%%%%%%%%%%%%%%%%%%%%%%%%%%%%%%%%%%%%%%%%%
%%%%%%%%%%%%%%%%%%%%%%%%%%%%%%%%%%%%%%%%%%%%%%%%%%%%%%%%%%%%%%%%%%%%%%%%%%%%%%%%
%%%%%%%%%%%%%%%%%%%%%%%%%%%%%%%%%%%%%%%%%%%%%%%%%%%%%%%%%%%%%%%%%%%%%%%%%%%%%%%%
\begin{mdframed}[style=darkQuesion]
16. Prove or disprove this statement. If $G$ is a group in which every proper subgroup is cyclic, then $G$ is cyclic.

\end{mdframed}

%%%%%%%%%%%%%%%%%%%%%%%%%%%%%%%%%%%%%%%%%%%%%%%%%%%%%%%%%%%%%%%%%%%%%%%%%%%%%%%%
\begin{mdframed}[style=darkAnswer,frametitle={Joe Starr}]
Select $G$ with the given property. Let $H$

\end{mdframed}
\newpage
%%%%%%%%%%%%%%%%%%%%%%%%%%%%%%%%%%%%%%%%%%%%%%%%%%%%%%%%%%%%%%%%%%%%%%%%%%%%%%%%
%%%%%%%%%%%%%%%%%%%%%%%%%%%%%%%%%%%%%%%%%%%%%%%%%%%%%%%%%%%%%%%%%%%%%%%%%%%%%%%%
%%%%%%%%%%%%%%%%%%%%%%%%%%%%%%%%%%%%%%%%%%%%%%%%%%%%%%%%%%%%%%%%%%%%%%%%%%%%%%%%
%%%%%%%%%%%%%%%%%%%%%%%%%%%%%%%%%%%%%%%%%%%%%%%%%%%%%%%%%%%%%%%%%%%%%%%%%%%%%%%%
\begin{mdframed}[style=darkQuesion]
17. Prove that the intersection of any collection of subgroups of a group is again a subgroup.

\end{mdframed}

%%%%%%%%%%%%%%%%%%%%%%%%%%%%%%%%%%%%%%%%%%%%%%%%%%%%%%%%%%%%%%%%%%%%%%%%%%%%%%%%
\begin{mdframed}[style=darkAnswer,frametitle={Joe Starr}]
Let $H$ and $K$ be subgroups of a group $G$, consider $a\in H\cap K$. Since
both $K$ and $H$ are groups $\inv{a} \in H$ and $\inv{a} \in K$ putting it in
the intersection. Next we consider $a,b\in H\cap K$, since
both $K$ and $H$ are groups $ab \in H$ and $ab \in K$ putting it in
the intersection. Showing $ H\cap K$ a subgroup.
\end{mdframed}
\newpage
%%%%%%%%%%%%%%%%%%%%%%%%%%%%%%%%%%%%%%%%%%%%%%%%%%%%%%%%%%%%%%%%%%%%%%%%%%%%%%%%
%%%%%%%%%%%%%%%%%%%%%%%%%%%%%%%%%%%%%%%%%%%%%%%%%%%%%%%%%%%%%%%%%%%%%%%%%%%%%%%%
%%%%%%%%%%%%%%%%%%%%%%%%%%%%%%%%%%%%%%%%%%%%%%%%%%%%%%%%%%%%%%%%%%%%%%%%%%%%%%%%
%%%%%%%%%%%%%%%%%%%%%%%%%%%%%%%%%%%%%%%%%%%%%%%%%%%%%%%%%%%%%%%%%%%%%%%%%%%%%%%%
\begin{mdframed}[style=darkQuesion]
18. Let $G$ be the group of rational numbers, under addition, and let $H, K$ be subgroups
of $G .$ Prove that if $H \neq\{0\}$ and $K \neq\{0\}$, then $H \cap K \neq\{0\}$.

\end{mdframed}

%%%%%%%%%%%%%%%%%%%%%%%%%%%%%%%%%%%%%%%%%%%%%%%%%%%%%%%%%%%%%%%%%%%%%%%%%%%%%%%%
\begin{mdframed}[style=darkAnswer,frametitle={Joe Starr}]
We have previously shown that the intersection of two subgroups is a subgroup.
Since neither $H$ nor $K $ are the trivial subgroup we have $\frac{a}{b}\in H$
and $\frac{m}{n}\in K$, $a,\ b,\ m,\ n$ with the usual properties. We observe
$b\frac{a}{b}\in H$ and $n\frac{m}{n}\in K$, we then can add these $m$ and
$a$ times respectively, yielding $bm\frac{a}{b}=ma$ and $na\frac{m}{n}=ma$
putting $ma\in H\cap K$.
\end{mdframed}
\newpage
%%%%%%%%%%%%%%%%%%%%%%%%%%%%%%%%%%%%%%%%%%%%%%%%%%%%%%%%%%%%%%%%%%%%%%%%%%%%%%%%
%%%%%%%%%%%%%%%%%%%%%%%%%%%%%%%%%%%%%%%%%%%%%%%%%%%%%%%%%%%%%%%%%%%%%%%%%%%%%%%%
%%%%%%%%%%%%%%%%%%%%%%%%%%%%%%%%%%%%%%%%%%%%%%%%%%%%%%%%%%%%%%%%%%%%%%%%%%%%%%%%
%%%%%%%%%%%%%%%%%%%%%%%%%%%%%%%%%%%%%%%%%%%%%%%%%%%%%%%%%%%%%%%%%%%%%%%%%%%%%%%%
\begin{mdframed}[style=darkQuesion]
19. Let $G$ be a group, and let $a \in G .$ The set
$C(a)=\{x \in G | x a=a x\}$ of all elements of $G$ that commute with $a$ is
called the centralizer of $a .$
\begin{itemize}
\item[]{(a) Show that  $C(a)$ is a subgroup of $G$}
\item[]{(b) Show that $\langle a\rangle \subseteq C(a)$}
\item[]{(c) Compute $C(a)$ if $G=S_{3}$ and $a=(1,2,3)$}
\item[]{(d) Compute $C(a)$ if $G=S_{3}$ and $a=(1,2)$}
\end{itemize}

\end{mdframed}

%%%%%%%%%%%%%%%%%%%%%%%%%%%%%%%%%%%%%%%%%%%%%%%%%%%%%%%%%%%%%%%%%%%%%%%%%%%%%%%%
\begin{mdframed}[style=darkAnswer,frametitle={Joe Starr}]
\begin{multicols}{2}
\begin{itemize}[align=left]
\item[(a)]{
    \begin{itemize}[align=left]
    \Invs{
        Let $x\in C\lrp{a}$, by construction we know $xa=ax$, by transitive proof
        \begin{align*}
        xa=ax &\rightarrow \inv{x}xa=\inv{x}ax\\
        &\rightarrow a\inv{x}=\inv{x}ax\inv{x}\\
        &\rightarrow a\inv{x}=\inv{x}a
        \end{align*}
        putting $\inv{x}\in C\lrp{a}$. 
      }
    \Clos{
        let $x,\ y\in C\lrp{a}$, by construction we know $xa=ax$ and $ya=ay$. 
        if we take $xa=ax$ and multiply by $y$ on the left we get $yxa=yax$ 
        then by commutativity we have $yxa=ayx$ putting $yx\in C\lrp{a}$ 
        similarly for $xy$. 
      }
    \end{itemize}}
\item[(b)]{
    \begin{itemize}[align=left]
    \Invs{
          If we consider $a\in \lra{a}$
      }
    \Clos{

      }
    \end{itemize}}
\item[(c)]{
    \begin{itemize}[align=left]
    \Invs{

      }
    \Clos{

      }
    \end{itemize}}
\item[(d)]{
    \begin{itemize}[align=left]
    \Invs{

      }
    \Clos{

      }
    \end{itemize}}
\end{itemize}
\end{multicols}
\end{mdframed}
\newpage
%%%%%%%%%%%%%%%%%%%%%%%%%%%%%%%%%%%%%%%%%%%%%%%%%%%%%%%%%%%%%%%%%%%%%%%%%%%%%%%%
%%%%%%%%%%%%%%%%%%%%%%%%%%%%%%%%%%%%%%%%%%%%%%%%%%%%%%%%%%%%%%%%%%%%%%%%%%%%%%%%
%%%%%%%%%%%%%%%%%%%%%%%%%%%%%%%%%%%%%%%%%%%%%%%%%%%%%%%%%%%%%%%%%%%%%%%%%%%%%%%%
%%%%%%%%%%%%%%%%%%%%%%%%%%%%%%%%%%%%%%%%%%%%%%%%%%%%%%%%%%%%%%%%%%%%%%%%%%%%%%%%
\begin{mdframed}[style=darkQuesion]
20. Compute the centralizer in $\mathrm{GL}_{2}$ ( $\mathbf{R}$ ) of the matrix $\left[\begin{array}{ll}1 & 1 \\ 0 & 1\end{array}\right]$

\end{mdframed}

%%%%%%%%%%%%%%%%%%%%%%%%%%%%%%%%%%%%%%%%%%%%%%%%%%%%%%%%%%%%%%%%%%%%%%%%%%%%%%%%
\begin{mdframed}[style=darkAnswer,frametitle={Joe Starr}]

\end{mdframed}
\newpage
%%%%%%%%%%%%%%%%%%%%%%%%%%%%%%%%%%%%%%%%%%%%%%%%%%%%%%%%%%%%%%%%%%%%%%%%%%%%%%%%
%%%%%%%%%%%%%%%%%%%%%%%%%%%%%%%%%%%%%%%%%%%%%%%%%%%%%%%%%%%%%%%%%%%%%%%%%%%%%%%%
%%%%%%%%%%%%%%%%%%%%%%%%%%%%%%%%%%%%%%%%%%%%%%%%%%%%%%%%%%%%%%%%%%%%%%%%%%%%%%%%
%%%%%%%%%%%%%%%%%%%%%%%%%%%%%%%%%%%%%%%%%%%%%%%%%%%%%%%%%%%%%%%%%%%%%%%%%%%%%%%%
\begin{mdframed}[style=darkQuesion]
22. Show that if a group $G$ has a unique element $a$ of order $2,$ then $a \in Z(G)$

\end{mdframed}

%%%%%%%%%%%%%%%%%%%%%%%%%%%%%%%%%%%%%%%%%%%%%%%%%%%%%%%%%%%%%%%%%%%%%%%%%%%%%%%%
\begin{mdframed}[style=darkAnswer,frametitle={Joe Starr}]

\end{mdframed}
\newpage
%%%%%%%%%%%%%%%%%%%%%%%%%%%%%%%%%%%%%%%%%%%%%%%%%%%%%%%%%%%%%%%%%%%%%%%%%%%%%%%%
%%%%%%%%%%%%%%%%%%%%%%%%%%%%%%%%%%%%%%%%%%%%%%%%%%%%%%%%%%%%%%%%%%%%%%%%%%%%%%%%
%%%%%%%%%%%%%%%%%%%%%%%%%%%%%%%%%%%%%%%%%%%%%%%%%%%%%%%%%%%%%%%%%%%%%%%%%%%%%%%%
%%%%%%%%%%%%%%%%%%%%%%%%%%%%%%%%%%%%%%%%%%%%%%%%%%%%%%%%%%%%%%%%%%%%%%%%%%%%%%%%
\begin{mdframed}[style=darkQuesion]
23. If the group $G$ is not abelian, show that its center $Z(G)$ is a proper subgroup of an abelian subgroup of $G .$

\end{mdframed}

%%%%%%%%%%%%%%%%%%%%%%%%%%%%%%%%%%%%%%%%%%%%%%%%%%%%%%%%%%%%%%%%%%%%%%%%%%%%%%%%
\begin{mdframed}[style=darkAnswer,frametitle={Joe Starr}]

\end{mdframed}
\newpage
%%%%%%%%%%%%%%%%%%%%%%%%%%%%%%%%%%%%%%%%%%%%%%%%%%%%%%%%%%%%%%%%%%%%%%%%%%%%%%%%
%%%%%%%%%%%%%%%%%%%%%%%%%%%%%%%%%%%%%%%%%%%%%%%%%%%%%%%%%%%%%%%%%%%%%%%%%%%%%%%%
%%%%%%%%%%%%%%%%%%%%%%%%%%%%%%%%%%%%%%%%%%%%%%%%%%%%%%%%%%%%%%%%%%%%%%%%%%%%%%%%
%%%%%%%%%%%%%%%%%%%%%%%%%%%%%%%%%%%%%%%%%%%%%%%%%%%%%%%%%%%%%%%%%%%%%%%%%%%%%%%%
\begin{mdframed}[style=darkQuesion]
26. Let $G$ be a group with $a, b \in G$.
(a) Show that $o\left(a^{-1}\right)=o(a)$
(b) Show that $o(a b)=o(b a)$
(c) Show that $o\left(a b a^{-1}\right)=o(b)$

\end{mdframed}

%%%%%%%%%%%%%%%%%%%%%%%%%%%%%%%%%%%%%%%%%%%%%%%%%%%%%%%%%%%%%%%%%%%%%%%%%%%%%%%%
\begin{mdframed}[style=darkAnswer,frametitle={Joe Starr}]

\end{mdframed}
\newpage
%%%%%%%%%%%%%%%%%%%%%%%%%%%%%%%%%%%%%%%%%%%%%%%%%%%%%%%%%%%%%%%%%%%%%%%%%%%%%%%%
%%%%%%%%%%%%%%%%%%%%%%%%%%%%%%%%%%%%%%%%%%%%%%%%%%%%%%%%%%%%%%%%%%%%%%%%%%%%%%%%
%%%%%%%%%%%%%%%%%%%%%%%%%%%%%%%%%%%%%%%%%%%%%%%%%%%%%%%%%%%%%%%%%%%%%%%%%%%%%%%%
%%%%%%%%%%%%%%%%%%%%%%%%%%%%%%%%%%%%%%%%%%%%%%%%%%%%%%%%%%%%%%%%%%%%%%%%%%%%%%%%
\begin{mdframed}[style=darkQuesion]
27. Let $G$ be a finite group, let $n>2$ be an integer, and let $S$ be the set of elements of
$G$ that have order $n .$ Show that $S$ has an even number of elements.

\end{mdframed}

%%%%%%%%%%%%%%%%%%%%%%%%%%%%%%%%%%%%%%%%%%%%%%%%%%%%%%%%%%%%%%%%%%%%%%%%%%%%%%%%
\begin{mdframed}[style=darkAnswer,frametitle={Joe Starr}]

\end{mdframed}
\newpage
%%%%%%%%%%%%%%%%%%%%%%%%%%%%%%%%%%%%%%%%%%%%%%%%%%%%%%%%%%%%%%%%%%%%%%%%%%%%%%%%
%%%%%%%%%%%%%%%%%%%%%%%%%%%%%%%%%%%%%%%%%%%%%%%%%%%%%%%%%%%%%%%%%%%%%%%%%%%%%%%%
%%%%%%%%%%%%%%%%%%%%%%%%%%%%%%%%%%%%%%%%%%%%%%%%%%%%%%%%%%%%%%%%%%%%%%%%%%%%%%%%
%%%%%%%%%%%%%%%%%%%%%%%%%%%%%%%%%%%%%%%%%%%%%%%%%%%%%%%%%%%%%%%%%%%%%%%%%%%%%%%%
\begin{mdframed}[style=darkQuesion]
28. Let $G$ be a group with $a, b \in G$. Assume that $o(a)$ and $o(b)$ are finite and relatively prime, and that $a b=b a$. Show that $o(a b)=o(a) o(b)$

\end{mdframed}

%%%%%%%%%%%%%%%%%%%%%%%%%%%%%%%%%%%%%%%%%%%%%%%%%%%%%%%%%%%%%%%%%%%%%%%%%%%%%%%%
\begin{mdframed}[style=darkAnswer,frametitle={Joe Starr}]

\end{mdframed}
\newpage
%%%%%%%%%%%%%%%%%%%%%%%%%%%%%%%%%%%%%%%%%%%%%%%%%%%%%%%%%%%%%%%%%%%%%%%%%%%%%%%%
%%%%%%%%%%%%%%%%%%%%%%%%%%%%%%%%%%%%%%%%%%%%%%%%%%%%%%%%%%%%%%%%%%%%%%%%%%%%%%%%
%%%%%%%%%%%%%%%%%%%%%%%%%%%%%%%%%%%%%%%%%%%%%%%%%%%%%%%%%%%%%%%%%%%%%%%%%%%%%%%%
%%%%%%%%%%%%%%%%%%%%%%%%%%%%%%%%%%%%%%%%%%%%%%%%%%%%%%%%%%%%%%%%%%%%%%%%%%%%%%%%
\begin{mdframed}[style=darkQuesion]
29. Find an example of a group $G$ and elements $a, b \in G$ such that $a$ and $b$ each have finite order but $a b$ does not.

\end{mdframed}

%%%%%%%%%%%%%%%%%%%%%%%%%%%%%%%%%%%%%%%%%%%%%%%%%%%%%%%%%%%%%%%%%%%%%%%%%%%%%%%%
\begin{mdframed}[style=darkAnswer,frametitle={Joe Starr}]

\end{mdframed}
%TODO: Section 3.3
\newpage
%%%%%%%%%%%%%%%%%%%%%%%%%%%%%%%%%%%%%%%%%%%%%%%%%%%%%%%%%%%%%%%%%%%%%%%%%%%%%%%%
%%%%%%%%%%%%%%%%%%%%%%%%%%%%%%%%%%%%%%%%%%%%%%%%%%%%%%%%%%%%%%%%%%%%%%%%%%%%%%%%
%%%%%%%%%%%%%%%%%%%%%%%%%%%%%%%%%%%%%%%%%%%%%%%%%%%%%%%%%%%%%%%%%%%%%%%%%%%%%%%%
%%%%%%%%%%%%%%%%%%%%%%%%%%%%%%%%%%%%%%%%%%%%%%%%%%%%%%%%%%%%%%%%%%%%%%%%%%%%%%%%
\begin{mdframed}[style=darkQuesion]
2.
\end{mdframed}

%%%%%%%%%%%%%%%%%%%%%%%%%%%%%%%%%%%%%%%%%%%%%%%%%%%%%%%%%%%%%%%%%%%%%%%%%%%%%%%%
\begin{mdframed}[style=darkAnswer,frametitle={Joe Starr}]

\end{mdframed}
\newpage
%%%%%%%%%%%%%%%%%%%%%%%%%%%%%%%%%%%%%%%%%%%%%%%%%%%%%%%%%%%%%%%%%%%%%%%%%%%%%%%%
%%%%%%%%%%%%%%%%%%%%%%%%%%%%%%%%%%%%%%%%%%%%%%%%%%%%%%%%%%%%%%%%%%%%%%%%%%%%%%%%
%%%%%%%%%%%%%%%%%%%%%%%%%%%%%%%%%%%%%%%%%%%%%%%%%%%%%%%%%%%%%%%%%%%%%%%%%%%%%%%%
%%%%%%%%%%%%%%%%%%%%%%%%%%%%%%%%%%%%%%%%%%%%%%%%%%%%%%%%%%%%%%%%%%%%%%%%%%%%%%%%
\begin{mdframed}[style=darkQuesion]
2.
\end{mdframed}

%%%%%%%%%%%%%%%%%%%%%%%%%%%%%%%%%%%%%%%%%%%%%%%%%%%%%%%%%%%%%%%%%%%%%%%%%%%%%%%%
\begin{mdframed}[style=darkAnswer,frametitle={Joe Starr}]

\end{mdframed}
\clearpage
\subsection{Constructing Examples}
%%%%%%%%%%%%%%%%%%%%%%%%%%%%%%%%%%%%%%%%%%%%%%%%%%%%%%%%%%%%%%%%%%%%%%%%%%%%%%%%
%%%%%%%%%%%%%%%%%%%%%%%%%%%%%%%%%%%%%%%%%%%%%%%%%%%%%%%%%%%%%%%%%%%%%%%%%%%%%%%%
%%%%%%%%%%%%%%%%%%%%%%%%%%%%%%%%%%%%%%%%%%%%%%%%%%%%%%%%%%%%%%%%%%%%%%%%%%%%%%%%
%%%%%%%%%%%%%%%%%%%%%%%%%%%%%%%%%%%%%%%%%%%%%%%%%%%%%%%%%%%%%%%%%%%%%%%%%%%%%%%%
\begin{mdframed}[style=darkQuesion]
  1. $\dagger$ Find $H K$ in $\mathbf{Z}_{16}^{\times}$, if $H=\langle[3]\rangle$ and $K=\langle[5]\rangle$
\end{mdframed}
%%%%%%%%%%%%%%%%%%%%%%%%%%%%%%%%%%%%%%%%%%%%%%%%%%%%%%%%%%%%%%%%%%%%%%%%%%%%%%%%
\begin{mdframed}[style=darkAnswer,frametitle={Joe Starr}]
  We begin by calculating $H=\lrs{1,\ 3,\ 9,\ 11}$ and $K=\lrs{1,\ 5,\ 9,\ 13}$. 
  now by observing $(3\cdot13)\%16=7$ and $(3\cdot5)\%16=15$ we get $\Z_{16}^\times$.
\end{mdframed}
\newpage
%%%%%%%%%%%%%%%%%%%%%%%%%%%%%%%%%%%%%%%%%%%%%%%%%%%%%%%%%%%%%%%%%%%%%%%%%%%%%%%%
%%%%%%%%%%%%%%%%%%%%%%%%%%%%%%%%%%%%%%%%%%%%%%%%%%%%%%%%%%%%%%%%%%%%%%%%%%%%%%%%
%%%%%%%%%%%%%%%%%%%%%%%%%%%%%%%%%%%%%%%%%%%%%%%%%%%%%%%%%%%%%%%%%%%%%%%%%%%%%%%%
%%%%%%%%%%%%%%%%%%%%%%%%%%%%%%%%%%%%%%%%%%%%%%%%%%%%%%%%%%%%%%%%%%%%%%%%%%%%%%%%
\begin{mdframed}[style=darkQuesion]
  2. Find $H K$ in $\mathbf{Z}_{21}^{\times},$ if $H=\{[1],[8]\}$ and $K=\{[1],[4],[10],[13],[16],[19]\}$
\end{mdframed}
%%%%%%%%%%%%%%%%%%%%%%%%%%%%%%%%%%%%%%%%%%%%%%%%%%%%%%%%%%%%%%%%%%%%%%%%%%%%%%%%
\begin{mdframed}[style=darkAnswer,frametitle={Joe Starr}]
  \begin{align*}
    \lrp{1\cdot1 }\% 21 &=1\\
\lrp{1\cdot4 }\% 21 &=4\\
\lrp{1\cdot10}\% 21 &= 10\\
\lrp{1\cdot13}\% 21 &= 13\\
\lrp{1\cdot16}\% 21 &= 16\\
\lrp{1\cdot19}\% 21 &= 19\\
\lrp{8\cdot1 }\% 21 &=8\\
\lrp{8\cdot4 }\% 21 &=11\\
\lrp{8\cdot10}\% 21 &= 17\\
\lrp{8\cdot13}\% 21 &= 20\\
\lrp{8\cdot16}\% 21 &= 2\\
\lrp{8\cdot19}\% 21 &= 5
  \end{align*}
  $H=\lrs{1,\ 2,\ 4,\ 5,\ 8,\ 10,\ 11,\ 13,\ 16,\ 17,\ 19,\ 20}$
\end{mdframed}
\newpage
%%%%%%%%%%%%%%%%%%%%%%%%%%%%%%%%%%%%%%%%%%%%%%%%%%%%%%%%%%%%%%%%%%%%%%%%%%%%%%%%
%%%%%%%%%%%%%%%%%%%%%%%%%%%%%%%%%%%%%%%%%%%%%%%%%%%%%%%%%%%%%%%%%%%%%%%%%%%%%%%%
%%%%%%%%%%%%%%%%%%%%%%%%%%%%%%%%%%%%%%%%%%%%%%%%%%%%%%%%%%%%%%%%%%%%%%%%%%%%%%%%
%%%%%%%%%%%%%%%%%%%%%%%%%%%%%%%%%%%%%%%%%%%%%%%%%%%%%%%%%%%%%%%%%%%%%%%%%%%%%%%%
\begin{mdframed}[style=darkQuesion]
  3.  Find an example of two subgroups $H$ and $K$ of $S_{3}$ for which $H K$ is not a subgroup.

\end{mdframed}
%%%%%%%%%%%%%%%%%%%%%%%%%%%%%%%%%%%%%%%%%%%%%%%%%%%%%%%%%%%%%%%%%%%%%%%%%%%%%%%%
\begin{mdframed}[style=darkAnswer,frametitle={Joe Starr}]
  Let $H=\lrs{\lrp{\ },\ \lrp{1,2}}$ and 
  $K=\lrs{\lrp{\ },\ \lrp{1,2,3},\ \lrp{1,3,2}}$ we then
  take \\
  $HK= \lrs{\lrp{\ },\ \lrp{1,2},\ \lrp{2,3},\ 
            \lrp{1,3},\ \lrp{1,2,3},\ \lrp{1,3,2}}$
  we observe this isn't a proper subgroup of $S_3$.
\end{mdframed}
\newpage
%%%%%%%%%%%%%%%%%%%%%%%%%%%%%%%%%%%%%%%%%%%%%%%%%%%%%%%%%%%%%%%%%%%%%%%%%%%%%%%%
%%%%%%%%%%%%%%%%%%%%%%%%%%%%%%%%%%%%%%%%%%%%%%%%%%%%%%%%%%%%%%%%%%%%%%%%%%%%%%%%
%%%%%%%%%%%%%%%%%%%%%%%%%%%%%%%%%%%%%%%%%%%%%%%%%%%%%%%%%%%%%%%%%%%%%%%%%%%%%%%%
%%%%%%%%%%%%%%%%%%%%%%%%%%%%%%%%%%%%%%%%%%%%%%%%%%%%%%%%%%%%%%%%%%%%%%%%%%%%%%%%
\begin{mdframed}[style=darkQuesion]
  4 Show that the list of elements of $\mathrm{GL}_{2}\left(\mathbf{Z}_{2}\right)$ given in Example 3.3.6 is correct.
\end{mdframed}
%%%%%%%%%%%%%%%%%%%%%%%%%%%%%%%%%%%%%%%%%%%%%%%%%%%%%%%%%%%%%%%%%%%%%%%%%%%%%%%%
\begin{mdframed}[style=darkAnswer,frametitle={Joe Starr}]
$$
\begin{matrix}
a=\begin{bmatrix}0&0\\0&0\end{bmatrix}&
b=\begin{bmatrix}0&0\\0&1\end{bmatrix}&
c=\begin{bmatrix}0&0\\1&0\end{bmatrix}&
d=\begin{bmatrix}0&0\\1&1\end{bmatrix}\\ \hspace{.5in} \\
e=\begin{bmatrix}0&1\\0&0\end{bmatrix}&
f=\begin{bmatrix}0&1\\0&1\end{bmatrix}&
g=\begin{bmatrix}0&1\\1&0\end{bmatrix}&
h=\begin{bmatrix}0&1\\1&1\end{bmatrix}\\ \hspace{.5in} \\
i=\begin{bmatrix}1&0\\0&0\end{bmatrix}&
j=\begin{bmatrix}1&0\\0&1\end{bmatrix}&
k=\begin{bmatrix}1&0\\1&0\end{bmatrix}&
l=\begin{bmatrix}1&0\\1&1\end{bmatrix}\\ \hspace{.5in} \\
m=\begin{bmatrix}1&1\\0&0\end{bmatrix}&
n=\begin{bmatrix}1&1\\0&1\end{bmatrix}&
o=\begin{bmatrix}1&1\\1&0\end{bmatrix}&
p=\begin{bmatrix}1&1\\1&1\end{bmatrix}\\ \hspace{.5in} \\
\end{matrix}
  $$
  Det of a is $  0$\\
  Det of b is $  0$\\
  Det of c is $  0$\\
  Det of d is $  0$\\
  Det of e is $  0$\\
  Det of f is $  0$\\
  Det of g is $\m1$\\
  Det of h is $\m1$\\
  Det of i is $  0$\\
  Det of j is $  1$\\
  Det of k is $  0$\\
  Det of l is $  1$\\
  Det of m is $  0$\\
  Det of n is $  1$\\
  Det of o is $\m1$\\
  Det of p is $  0$\\
\end{mdframed}
\newpage
%%%%%%%%%%%%%%%%%%%%%%%%%%%%%%%%%%%%%%%%%%%%%%%%%%%%%%%%%%%%%%%%%%%%%%%%%%%%%%%%
%%%%%%%%%%%%%%%%%%%%%%%%%%%%%%%%%%%%%%%%%%%%%%%%%%%%%%%%%%%%%%%%%%%%%%%%%%%%%%%%
%%%%%%%%%%%%%%%%%%%%%%%%%%%%%%%%%%%%%%%%%%%%%%%%%%%%%%%%%%%%%%%%%%%%%%%%%%%%%%%%
%%%%%%%%%%%%%%%%%%%%%%%%%%%%%%%%%%%%%%%%%%%%%%%%%%%%%%%%%%%%%%%%%%%%%%%%%%%%%%%%
\begin{mdframed}[style=darkQuesion]
  5 Find $\left|\mathrm{GL}_{2}\left(\mathbf{Z}_{3}\right)\right|$
\end{mdframed}
%%%%%%%%%%%%%%%%%%%%%%%%%%%%%%%%%%%%%%%%%%%%%%%%%%%%%%%%%%%%%%%%%%%%%%%%%%%%%%%%
\begin{mdframed}[style=darkAnswer,frametitle={Joe Starr}]
  $$
\begin{matrix}
\text{det}\lrp{\begin{bmatrix} 0 & 1 \\ 1 & 0 \end{bmatrix}}= 2 &
\text{det}\lrp{\begin{bmatrix} 0 & 1 \\ 1 & 1 \end{bmatrix}}= 2 &
\text{det}\lrp{\begin{bmatrix} 0 & 1 \\ 1 & 2 \end{bmatrix}}= 2 &
\text{det}\lrp{\begin{bmatrix} 0 & 1 \\ 2 & 0 \end{bmatrix}}= 1 \\ \hspace{.5in} \\
\text{det}\lrp{\begin{bmatrix} 0 & 1 \\ 2 & 1 \end{bmatrix}}= 1 &
\text{det}\lrp{\begin{bmatrix} 0 & 1 \\ 2 & 2 \end{bmatrix}}= 1 &
\text{det}\lrp{\begin{bmatrix} 0 & 2 \\ 1 & 0 \end{bmatrix}}= 1 &
\text{det}\lrp{\begin{bmatrix} 0 & 2 \\ 1 & 1 \end{bmatrix}}= 1 \\ \hspace{.5in} \\
\text{det}\lrp{\begin{bmatrix} 0 & 2 \\ 1 & 2 \end{bmatrix}}= 1 &
\text{det}\lrp{\begin{bmatrix} 0 & 2 \\ 2 & 0 \end{bmatrix}}= 2 &
\text{det}\lrp{\begin{bmatrix} 0 & 2 \\ 2 & 1 \end{bmatrix}}= 2 &
\text{det}\lrp{\begin{bmatrix} 0 & 2 \\ 2 & 2 \end{bmatrix}}= 2 \\ \hspace{.5in} \\
\text{det}\lrp{\begin{bmatrix} 1 & 0 \\ 0 & 1 \end{bmatrix}}= 1 &
\text{det}\lrp{\begin{bmatrix} 1 & 0 \\ 0 & 2 \end{bmatrix}}= 2 &
\text{det}\lrp{\begin{bmatrix} 1 & 0 \\ 1 & 1 \end{bmatrix}}= 1 &
\text{det}\lrp{\begin{bmatrix} 1 & 0 \\ 1 & 2 \end{bmatrix}}= 2 \\ \hspace{.5in} \\
\text{det}\lrp{\begin{bmatrix} 1 & 0 \\ 2 & 1 \end{bmatrix}}= 1 &
\text{det}\lrp{\begin{bmatrix} 1 & 0 \\ 2 & 2 \end{bmatrix}}= 2 &
\text{det}\lrp{\begin{bmatrix} 1 & 1 \\ 0 & 1 \end{bmatrix}}= 1 &
\text{det}\lrp{\begin{bmatrix} 1 & 1 \\ 0 & 2 \end{bmatrix}}= 2 \\ \hspace{.5in} \\
\text{det}\lrp{\begin{bmatrix} 1 & 1 \\ 1 & 0 \end{bmatrix}}= 2 &
\text{det}\lrp{\begin{bmatrix} 1 & 1 \\ 1 & 2 \end{bmatrix}}= 1 &
\text{det}\lrp{\begin{bmatrix} 1 & 1 \\ 2 & 0 \end{bmatrix}}= 1 &
\text{det}\lrp{\begin{bmatrix} 1 & 1 \\ 2 & 1 \end{bmatrix}}= 2 \\ \hspace{.5in} \\
\text{det}\lrp{\begin{bmatrix} 1 & 2 \\ 0 & 1 \end{bmatrix}}= 1 &
\text{det}\lrp{\begin{bmatrix} 1 & 2 \\ 0 & 2 \end{bmatrix}}= 2 &
\text{det}\lrp{\begin{bmatrix} 1 & 2 \\ 1 & 0 \end{bmatrix}}= 1 &
\text{det}\lrp{\begin{bmatrix} 1 & 2 \\ 1 & 1 \end{bmatrix}}= 2 \\ \hspace{.5in} \\
\text{det}\lrp{\begin{bmatrix} 1 & 2 \\ 2 & 0 \end{bmatrix}}= 2 &
\text{det}\lrp{\begin{bmatrix} 1 & 2 \\ 2 & 2 \end{bmatrix}}= 1 &
\text{det}\lrp{\begin{bmatrix} 2 & 0 \\ 0 & 1 \end{bmatrix}}= 2 &
\text{det}\lrp{\begin{bmatrix} 2 & 0 \\ 0 & 2 \end{bmatrix}}= 1 \\ \hspace{.5in} \\
\text{det}\lrp{\begin{bmatrix} 2 & 0 \\ 1 & 1 \end{bmatrix}}= 2 &
\text{det}\lrp{\begin{bmatrix} 2 & 0 \\ 1 & 2 \end{bmatrix}}= 1 &
\text{det}\lrp{\begin{bmatrix} 2 & 0 \\ 2 & 1 \end{bmatrix}}= 2 &
\text{det}\lrp{\begin{bmatrix} 2 & 0 \\ 2 & 2 \end{bmatrix}}= 1 \\ \hspace{.5in} \\
\text{det}\lrp{\begin{bmatrix} 2 & 1 \\ 0 & 1 \end{bmatrix}}= 2 &
\text{det}\lrp{\begin{bmatrix} 2 & 1 \\ 0 & 2 \end{bmatrix}}= 1 &
\text{det}\lrp{\begin{bmatrix} 2 & 1 \\ 1 & 0 \end{bmatrix}}= 2 &
\text{det}\lrp{\begin{bmatrix} 2 & 1 \\ 1 & 1 \end{bmatrix}}= 1 \\ \hspace{.5in} \\
\text{det}\lrp{\begin{bmatrix} 2 & 1 \\ 2 & 0 \end{bmatrix}}= 1 &
\text{det}\lrp{\begin{bmatrix} 2 & 1 \\ 2 & 2 \end{bmatrix}}= 2 &
\text{det}\lrp{\begin{bmatrix} 2 & 2 \\ 0 & 1 \end{bmatrix}}= 2 &
\text{det}\lrp{\begin{bmatrix} 2 & 2 \\ 0 & 2 \end{bmatrix}}= 1 \\ \hspace{.5in} \\
\text{det}\lrp{\begin{bmatrix} 2 & 2 \\ 1 & 0 \end{bmatrix}}= 1 &
\text{det}\lrp{\begin{bmatrix} 2 & 2 \\ 1 & 2 \end{bmatrix}}= 2 &
\text{det}\lrp{\begin{bmatrix} 2 & 2 \\ 2 & 0 \end{bmatrix}}= 2 &
\text{det}\lrp{\begin{bmatrix} 2 & 2 \\ 2 & 1 \end{bmatrix}}= 1 \\ \hspace{.5in} \\
\end{matrix}
$$
$\left|\mathrm{GL}_{2}\left(\mathbf{\Z}_{3}\right)\right|=48$
\end{mdframed}
\newpage
%%%%%%%%%%%%%%%%%%%%%%%%%%%%%%%%%%%%%%%%%%%%%%%%%%%%%%%%%%%%%%%%%%%%%%%%%%%%%%%%
%%%%%%%%%%%%%%%%%%%%%%%%%%%%%%%%%%%%%%%%%%%%%%%%%%%%%%%%%%%%%%%%%%%%%%%%%%%%%%%%
%%%%%%%%%%%%%%%%%%%%%%%%%%%%%%%%%%%%%%%%%%%%%%%%%%%%%%%%%%%%%%%%%%%%%%%%%%%%%%%%
%%%%%%%%%%%%%%%%%%%%%%%%%%%%%%%%%%%%%%%%%%%%%%%%%%%%%%%%%%%%%%%%%%%%%%%%%%%%%%%%
\begin{mdframed}[style=darkQuesion]
  6. Find the cyclic subgroup generated by $\left[\begin{array}{ll}2 & 1 \\ 0 & 2\end{array}\right]$ in $\mathrm{GL}_{2}\left(\mathbf{Z}_{3}\right)$
\end{mdframed}
%%%%%%%%%%%%%%%%%%%%%%%%%%%%%%%%%%%%%%%%%%%%%%%%%%%%%%%%%%%%%%%%%%%%%%%%%%%%%%%%
\begin{mdframed}[style=darkAnswer,frametitle={Joe Starr}]
  $$
\begin{matrix}
a^{ 1 }=\lrp{\begin{bmatrix}2 & 1 \\ 0 & 2\end{bmatrix}}&
a^{ 2 }=\lrp{\begin{bmatrix} 1 & 1 \\ 0 & 1 \end{bmatrix}}&
a^{ 3 }=\lrp{\begin{bmatrix} 2 & 0 \\ 0 & 2 \end{bmatrix}}\\ \hspace{.5in} \\
a^{ 4 }=\lrp{\begin{bmatrix} 1 & 2 \\ 0 & 1 \end{bmatrix}}&
a^{ 5 }=\lrp{\begin{bmatrix} 2 & 2 \\ 0 & 2 \end{bmatrix}}&
a^{ 6 }=\lrp{\begin{bmatrix} 1 & 0 \\ 0 & 1 \end{bmatrix}}&
\end{matrix}
$$
\end{mdframed}
\newpage
%%%%%%%%%%%%%%%%%%%%%%%%%%%%%%%%%%%%%%%%%%%%%%%%%%%%%%%%%%%%%%%%%%%%%%%%%%%%%%%%
%%%%%%%%%%%%%%%%%%%%%%%%%%%%%%%%%%%%%%%%%%%%%%%%%%%%%%%%%%%%%%%%%%%%%%%%%%%%%%%%
%%%%%%%%%%%%%%%%%%%%%%%%%%%%%%%%%%%%%%%%%%%%%%%%%%%%%%%%%%%%%%%%%%%%%%%%%%%%%%%%
%%%%%%%%%%%%%%%%%%%%%%%%%%%%%%%%%%%%%%%%%%%%%%%%%%%%%%%%%%%%%%%%%%%%%%%%%%%%%%%%
\begin{mdframed}[style=darkQuesion]
  7. thet $F$ be a field. Compute the center of $\mathrm{GL}_{2}(F)$
\end{mdframed}
%%%%%%%%%%%%%%%%%%%%%%%%%%%%%%%%%%%%%%%%%%%%%%%%%%%%%%%%%%%%%%%%%%%%%%%%%%%%%%%%
\begin{mdframed}[style=darkAnswer,frametitle={Joe Starr}]
  %TODO 3.3 Q7 not done yet
\end{mdframed}
\newpage
%%%%%%%%%%%%%%%%%%%%%%%%%%%%%%%%%%%%%%%%%%%%%%%%%%%%%%%%%%%%%%%%%%%%%%%%%%%%%%%%
%%%%%%%%%%%%%%%%%%%%%%%%%%%%%%%%%%%%%%%%%%%%%%%%%%%%%%%%%%%%%%%%%%%%%%%%%%%%%%%%
%%%%%%%%%%%%%%%%%%%%%%%%%%%%%%%%%%%%%%%%%%%%%%%%%%%%%%%%%%%%%%%%%%%%%%%%%%%%%%%%
%%%%%%%%%%%%%%%%%%%%%%%%%%%%%%%%%%%%%%%%%%%%%%%%%%%%%%%%%%%%%%%%%%%%%%%%%%%%%%%%
\begin{mdframed}[style=darkQuesion]
  8. Prove that if $G_{1}$ and $G_{2}$ are abelian groups, then the direct product $G_{1} \times G_{2}$ is abelian.
\end{mdframed}
%%%%%%%%%%%%%%%%%%%%%%%%%%%%%%%%%%%%%%%%%%%%%%%%%%%%%%%%%%%%%%%%%%%%%%%%%%%%%%%%
\begin{mdframed}[style=darkAnswer,frametitle={Joe Starr}]
   %TODO 3.3 Q8 not done yet
\end{mdframed}
\newpage
%%%%%%%%%%%%%%%%%%%%%%%%%%%%%%%%%%%%%%%%%%%%%%%%%%%%%%%%%%%%%%%%%%%%%%%%%%%%%%%%
%%%%%%%%%%%%%%%%%%%%%%%%%%%%%%%%%%%%%%%%%%%%%%%%%%%%%%%%%%%%%%%%%%%%%%%%%%%%%%%%
%%%%%%%%%%%%%%%%%%%%%%%%%%%%%%%%%%%%%%%%%%%%%%%%%%%%%%%%%%%%%%%%%%%%%%%%%%%%%%%%
%%%%%%%%%%%%%%%%%%%%%%%%%%%%%%%%%%%%%%%%%%%%%%%%%%%%%%%%%%%%%%%%%%%%%%%%%%%%%%%%
\begin{mdframed}[style=darkQuesion]
  9. Construct an abelian group of order 12 that is not cyclic.
\end{mdframed}
%%%%%%%%%%%%%%%%%%%%%%%%%%%%%%%%%%%%%%%%%%%%%%%%%%%%%%%%%%%%%%%%%%%%%%%%%%%%%%%%
\begin{mdframed}[style=darkAnswer,frametitle={Joe Starr}]
   %TODO 3.3 Q9 not done yet
\end{mdframed}
\newpage
%%%%%%%%%%%%%%%%%%%%%%%%%%%%%%%%%%%%%%%%%%%%%%%%%%%%%%%%%%%%%%%%%%%%%%%%%%%%%%%%
%%%%%%%%%%%%%%%%%%%%%%%%%%%%%%%%%%%%%%%%%%%%%%%%%%%%%%%%%%%%%%%%%%%%%%%%%%%%%%%%
%%%%%%%%%%%%%%%%%%%%%%%%%%%%%%%%%%%%%%%%%%%%%%%%%%%%%%%%%%%%%%%%%%%%%%%%%%%%%%%%
%%%%%%%%%%%%%%%%%%%%%%%%%%%%%%%%%%%%%%%%%%%%%%%%%%%%%%%%%%%%%%%%%%%%%%%%%%%%%%%%
\begin{mdframed}[style=darkQuesion]
  10 . F Construct a group of order 12 that is not abelian.
\end{mdframed}
%%%%%%%%%%%%%%%%%%%%%%%%%%%%%%%%%%%%%%%%%%%%%%%%%%%%%%%%%%%%%%%%%%%%%%%%%%%%%%%%
\begin{mdframed}[style=darkAnswer,frametitle={Joe Starr}]
   %TODO 3.3 Q10 not done yet
\end{mdframed}
\newpage
%%%%%%%%%%%%%%%%%%%%%%%%%%%%%%%%%%%%%%%%%%%%%%%%%%%%%%%%%%%%%%%%%%%%%%%%%%%%%%%%
%%%%%%%%%%%%%%%%%%%%%%%%%%%%%%%%%%%%%%%%%%%%%%%%%%%%%%%%%%%%%%%%%%%%%%%%%%%%%%%%
%%%%%%%%%%%%%%%%%%%%%%%%%%%%%%%%%%%%%%%%%%%%%%%%%%%%%%%%%%%%%%%%%%%%%%%%%%%%%%%%
%%%%%%%%%%%%%%%%%%%%%%%%%%%%%%%%%%%%%%%%%%%%%%%%%%%%%%%%%%%%%%%%%%%%%%%%%%%%%%%%
\begin{mdframed}[style=darkQuesion]
  13. Let $n>2$ be an integer, and let $X \subseteq S_{n} \times S_{n}$ be the set $X=\{(\sigma, \tau) | \sigma(1)=\tau(1)\}$ Show that $X$ is not a subgroup of $S_{n} \times S_{n}$
\end{mdframed}
%%%%%%%%%%%%%%%%%%%%%%%%%%%%%%%%%%%%%%%%%%%%%%%%%%%%%%%%%%%%%%%%%%%%%%%%%%%%%%%%
\begin{mdframed}[style=darkAnswer,frametitle={Joe Starr}]
   %TODO 3.3 Q13 not done yet
\end{mdframed}
\newpage
%%%%%%%%%%%%%%%%%%%%%%%%%%%%%%%%%%%%%%%%%%%%%%%%%%%%%%%%%%%%%%%%%%%%%%%%%%%%%%%%
%%%%%%%%%%%%%%%%%%%%%%%%%%%%%%%%%%%%%%%%%%%%%%%%%%%%%%%%%%%%%%%%%%%%%%%%%%%%%%%%
%%%%%%%%%%%%%%%%%%%%%%%%%%%%%%%%%%%%%%%%%%%%%%%%%%%%%%%%%%%%%%%%%%%%%%%%%%%%%%%%
%%%%%%%%%%%%%%%%%%%%%%%%%%%%%%%%%%%%%%%%%%%%%%%%%%%%%%%%%%%%%%%%%%%%%%%%%%%%%%%%
\begin{mdframed}[style=darkQuesion]
  17. Let $G$ be a finite group, and let $H, K$ be subgroups of $G .$ Prove that
$$
|H K|=\frac{|H||K|}{|H \cap K|}
$$ 
\end{mdframed}
%%%%%%%%%%%%%%%%%%%%%%%%%%%%%%%%%%%%%%%%%%%%%%%%%%%%%%%%%%%%%%%%%%%%%%%%%%%%%%%%
\begin{mdframed}[style=darkAnswer,frametitle={Joe Starr}]
   %TODO 3.3 Q17 not done yet
\end{mdframed}
\newpage
%%%%%%%%%%%%%%%%%%%%%%%%%%%%%%%%%%%%%%%%%%%%%%%%%%%%%%%%%%%%%%%%%%%%%%%%%%%%%%%%
%%%%%%%%%%%%%%%%%%%%%%%%%%%%%%%%%%%%%%%%%%%%%%%%%%%%%%%%%%%%%%%%%%%%%%%%%%%%%%%%
%%%%%%%%%%%%%%%%%%%%%%%%%%%%%%%%%%%%%%%%%%%%%%%%%%%%%%%%%%%%%%%%%%%%%%%%%%%%%%%%
%%%%%%%%%%%%%%%%%%%%%%%%%%%%%%%%%%%%%%%%%%%%%%%%%%%%%%%%%%%%%%%%%%%%%%%%%%%%%%%%
\begin{mdframed}[style=darkQuesion]
20. thet $G$ be a group of order $6,$ and suppose that $a, b \in G$ with $a$ of order 3 and $b$ of order $2 .$ Show that either $G$ is cyclic or $a b \neq b a$
\end{mdframed}
%%%%%%%%%%%%%%%%%%%%%%%%%%%%%%%%%%%%%%%%%%%%%%%%%%%%%%%%%%%%%%%%%%%%%%%%%%%%%%%%
\begin{mdframed}[style=darkAnswer,frametitle={Joe Starr}]
   %TODO 3.3 Q20 not done yet
\end{mdframed}

\clearpage
\subsection{Isomorphisms}
%%%%%%%%%%%%%%%%%%%%%%%%%%%%%%%%%%%%%%%%%%%%%%%%%%%%%%%%%%%%%%%%%%%%%%%%%%%%%%%%
%%%%%%%%%%%%%%%%%%%%%%%%%%%%%%%%%%%%%%%%%%%%%%%%%%%%%%%%%%%%%%%%%%%%%%%%%%%%%%%%
%%%%%%%%%%%%%%%%%%%%%%%%%%%%%%%%%%%%%%%%%%%%%%%%%%%%%%%%%%%%%%%%%%%%%%%%%%%%%%%%
%%%%%%%%%%%%%%%%%%%%%%%%%%%%%%%%%%%%%%%%%%%%%%%%%%%%%%%%%%%%%%%%%%%%%%%%%%%%%%%%
\begin{mdframed}[style=darkQuesion]
1. Show that the multiplicative group $\mathbf{Z}_{10}^{\times}$ is isomorphic to the additive group $\mathbf{Z}_{4}$. Hint: Find a generator $[a]_{10}$ of $\mathbf{Z}_{10}^{\mathrm{x}}$ and define $\phi: \mathbf{Z}_{4} \rightarrow \mathbf{Z}_{10}^{\times}$ by $\phi\left([n]_{4}\right)=[a]_{10}^{n}$
\end{mdframed}
%%%%%%%%%%%%%%%%%%%%%%%%%%%%%%%%%%%%%%%%%%%%%%%%%%%%%%%%%%%%%%%%%%%%%%%%%%%%%%%%
\begin{mdframed}[style=darkAnswer,frametitle={Joe Starr}]
We begin by considering $\mathbf{\Z}_{10}^{\times}=\lrs{1, 3, 7, 9}$, observe
$3=3,\ 3^2=9,\ 3^3=7,\ 3^4=1$ showing $\mathbf{Z}_{10}^{\times}=\lra{3}$. We
can then let $\varphi:\Z_4\to \Z_{10}^\times$ be $\pof{n}=3^n$ we can construct
$\pofinv{3^n}=n$ and observe $\pof{\pofinv{3^n}}=\pof{n}=3^n$ showing
$\varphi$ a bijection. We will now show $\varphi$ is a homomorphism.
Let $n,k\in \Z_4$, and $\pof{n+k}=3^{n+k}=3^{n}3^{n}=\pof{n}\pof{k}$, showing
$\varphi$ an isomorphism.
\end{mdframed}
\newpage
%%%%%%%%%%%%%%%%%%%%%%%%%%%%%%%%%%%%%%%%%%%%%%%%%%%%%%%%%%%%%%%%%%%%%%%%%%%%%%%%
%%%%%%%%%%%%%%%%%%%%%%%%%%%%%%%%%%%%%%%%%%%%%%%%%%%%%%%%%%%%%%%%%%%%%%%%%%%%%%%%
%%%%%%%%%%%%%%%%%%%%%%%%%%%%%%%%%%%%%%%%%%%%%%%%%%%%%%%%%%%%%%%%%%%%%%%%%%%%%%%%
%%%%%%%%%%%%%%%%%%%%%%%%%%%%%%%%%%%%%%%%%%%%%%%%%%%%%%%%%%%%%%%%%%%%%%%%%%%%%%%%
\begin{mdframed}[style=darkQuesion]
2. Show that the multiplicative group $\mathbf{Z}_{7}^{\times}$ is isomorphic to the additive group $\mathbf{Z}_{6}$.
\end{mdframed}
%%%%%%%%%%%%%%%%%%%%%%%%%%%%%%%%%%%%%%%%%%%%%%%%%%%%%%%%%%%%%%%%%%%%%%%%%%%%%%%%
\begin{mdframed}[style=darkAnswer,frametitle={Joe Starr}]
We begin by considering $\mathbf{\Z}_{7}^{\times}=\lrs{1,2, 3,4,5,6}$, observe
$3=3,\ 3^2=2,\ 3^3=6,\ 3^4=4,\ 3^5=5,\ 3^6=1$ showing $\mathbf{\Z}_{7}^{\times}=\lra{3}$. We
can then let $\varphi:\Z_6\to \Z_{7}^\times$ be $\pof{n}=3^n$ we can construct
$\pofinv{3^n}=n$ and observe $\pof{\pofinv{3^n}}=\pof{n}=3^n$ showing
$\varphi$ a bijection. We will now show $\varphi$ is a homomorphism.
Let $n,k\in \Z_6$, and $\pof{n+k}=3^{n+k}=3^{n}3^{n}=\pof{n}\pof{k}$, showing
$\varphi$ an isomorphism.
\end{mdframed}
\newpage
%%%%%%%%%%%%%%%%%%%%%%%%%%%%%%%%%%%%%%%%%%%%%%%%%%%%%%%%%%%%%%%%%%%%%%%%%%%%%%%%
%%%%%%%%%%%%%%%%%%%%%%%%%%%%%%%%%%%%%%%%%%%%%%%%%%%%%%%%%%%%%%%%%%%%%%%%%%%%%%%%
%%%%%%%%%%%%%%%%%%%%%%%%%%%%%%%%%%%%%%%%%%%%%%%%%%%%%%%%%%%%%%%%%%%%%%%%%%%%%%%%
%%%%%%%%%%%%%%%%%%%%%%%%%%%%%%%%%%%%%%%%%%%%%%%%%%%%%%%%%%%%%%%%%%%%%%%%%%%%%%%%
\begin{mdframed}[style=darkQuesion]
3. Show that the multiplicative group $\mathbf{Z}_{8}^{\times}$ is isomorphic to the group $\mathbf{Z}_{2} \times \mathbf{Z}_{2}$
\end{mdframed}
%%%%%%%%%%%%%%%%%%%%%%%%%%%%%%%%%%%%%%%%%%%%%%%%%%%%%%%%%%%%%%%%%%%%%%%%%%%%%%%%
\begin{mdframed}[style=darkAnswer,frametitle={Joe Starr}]
We let $\varphi: \mathbf{Z}_{8}^{\times} \to \mathbf{Z}_{2} \times \mathbf{Z}_{2}$
 by  
 \begin{equation}
  \pof{a}=
  \begin{cases}
      \lrp{0,0}  & \text{ For } a=1\\
      \lrp{0,1}  & \text{ For } a=3\\
      \lrp{1,0}  & \text{ For } a=5\\
      \lrp{1,1}  & \text{ For } a=7
  \end{cases}
\end{equation}
observe the tables for the two groups: 
$$\vbox{\tabskip0.5em\offinterlineskip
      \halign{\strut$#$\hfil\ \tabskip1em\vrule&&$#$\hfil\cr
          \cdot      & \lrp{0,0}  & \lrp{0,1}   & \lrp{1,0}   & \lrp{1,1}  \cr
          \noalign{\hrule}\vrule height 12pt width 0pt
          \lrp{0,0}  & \lrp{0,0}  & \lrp{0,1}   & \lrp{1,0}   & \lrp{1,1}    \cr
          \lrp{0,1}  & \lrp{0,1}  & \lrp{0,0}   & \lrp{1,1}   & \lrp{1,0}    \cr
          \lrp{1,0}  & \lrp{1,0}  & \lrp{1,1}   & \lrp{0,0}   & \lrp{0,1}    \cr
          \lrp{1,1}  & \lrp{1,1}  & \lrp{1,0}   & \lrp{0,1}   & \lrp{0,0}    \cr
        }}$$
        $$\vbox{\tabskip0.5em\offinterlineskip
      \halign{\strut$#$\hfil\ \tabskip1em\vrule&&$#$\hfil\cr
          \cdot      & 1  & 3   & 5   & 7  \cr
          \noalign{\hrule}\vrule height 12pt width 0pt
          1  & 1  & 3   & 5   & 7    \cr
          3  & 3  & 1   & 7   & 5    \cr
          5  & 5  & 7   & 1   & 3    \cr
          7  & 7  & 5   & 3   & 1    \cr
        }}$$
\end{mdframed}
\newpage
%%%%%%%%%%%%%%%%%%%%%%%%%%%%%%%%%%%%%%%%%%%%%%%%%%%%%%%%%%%%%%%%%%%%%%%%%%%%%%%%
%%%%%%%%%%%%%%%%%%%%%%%%%%%%%%%%%%%%%%%%%%%%%%%%%%%%%%%%%%%%%%%%%%%%%%%%%%%%%%%%
%%%%%%%%%%%%%%%%%%%%%%%%%%%%%%%%%%%%%%%%%%%%%%%%%%%%%%%%%%%%%%%%%%%%%%%%%%%%%%%%
%%%%%%%%%%%%%%%%%%%%%%%%%%%%%%%%%%%%%%%%%%%%%%%%%%%%%%%%%%%%%%%%%%%%%%%%%%%%%%%%
\begin{mdframed}[style=darkQuesion]
4. Show that $\mathbf{Z}_{5}^{\times}$ is not isomorphic to $\mathbf{Z}_{8}^{\times}$ by showing that the first group has an element of order 4 but the second group does not.
\end{mdframed}
%%%%%%%%%%%%%%%%%%%%%%%%%%%%%%%%%%%%%%%%%%%%%%%%%%%%%%%%%%%%%%%%%%%%%%%%%%%%%%%%
\begin{mdframed}[style=darkAnswer,frametitle={Joe Starr}]
  We first consider $2^{1}=2,\ 2^{2}=4,\ 2^{3}=3,\ 2^{4}=1$ showing $\lra{2}$ is of
  order $4$. We can then observe the order of the elements of 
  $\mathbf{Z}_{8}^{\times}$, $3^2=1,\ 5^2=1,\ 7^2=1$ since none of the elements 
  of $\mathbf{Z}_{8}^{\times}$ have order $4$ it cant me that 
  $\mathbf{Z}_{8}^{\times}$ and $\mathbf{Z}_{5}^{\times}$ are isomorphic. 

\end{mdframed}
\newpage
%%%%%%%%%%%%%%%%%%%%%%%%%%%%%%%%%%%%%%%%%%%%%%%%%%%%%%%%%%%%%%%%%%%%%%%%%%%%%%%%
%%%%%%%%%%%%%%%%%%%%%%%%%%%%%%%%%%%%%%%%%%%%%%%%%%%%%%%%%%%%%%%%%%%%%%%%%%%%%%%%
%%%%%%%%%%%%%%%%%%%%%%%%%%%%%%%%%%%%%%%%%%%%%%%%%%%%%%%%%%%%%%%%%%%%%%%%%%%%%%%%
%%%%%%%%%%%%%%%%%%%%%%%%%%%%%%%%%%%%%%%%%%%%%%%%%%%%%%%%%%%%%%%%%%%%%%%%%%%%%%%%
\begin{mdframed}[style=darkQuesion]
6. Is the additive group $\mathbf{C}$ of complex numbers isomorphic to the multiplicative group $\mathbf{C}^{\times}$ of nonzero complex numbers?
\end{mdframed}
%%%%%%%%%%%%%%%%%%%%%%%%%%%%%%%%%%%%%%%%%%%%%%%%%%%%%%%%%%%%%%%%%%%%%%%%%%%%%%%%
\begin{mdframed}[style=darkAnswer,frametitle={Joe Starr}]
%TODO not done yet
\end{mdframed}
\newpage
%%%%%%%%%%%%%%%%%%%%%%%%%%%%%%%%%%%%%%%%%%%%%%%%%%%%%%%%%%%%%%%%%%%%%%%%%%%%%%%%
%%%%%%%%%%%%%%%%%%%%%%%%%%%%%%%%%%%%%%%%%%%%%%%%%%%%%%%%%%%%%%%%%%%%%%%%%%%%%%%%
%%%%%%%%%%%%%%%%%%%%%%%%%%%%%%%%%%%%%%%%%%%%%%%%%%%%%%%%%%%%%%%%%%%%%%%%%%%%%%%%
%%%%%%%%%%%%%%%%%%%%%%%%%%%%%%%%%%%%%%%%%%%%%%%%%%%%%%%%%%%%%%%%%%%%%%%%%%%%%%%%
\begin{mdframed}[style=darkQuesion]
7. Let $G_{1}$ and $G_{2}$ be groups. Show that $G_{2} \times G_{1}$ is isomorphic to $G_{1} \times G_{2}$
\end{mdframed}
%%%%%%%%%%%%%%%%%%%%%%%%%%%%%%%%%%%%%%%%%%%%%%%%%%%%%%%%%%%%%%%%%%%%%%%%%%%%%%%%
\begin{mdframed}[style=darkAnswer,frametitle={Joe Starr}]
Let $\varphi:G_{2} \times G_{1} \to G_{1} \times G_{2}$ be 
$\pof{\lrp{a,b}}=\lrp{b,a}$ and $\pofinv{\lrp{b,a}}=\lrp{a,b}$. We first 
establish $\varphi$ as a bijection, 
$\pof{\pofinv{(b,a)}}=\pof{\lrp{a,b}}=\lrp{b,a}$. Now we will establish $\varphi$
as a homomorphism, let $\lrp{a,b},\lrp{x,y}\in G_{2} \times G_{1}$
consider 
\begin{align*}
  \pof{\lrp{a,b}\lrp{x,y}}&=\pof{\lrp{ax,by}}\\
  &=\lrp{by,ax}\\
  &=\lrp{b,a}\lrp{y,x}\\
  &=\pof{\lrp{a,b}}\pof{\lrp{x,y}}\\
\end{align*}
showing the groups to be isomorphic. 
\end{mdframed}
\newpage
%%%%%%%%%%%%%%%%%%%%%%%%%%%%%%%%%%%%%%%%%%%%%%%%%%%%%%%%%%%%%%%%%%%%%%%%%%%%%%%%
%%%%%%%%%%%%%%%%%%%%%%%%%%%%%%%%%%%%%%%%%%%%%%%%%%%%%%%%%%%%%%%%%%%%%%%%%%%%%%%%
%%%%%%%%%%%%%%%%%%%%%%%%%%%%%%%%%%%%%%%%%%%%%%%%%%%%%%%%%%%%%%%%%%%%%%%%%%%%%%%%
%%%%%%%%%%%%%%%%%%%%%%%%%%%%%%%%%%%%%%%%%%%%%%%%%%%%%%%%%%%%%%%%%%%%%%%%%%%%%%%%
\begin{mdframed}[style=darkQuesion]
8. Let $G$ be a group. Show that the group $(G, *)$ defined in Exercise 3 of Section 3.1 is isomorphic to $G .$
\end{mdframed}
%%%%%%%%%%%%%%%%%%%%%%%%%%%%%%%%%%%%%%%%%%%%%%%%%%%%%%%%%%%%%%%%%%%%%%%%%%%%%%%%
\begin{mdframed}[style=darkAnswer,frametitle={Joe Starr}]
  Let $\grp{G}{\cdot}$ be a group. Define a new binary operation $\ast$ on
  $G$ by the formula $a \ast b=b\cdot a$, for all $a,b\in G$. Define 
  $\varphi:(G, \cdot)\to(G, *) $, with $\pof{a}=a$ the trivial map is a bijection. 
  Next we will show $\varphi$ to be a homomorphism let $a,b\in G$, observe 
  $\pof{a\cdot b}= a\cdot b=b\ast a=\pof{b}\ast\pof{a}$.
\end{mdframed}
\newpage
%%%%%%%%%%%%%%%%%%%%%%%%%%%%%%%%%%%%%%%%%%%%%%%%%%%%%%%%%%%%%%%%%%%%%%%%%%%%%%%%
%%%%%%%%%%%%%%%%%%%%%%%%%%%%%%%%%%%%%%%%%%%%%%%%%%%%%%%%%%%%%%%%%%%%%%%%%%%%%%%%
%%%%%%%%%%%%%%%%%%%%%%%%%%%%%%%%%%%%%%%%%%%%%%%%%%%%%%%%%%%%%%%%%%%%%%%%%%%%%%%%
%%%%%%%%%%%%%%%%%%%%%%%%%%%%%%%%%%%%%%%%%%%%%%%%%%%%%%%%%%%%%%%%%%%%%%%%%%%%%%%%
\begin{mdframed}[style=darkQuesion]
9. Prove that any group with three elements must be isomorphic to $\mathbf{Z}_{3}$.
\end{mdframed}
%%%%%%%%%%%%%%%%%%%%%%%%%%%%%%%%%%%%%%%%%%%%%%%%%%%%%%%%%%%%%%%%%%%%%%%%%%%%%%%%
\begin{mdframed}[style=darkAnswer,frametitle={Joe Starr}]
Let $G$ be a group of three elements. Since $G$ is a group there exists an 
identity element $e$ in $G$. This makes $a,b\in G$ with $a\neq b \neq e$ and WOLG
we are left with two options either $aa=e$ or $ab=e$. If $aa=e$ we get the table
$$\vbox{\tabskip0.5em\offinterlineskip
\halign{\strut$#$\hfil\ \tabskip1em\vrule&&$#$\hfil\cr
    \cdot      & e  & a   & b   \cr
    \noalign{\hrule}\vrule height 12pt width 0pt
    e  & e  & a   & b      \cr
    a  & a  & e   & b      \cr
    b  & b  & b   & e      \cr
  }}$$
observe $a\cdot\lrp{b\cdot b}=a$ but $\lrp{a\cdot b}\cdot b=e$ showing this is 
not a group under $\cdot$. 
Leaving us with one option given by the table below, which by comparison is 
isomorphic to $\Z_3$
$$\vbox{\tabskip0.5em\offinterlineskip
\halign{\strut$#$\hfil\ \tabskip1em\vrule&&$#$\hfil\cr
    \cdot      & e  & a   & b   \cr
    \noalign{\hrule}\vrule height 12pt width 0pt
    e  & e  & a   & b      \cr
    a  & a  & b   & e      \cr
    b  & b  & e   & a      \cr
  }}$$

$$\vbox{\tabskip0.5em\offinterlineskip
\halign{\strut$#$\hfil\ \tabskip1em\vrule&&$#$\hfil\cr
    \cdot      & 0  & 1   & 2   \cr
    \noalign{\hrule}\vrule height 12pt width 0pt
    0  & 0  & 1   & 2      \cr
    1  & 1  & 2   & 0      \cr
    2  & 2  & 0   & 1      \cr
  }}$$
\end{mdframed}
\newpage
%%%%%%%%%%%%%%%%%%%%%%%%%%%%%%%%%%%%%%%%%%%%%%%%%%%%%%%%%%%%%%%%%%%%%%%%%%%%%%%%
%%%%%%%%%%%%%%%%%%%%%%%%%%%%%%%%%%%%%%%%%%%%%%%%%%%%%%%%%%%%%%%%%%%%%%%%%%%%%%%%
%%%%%%%%%%%%%%%%%%%%%%%%%%%%%%%%%%%%%%%%%%%%%%%%%%%%%%%%%%%%%%%%%%%%%%%%%%%%%%%%
%%%%%%%%%%%%%%%%%%%%%%%%%%%%%%%%%%%%%%%%%%%%%%%%%%%%%%%%%%%%%%%%%%%%%%%%%%%%%%%%
\begin{mdframed}[style=darkQuesion]
13. Let $G$ be the set of all matrices in $\mathrm{GL}_{2}\left(\mathbf{Z}_{3}\right)$ of the form $\left[\begin{array}{ll}1 & 0 \\ c & d\end{array}\right] .$ That is, $c, d \in \mathbf{Z}_{3}$ and $d \neq[0]_{3} .$ Show that $G$ is isomorphic to $S_{3}$
\end{mdframed}
%%%%%%%%%%%%%%%%%%%%%%%%%%%%%%%%%%%%%%%%%%%%%%%%%%%%%%%%%%%%%%%%%%%%%%%%%%%%%%%%
\begin{mdframed}[style=darkAnswer,frametitle={Joe Starr}]
$$\begin{matrix}
  I=\begin{bmatrix}
    1 & 0 \\ 0 & 1 
  \end{bmatrix} & 
  A=\begin{bmatrix}
    1 & 0 \\ 1 & 1 
  \end{bmatrix} &
  B=\begin{bmatrix}
    1 & 0 \\ 1 & 2 
  \end{bmatrix} \\
  C=\begin{bmatrix}
    1 & 0 \\ 0 & 2 
  \end{bmatrix} & 
  D=\begin{bmatrix}
    1 & 0 \\ 2 & 1 
  \end{bmatrix} &
  E=\begin{bmatrix}
    1 & 0 \\ 2 & 2 
  \end{bmatrix} \\
\end{matrix}$$
\end{mdframed}
$$\vbox{\tabskip0.5em\offinterlineskip
      \halign{\strut$#$\hfil\ \tabskip1em\vrule&&$#$\hfil\cr
          \cdot        & \lrp{}       & \lrp{1,2}   & \lrp{2,3}   & \lrp{1,3}   & \lrp{1,2,3}  & \lrp{1,3,2}  \cr
          \noalign{\hrule}\vrule height 12pt width 0pt
          \lrp{}       & \lrp{}       & \lrp{1,2}   & \lrp{2,3}   & \lrp{1,3}   & \lrp{1,2,3}  & \lrp{1,3,2} \cr
          \lrp{1,2}    & \lrp{1,2}    & \lrp{}      & \lrp{1,2,3} & \lrp{1,3,2} & \lrp{2,3}    & \lrp{1,3} \cr
          \lrp{2,3}    & \lrp{2,3}    & \lrp{1,3,2} & \lrp{}      & \lrp{1,2,3} & \lrp{1,3}    & \lrp{1,2} \cr
          \lrp{1,3}    & \lrp{1,3}    & \lrp{1,2,3} & \lrp{1,3,2} & \lrp{}      & \lrp{1,2}    & \lrp{2,3} \cr
          \lrp{1,2,3}  & \lrp{1,2,3}  & \lrp{1,3}   & \lrp{1,2}   & \lrp{2,3}   & \lrp{1,3,2}  & \lrp{}    \cr
          \lrp{1,3,2}  & \lrp{1,3,2}  & \lrp{2,3}   & \lrp{1,3}   & \lrp{1,2}   & \lrp{}       & \lrp{1,2,3} \cr
        }}$$
$$\begin{matrix}
  I \times I = I &
  I \times A = A &
  I \times B = B &
  I \times C = C \\ \hspace{.5in} \\
  I \times D = D &
  I \times E = E &
  A \times I = A &
  A \times A = D \\ \hspace{.5in} \\
  A \times B = E &
  A \times C = B &
  A \times D = I &
  A \times E = C \\ \hspace{.5in} \\
  B \times I = B &
  B \times A = C &
  B \times B = I &
  B \times C = A \\ \hspace{.5in} \\
  B \times D = E &
  B \times E = D &
  C \times I = C &
  C \times A = E \\ \hspace{.5in} \\
  C \times B = D &
  C \times C = I &
  C \times D = B &
  C \times E = A \\ \hspace{.5in} \\
  D \times I = D &
  D \times A = I &
  D \times B = C &
  D \times C = E \\ \hspace{.5in} \\
  D \times D = A &
  D \times E = B &
  E \times I = E &
  E \times A = B \\ \hspace{.5in} \\
  E \times B = A &
  E \times C = D &
  E \times D = C &
  E \times E = I \\ \hspace{.5in} \\
\end{matrix}$$
%TODO finish this
\newpage
%%%%%%%%%%%%%%%%%%%%%%%%%%%%%%%%%%%%%%%%%%%%%%%%%%%%%%%%%%%%%%%%%%%%%%%%%%%%%%%%
%%%%%%%%%%%%%%%%%%%%%%%%%%%%%%%%%%%%%%%%%%%%%%%%%%%%%%%%%%%%%%%%%%%%%%%%%%%%%%%%
%%%%%%%%%%%%%%%%%%%%%%%%%%%%%%%%%%%%%%%%%%%%%%%%%%%%%%%%%%%%%%%%%%%%%%%%%%%%%%%%
%%%%%%%%%%%%%%%%%%%%%%%%%%%%%%%%%%%%%%%%%%%%%%%%%%%%%%%%%%%%%%%%%%%%%%%%%%%%%%%%
\begin{mdframed}[style=darkQuesion]
15. Let $C_{2}$ be the subgroup $\{\pm 1\}$ of the multiplicative group
$\mathbf{R}^{\times} .$ Show that $\mathbf{R}^{\times}$ is isomorphic to
$\mathbf{R}^{+} \times C_{2}$
\end{mdframed}
%%%%%%%%%%%%%%%%%%%%%%%%%%%%%%%%%%%%%%%%%%%%%%%%%%%%%%%%%%%%%%%%%%%%%%%%%%%%%%%%
\begin{mdframed}[style=darkAnswer,frametitle={Joe Starr}]
Let $\pof{\lrp{x,a}}=ae^x$, and $\pofinv{ae^x}=\lrp{x,a}$ observe 
$\pof{\pofinv{ae^x}}=\pof{\lrp{x,a}}=ae^x$.

Next consider $\pof{\lrp{x,a}\lrp{y,b}}= \lrp{x+y,ab}=abe^{x+y}
=ae^xbe^y=\pof{\lrp{x,a}}\pof{\lrp{y,b}}$ as desired. 

\end{mdframed}
\newpage
%%%%%%%%%%%%%%%%%%%%%%%%%%%%%%%%%%%%%%%%%%%%%%%%%%%%%%%%%%%%%%%%%%%%%%%%%%%%%%%%
%%%%%%%%%%%%%%%%%%%%%%%%%%%%%%%%%%%%%%%%%%%%%%%%%%%%%%%%%%%%%%%%%%%%%%%%%%%%%%%%
%%%%%%%%%%%%%%%%%%%%%%%%%%%%%%%%%%%%%%%%%%%%%%%%%%%%%%%%%%%%%%%%%%%%%%%%%%%%%%%%
%%%%%%%%%%%%%%%%%%%%%%%%%%%%%%%%%%%%%%%%%%%%%%%%%%%%%%%%%%%%%%%%%%%%%%%%%%%%%%%%
\begin{mdframed}[style=darkQuesion]
17. Let $G$ be any group, and let $a$ be a fixed element of $G .$ Define a function $\phi_{a}: G \rightarrow G$
by $\phi_{a}(x)=a x a^{-1},$ for all $x \in G .$ Show that $\phi_{a}$ is an isomorphism.
\end{mdframed}
%%%%%%%%%%%%%%%%%%%%%%%%%%%%%%%%%%%%%%%%%%%%%%%%%%%%%%%%%%%%%%%%%%%%%%%%%%%%%%%%
\begin{mdframed}[style=darkAnswer,frametitle={Joe Starr}]
Define $\pofinv{x}=\inv{a}xa$, consider $\pofinv{\pof{x}}=\pofinv{a x a^{-1}}=\inv{a}a x \inv{a}a=x$.

Next consider $\pof{xy}= a xy \inv{a}=ax\inv{a}ay\inv{a}=\pof{x}\pof{y}$.
\end{mdframed}
\newpage
%%%%%%%%%%%%%%%%%%%%%%%%%%%%%%%%%%%%%%%%%%%%%%%%%%%%%%%%%%%%%%%%%%%%%%%%%%%%%%%%
%%%%%%%%%%%%%%%%%%%%%%%%%%%%%%%%%%%%%%%%%%%%%%%%%%%%%%%%%%%%%%%%%%%%%%%%%%%%%%%%
%%%%%%%%%%%%%%%%%%%%%%%%%%%%%%%%%%%%%%%%%%%%%%%%%%%%%%%%%%%%%%%%%%%%%%%%%%%%%%%%
%%%%%%%%%%%%%%%%%%%%%%%%%%%%%%%%%%%%%%%%%%%%%%%%%%%%%%%%%%%%%%%%%%%%%%%%%%%%%%%%
\begin{mdframed}[style=darkQuesion]
18. Let $G$ be any group. Define $\phi: G \rightarrow G$ by $\phi(x)=x^{-1},$ for all $x \in G$
(a) Prove that $\phi$ is one-to-one and onto.
(b) Prove that $\phi$ is an isomorphism if and only if $G$ is abelian.
\end{mdframed}
%%%%%%%%%%%%%%%%%%%%%%%%%%%%%%%%%%%%%%%%%%%%%%%%%%%%%%%%%%%%%%%%%%%%%%%%%%%%%%%%
\begin{mdframed}[style=darkAnswer,frametitle={Joe Starr}]
We've shown uniqueness of inverses, and since $G$ is a group $\varphi$ is a 
bijection. 
$G$ is abelian: \\
$\pof{ab}=\inv{ab}=\inv{a}\inv{b}=\pof{a}\pof{b}$
$G$ is non-abelian:\\
$\pof{ab}=\inv{ab}\neq\inv{a}\inv{b}$ so is not a homomorphism. 

\end{mdframed}
\newpage
%%%%%%%%%%%%%%%%%%%%%%%%%%%%%%%%%%%%%%%%%%%%%%%%%%%%%%%%%%%%%%%%%%%%%%%%%%%%%%%%
%%%%%%%%%%%%%%%%%%%%%%%%%%%%%%%%%%%%%%%%%%%%%%%%%%%%%%%%%%%%%%%%%%%%%%%%%%%%%%%%
%%%%%%%%%%%%%%%%%%%%%%%%%%%%%%%%%%%%%%%%%%%%%%%%%%%%%%%%%%%%%%%%%%%%%%%%%%%%%%%%
%%%%%%%%%%%%%%%%%%%%%%%%%%%%%%%%%%%%%%%%%%%%%%%%%%%%%%%%%%%%%%%%%%%%%%%%%%%%%%%%
\begin{mdframed}[style=darkQuesion]
22. Let $G_{1}$ and $G_{2}$ be groups. Show that $G_{1}$ is isomorphic to the subgroup of the direct product $G_{1} \times G_{2}$ defined by $\left\{\left(x_{1}, x_{2}\right) | x_{2}=e\right\}$
\end{mdframed}
%%%%%%%%%%%%%%%%%%%%%%%%%%%%%%%%%%%%%%%%%%%%%%%%%%%%%%%%%%%%%%%%%%%%%%%%%%%%%%%%
\begin{mdframed}[style=darkAnswer,frametitle={Joe Starr}]
  Define $\pof{a}=\lrp{a,e}$, and $\pofinv{\lrp{a,e}}=a$, consider 
  $\pofinv{\pof{a}}=\pofinv{\lrp{a,e}}=a$. 
  
  Now consider $\pof{ab}=\lrp{ab,e}=\lrp{a,e}\lrp{b,e}=\pof{a}\pof{b}$
\end{mdframed}
\newpage
%%%%%%%%%%%%%%%%%%%%%%%%%%%%%%%%%%%%%%%%%%%%%%%%%%%%%%%%%%%%%%%%%%%%%%%%%%%%%%%%
%%%%%%%%%%%%%%%%%%%%%%%%%%%%%%%%%%%%%%%%%%%%%%%%%%%%%%%%%%%%%%%%%%%%%%%%%%%%%%%%
%%%%%%%%%%%%%%%%%%%%%%%%%%%%%%%%%%%%%%%%%%%%%%%%%%%%%%%%%%%%%%%%%%%%%%%%%%%%%%%%
%%%%%%%%%%%%%%%%%%%%%%%%%%%%%%%%%%%%%%%%%%%%%%%%%%%%%%%%%%%%%%%%%%%%%%%%%%%%%%%%
\begin{mdframed}[style=darkQuesion]
23. Prove that if $m, n$ are positive integers such that $\operatorname{gcd}(m, n)=1,$ then $\mathbf{Z}_{m n}^{\times}$ is isomorphic to $\mathbf{Z}_{m}^{\times} \times \mathbf{Z}_{n}^{\times}$
\end{mdframed}
%%%%%%%%%%%%%%%%%%%%%%%%%%%%%%%%%%%%%%%%%%%%%%%%%%%%%%%%%%%%%%%%%%%%%%%%%%%%%%%%
\begin{mdframed}[style=darkAnswer,frametitle={Joe Starr}]
%TODO not done yet
\end{mdframed}
\newpage
%%%%%%%%%%%%%%%%%%%%%%%%%%%%%%%%%%%%%%%%%%%%%%%%%%%%%%%%%%%%%%%%%%%%%%%%%%%%%%%%
%%%%%%%%%%%%%%%%%%%%%%%%%%%%%%%%%%%%%%%%%%%%%%%%%%%%%%%%%%%%%%%%%%%%%%%%%%%%%%%%
%%%%%%%%%%%%%%%%%%%%%%%%%%%%%%%%%%%%%%%%%%%%%%%%%%%%%%%%%%%%%%%%%%%%%%%%%%%%%%%%
%%%%%%%%%%%%%%%%%%%%%%%%%%%%%%%%%%%%%%%%%%%%%%%%%%%%%%%%%%%%%%%%%%%%%%%%%%%%%%%%
\begin{mdframed}[style=darkQuesion]
30. Let $G_{1}$ and $G_{2}$ be groups. A function from $G_{1}$ into $G_{2}$ that preserves products but
is not necessarily a one-to-one correspondence will be called a group homomorphism, from the Greek word homos meaning same. Show that $\phi: \mathrm{GL}_{2}(\mathbf{R}) \rightarrow \mathbf{R}^{\times}$ defined by $\phi(A)=\operatorname{det}(A)$ for all matrices $A \in \mathrm{GL}_{2}(\mathbf{R})$ is a group homomorphism.
\end{mdframed}
%%%%%%%%%%%%%%%%%%%%%%%%%%%%%%%%%%%%%%%%%%%%%%%%%%%%%%%%%%%%%%%%%%%%%%%%%%%%%%%%
\begin{mdframed}[style=darkAnswer,frametitle={Joe Starr}]
%TODO not done yet
\end{mdframed}
\newpage

\clearpage
\subsection{Constructing Examples}
%%%%%%%%%%%%%%%%%%%%%%%%%%%%%%%%%%%%%%%%%%%%%%%%%%%%%%%%%%%%%%%%%%%%%%%%%%%%%%%%
%%%%%%%%%%%%%%%%%%%%%%%%%%%%%%%%%%%%%%%%%%%%%%%%%%%%%%%%%%%%%%%%%%%%%%%%%%%%%%%%
%%%%%%%%%%%%%%%%%%%%%%%%%%%%%%%%%%%%%%%%%%%%%%%%%%%%%%%%%%%%%%%%%%%%%%%%%%%%%%%%
%%%%%%%%%%%%%%%%%%%%%%%%%%%%%%%%%%%%%%%%%%%%%%%%%%%%%%%%%%%%%%%%%%%%%%%%%%%%%%%%
\begin{mdframed}[style=darkQuesion]
  1. FLet $G$ be a group and let $a \in G$ be an element of order $12 .$ What is the order of $a^{j}$ for $j=2, \ldots, 11 ?$
\end{mdframed}
%%%%%%%%%%%%%%%%%%%%%%%%%%%%%%%%%%%%%%%%%%%%%%%%%%%%%%%%%%%%%%%%%%%%%%%%%%%%%%%%
\begin{mdframed}[style=darkAnswer,frametitle={Joe Starr}]
  
\end{mdframed}
\newpage
%%%%%%%%%%%%%%%%%%%%%%%%%%%%%%%%%%%%%%%%%%%%%%%%%%%%%%%%%%%%%%%%%%%%%%%%%%%%%%%%
%%%%%%%%%%%%%%%%%%%%%%%%%%%%%%%%%%%%%%%%%%%%%%%%%%%%%%%%%%%%%%%%%%%%%%%%%%%%%%%%
%%%%%%%%%%%%%%%%%%%%%%%%%%%%%%%%%%%%%%%%%%%%%%%%%%%%%%%%%%%%%%%%%%%%%%%%%%%%%%%%
%%%%%%%%%%%%%%%%%%%%%%%%%%%%%%%%%%%%%%%%%%%%%%%%%%%%%%%%%%%%%%%%%%%%%%%%%%%%%%%%
\begin{mdframed}[style=darkQuesion]
2. Let $G$ be a group and let $a \in G$ be an element of order $30 .$ List the powers of $a$ that have order $2,$ order 3 or order 5
\end{mdframed}
%%%%%%%%%%%%%%%%%%%%%%%%%%%%%%%%%%%%%%%%%%%%%%%%%%%%%%%%%%%%%%%%%%%%%%%%%%%%%%%%
\begin{mdframed}[style=darkAnswer,frametitle={Joe Starr}]
  
\end{mdframed}
\newpage
%%%%%%%%%%%%%%%%%%%%%%%%%%%%%%%%%%%%%%%%%%%%%%%%%%%%%%%%%%%%%%%%%%%%%%%%%%%%%%%%
%%%%%%%%%%%%%%%%%%%%%%%%%%%%%%%%%%%%%%%%%%%%%%%%%%%%%%%%%%%%%%%%%%%%%%%%%%%%%%%%
%%%%%%%%%%%%%%%%%%%%%%%%%%%%%%%%%%%%%%%%%%%%%%%%%%%%%%%%%%%%%%%%%%%%%%%%%%%%%%%%
%%%%%%%%%%%%%%%%%%%%%%%%%%%%%%%%%%%%%%%%%%%%%%%%%%%%%%%%%%%%%%%%%%%%%%%%%%%%%%%%
\begin{mdframed}[style=darkQuesion]
3. Give the subgroup diagrams of the following groups.
(a) $\mathbf{Z}_{24}$
(b) $\mathbf{Z}_{36}$
\end{mdframed}
%%%%%%%%%%%%%%%%%%%%%%%%%%%%%%%%%%%%%%%%%%%%%%%%%%%%%%%%%%%%%%%%%%%%%%%%%%%%%%%%
\begin{mdframed}[style=darkAnswer,frametitle={Joe Starr}]
  
\end{mdframed}
\newpage%%%%%%%%%%%%%%%%%%%%%%%%%%%%%%%%%%%%%%%%%%%%%%%%%%%%%%%%%%%%%%%%%%%%%%%%%%%%%%%%
%%%%%%%%%%%%%%%%%%%%%%%%%%%%%%%%%%%%%%%%%%%%%%%%%%%%%%%%%%%%%%%%%%%%%%%%%%%%%%%%
%%%%%%%%%%%%%%%%%%%%%%%%%%%%%%%%%%%%%%%%%%%%%%%%%%%%%%%%%%%%%%%%%%%%%%%%%%%%%%%%
%%%%%%%%%%%%%%%%%%%%%%%%%%%%%%%%%%%%%%%%%%%%%%%%%%%%%%%%%%%%%%%%%%%%%%%%%%%%%%%%
\begin{mdframed}[style=darkQuesion]
4. Give the subgroup diagram of $\mathbf{Z}_{60}$.
\end{mdframed}
%%%%%%%%%%%%%%%%%%%%%%%%%%%%%%%%%%%%%%%%%%%%%%%%%%%%%%%%%%%%%%%%%%%%%%%%%%%%%%%%
\begin{mdframed}[style=darkAnswer,frametitle={Joe Starr}]
  
\end{mdframed}
\newpage
%%%%%%%%%%%%%%%%%%%%%%%%%%%%%%%%%%%%%%%%%%%%%%%%%%%%%%%%%%%%%%%%%%%%%%%%%%%%%%%%
%%%%%%%%%%%%%%%%%%%%%%%%%%%%%%%%%%%%%%%%%%%%%%%%%%%%%%%%%%%%%%%%%%%%%%%%%%%%%%%%
%%%%%%%%%%%%%%%%%%%%%%%%%%%%%%%%%%%%%%%%%%%%%%%%%%%%%%%%%%%%%%%%%%%%%%%%%%%%%%%%
%%%%%%%%%%%%%%%%%%%%%%%%%%%%%%%%%%%%%%%%%%%%%%%%%%%%%%%%%%%%%%%%%%%%%%%%%%%%%%%%
\begin{mdframed}[style=darkQuesion]
6. Find the order of the cyclic subgroup of $\mathbf{C}^{\times}$ generated by $1+i$
\end{mdframed}
%%%%%%%%%%%%%%%%%%%%%%%%%%%%%%%%%%%%%%%%%%%%%%%%%%%%%%%%%%%%%%%%%%%%%%%%%%%%%%%%
\begin{mdframed}[style=darkAnswer,frametitle={Joe Starr}]
  
\end{mdframed}
\newpage
%%%%%%%%%%%%%%%%%%%%%%%%%%%%%%%%%%%%%%%%%%%%%%%%%%%%%%%%%%%%%%%%%%%%%%%%%%%%%%%%
%%%%%%%%%%%%%%%%%%%%%%%%%%%%%%%%%%%%%%%%%%%%%%%%%%%%%%%%%%%%%%%%%%%%%%%%%%%%%%%%
%%%%%%%%%%%%%%%%%%%%%%%%%%%%%%%%%%%%%%%%%%%%%%%%%%%%%%%%%%%%%%%%%%%%%%%%%%%%%%%%
%%%%%%%%%%%%%%%%%%%%%%%%%%%%%%%%%%%%%%%%%%%%%%%%%%%%%%%%%%%%%%%%%%%%%%%%%%%%%%%%
\begin{mdframed}[style=darkQuesion]
7. $+$ Which of the multiplicative groups $\mathbf{Z}_{15}^{\times}, \mathbf{Z}_{18}^{\times}, \mathbf{Z}_{20}^{\times}, \mathbf{Z}_{27}^{\times}$ are cyclic?
\end{mdframed}
%%%%%%%%%%%%%%%%%%%%%%%%%%%%%%%%%%%%%%%%%%%%%%%%%%%%%%%%%%%%%%%%%%%%%%%%%%%%%%%%
\begin{mdframed}[style=darkAnswer,frametitle={Joe Starr}]
  
\end{mdframed}
\newpage
%%%%%%%%%%%%%%%%%%%%%%%%%%%%%%%%%%%%%%%%%%%%%%%%%%%%%%%%%%%%%%%%%%%%%%%%%%%%%%%%
%%%%%%%%%%%%%%%%%%%%%%%%%%%%%%%%%%%%%%%%%%%%%%%%%%%%%%%%%%%%%%%%%%%%%%%%%%%%%%%%
%%%%%%%%%%%%%%%%%%%%%%%%%%%%%%%%%%%%%%%%%%%%%%%%%%%%%%%%%%%%%%%%%%%%%%%%%%%%%%%%
%%%%%%%%%%%%%%%%%%%%%%%%%%%%%%%%%%%%%%%%%%%%%%%%%%%%%%%%%%%%%%%%%%%%%%%%%%%%%%%%
\begin{mdframed}[style=darkQuesion]
8. Find $\langle\pi\rangle$ in $\mathbf{R}^{\times}$
\end{mdframed}
%%%%%%%%%%%%%%%%%%%%%%%%%%%%%%%%%%%%%%%%%%%%%%%%%%%%%%%%%%%%%%%%%%%%%%%%%%%%%%%%
\begin{mdframed}[style=darkAnswer,frametitle={Joe Starr}]
  
\end{mdframed}
\newpage
%%%%%%%%%%%%%%%%%%%%%%%%%%%%%%%%%%%%%%%%%%%%%%%%%%%%%%%%%%%%%%%%%%%%%%%%%%%%%%%%
%%%%%%%%%%%%%%%%%%%%%%%%%%%%%%%%%%%%%%%%%%%%%%%%%%%%%%%%%%%%%%%%%%%%%%%%%%%%%%%%
%%%%%%%%%%%%%%%%%%%%%%%%%%%%%%%%%%%%%%%%%%%%%%%%%%%%%%%%%%%%%%%%%%%%%%%%%%%%%%%%
%%%%%%%%%%%%%%%%%%%%%%%%%%%%%%%%%%%%%%%%%%%%%%%%%%%%%%%%%%%%%%%%%%%%%%%%%%%%%%%%
\begin{mdframed}[style=darkQuesion]
9. $+$ Find all cyclic subgroups of $\mathbf{Z}_{4} \times \mathbf{Z}_{2}$
\end{mdframed}
%%%%%%%%%%%%%%%%%%%%%%%%%%%%%%%%%%%%%%%%%%%%%%%%%%%%%%%%%%%%%%%%%%%%%%%%%%%%%%%%
\begin{mdframed}[style=darkAnswer,frametitle={Joe Starr}]
  
\end{mdframed}
\newpage
%13. # Show that in a finite cyclic group of order $n$, the equation $x^{m}=e$ has exactly $m$ solutions, for each positive integer $m$ that is a divisor of $n$
%15. Prove that any finite cyclic group with more than two elements has an even number of distinct generators.
%17. thet $G$ be a finite group, and suppose that for any two subgroups $H$ and $K$ either $H \subseteq K$ or $K \subseteq H .$ Prove that $G$ is cyclic of prime power order.
%18. Let $G$ be the set of all 3 $\times 3$ matrices of the form $\left[\begin{array}{lll}1 & 0 & 0 \\ a & 1 & 0 \\ b & c & 1\end{array}\right]$
%(a) Show that if $a, b, c \in \mathbf{Z}_{3},$ then $G$ is a group with exponent 3
%(b) Show that if $a, b, c \in \mathbf{Z}_{2},$ then $G$ is a group with exponent 4.
%19. Prove that $\sum_{d | n} \varphi(d)=n$ for any positive integer $n$ Hint: Interpret the equation in the cyclic group $\mathbf{Z}_{n},$ by considering all of its sub-
%groups.
%20. Let $n=2^{k}$ for $k>2 .$ Prove that $\mathbf{Z}_{n}^{\times}$ is not cyclic. Hint: Show that \pm 1 and $(n / 2) \pm 1$ satisfy the equation $x^{2}=1,$ and that this is impossible in any cyclic group.
%21. # Prove that if $p$ and $q$ are different odd primes, then $\mathbf{Z}_{p q}^{\times}$ is not a cyclic group.
%22. Let $G$ be a group with $p^{k}$ elements, where $p$ is a prime number and $k \geq 1 .$ Prove that $G$ has a subgroup of order $p$
\clearpage
\subsection{Cyclic Groups}
%%%%%%%%%%%%%%%%%%%%%%%%%%%%%%%%%%%%%%%%%%%%%%%%%%%%%%%%%%%%%%%%%%%%%%%%%%%%%%%%
%%%%%%%%%%%%%%%%%%%%%%%%%%%%%%%%%%%%%%%%%%%%%%%%%%%%%%%%%%%%%%%%%%%%%%%%%%%%%%%%
%%%%%%%%%%%%%%%%%%%%%%%%%%%%%%%%%%%%%%%%%%%%%%%%%%%%%%%%%%%%%%%%%%%%%%%%%%%%%%%%
%%%%%%%%%%%%%%%%%%%%%%%%%%%%%%%%%%%%%%%%%%%%%%%%%%%%%%%%%%%%%%%%%%%%%%%%%%%%%%%%
\begin{mdframed}[style=darkQuesion]
  1. Let $G$ be a group and let $a \in G$ be an element of order $12 .$ 
  What is the order of $a^{j}$ for $j=2, \ldots, 11 ?$
\end{mdframed}
%%%%%%%%%%%%%%%%%%%%%%%%%%%%%%%%%%%%%%%%%%%%%%%%%%%%%%%%%%%%%%%%%%%%%%%%%%%%%%%%
\begin{mdframed}[style=darkAnswer,frametitle={Joe Starr}]
  By applying 3.5.3 and 3.5.4 we get
$\lrp{a^{1}}^{12},\lrp{a^{2}}^{6},\lrp{a^{3}}^{4},\lrp{a^{4}}^{3},\lrp{a^{5}}^{12},
\lrp{a^{6}}^{2},\lrp{a^{7}}^{12},\lrp{a^{8}}^{3},\\\lrp{a^{9}}^{4},\lrp{a^{10}}^{6},
\lrp{a^{11}}^{12}$ 
\end{mdframed}
\newpage
%%%%%%%%%%%%%%%%%%%%%%%%%%%%%%%%%%%%%%%%%%%%%%%%%%%%%%%%%%%%%%%%%%%%%%%%%%%%%%%%
%%%%%%%%%%%%%%%%%%%%%%%%%%%%%%%%%%%%%%%%%%%%%%%%%%%%%%%%%%%%%%%%%%%%%%%%%%%%%%%%
%%%%%%%%%%%%%%%%%%%%%%%%%%%%%%%%%%%%%%%%%%%%%%%%%%%%%%%%%%%%%%%%%%%%%%%%%%%%%%%%
%%%%%%%%%%%%%%%%%%%%%%%%%%%%%%%%%%%%%%%%%%%%%%%%%%%%%%%%%%%%%%%%%%%%%%%%%%%%%%%%
\begin{mdframed}[style=darkQuesion]
2. Let $G$ be a group and let $a \in G$ be an element of order $30 .$ 
List the powers of $a$ that have order $2,$ order 3 or order 5
\end{mdframed}
%%%%%%%%%%%%%%%%%%%%%%%%%%%%%%%%%%%%%%%%%%%%%%%%%%%%%%%%%%%%%%%%%%%%%%%%%%%%%%%%
\begin{mdframed}[style=darkAnswer,frametitle={Joe Starr}]
\begin{align*}
  2&: 15\\
  3&: 10, 20\\
  5&: 6,12,18,24\\
\end{align*}
\end{mdframed}
\newpage
%%%%%%%%%%%%%%%%%%%%%%%%%%%%%%%%%%%%%%%%%%%%%%%%%%%%%%%%%%%%%%%%%%%%%%%%%%%%%%%%
%%%%%%%%%%%%%%%%%%%%%%%%%%%%%%%%%%%%%%%%%%%%%%%%%%%%%%%%%%%%%%%%%%%%%%%%%%%%%%%%
%%%%%%%%%%%%%%%%%%%%%%%%%%%%%%%%%%%%%%%%%%%%%%%%%%%%%%%%%%%%%%%%%%%%%%%%%%%%%%%%
%%%%%%%%%%%%%%%%%%%%%%%%%%%%%%%%%%%%%%%%%%%%%%%%%%%%%%%%%%%%%%%%%%%%%%%%%%%%%%%%
\begin{mdframed}[style=darkQuesion]
5. Find the cyclic subgroup of $C^{\times}$ generated by $\frac{\sqrt{2}+\sqrt{2}i}{2}$.
\end{mdframed}
%%%%%%%%%%%%%%%%%%%%%%%%%%%%%%%%%%%%%%%%%%%%%%%%%%%%%%%%%%%%%%%%%%%%%%%%%%%%%%%%
\begin{mdframed}[style=darkAnswer,frametitle={Joe Starr}]
 \begin{align*}
   a^{1}&=\frac{\sqrt{2}+\sqrt{2}i}{2}\\
   a^{2}&=\frac{\sqrt{2}+\sqrt{2}i}{2}\frac{\sqrt{2}+\sqrt{2}i}{2}=i\\
   a^{4}&=i=-1\\
   a^{8}&=i=1\\
 \end{align*}
\end{mdframed}
\newpage
%%%%%%%%%%%%%%%%%%%%%%%%%%%%%%%%%%%%%%%%%%%%%%%%%%%%%%%%%%%%%%%%%%%%%%%%%%%%%%%%
%%%%%%%%%%%%%%%%%%%%%%%%%%%%%%%%%%%%%%%%%%%%%%%%%%%%%%%%%%%%%%%%%%%%%%%%%%%%%%%%
%%%%%%%%%%%%%%%%%%%%%%%%%%%%%%%%%%%%%%%%%%%%%%%%%%%%%%%%%%%%%%%%%%%%%%%%%%%%%%%%
%%%%%%%%%%%%%%%%%%%%%%%%%%%%%%%%%%%%%%%%%%%%%%%%%%%%%%%%%%%%%%%%%%%%%%%%%%%%%%%%
\begin{mdframed}[style=darkQuesion]
6. Find the order of the cyclic subgroup of $\mathbf{C}^{\times}$ generated by $1+i$
\end{mdframed}
%%%%%%%%%%%%%%%%%%%%%%%%%%%%%%%%%%%%%%%%%%%%%%%%%%%%%%%%%%%%%%%%%%%%%%%%%%%%%%%%
\begin{mdframed}[style=darkAnswer,frametitle={Joe Starr}]
 \begin{align*}
  a^{1}&=1+i\\
  a^{2}&=\lrp{1+i}\lrp{1+i}=2i\\
  a^{4}&=-4\\
  a^{8}&=16\\
  a^{2^{3+n}}&={2^{4+n}}\\
 \end{align*}
 we observe $2^{4+n}$ has infinite order so $a$ must have infinite order. 
\end{mdframed}
\newpage
%%%%%%%%%%%%%%%%%%%%%%%%%%%%%%%%%%%%%%%%%%%%%%%%%%%%%%%%%%%%%%%%%%%%%%%%%%%%%%%%
%%%%%%%%%%%%%%%%%%%%%%%%%%%%%%%%%%%%%%%%%%%%%%%%%%%%%%%%%%%%%%%%%%%%%%%%%%%%%%%%
%%%%%%%%%%%%%%%%%%%%%%%%%%%%%%%%%%%%%%%%%%%%%%%%%%%%%%%%%%%%%%%%%%%%%%%%%%%%%%%%
%%%%%%%%%%%%%%%%%%%%%%%%%%%%%%%%%%%%%%%%%%%%%%%%%%%%%%%%%%%%%%%%%%%%%%%%%%%%%%%%
\begin{mdframed}[style=darkQuesion]
7. Which of the multiplicative groups $\mathbf{Z}_{15}^{\times}, \mathbf{Z}_{18}^{\times}, \mathbf{Z}_{20}^{\times}, \mathbf{Z}_{27}^{\times}$ are cyclic?
\end{mdframed}
%%%%%%%%%%%%%%%%%%%%%%%%%%%%%%%%%%%%%%%%%%%%%%%%%%%%%%%%%%%%%%%%%%%%%%%%%%%%%%%%
\begin{mdframed}[style=darkAnswer,frametitle={Joe Starr}]
 \begin{align*}
  \mathbf{Z}_{15}^{\times}&:15=5\cdot3 \text{ The product of two odd primes, not cyclic}\\
  \mathbf{Z}_{18}^{\times}&:18=2\cdot3^2 \text{ The product of two and power of an odd prime, cyclic}\\
  \mathbf{Z}_{20}^{\times}&:20=2^2\cdot5 \text{ The product of four and an odd prime, not cyclic}\\
  \mathbf{Z}_{27}^{\times}&:15=3^3 \text{ The power of an odd prime, cyclic}\\
 \end{align*}
\end{mdframed}
\newpage
%%%%%%%%%%%%%%%%%%%%%%%%%%%%%%%%%%%%%%%%%%%%%%%%%%%%%%%%%%%%%%%%%%%%%%%%%%%%%%%%
%%%%%%%%%%%%%%%%%%%%%%%%%%%%%%%%%%%%%%%%%%%%%%%%%%%%%%%%%%%%%%%%%%%%%%%%%%%%%%%%
%%%%%%%%%%%%%%%%%%%%%%%%%%%%%%%%%%%%%%%%%%%%%%%%%%%%%%%%%%%%%%%%%%%%%%%%%%%%%%%%
%%%%%%%%%%%%%%%%%%%%%%%%%%%%%%%%%%%%%%%%%%%%%%%%%%%%%%%%%%%%%%%%%%%%%%%%%%%%%%%%
\begin{mdframed}[style=darkQuesion]
11. Which of the multiplicative groups  $\Z^{\times}_{7},\ \Z^{\times}_{10},\ \Z^{\times}_{12},\ \Z^{\times}_{14}$ are isomorphic.
\end{mdframed}
%%%%%%%%%%%%%%%%%%%%%%%%%%%%%%%%%%%%%%%%%%%%%%%%%%%%%%%%%%%%%%%%%%%%%%%%%%%%%%%%
\begin{mdframed}[style=darkAnswer,frametitle={Joe Starr}]
  \begin{align*}
    \mathbf{\Z}_{7 }^{\times}&:7 =7 \text{ The power of an odd prime, cyclic}\\
    \mathbf{\Z}_{10}^{\times}&:10=2\cdot5 \text{ The product of two and power of an odd prime, cyclic}\\
    \mathbf{\Z}_{12}^{\times}&:12=2^2\cdot3\cdot5 \text{ The product of four and an odd prime, not cyclic}\\
    \mathbf{\Z}_{14}^{\times}&:14=2\cdot7 \text{ The product of two and power of an odd prime, cyclic}\\
   \hline \\
    \mathbf{\Z}_{7 }^{\times}&=\lrs{1, 2, 3, 4, 5, 6} \\
    \mathbf{\Z}_{10}^{\times}&=\lrs{1, 3, 7, 9} \\
    \mathbf{\Z}_{14}^{\times}&=\lrs{1, 3, 5, 9, 11, 13} \\
   \end{align*}
   Since $\Z_{14}^{\times}$ has order six and and $\Zmx{7}$ has order six, 
   and both are cyclic they must both be isomorphic to $\Zm{6}$.

\end{mdframed}
\newpage
%%%%%%%%%%%%%%%%%%%%%%%%%%%%%%%%%%%%%%%%%%%%%%%%%%%%%%%%%%%%%%%%%%%%%%%%%%%%%%%%
%%%%%%%%%%%%%%%%%%%%%%%%%%%%%%%%%%%%%%%%%%%%%%%%%%%%%%%%%%%%%%%%%%%%%%%%%%%%%%%%
%%%%%%%%%%%%%%%%%%%%%%%%%%%%%%%%%%%%%%%%%%%%%%%%%%%%%%%%%%%%%%%%%%%%%%%%%%%%%%%%
%%%%%%%%%%%%%%%%%%%%%%%%%%%%%%%%%%%%%%%%%%%%%%%%%%%%%%%%%%%%%%%%%%%%%%%%%%%%%%%%
\begin{mdframed}[style=darkQuesion]
12. Let $a,b$ be positive integers, and let $d=\ngcd{a}{b}$ and $m=\nlcm{a}{b}$. 
Use proposition 3.5.5 to prove that $\Z_a\times \Z_b\cong \Z_d\times \Z_m$.
\end{mdframed}
%%%%%%%%%%%%%%%%%%%%%%%%%%%%%%%%%%%%%%%%%%%%%%%%%%%%%%%%%%%%%%%%%%%%%%%%%%%%%%%%
\begin{mdframed}[style=darkAnswer,frametitle={Joe Starr}]
 By $3.5.5$ we have that $\Zm{ab}\cong \Z_a\times \Z_b$, and 
 $\Zm{dm}\cong \Z_d\times \Z_m$, we've previously shown $\ngcd{a}{b}\cdot \nlcm{a}{b}=ab$
 making $\Zm{dm}=\Zm{ab}$ showing $\Z_a\times \Z_b\cong \Z_d\times \Z_m$.
\end{mdframed}
\newpage
%%%%%%%%%%%%%%%%%%%%%%%%%%%%%%%%%%%%%%%%%%%%%%%%%%%%%%%%%%%%%%%%%%%%%%%%%%%%%%%%
%%%%%%%%%%%%%%%%%%%%%%%%%%%%%%%%%%%%%%%%%%%%%%%%%%%%%%%%%%%%%%%%%%%%%%%%%%%%%%%%
%%%%%%%%%%%%%%%%%%%%%%%%%%%%%%%%%%%%%%%%%%%%%%%%%%%%%%%%%%%%%%%%%%%%%%%%%%%%%%%%
%%%%%%%%%%%%%%%%%%%%%%%%%%%%%%%%%%%%%%%%%%%%%%%%%%%%%%%%%%%%%%%%%%%%%%%%%%%%%%%%
\begin{mdframed}[style=darkQuesion]
18. Let $G$ be the set of all 3 $\times 3$ matrices of the form $\left[\begin{array}{lll}1 & 0 & 0 \\ a & 1 & 0 \\ b & c & 1\end{array}\right]$
  \begin{itemize}
    \item []{(a) Show that if $a, b, c \in \mathbf{Z}_{3},$ then $G$ is a group with exponent 3}
    \item []{(b) Show that if $a, b, c \in \mathbf{Z}_{2},$ then $G$ is a group with exponent 4.}
  \end{itemize}
  
  
\end{mdframed}
%%%%%%%%%%%%%%%%%%%%%%%%%%%%%%%%%%%%%%%%%%%%%%%%%%%%%%%%%%%%%%%%%%%%%%%%%%%%%%%%
\begin{mdframed}[style=darkAnswer,frametitle={Joe Starr}]
  \begin{itemize}\item[(a)]{
$$
a= \begin{bmatrix}1&0&0\\0&1&0\\0&0&1\end{bmatrix} \\
$$ $$
a= \begin{bmatrix}1&0&0\\0&1&0\\0&1&1\end{bmatrix} \\
a^{ 2 }=\lrp{ \begin{bmatrix}1&0&0\\0&1&0\\0&2&1\end{bmatrix} }
a^{ 3 }=\lrp{ \begin{bmatrix}1&0&0\\0&1&0\\0&0&1\end{bmatrix} }
$$ $$
a= \begin{bmatrix}1&0&0\\0&1&0\\0&2&1\end{bmatrix} \\
a^{ 2 }=\lrp{ \begin{bmatrix}1&0&0\\0&1&0\\0&1&1\end{bmatrix} }
a^{ 3 }=\lrp{ \begin{bmatrix}1&0&0\\0&1&0\\0&0&1\end{bmatrix} }
$$ $$
a= \begin{bmatrix}1&0&0\\0&1&0\\1&0&1\end{bmatrix} \\
a^{ 2 }=\lrp{ \begin{bmatrix}1&0&0\\0&1&0\\2&0&1\end{bmatrix} }
a^{ 3 }=\lrp{ \begin{bmatrix}1&0&0\\0&1&0\\0&0&1\end{bmatrix} }
$$ $$
a= \begin{bmatrix}1&0&0\\0&1&0\\1&1&1\end{bmatrix} \\
a^{ 2 }=\lrp{ \begin{bmatrix}1&0&0\\0&1&0\\2&2&1\end{bmatrix} }
a^{ 3 }=\lrp{ \begin{bmatrix}1&0&0\\0&1&0\\0&0&1\end{bmatrix} }
$$ $$
a= \begin{bmatrix}1&0&0\\0&1&0\\1&2&1\end{bmatrix} \\
a^{ 2 }=\lrp{ \begin{bmatrix}1&0&0\\0&1&0\\2&1&1\end{bmatrix} }
a^{ 3 }=\lrp{ \begin{bmatrix}1&0&0\\0&1&0\\0&0&1\end{bmatrix} }
$$ $$
a= \begin{bmatrix}1&0&0\\0&1&0\\2&0&1\end{bmatrix} \\
a^{ 2 }=\lrp{ \begin{bmatrix}1&0&0\\0&1&0\\1&0&1\end{bmatrix} }
a^{ 3 }=\lrp{ \begin{bmatrix}1&0&0\\0&1&0\\0&0&1\end{bmatrix} }
$$ $$
a= \begin{bmatrix}1&0&0\\0&1&0\\2&1&1\end{bmatrix} \\
a^{ 2 }=\lrp{ \begin{bmatrix}1&0&0\\0&1&0\\1&2&1\end{bmatrix} }
a^{ 3 }=\lrp{ \begin{bmatrix}1&0&0\\0&1&0\\0&0&1\end{bmatrix} }
$$ $$
a= \begin{bmatrix}1&0&0\\0&1&0\\2&2&1\end{bmatrix} \\
a^{ 2 }=\lrp{ \begin{bmatrix}1&0&0\\0&1&0\\1&1&1\end{bmatrix} }
a^{ 3 }=\lrp{ \begin{bmatrix}1&0&0\\0&1&0\\0&0&1\end{bmatrix} }
$$ $$
a= \begin{bmatrix}1&0&0\\1&1&0\\0&0&1\end{bmatrix} \\
a^{ 2 }=\lrp{ \begin{bmatrix}1&0&0\\2&1&0\\0&0&1\end{bmatrix} }
a^{ 3 }=\lrp{ \begin{bmatrix}1&0&0\\0&1&0\\0&0&1\end{bmatrix} }
$$ $$
a= \begin{bmatrix}1&0&0\\1&1&0\\0&1&1\end{bmatrix} \\
a^{ 2 }=\lrp{ \begin{bmatrix}1&0&0\\2&1&0\\1&2&1\end{bmatrix} }
a^{ 3 }=\lrp{ \begin{bmatrix}1&0&0\\0&1&0\\0&0&1\end{bmatrix} }
$$ $$
a= \begin{bmatrix}1&0&0\\1&1&0\\0&2&1\end{bmatrix} \\
a^{ 2 }=\lrp{ \begin{bmatrix}1&0&0\\2&1&0\\2&1&1\end{bmatrix} }
a^{ 3 }=\lrp{ \begin{bmatrix}1&0&0\\0&1&0\\0&0&1\end{bmatrix} }
$$ $$
a= \begin{bmatrix}1&0&0\\1&1&0\\1&0&1\end{bmatrix} \\
a^{ 2 }=\lrp{ \begin{bmatrix}1&0&0\\2&1&0\\2&0&1\end{bmatrix} }
a^{ 3 }=\lrp{ \begin{bmatrix}1&0&0\\0&1&0\\0&0&1\end{bmatrix} }
$$ $$
a= \begin{bmatrix}1&0&0\\1&1&0\\1&1&1\end{bmatrix} \\
a^{ 2 }=\lrp{ \begin{bmatrix}1&0&0\\2&1&0\\0&2&1\end{bmatrix} }
a^{ 3 }=\lrp{ \begin{bmatrix}1&0&0\\0&1&0\\0&0&1\end{bmatrix} }
$$ $$
a= \begin{bmatrix}1&0&0\\1&1&0\\1&2&1\end{bmatrix} \\
a^{ 2 }=\lrp{ \begin{bmatrix}1&0&0\\2&1&0\\1&1&1\end{bmatrix} }
a^{ 3 }=\lrp{ \begin{bmatrix}1&0&0\\0&1&0\\0&0&1\end{bmatrix} }
$$ $$
a= \begin{bmatrix}1&0&0\\1&1&0\\2&0&1\end{bmatrix} \\
a^{ 2 }=\lrp{ \begin{bmatrix}1&0&0\\2&1&0\\1&0&1\end{bmatrix} }
a^{ 3 }=\lrp{ \begin{bmatrix}1&0&0\\0&1&0\\0&0&1\end{bmatrix} }
$$ $$
a= \begin{bmatrix}1&0&0\\1&1&0\\2&1&1\end{bmatrix} \\
a^{ 2 }=\lrp{ \begin{bmatrix}1&0&0\\2&1&0\\2&2&1\end{bmatrix} }
a^{ 3 }=\lrp{ \begin{bmatrix}1&0&0\\0&1&0\\0&0&1\end{bmatrix} }
$$ $$
a= \begin{bmatrix}1&0&0\\1&1&0\\2&2&1\end{bmatrix} \\
a^{ 2 }=\lrp{ \begin{bmatrix}1&0&0\\2&1&0\\0&1&1\end{bmatrix} }
a^{ 3 }=\lrp{ \begin{bmatrix}1&0&0\\0&1&0\\0&0&1\end{bmatrix} }
$$ $$
a= \begin{bmatrix}1&0&0\\2&1&0\\0&0&1\end{bmatrix} \\
a^{ 2 }=\lrp{ \begin{bmatrix}1&0&0\\1&1&0\\0&0&1\end{bmatrix} }
a^{ 3 }=\lrp{ \begin{bmatrix}1&0&0\\0&1&0\\0&0&1\end{bmatrix} }
$$ $$
a= \begin{bmatrix}1&0&0\\2&1&0\\0&1&1\end{bmatrix} \\
a^{ 2 }=\lrp{ \begin{bmatrix}1&0&0\\1&1&0\\2&2&1\end{bmatrix} }
a^{ 3 }=\lrp{ \begin{bmatrix}1&0&0\\0&1&0\\0&0&1\end{bmatrix} }
$$ $$
a= \begin{bmatrix}1&0&0\\2&1&0\\0&2&1\end{bmatrix} \\
a^{ 2 }=\lrp{ \begin{bmatrix}1&0&0\\1&1&0\\1&1&1\end{bmatrix} }
a^{ 3 }=\lrp{ \begin{bmatrix}1&0&0\\0&1&0\\0&0&1\end{bmatrix} }
$$ $$
a= \begin{bmatrix}1&0&0\\2&1&0\\1&0&1\end{bmatrix} \\
a^{ 2 }=\lrp{ \begin{bmatrix}1&0&0\\1&1&0\\2&0&1\end{bmatrix} }
a^{ 3 }=\lrp{ \begin{bmatrix}1&0&0\\0&1&0\\0&0&1\end{bmatrix} }
$$ $$
a= \begin{bmatrix}1&0&0\\2&1&0\\1&1&1\end{bmatrix} \\
a^{ 2 }=\lrp{ \begin{bmatrix}1&0&0\\1&1&0\\1&2&1\end{bmatrix} }
a^{ 3 }=\lrp{ \begin{bmatrix}1&0&0\\0&1&0\\0&0&1\end{bmatrix} }
$$ $$
a= \begin{bmatrix}1&0&0\\2&1&0\\1&2&1\end{bmatrix} \\
a^{ 2 }=\lrp{ \begin{bmatrix}1&0&0\\1&1&0\\0&1&1\end{bmatrix} }
a^{ 3 }=\lrp{ \begin{bmatrix}1&0&0\\0&1&0\\0&0&1\end{bmatrix} }
$$ $$
a= \begin{bmatrix}1&0&0\\2&1&0\\2&0&1\end{bmatrix} \\
a^{ 2 }=\lrp{ \begin{bmatrix}1&0&0\\1&1&0\\1&0&1\end{bmatrix} }
a^{ 3 }=\lrp{ \begin{bmatrix}1&0&0\\0&1&0\\0&0&1\end{bmatrix} }
$$ $$
a= \begin{bmatrix}1&0&0\\2&1&0\\2&1&1\end{bmatrix} \\
a^{ 2 }=\lrp{ \begin{bmatrix}1&0&0\\1&1&0\\0&2&1\end{bmatrix} }
a^{ 3 }=\lrp{ \begin{bmatrix}1&0&0\\0&1&0\\0&0&1\end{bmatrix} }
$$ $$
a= \begin{bmatrix}1&0&0\\2&1&0\\2&2&1\end{bmatrix} \\
a^{ 2 }=\lrp{ \begin{bmatrix}1&0&0\\1&1&0\\2&1&1\end{bmatrix} }
a^{ 3 }=\lrp{ \begin{bmatrix}1&0&0\\0&1&0\\0&0&1\end{bmatrix} }
$$

We observe that every element is order 3 meaning exponent of $G$ is 3. 
}
  \item[(b)]{
  $$a= \begin{bmatrix}1&0&0\\0&1&0\\0&0&1\end{bmatrix} \\
    $$
    $$a= \begin{bmatrix}1&0&0\\0&1&0\\0&1&1\end{bmatrix} \\
    a^{ 2 }={ \begin{bmatrix}1&0&0\\0&1&0\\0&0&1\end{bmatrix} }
    $$
    $$a= \begin{bmatrix}1&0&0\\0&1&0\\0&0&1\end{bmatrix} \\
    a^{ 2 }={ \begin{bmatrix}1&0&0\\0&1&0\\0&0&1\end{bmatrix} }
    $$
    $$a= \begin{bmatrix}1&0&0\\0&1&0\\1&0&1\end{bmatrix} \\
    a^{ 2 }={ \begin{bmatrix}1&0&0\\0&1&0\\0&0&1\end{bmatrix} }
    $$
    $$a= \begin{bmatrix}1&0&0\\0&1&0\\1&1&1\end{bmatrix} \\
    a^{ 2 }={ \begin{bmatrix}1&0&0\\0&1&0\\0&0&1\end{bmatrix} }
    $$
    $$a= \begin{bmatrix}1&0&0\\0&1&0\\1&0&1\end{bmatrix} \\
    a^{ 2 }={ \begin{bmatrix}1&0&0\\0&1&0\\0&0&1\end{bmatrix} }
    $$
    $$a= \begin{bmatrix}1&0&0\\0&1&0\\0&0&1\end{bmatrix} \\
    a^{ 2 }={ \begin{bmatrix}1&0&0\\0&1&0\\0&0&1\end{bmatrix} }
    $$
    $$a= \begin{bmatrix}1&0&0\\0&1&0\\0&1&1\end{bmatrix} \\
    a^{ 2 }={ \begin{bmatrix}1&0&0\\0&1&0\\0&0&1\end{bmatrix} }
    $$
    $$a= \begin{bmatrix}1&0&0\\0&1&0\\0&0&1\end{bmatrix} \\
    a^{ 2 }={ \begin{bmatrix}1&0&0\\0&1&0\\0&0&1\end{bmatrix} }
    $$
    $$a= \begin{bmatrix}1&0&0\\1&1&0\\0&0&1\end{bmatrix} \\
    a^{ 2 }={ \begin{bmatrix}1&0&0\\0&1&0\\0&0&1\end{bmatrix} }
    $$
    $$a= \begin{bmatrix}1&0&0\\1&1&0\\0&1&1\end{bmatrix} \\
    a^{ 2 }={ \begin{bmatrix}1&0&0\\0&1&0\\1&0&1\end{bmatrix} }
    a^{ 3 }={ \begin{bmatrix}1&0&0\\1&1&0\\1&1&1\end{bmatrix} }
    a^{ 4 }={ \begin{bmatrix}1&0&0\\0&1&0\\0&0&1\end{bmatrix} }
    $$
    $$a= \begin{bmatrix}1&0&0\\1&1&0\\0&0&1\end{bmatrix} \\
    a^{ 2 }={ \begin{bmatrix}1&0&0\\0&1&0\\0&0&1\end{bmatrix} }
    $$
    $$a= \begin{bmatrix}1&0&0\\1&1&0\\1&0&1\end{bmatrix} \\
    a^{ 2 }={ \begin{bmatrix}1&0&0\\0&1&0\\0&0&1\end{bmatrix} }
    $$
    $$a= \begin{bmatrix}1&0&0\\1&1&0\\1&1&1\end{bmatrix} \\
    a^{ 2 }={ \begin{bmatrix}1&0&0\\0&1&0\\1&0&1\end{bmatrix} }
    a^{ 3 }={ \begin{bmatrix}1&0&0\\1&1&0\\0&1&1\end{bmatrix} }
    a^{ 4 }={ \begin{bmatrix}1&0&0\\0&1&0\\0&0&1\end{bmatrix} }
    $$
    $$a= \begin{bmatrix}1&0&0\\1&1&0\\1&0&1\end{bmatrix} \\
    a^{ 2 }={ \begin{bmatrix}1&0&0\\0&1&0\\0&0&1\end{bmatrix} }
    $$
    $$a= \begin{bmatrix}1&0&0\\1&1&0\\0&0&1\end{bmatrix} \\
    a^{ 2 }={ \begin{bmatrix}1&0&0\\0&1&0\\0&0&1\end{bmatrix} }
    $$
    $$a= \begin{bmatrix}1&0&0\\1&1&0\\0&1&1\end{bmatrix} \\
    a^{ 2 }={ \begin{bmatrix}1&0&0\\0&1&0\\1&0&1\end{bmatrix} }
    a^{ 3 }={ \begin{bmatrix}1&0&0\\1&1&0\\1&1&1\end{bmatrix} }
    a^{ 4 }={ \begin{bmatrix}1&0&0\\0&1&0\\0&0&1\end{bmatrix} }
    $$
    $$a= \begin{bmatrix}1&0&0\\1&1&0\\0&0&1\end{bmatrix} \\
    a^{ 2 }={ \begin{bmatrix}1&0&0\\0&1&0\\0&0&1\end{bmatrix} }
    $$
    $$a= \begin{bmatrix}1&0&0\\0&1&0\\0&0&1\end{bmatrix} \\
    a^{ 2 }={ \begin{bmatrix}1&0&0\\0&1&0\\0&0&1\end{bmatrix} }
    $$
    $$a= \begin{bmatrix}1&0&0\\0&1&0\\0&1&1\end{bmatrix} \\
    a^{ 2 }={ \begin{bmatrix}1&0&0\\0&1&0\\0&0&1\end{bmatrix} }
    $$
    $$a= \begin{bmatrix}1&0&0\\0&1&0\\0&0&1\end{bmatrix} \\
    a^{ 2 }={ \begin{bmatrix}1&0&0\\0&1&0\\0&0&1\end{bmatrix} }
    $$
    $$a= \begin{bmatrix}1&0&0\\0&1&0\\1&0&1\end{bmatrix} \\
    a^{ 2 }={ \begin{bmatrix}1&0&0\\0&1&0\\0&0&1\end{bmatrix} }
    $$
    $$a= \begin{bmatrix}1&0&0\\0&1&0\\1&1&1\end{bmatrix} \\
    a^{ 2 }={ \begin{bmatrix}1&0&0\\0&1&0\\0&0&1\end{bmatrix} }
    $$
    $$a= \begin{bmatrix}1&0&0\\0&1&0\\1&0&1\end{bmatrix} \\
    a^{ 2 }={ \begin{bmatrix}1&0&0\\0&1&0\\0&0&1\end{bmatrix} }
    $$
    $$a= \begin{bmatrix}1&0&0\\0&1&0\\0&0&1\end{bmatrix} \\
    a^{ 2 }={ \begin{bmatrix}1&0&0\\0&1&0\\0&0&1\end{bmatrix} }
    $$
    $$a= \begin{bmatrix}1&0&0\\0&1&0\\0&1&1\end{bmatrix} \\
    a^{ 2 }={ \begin{bmatrix}1&0&0\\0&1&0\\0&0&1\end{bmatrix} }
    $$
    $$a= \begin{bmatrix}1&0&0\\0&1&0\\0&0&1\end{bmatrix} \\
    a^{ 2 }={ \begin{bmatrix}1&0&0\\0&1&0\\0&0&1\end{bmatrix} }
    $$
    We observe that every element is either order two or four. 
    This makes exponent of $G$ 4. 
  }
\end{itemize}
\end{mdframed}
\newpage
%%%%%%%%%%%%%%%%%%%%%%%%%%%%%%%%%%%%%%%%%%%%%%%%%%%%%%%%%%%%%%%%%%%%%%%%%%%%%%%%
%%%%%%%%%%%%%%%%%%%%%%%%%%%%%%%%%%%%%%%%%%%%%%%%%%%%%%%%%%%%%%%%%%%%%%%%%%%%%%%%
%%%%%%%%%%%%%%%%%%%%%%%%%%%%%%%%%%%%%%%%%%%%%%%%%%%%%%%%%%%%%%%%%%%%%%%%%%%%%%%%
%%%%%%%%%%%%%%%%%%%%%%%%%%%%%%%%%%%%%%%%%%%%%%%%%%%%%%%%%%%%%%%%%%%%%%%%%%%%%%%%
\begin{mdframed}[style=darkQuesion]
  19. Prove that $\sum_{d | n} \varphi(d)=n$ for any positive integer $n$ 
  Hint: Interpret the equation in the cyclic group $\mathbf{Z}_{n},$ 
  by considering all of its sub-groups.
\end{mdframed}
%%%%%%%%%%%%%%%%%%%%%%%%%%%%%%%%%%%%%%%%%%%%%%%%%%%%%%%%%%%%%%%%%%%%%%%%%%%%%%%%
\begin{mdframed}[style=darkAnswer,frametitle={Joe Starr}]
%TODO this one doesn't make sense. 
\end{mdframed}
\newpage
%%%%%%%%%%%%%%%%%%%%%%%%%%%%%%%%%%%%%%%%%%%%%%%%%%%%%%%%%%%%%%%%%%%%%%%%%%%%%%%%
%%%%%%%%%%%%%%%%%%%%%%%%%%%%%%%%%%%%%%%%%%%%%%%%%%%%%%%%%%%%%%%%%%%%%%%%%%%%%%%%
%%%%%%%%%%%%%%%%%%%%%%%%%%%%%%%%%%%%%%%%%%%%%%%%%%%%%%%%%%%%%%%%%%%%%%%%%%%%%%%%
%%%%%%%%%%%%%%%%%%%%%%%%%%%%%%%%%%%%%%%%%%%%%%%%%%%%%%%%%%%%%%%%%%%%%%%%%%%%%%%%
\begin{mdframed}[style=darkQuesion]
  20. Let $n=2^{k}$ for $k>2 .$ Prove that $\mathbf{Z}_{n}^{\times}$ is not cyclic. 
  Hint: Show that $\pm 1$ and $(n / 2) \pm 1$ satisfy the equation $x^{2}=1,$ and that this is impossible in any cyclic group.
\end{mdframed}
%%%%%%%%%%%%%%%%%%%%%%%%%%%%%%%%%%%%%%%%%%%%%%%%%%%%%%%%%%%%%%%%%%%%%%%%%%%%%%%%
\begin{mdframed}[style=darkAnswer,frametitle={Joe Starr}]
  %TODO not done
\end{mdframed}
\newpage
%%%%%%%%%%%%%%%%%%%%%%%%%%%%%%%%%%%%%%%%%%%%%%%%%%%%%%%%%%%%%%%%%%%%%%%%%%%%%%%%
%%%%%%%%%%%%%%%%%%%%%%%%%%%%%%%%%%%%%%%%%%%%%%%%%%%%%%%%%%%%%%%%%%%%%%%%%%%%%%%%
%%%%%%%%%%%%%%%%%%%%%%%%%%%%%%%%%%%%%%%%%%%%%%%%%%%%%%%%%%%%%%%%%%%%%%%%%%%%%%%%
%%%%%%%%%%%%%%%%%%%%%%%%%%%%%%%%%%%%%%%%%%%%%%%%%%%%%%%%%%%%%%%%%%%%%%%%%%%%%%%%
\begin{mdframed}[style=darkQuesion]
  21.  Prove that if $p$ and $q$ are different odd primes, then $\mathbf{Z}_{p q}^{\times}$ is not a cyclic group.
\end{mdframed}
%%%%%%%%%%%%%%%%%%%%%%%%%%%%%%%%%%%%%%%%%%%%%%%%%%%%%%%%%%%%%%%%%%%%%%%%%%%%%%%%
\begin{mdframed}[style=darkAnswer,frametitle={Joe Starr}]
 %TODO not done
\end{mdframed}
\newpage
\clearpage
\subsection{Homomorphism}
%%%%%%%%%%%%%%%%%%%%%%%%%%%%%%%%%%%%%%%%%%%%%%%%%%%%%%%%%%%%%%%%%%%%%%%%%%%%%%%%
%%%%%%%%%%%%%%%%%%%%%%%%%%%%%%%%%%%%%%%%%%%%%%%%%%%%%%%%%%%%%%%%%%%%%%%%%%%%%%%%
%%%%%%%%%%%%%%%%%%%%%%%%%%%%%%%%%%%%%%%%%%%%%%%%%%%%%%%%%%%%%%%%%%%%%%%%%%%%%%%%
%%%%%%%%%%%%%%%%%%%%%%%%%%%%%%%%%%%%%%%%%%%%%%%%%%%%%%%%%%%%%%%%%%%%%%%%%%%%%%%%
\begin{mdframed}[style=darkQuesion]
  
\end{mdframed}
%%%%%%%%%%%%%%%%%%%%%%%%%%%%%%%%%%%%%%%%%%%%%%%%%%%%%%%%%%%%%%%%%%%%%%%%%%%%%%%%
\begin{mdframed}[style=darkAnswer,frametitle={Joe Starr}]
%TODO Question not started
\end{mdframed}
\newpage
%%%%%%%%%%%%%%%%%%%%%%%%%%%%%%%%%%%%%%%%%%%%%%%%%%%%%%%%%%%%%%%%%%%%%%%%%%%%%%%%
%%%%%%%%%%%%%%%%%%%%%%%%%%%%%%%%%%%%%%%%%%%%%%%%%%%%%%%%%%%%%%%%%%%%%%%%%%%%%%%%
%%%%%%%%%%%%%%%%%%%%%%%%%%%%%%%%%%%%%%%%%%%%%%%%%%%%%%%%%%%%%%%%%%%%%%%%%%%%%%%%
%%%%%%%%%%%%%%%%%%%%%%%%%%%%%%%%%%%%%%%%%%%%%%%%%%%%%%%%%%%%%%%%%%%%%%%%%%%%%%%%
\begin{mdframed}[style=darkQuesion]
  
\end{mdframed}
%%%%%%%%%%%%%%%%%%%%%%%%%%%%%%%%%%%%%%%%%%%%%%%%%%%%%%%%%%%%%%%%%%%%%%%%%%%%%%%%
\begin{mdframed}[style=darkAnswer,frametitle={Joe Starr}]
%TODO Question not started
\end{mdframed}
\newpage
%%%%%%%%%%%%%%%%%%%%%%%%%%%%%%%%%%%%%%%%%%%%%%%%%%%%%%%%%%%%%%%%%%%%%%%%%%%%%%%%
%%%%%%%%%%%%%%%%%%%%%%%%%%%%%%%%%%%%%%%%%%%%%%%%%%%%%%%%%%%%%%%%%%%%%%%%%%%%%%%%
%%%%%%%%%%%%%%%%%%%%%%%%%%%%%%%%%%%%%%%%%%%%%%%%%%%%%%%%%%%%%%%%%%%%%%%%%%%%%%%%
%%%%%%%%%%%%%%%%%%%%%%%%%%%%%%%%%%%%%%%%%%%%%%%%%%%%%%%%%%%%%%%%%%%%%%%%%%%%%%%%
\begin{mdframed}[style=darkQuesion]
  
\end{mdframed}
%%%%%%%%%%%%%%%%%%%%%%%%%%%%%%%%%%%%%%%%%%%%%%%%%%%%%%%%%%%%%%%%%%%%%%%%%%%%%%%%
\begin{mdframed}[style=darkAnswer,frametitle={Joe Starr}]
%TODO Question not started
\end{mdframed}
\newpage
%%%%%%%%%%%%%%%%%%%%%%%%%%%%%%%%%%%%%%%%%%%%%%%%%%%%%%%%%%%%%%%%%%%%%%%%%%%%%%%%
%%%%%%%%%%%%%%%%%%%%%%%%%%%%%%%%%%%%%%%%%%%%%%%%%%%%%%%%%%%%%%%%%%%%%%%%%%%%%%%%
%%%%%%%%%%%%%%%%%%%%%%%%%%%%%%%%%%%%%%%%%%%%%%%%%%%%%%%%%%%%%%%%%%%%%%%%%%%%%%%%
%%%%%%%%%%%%%%%%%%%%%%%%%%%%%%%%%%%%%%%%%%%%%%%%%%%%%%%%%%%%%%%%%%%%%%%%%%%%%%%%
\begin{mdframed}[style=darkQuesion]
  
\end{mdframed}
%%%%%%%%%%%%%%%%%%%%%%%%%%%%%%%%%%%%%%%%%%%%%%%%%%%%%%%%%%%%%%%%%%%%%%%%%%%%%%%%
\begin{mdframed}[style=darkAnswer,frametitle={Joe Starr}]
%TODO Question not started
\end{mdframed}
\newpage
%%%%%%%%%%%%%%%%%%%%%%%%%%%%%%%%%%%%%%%%%%%%%%%%%%%%%%%%%%%%%%%%%%%%%%%%%%%%%%%%
%%%%%%%%%%%%%%%%%%%%%%%%%%%%%%%%%%%%%%%%%%%%%%%%%%%%%%%%%%%%%%%%%%%%%%%%%%%%%%%%
%%%%%%%%%%%%%%%%%%%%%%%%%%%%%%%%%%%%%%%%%%%%%%%%%%%%%%%%%%%%%%%%%%%%%%%%%%%%%%%%
%%%%%%%%%%%%%%%%%%%%%%%%%%%%%%%%%%%%%%%%%%%%%%%%%%%%%%%%%%%%%%%%%%%%%%%%%%%%%%%%
\begin{mdframed}[style=darkQuesion]
  
\end{mdframed}
%%%%%%%%%%%%%%%%%%%%%%%%%%%%%%%%%%%%%%%%%%%%%%%%%%%%%%%%%%%%%%%%%%%%%%%%%%%%%%%%
\begin{mdframed}[style=darkAnswer,frametitle={Joe Starr}]
%TODO Question not started
\end{mdframed}
\newpage
%%%%%%%%%%%%%%%%%%%%%%%%%%%%%%%%%%%%%%%%%%%%%%%%%%%%%%%%%%%%%%%%%%%%%%%%%%%%%%%%
%%%%%%%%%%%%%%%%%%%%%%%%%%%%%%%%%%%%%%%%%%%%%%%%%%%%%%%%%%%%%%%%%%%%%%%%%%%%%%%%
%%%%%%%%%%%%%%%%%%%%%%%%%%%%%%%%%%%%%%%%%%%%%%%%%%%%%%%%%%%%%%%%%%%%%%%%%%%%%%%%
%%%%%%%%%%%%%%%%%%%%%%%%%%%%%%%%%%%%%%%%%%%%%%%%%%%%%%%%%%%%%%%%%%%%%%%%%%%%%%%%
\begin{mdframed}[style=darkQuesion]
  
\end{mdframed}
%%%%%%%%%%%%%%%%%%%%%%%%%%%%%%%%%%%%%%%%%%%%%%%%%%%%%%%%%%%%%%%%%%%%%%%%%%%%%%%%
\begin{mdframed}[style=darkAnswer,frametitle={Joe Starr}]
%TODO Question not started
\end{mdframed}
\newpage
\clearpage
\subsection{Cyclic Groups}
%%%%%%%%%%%%%%%%%%%%%%%%%%%%%%%%%%%%%%%%%%%%%%%%%%%%%%%%%%%%%%%%%%%%%%%%%%%%%%%%
%%%%%%%%%%%%%%%%%%%%%%%%%%%%%%%%%%%%%%%%%%%%%%%%%%%%%%%%%%%%%%%%%%%%%%%%%%%%%%%%
%%%%%%%%%%%%%%%%%%%%%%%%%%%%%%%%%%%%%%%%%%%%%%%%%%%%%%%%%%%%%%%%%%%%%%%%%%%%%%%%
%%%%%%%%%%%%%%%%%%%%%%%%%%%%%%%%%%%%%%%%%%%%%%%%%%%%%%%%%%%%%%%%%%%%%%%%%%%%%%%%
\begin{mdframed}[style=darkQuesion]
  1. Let $G$ be a group and let $a \in G$ be an element of order $12 .$ 
  What is the order of $a^{j}$ for $j=2, \ldots, 11 ?$
\end{mdframed}
%%%%%%%%%%%%%%%%%%%%%%%%%%%%%%%%%%%%%%%%%%%%%%%%%%%%%%%%%%%%%%%%%%%%%%%%%%%%%%%%
\begin{mdframed}[style=darkAnswer,frametitle={Joe Starr}]
  By applying 3.5.3 and 3.5.4 we get
$\lrp{a^{1}}^{12},\lrp{a^{2}}^{6},\lrp{a^{3}}^{4},\lrp{a^{4}}^{3},\lrp{a^{5}}^{12},
\lrp{a^{6}}^{2},\lrp{a^{7}}^{12},\lrp{a^{8}}^{3},\\\lrp{a^{9}}^{4},\lrp{a^{10}}^{6},
\lrp{a^{11}}^{12}$ 
\end{mdframed}
\newpage
%%%%%%%%%%%%%%%%%%%%%%%%%%%%%%%%%%%%%%%%%%%%%%%%%%%%%%%%%%%%%%%%%%%%%%%%%%%%%%%%
%%%%%%%%%%%%%%%%%%%%%%%%%%%%%%%%%%%%%%%%%%%%%%%%%%%%%%%%%%%%%%%%%%%%%%%%%%%%%%%%
%%%%%%%%%%%%%%%%%%%%%%%%%%%%%%%%%%%%%%%%%%%%%%%%%%%%%%%%%%%%%%%%%%%%%%%%%%%%%%%%
%%%%%%%%%%%%%%%%%%%%%%%%%%%%%%%%%%%%%%%%%%%%%%%%%%%%%%%%%%%%%%%%%%%%%%%%%%%%%%%%
\begin{mdframed}[style=darkQuesion]
2. Let $G$ be a group and let $a \in G$ be an element of order $30 .$ 
List the powers of $a$ that have order $2,$ order 3 or order 5
\end{mdframed}
%%%%%%%%%%%%%%%%%%%%%%%%%%%%%%%%%%%%%%%%%%%%%%%%%%%%%%%%%%%%%%%%%%%%%%%%%%%%%%%%
\begin{mdframed}[style=darkAnswer,frametitle={Joe Starr}]
\begin{align*}
  2&: 15\\
  3&: 10, 20\\
  5&: 6,12,18,24\\
\end{align*}
\end{mdframed}
\newpage
%%%%%%%%%%%%%%%%%%%%%%%%%%%%%%%%%%%%%%%%%%%%%%%%%%%%%%%%%%%%%%%%%%%%%%%%%%%%%%%%
%%%%%%%%%%%%%%%%%%%%%%%%%%%%%%%%%%%%%%%%%%%%%%%%%%%%%%%%%%%%%%%%%%%%%%%%%%%%%%%%
%%%%%%%%%%%%%%%%%%%%%%%%%%%%%%%%%%%%%%%%%%%%%%%%%%%%%%%%%%%%%%%%%%%%%%%%%%%%%%%%
%%%%%%%%%%%%%%%%%%%%%%%%%%%%%%%%%%%%%%%%%%%%%%%%%%%%%%%%%%%%%%%%%%%%%%%%%%%%%%%%
\begin{mdframed}[style=darkQuesion]
5. Find the cyclic subgroup of $C^{\times}$ generated by $\frac{\sqrt{2}+\sqrt{2}i}{2}$.
\end{mdframed}
%%%%%%%%%%%%%%%%%%%%%%%%%%%%%%%%%%%%%%%%%%%%%%%%%%%%%%%%%%%%%%%%%%%%%%%%%%%%%%%%
\begin{mdframed}[style=darkAnswer,frametitle={Joe Starr}]
 \begin{align*}
   a^{1}&=\frac{\sqrt{2}+\sqrt{2}i}{2}\\
   a^{2}&=\frac{\sqrt{2}+\sqrt{2}i}{2}\frac{\sqrt{2}+\sqrt{2}i}{2}=i\\
   a^{4}&=i=-1\\
   a^{8}&=i=1\\
 \end{align*}
\end{mdframed}
\newpage
%%%%%%%%%%%%%%%%%%%%%%%%%%%%%%%%%%%%%%%%%%%%%%%%%%%%%%%%%%%%%%%%%%%%%%%%%%%%%%%%
%%%%%%%%%%%%%%%%%%%%%%%%%%%%%%%%%%%%%%%%%%%%%%%%%%%%%%%%%%%%%%%%%%%%%%%%%%%%%%%%
%%%%%%%%%%%%%%%%%%%%%%%%%%%%%%%%%%%%%%%%%%%%%%%%%%%%%%%%%%%%%%%%%%%%%%%%%%%%%%%%
%%%%%%%%%%%%%%%%%%%%%%%%%%%%%%%%%%%%%%%%%%%%%%%%%%%%%%%%%%%%%%%%%%%%%%%%%%%%%%%%
\begin{mdframed}[style=darkQuesion]
6. Find the order of the cyclic subgroup of $\mathbf{C}^{\times}$ generated by $1+i$
\end{mdframed}
%%%%%%%%%%%%%%%%%%%%%%%%%%%%%%%%%%%%%%%%%%%%%%%%%%%%%%%%%%%%%%%%%%%%%%%%%%%%%%%%
\begin{mdframed}[style=darkAnswer,frametitle={Joe Starr}]
 \begin{align*}
  a^{1}&=1+i\\
  a^{2}&=\lrp{1+i}\lrp{1+i}=2i\\
  a^{4}&=-4\\
  a^{8}&=16\\
  a^{2^{3+n}}&={2^{4+n}}\\
 \end{align*}
 we observe $2^{4+n}$ has infinite order so $a$ must have infinite order. 
\end{mdframed}
\newpage
%%%%%%%%%%%%%%%%%%%%%%%%%%%%%%%%%%%%%%%%%%%%%%%%%%%%%%%%%%%%%%%%%%%%%%%%%%%%%%%%
%%%%%%%%%%%%%%%%%%%%%%%%%%%%%%%%%%%%%%%%%%%%%%%%%%%%%%%%%%%%%%%%%%%%%%%%%%%%%%%%
%%%%%%%%%%%%%%%%%%%%%%%%%%%%%%%%%%%%%%%%%%%%%%%%%%%%%%%%%%%%%%%%%%%%%%%%%%%%%%%%
%%%%%%%%%%%%%%%%%%%%%%%%%%%%%%%%%%%%%%%%%%%%%%%%%%%%%%%%%%%%%%%%%%%%%%%%%%%%%%%%
\begin{mdframed}[style=darkQuesion]
7. Which of the multiplicative groups $\mathbf{Z}_{15}^{\times}, \mathbf{Z}_{18}^{\times}, \mathbf{Z}_{20}^{\times}, \mathbf{Z}_{27}^{\times}$ are cyclic?
\end{mdframed}
%%%%%%%%%%%%%%%%%%%%%%%%%%%%%%%%%%%%%%%%%%%%%%%%%%%%%%%%%%%%%%%%%%%%%%%%%%%%%%%%
\begin{mdframed}[style=darkAnswer,frametitle={Joe Starr}]
 \begin{align*}
  \mathbf{Z}_{15}^{\times}&:15=5\cdot3 \text{ The product of two odd primes, not cyclic}\\
  \mathbf{Z}_{18}^{\times}&:18=2\cdot3^2 \text{ The product of two and power of an odd prime, cyclic}\\
  \mathbf{Z}_{20}^{\times}&:20=2^2\cdot5 \text{ The product of four and an odd prime, not cyclic}\\
  \mathbf{Z}_{27}^{\times}&:15=3^3 \text{ The power of an odd prime, cyclic}\\
 \end{align*}
\end{mdframed}
\newpage
%%%%%%%%%%%%%%%%%%%%%%%%%%%%%%%%%%%%%%%%%%%%%%%%%%%%%%%%%%%%%%%%%%%%%%%%%%%%%%%%
%%%%%%%%%%%%%%%%%%%%%%%%%%%%%%%%%%%%%%%%%%%%%%%%%%%%%%%%%%%%%%%%%%%%%%%%%%%%%%%%
%%%%%%%%%%%%%%%%%%%%%%%%%%%%%%%%%%%%%%%%%%%%%%%%%%%%%%%%%%%%%%%%%%%%%%%%%%%%%%%%
%%%%%%%%%%%%%%%%%%%%%%%%%%%%%%%%%%%%%%%%%%%%%%%%%%%%%%%%%%%%%%%%%%%%%%%%%%%%%%%%
\begin{mdframed}[style=darkQuesion]
11. Which of the multiplicative groups  $\Z^{\times}_{7},\ \Z^{\times}_{10},\ \Z^{\times}_{12},\ \Z^{\times}_{14}$ are isomorphic.
\end{mdframed}
%%%%%%%%%%%%%%%%%%%%%%%%%%%%%%%%%%%%%%%%%%%%%%%%%%%%%%%%%%%%%%%%%%%%%%%%%%%%%%%%
\begin{mdframed}[style=darkAnswer,frametitle={Joe Starr}]
  \begin{align*}
    \mathbf{\Z}_{7 }^{\times}&:7 =7 \text{ The power of an odd prime, cyclic}\\
    \mathbf{\Z}_{10}^{\times}&:10=2\cdot5 \text{ The product of two and power of an odd prime, cyclic}\\
    \mathbf{\Z}_{12}^{\times}&:12=2^2\cdot3\cdot5 \text{ The product of four and an odd prime, not cyclic}\\
    \mathbf{\Z}_{14}^{\times}&:14=2\cdot7 \text{ The product of two and power of an odd prime, cyclic}\\
   \hline \\
    \mathbf{\Z}_{7 }^{\times}&=\lrs{1, 2, 3, 4, 5, 6} \\
    \mathbf{\Z}_{10}^{\times}&=\lrs{1, 3, 7, 9} \\
    \mathbf{\Z}_{14}^{\times}&=\lrs{1, 3, 5, 9, 11, 13} \\
   \end{align*}
   Since $\Z_{14}^{\times}$ has order six and and $\Zmx{7}$ has order six, 
   and both are cyclic they must both be isomorphic to $\Zm{6}$.

\end{mdframed}
\newpage
%%%%%%%%%%%%%%%%%%%%%%%%%%%%%%%%%%%%%%%%%%%%%%%%%%%%%%%%%%%%%%%%%%%%%%%%%%%%%%%%
%%%%%%%%%%%%%%%%%%%%%%%%%%%%%%%%%%%%%%%%%%%%%%%%%%%%%%%%%%%%%%%%%%%%%%%%%%%%%%%%
%%%%%%%%%%%%%%%%%%%%%%%%%%%%%%%%%%%%%%%%%%%%%%%%%%%%%%%%%%%%%%%%%%%%%%%%%%%%%%%%
%%%%%%%%%%%%%%%%%%%%%%%%%%%%%%%%%%%%%%%%%%%%%%%%%%%%%%%%%%%%%%%%%%%%%%%%%%%%%%%%
\begin{mdframed}[style=darkQuesion]
12. Let $a,b$ be positive integers, and let $d=\ngcd{a}{b}$ and $m=\nlcm{a}{b}$. 
Use proposition 3.5.5 to prove that $\Z_a\times \Z_b\cong \Z_d\times \Z_m$.
\end{mdframed}
%%%%%%%%%%%%%%%%%%%%%%%%%%%%%%%%%%%%%%%%%%%%%%%%%%%%%%%%%%%%%%%%%%%%%%%%%%%%%%%%
\begin{mdframed}[style=darkAnswer,frametitle={Joe Starr}]
 By $3.5.5$ we have that $\Zm{ab}\cong \Z_a\times \Z_b$, and 
 $\Zm{dm}\cong \Z_d\times \Z_m$, we've previously shown $\ngcd{a}{b}\cdot \nlcm{a}{b}=ab$
 making $\Zm{dm}=\Zm{ab}$ showing $\Z_a\times \Z_b\cong \Z_d\times \Z_m$.
\end{mdframed}
\newpage
%%%%%%%%%%%%%%%%%%%%%%%%%%%%%%%%%%%%%%%%%%%%%%%%%%%%%%%%%%%%%%%%%%%%%%%%%%%%%%%%
%%%%%%%%%%%%%%%%%%%%%%%%%%%%%%%%%%%%%%%%%%%%%%%%%%%%%%%%%%%%%%%%%%%%%%%%%%%%%%%%
%%%%%%%%%%%%%%%%%%%%%%%%%%%%%%%%%%%%%%%%%%%%%%%%%%%%%%%%%%%%%%%%%%%%%%%%%%%%%%%%
%%%%%%%%%%%%%%%%%%%%%%%%%%%%%%%%%%%%%%%%%%%%%%%%%%%%%%%%%%%%%%%%%%%%%%%%%%%%%%%%
\begin{mdframed}[style=darkQuesion]
18. Let $G$ be the set of all 3 $\times 3$ matrices of the form $\left[\begin{array}{lll}1 & 0 & 0 \\ a & 1 & 0 \\ b & c & 1\end{array}\right]$
  \begin{itemize}
    \item []{(a) Show that if $a, b, c \in \mathbf{Z}_{3},$ then $G$ is a group with exponent 3}
    \item []{(b) Show that if $a, b, c \in \mathbf{Z}_{2},$ then $G$ is a group with exponent 4.}
  \end{itemize}
  
  
\end{mdframed}
%%%%%%%%%%%%%%%%%%%%%%%%%%%%%%%%%%%%%%%%%%%%%%%%%%%%%%%%%%%%%%%%%%%%%%%%%%%%%%%%
\begin{mdframed}[style=darkAnswer,frametitle={Joe Starr}]
  \begin{itemize}\item[(a)]{
$$
a= \begin{bmatrix}1&0&0\\0&1&0\\0&0&1\end{bmatrix} \\
$$ $$
a= \begin{bmatrix}1&0&0\\0&1&0\\0&1&1\end{bmatrix} \\
a^{ 2 }=\lrp{ \begin{bmatrix}1&0&0\\0&1&0\\0&2&1\end{bmatrix} }
a^{ 3 }=\lrp{ \begin{bmatrix}1&0&0\\0&1&0\\0&0&1\end{bmatrix} }
$$ $$
a= \begin{bmatrix}1&0&0\\0&1&0\\0&2&1\end{bmatrix} \\
a^{ 2 }=\lrp{ \begin{bmatrix}1&0&0\\0&1&0\\0&1&1\end{bmatrix} }
a^{ 3 }=\lrp{ \begin{bmatrix}1&0&0\\0&1&0\\0&0&1\end{bmatrix} }
$$ $$
a= \begin{bmatrix}1&0&0\\0&1&0\\1&0&1\end{bmatrix} \\
a^{ 2 }=\lrp{ \begin{bmatrix}1&0&0\\0&1&0\\2&0&1\end{bmatrix} }
a^{ 3 }=\lrp{ \begin{bmatrix}1&0&0\\0&1&0\\0&0&1\end{bmatrix} }
$$ $$
a= \begin{bmatrix}1&0&0\\0&1&0\\1&1&1\end{bmatrix} \\
a^{ 2 }=\lrp{ \begin{bmatrix}1&0&0\\0&1&0\\2&2&1\end{bmatrix} }
a^{ 3 }=\lrp{ \begin{bmatrix}1&0&0\\0&1&0\\0&0&1\end{bmatrix} }
$$ $$
a= \begin{bmatrix}1&0&0\\0&1&0\\1&2&1\end{bmatrix} \\
a^{ 2 }=\lrp{ \begin{bmatrix}1&0&0\\0&1&0\\2&1&1\end{bmatrix} }
a^{ 3 }=\lrp{ \begin{bmatrix}1&0&0\\0&1&0\\0&0&1\end{bmatrix} }
$$ $$
a= \begin{bmatrix}1&0&0\\0&1&0\\2&0&1\end{bmatrix} \\
a^{ 2 }=\lrp{ \begin{bmatrix}1&0&0\\0&1&0\\1&0&1\end{bmatrix} }
a^{ 3 }=\lrp{ \begin{bmatrix}1&0&0\\0&1&0\\0&0&1\end{bmatrix} }
$$ $$
a= \begin{bmatrix}1&0&0\\0&1&0\\2&1&1\end{bmatrix} \\
a^{ 2 }=\lrp{ \begin{bmatrix}1&0&0\\0&1&0\\1&2&1\end{bmatrix} }
a^{ 3 }=\lrp{ \begin{bmatrix}1&0&0\\0&1&0\\0&0&1\end{bmatrix} }
$$ $$
a= \begin{bmatrix}1&0&0\\0&1&0\\2&2&1\end{bmatrix} \\
a^{ 2 }=\lrp{ \begin{bmatrix}1&0&0\\0&1&0\\1&1&1\end{bmatrix} }
a^{ 3 }=\lrp{ \begin{bmatrix}1&0&0\\0&1&0\\0&0&1\end{bmatrix} }
$$ $$
a= \begin{bmatrix}1&0&0\\1&1&0\\0&0&1\end{bmatrix} \\
a^{ 2 }=\lrp{ \begin{bmatrix}1&0&0\\2&1&0\\0&0&1\end{bmatrix} }
a^{ 3 }=\lrp{ \begin{bmatrix}1&0&0\\0&1&0\\0&0&1\end{bmatrix} }
$$ $$
a= \begin{bmatrix}1&0&0\\1&1&0\\0&1&1\end{bmatrix} \\
a^{ 2 }=\lrp{ \begin{bmatrix}1&0&0\\2&1&0\\1&2&1\end{bmatrix} }
a^{ 3 }=\lrp{ \begin{bmatrix}1&0&0\\0&1&0\\0&0&1\end{bmatrix} }
$$ $$
a= \begin{bmatrix}1&0&0\\1&1&0\\0&2&1\end{bmatrix} \\
a^{ 2 }=\lrp{ \begin{bmatrix}1&0&0\\2&1&0\\2&1&1\end{bmatrix} }
a^{ 3 }=\lrp{ \begin{bmatrix}1&0&0\\0&1&0\\0&0&1\end{bmatrix} }
$$ $$
a= \begin{bmatrix}1&0&0\\1&1&0\\1&0&1\end{bmatrix} \\
a^{ 2 }=\lrp{ \begin{bmatrix}1&0&0\\2&1&0\\2&0&1\end{bmatrix} }
a^{ 3 }=\lrp{ \begin{bmatrix}1&0&0\\0&1&0\\0&0&1\end{bmatrix} }
$$ $$
a= \begin{bmatrix}1&0&0\\1&1&0\\1&1&1\end{bmatrix} \\
a^{ 2 }=\lrp{ \begin{bmatrix}1&0&0\\2&1&0\\0&2&1\end{bmatrix} }
a^{ 3 }=\lrp{ \begin{bmatrix}1&0&0\\0&1&0\\0&0&1\end{bmatrix} }
$$ $$
a= \begin{bmatrix}1&0&0\\1&1&0\\1&2&1\end{bmatrix} \\
a^{ 2 }=\lrp{ \begin{bmatrix}1&0&0\\2&1&0\\1&1&1\end{bmatrix} }
a^{ 3 }=\lrp{ \begin{bmatrix}1&0&0\\0&1&0\\0&0&1\end{bmatrix} }
$$ $$
a= \begin{bmatrix}1&0&0\\1&1&0\\2&0&1\end{bmatrix} \\
a^{ 2 }=\lrp{ \begin{bmatrix}1&0&0\\2&1&0\\1&0&1\end{bmatrix} }
a^{ 3 }=\lrp{ \begin{bmatrix}1&0&0\\0&1&0\\0&0&1\end{bmatrix} }
$$ $$
a= \begin{bmatrix}1&0&0\\1&1&0\\2&1&1\end{bmatrix} \\
a^{ 2 }=\lrp{ \begin{bmatrix}1&0&0\\2&1&0\\2&2&1\end{bmatrix} }
a^{ 3 }=\lrp{ \begin{bmatrix}1&0&0\\0&1&0\\0&0&1\end{bmatrix} }
$$ $$
a= \begin{bmatrix}1&0&0\\1&1&0\\2&2&1\end{bmatrix} \\
a^{ 2 }=\lrp{ \begin{bmatrix}1&0&0\\2&1&0\\0&1&1\end{bmatrix} }
a^{ 3 }=\lrp{ \begin{bmatrix}1&0&0\\0&1&0\\0&0&1\end{bmatrix} }
$$ $$
a= \begin{bmatrix}1&0&0\\2&1&0\\0&0&1\end{bmatrix} \\
a^{ 2 }=\lrp{ \begin{bmatrix}1&0&0\\1&1&0\\0&0&1\end{bmatrix} }
a^{ 3 }=\lrp{ \begin{bmatrix}1&0&0\\0&1&0\\0&0&1\end{bmatrix} }
$$ $$
a= \begin{bmatrix}1&0&0\\2&1&0\\0&1&1\end{bmatrix} \\
a^{ 2 }=\lrp{ \begin{bmatrix}1&0&0\\1&1&0\\2&2&1\end{bmatrix} }
a^{ 3 }=\lrp{ \begin{bmatrix}1&0&0\\0&1&0\\0&0&1\end{bmatrix} }
$$ $$
a= \begin{bmatrix}1&0&0\\2&1&0\\0&2&1\end{bmatrix} \\
a^{ 2 }=\lrp{ \begin{bmatrix}1&0&0\\1&1&0\\1&1&1\end{bmatrix} }
a^{ 3 }=\lrp{ \begin{bmatrix}1&0&0\\0&1&0\\0&0&1\end{bmatrix} }
$$ $$
a= \begin{bmatrix}1&0&0\\2&1&0\\1&0&1\end{bmatrix} \\
a^{ 2 }=\lrp{ \begin{bmatrix}1&0&0\\1&1&0\\2&0&1\end{bmatrix} }
a^{ 3 }=\lrp{ \begin{bmatrix}1&0&0\\0&1&0\\0&0&1\end{bmatrix} }
$$ $$
a= \begin{bmatrix}1&0&0\\2&1&0\\1&1&1\end{bmatrix} \\
a^{ 2 }=\lrp{ \begin{bmatrix}1&0&0\\1&1&0\\1&2&1\end{bmatrix} }
a^{ 3 }=\lrp{ \begin{bmatrix}1&0&0\\0&1&0\\0&0&1\end{bmatrix} }
$$ $$
a= \begin{bmatrix}1&0&0\\2&1&0\\1&2&1\end{bmatrix} \\
a^{ 2 }=\lrp{ \begin{bmatrix}1&0&0\\1&1&0\\0&1&1\end{bmatrix} }
a^{ 3 }=\lrp{ \begin{bmatrix}1&0&0\\0&1&0\\0&0&1\end{bmatrix} }
$$ $$
a= \begin{bmatrix}1&0&0\\2&1&0\\2&0&1\end{bmatrix} \\
a^{ 2 }=\lrp{ \begin{bmatrix}1&0&0\\1&1&0\\1&0&1\end{bmatrix} }
a^{ 3 }=\lrp{ \begin{bmatrix}1&0&0\\0&1&0\\0&0&1\end{bmatrix} }
$$ $$
a= \begin{bmatrix}1&0&0\\2&1&0\\2&1&1\end{bmatrix} \\
a^{ 2 }=\lrp{ \begin{bmatrix}1&0&0\\1&1&0\\0&2&1\end{bmatrix} }
a^{ 3 }=\lrp{ \begin{bmatrix}1&0&0\\0&1&0\\0&0&1\end{bmatrix} }
$$ $$
a= \begin{bmatrix}1&0&0\\2&1&0\\2&2&1\end{bmatrix} \\
a^{ 2 }=\lrp{ \begin{bmatrix}1&0&0\\1&1&0\\2&1&1\end{bmatrix} }
a^{ 3 }=\lrp{ \begin{bmatrix}1&0&0\\0&1&0\\0&0&1\end{bmatrix} }
$$

We observe that every element is order 3 meaning exponent of $G$ is 3. 
}
  \item[(b)]{
  $$a= \begin{bmatrix}1&0&0\\0&1&0\\0&0&1\end{bmatrix} \\
    $$
    $$a= \begin{bmatrix}1&0&0\\0&1&0\\0&1&1\end{bmatrix} \\
    a^{ 2 }={ \begin{bmatrix}1&0&0\\0&1&0\\0&0&1\end{bmatrix} }
    $$
    $$a= \begin{bmatrix}1&0&0\\0&1&0\\0&0&1\end{bmatrix} \\
    a^{ 2 }={ \begin{bmatrix}1&0&0\\0&1&0\\0&0&1\end{bmatrix} }
    $$
    $$a= \begin{bmatrix}1&0&0\\0&1&0\\1&0&1\end{bmatrix} \\
    a^{ 2 }={ \begin{bmatrix}1&0&0\\0&1&0\\0&0&1\end{bmatrix} }
    $$
    $$a= \begin{bmatrix}1&0&0\\0&1&0\\1&1&1\end{bmatrix} \\
    a^{ 2 }={ \begin{bmatrix}1&0&0\\0&1&0\\0&0&1\end{bmatrix} }
    $$
    $$a= \begin{bmatrix}1&0&0\\0&1&0\\1&0&1\end{bmatrix} \\
    a^{ 2 }={ \begin{bmatrix}1&0&0\\0&1&0\\0&0&1\end{bmatrix} }
    $$
    $$a= \begin{bmatrix}1&0&0\\0&1&0\\0&0&1\end{bmatrix} \\
    a^{ 2 }={ \begin{bmatrix}1&0&0\\0&1&0\\0&0&1\end{bmatrix} }
    $$
    $$a= \begin{bmatrix}1&0&0\\0&1&0\\0&1&1\end{bmatrix} \\
    a^{ 2 }={ \begin{bmatrix}1&0&0\\0&1&0\\0&0&1\end{bmatrix} }
    $$
    $$a= \begin{bmatrix}1&0&0\\0&1&0\\0&0&1\end{bmatrix} \\
    a^{ 2 }={ \begin{bmatrix}1&0&0\\0&1&0\\0&0&1\end{bmatrix} }
    $$
    $$a= \begin{bmatrix}1&0&0\\1&1&0\\0&0&1\end{bmatrix} \\
    a^{ 2 }={ \begin{bmatrix}1&0&0\\0&1&0\\0&0&1\end{bmatrix} }
    $$
    $$a= \begin{bmatrix}1&0&0\\1&1&0\\0&1&1\end{bmatrix} \\
    a^{ 2 }={ \begin{bmatrix}1&0&0\\0&1&0\\1&0&1\end{bmatrix} }
    a^{ 3 }={ \begin{bmatrix}1&0&0\\1&1&0\\1&1&1\end{bmatrix} }
    a^{ 4 }={ \begin{bmatrix}1&0&0\\0&1&0\\0&0&1\end{bmatrix} }
    $$
    $$a= \begin{bmatrix}1&0&0\\1&1&0\\0&0&1\end{bmatrix} \\
    a^{ 2 }={ \begin{bmatrix}1&0&0\\0&1&0\\0&0&1\end{bmatrix} }
    $$
    $$a= \begin{bmatrix}1&0&0\\1&1&0\\1&0&1\end{bmatrix} \\
    a^{ 2 }={ \begin{bmatrix}1&0&0\\0&1&0\\0&0&1\end{bmatrix} }
    $$
    $$a= \begin{bmatrix}1&0&0\\1&1&0\\1&1&1\end{bmatrix} \\
    a^{ 2 }={ \begin{bmatrix}1&0&0\\0&1&0\\1&0&1\end{bmatrix} }
    a^{ 3 }={ \begin{bmatrix}1&0&0\\1&1&0\\0&1&1\end{bmatrix} }
    a^{ 4 }={ \begin{bmatrix}1&0&0\\0&1&0\\0&0&1\end{bmatrix} }
    $$
    $$a= \begin{bmatrix}1&0&0\\1&1&0\\1&0&1\end{bmatrix} \\
    a^{ 2 }={ \begin{bmatrix}1&0&0\\0&1&0\\0&0&1\end{bmatrix} }
    $$
    $$a= \begin{bmatrix}1&0&0\\1&1&0\\0&0&1\end{bmatrix} \\
    a^{ 2 }={ \begin{bmatrix}1&0&0\\0&1&0\\0&0&1\end{bmatrix} }
    $$
    $$a= \begin{bmatrix}1&0&0\\1&1&0\\0&1&1\end{bmatrix} \\
    a^{ 2 }={ \begin{bmatrix}1&0&0\\0&1&0\\1&0&1\end{bmatrix} }
    a^{ 3 }={ \begin{bmatrix}1&0&0\\1&1&0\\1&1&1\end{bmatrix} }
    a^{ 4 }={ \begin{bmatrix}1&0&0\\0&1&0\\0&0&1\end{bmatrix} }
    $$
    $$a= \begin{bmatrix}1&0&0\\1&1&0\\0&0&1\end{bmatrix} \\
    a^{ 2 }={ \begin{bmatrix}1&0&0\\0&1&0\\0&0&1\end{bmatrix} }
    $$
    $$a= \begin{bmatrix}1&0&0\\0&1&0\\0&0&1\end{bmatrix} \\
    a^{ 2 }={ \begin{bmatrix}1&0&0\\0&1&0\\0&0&1\end{bmatrix} }
    $$
    $$a= \begin{bmatrix}1&0&0\\0&1&0\\0&1&1\end{bmatrix} \\
    a^{ 2 }={ \begin{bmatrix}1&0&0\\0&1&0\\0&0&1\end{bmatrix} }
    $$
    $$a= \begin{bmatrix}1&0&0\\0&1&0\\0&0&1\end{bmatrix} \\
    a^{ 2 }={ \begin{bmatrix}1&0&0\\0&1&0\\0&0&1\end{bmatrix} }
    $$
    $$a= \begin{bmatrix}1&0&0\\0&1&0\\1&0&1\end{bmatrix} \\
    a^{ 2 }={ \begin{bmatrix}1&0&0\\0&1&0\\0&0&1\end{bmatrix} }
    $$
    $$a= \begin{bmatrix}1&0&0\\0&1&0\\1&1&1\end{bmatrix} \\
    a^{ 2 }={ \begin{bmatrix}1&0&0\\0&1&0\\0&0&1\end{bmatrix} }
    $$
    $$a= \begin{bmatrix}1&0&0\\0&1&0\\1&0&1\end{bmatrix} \\
    a^{ 2 }={ \begin{bmatrix}1&0&0\\0&1&0\\0&0&1\end{bmatrix} }
    $$
    $$a= \begin{bmatrix}1&0&0\\0&1&0\\0&0&1\end{bmatrix} \\
    a^{ 2 }={ \begin{bmatrix}1&0&0\\0&1&0\\0&0&1\end{bmatrix} }
    $$
    $$a= \begin{bmatrix}1&0&0\\0&1&0\\0&1&1\end{bmatrix} \\
    a^{ 2 }={ \begin{bmatrix}1&0&0\\0&1&0\\0&0&1\end{bmatrix} }
    $$
    $$a= \begin{bmatrix}1&0&0\\0&1&0\\0&0&1\end{bmatrix} \\
    a^{ 2 }={ \begin{bmatrix}1&0&0\\0&1&0\\0&0&1\end{bmatrix} }
    $$
    We observe that every element is either order two or four. 
    This makes exponent of $G$ 4. 
  }
\end{itemize}
\end{mdframed}
\newpage
%%%%%%%%%%%%%%%%%%%%%%%%%%%%%%%%%%%%%%%%%%%%%%%%%%%%%%%%%%%%%%%%%%%%%%%%%%%%%%%%
%%%%%%%%%%%%%%%%%%%%%%%%%%%%%%%%%%%%%%%%%%%%%%%%%%%%%%%%%%%%%%%%%%%%%%%%%%%%%%%%
%%%%%%%%%%%%%%%%%%%%%%%%%%%%%%%%%%%%%%%%%%%%%%%%%%%%%%%%%%%%%%%%%%%%%%%%%%%%%%%%
%%%%%%%%%%%%%%%%%%%%%%%%%%%%%%%%%%%%%%%%%%%%%%%%%%%%%%%%%%%%%%%%%%%%%%%%%%%%%%%%
\begin{mdframed}[style=darkQuesion]
  19. Prove that $\sum_{d | n} \varphi(d)=n$ for any positive integer $n$ 
  Hint: Interpret the equation in the cyclic group $\mathbf{Z}_{n},$ 
  by considering all of its sub-groups.
\end{mdframed}
%%%%%%%%%%%%%%%%%%%%%%%%%%%%%%%%%%%%%%%%%%%%%%%%%%%%%%%%%%%%%%%%%%%%%%%%%%%%%%%%
\begin{mdframed}[style=darkAnswer,frametitle={Joe Starr}]
%TODO this one doesn't make sense. 
\end{mdframed}
\newpage
%%%%%%%%%%%%%%%%%%%%%%%%%%%%%%%%%%%%%%%%%%%%%%%%%%%%%%%%%%%%%%%%%%%%%%%%%%%%%%%%
%%%%%%%%%%%%%%%%%%%%%%%%%%%%%%%%%%%%%%%%%%%%%%%%%%%%%%%%%%%%%%%%%%%%%%%%%%%%%%%%
%%%%%%%%%%%%%%%%%%%%%%%%%%%%%%%%%%%%%%%%%%%%%%%%%%%%%%%%%%%%%%%%%%%%%%%%%%%%%%%%
%%%%%%%%%%%%%%%%%%%%%%%%%%%%%%%%%%%%%%%%%%%%%%%%%%%%%%%%%%%%%%%%%%%%%%%%%%%%%%%%
\begin{mdframed}[style=darkQuesion]
  20. Let $n=2^{k}$ for $k>2 .$ Prove that $\mathbf{Z}_{n}^{\times}$ is not cyclic. 
  Hint: Show that $\pm 1$ and $(n / 2) \pm 1$ satisfy the equation $x^{2}=1,$ and that this is impossible in any cyclic group.
\end{mdframed}
%%%%%%%%%%%%%%%%%%%%%%%%%%%%%%%%%%%%%%%%%%%%%%%%%%%%%%%%%%%%%%%%%%%%%%%%%%%%%%%%
\begin{mdframed}[style=darkAnswer,frametitle={Joe Starr}]
  %TODO not done
\end{mdframed}
\newpage
%%%%%%%%%%%%%%%%%%%%%%%%%%%%%%%%%%%%%%%%%%%%%%%%%%%%%%%%%%%%%%%%%%%%%%%%%%%%%%%%
%%%%%%%%%%%%%%%%%%%%%%%%%%%%%%%%%%%%%%%%%%%%%%%%%%%%%%%%%%%%%%%%%%%%%%%%%%%%%%%%
%%%%%%%%%%%%%%%%%%%%%%%%%%%%%%%%%%%%%%%%%%%%%%%%%%%%%%%%%%%%%%%%%%%%%%%%%%%%%%%%
%%%%%%%%%%%%%%%%%%%%%%%%%%%%%%%%%%%%%%%%%%%%%%%%%%%%%%%%%%%%%%%%%%%%%%%%%%%%%%%%
\begin{mdframed}[style=darkQuesion]
  21.  Prove that if $p$ and $q$ are different odd primes, then $\mathbf{Z}_{p q}^{\times}$ is not a cyclic group.
\end{mdframed}
%%%%%%%%%%%%%%%%%%%%%%%%%%%%%%%%%%%%%%%%%%%%%%%%%%%%%%%%%%%%%%%%%%%%%%%%%%%%%%%%
\begin{mdframed}[style=darkAnswer,frametitle={Joe Starr}]
 %TODO not done
\end{mdframed}
\newpage
\clearpage
%%%%%%%%%%%%%%%%%%%%%%%%%%%%%%%%%%%%%%%%%%%%%%%%%%%%%%%%%%%%%%%%%%%%%%%%%%%%%%%%
%Compile Chapter 4
\section{Polynomials}
\subsection{Fields; Roots of Polynomials}
%%%%%%%%%%%%%%%%%%%%%%%%%%%%%%%%%%%%%%%%%%%%%%%%%%%%%%%%%%%%%%%%%%%%%%%%%%%%%%%%
%%%%%%%%%%%%%%%%%%%%%%%%%%%%%%%%%%%%%%%%%%%%%%%%%%%%%%%%%%%%%%%%%%%%%%%%%%%%%%%%
%%%%%%%%%%%%%%%%%%%%%%%%%%%%%%%%%%%%%%%%%%%%%%%%%%%%%%%%%%%%%%%%%%%%%%%%%%%%%%%%
%%%%%%%%%%%%%%%%%%%%%%%%%%%%%%%%%%%%%%%%%%%%%%%%%%%%%%%%%%%%%%%%%%%%%%%%%%%%%%%%
\begin{mdframed}[style=darkQuesion]
  
\end{mdframed}
%%%%%%%%%%%%%%%%%%%%%%%%%%%%%%%%%%%%%%%%%%%%%%%%%%%%%%%%%%%%%%%%%%%%%%%%%%%%%%%%
\begin{mdframed}[style=darkAnswer,frametitle={Joe Starr}]
%TODO Question not started
\end{mdframed}
\newpage
%%%%%%%%%%%%%%%%%%%%%%%%%%%%%%%%%%%%%%%%%%%%%%%%%%%%%%%%%%%%%%%%%%%%%%%%%%%%%%%%
%%%%%%%%%%%%%%%%%%%%%%%%%%%%%%%%%%%%%%%%%%%%%%%%%%%%%%%%%%%%%%%%%%%%%%%%%%%%%%%%
%%%%%%%%%%%%%%%%%%%%%%%%%%%%%%%%%%%%%%%%%%%%%%%%%%%%%%%%%%%%%%%%%%%%%%%%%%%%%%%%
%%%%%%%%%%%%%%%%%%%%%%%%%%%%%%%%%%%%%%%%%%%%%%%%%%%%%%%%%%%%%%%%%%%%%%%%%%%%%%%%
\begin{mdframed}[style=darkQuesion]
  
\end{mdframed}
%%%%%%%%%%%%%%%%%%%%%%%%%%%%%%%%%%%%%%%%%%%%%%%%%%%%%%%%%%%%%%%%%%%%%%%%%%%%%%%%
\begin{mdframed}[style=darkAnswer,frametitle={Joe Starr}]
%TODO Question not started
\end{mdframed}
\newpage
%%%%%%%%%%%%%%%%%%%%%%%%%%%%%%%%%%%%%%%%%%%%%%%%%%%%%%%%%%%%%%%%%%%%%%%%%%%%%%%%
%%%%%%%%%%%%%%%%%%%%%%%%%%%%%%%%%%%%%%%%%%%%%%%%%%%%%%%%%%%%%%%%%%%%%%%%%%%%%%%%
%%%%%%%%%%%%%%%%%%%%%%%%%%%%%%%%%%%%%%%%%%%%%%%%%%%%%%%%%%%%%%%%%%%%%%%%%%%%%%%%
%%%%%%%%%%%%%%%%%%%%%%%%%%%%%%%%%%%%%%%%%%%%%%%%%%%%%%%%%%%%%%%%%%%%%%%%%%%%%%%%
\begin{mdframed}[style=darkQuesion]
  
\end{mdframed}
%%%%%%%%%%%%%%%%%%%%%%%%%%%%%%%%%%%%%%%%%%%%%%%%%%%%%%%%%%%%%%%%%%%%%%%%%%%%%%%%
\begin{mdframed}[style=darkAnswer,frametitle={Joe Starr}]
%TODO Question not started
\end{mdframed}
\newpage
%%%%%%%%%%%%%%%%%%%%%%%%%%%%%%%%%%%%%%%%%%%%%%%%%%%%%%%%%%%%%%%%%%%%%%%%%%%%%%%%
%%%%%%%%%%%%%%%%%%%%%%%%%%%%%%%%%%%%%%%%%%%%%%%%%%%%%%%%%%%%%%%%%%%%%%%%%%%%%%%%
%%%%%%%%%%%%%%%%%%%%%%%%%%%%%%%%%%%%%%%%%%%%%%%%%%%%%%%%%%%%%%%%%%%%%%%%%%%%%%%%
%%%%%%%%%%%%%%%%%%%%%%%%%%%%%%%%%%%%%%%%%%%%%%%%%%%%%%%%%%%%%%%%%%%%%%%%%%%%%%%%
\begin{mdframed}[style=darkQuesion]
  
\end{mdframed}
%%%%%%%%%%%%%%%%%%%%%%%%%%%%%%%%%%%%%%%%%%%%%%%%%%%%%%%%%%%%%%%%%%%%%%%%%%%%%%%%
\begin{mdframed}[style=darkAnswer,frametitle={Joe Starr}]
%TODO Question not started
\end{mdframed}
\newpage
%%%%%%%%%%%%%%%%%%%%%%%%%%%%%%%%%%%%%%%%%%%%%%%%%%%%%%%%%%%%%%%%%%%%%%%%%%%%%%%%
%%%%%%%%%%%%%%%%%%%%%%%%%%%%%%%%%%%%%%%%%%%%%%%%%%%%%%%%%%%%%%%%%%%%%%%%%%%%%%%%
%%%%%%%%%%%%%%%%%%%%%%%%%%%%%%%%%%%%%%%%%%%%%%%%%%%%%%%%%%%%%%%%%%%%%%%%%%%%%%%%
%%%%%%%%%%%%%%%%%%%%%%%%%%%%%%%%%%%%%%%%%%%%%%%%%%%%%%%%%%%%%%%%%%%%%%%%%%%%%%%%
\begin{mdframed}[style=darkQuesion]
  
\end{mdframed}
%%%%%%%%%%%%%%%%%%%%%%%%%%%%%%%%%%%%%%%%%%%%%%%%%%%%%%%%%%%%%%%%%%%%%%%%%%%%%%%%
\begin{mdframed}[style=darkAnswer,frametitle={Joe Starr}]
%TODO Question not started
\end{mdframed}
\newpage
%%%%%%%%%%%%%%%%%%%%%%%%%%%%%%%%%%%%%%%%%%%%%%%%%%%%%%%%%%%%%%%%%%%%%%%%%%%%%%%%
%%%%%%%%%%%%%%%%%%%%%%%%%%%%%%%%%%%%%%%%%%%%%%%%%%%%%%%%%%%%%%%%%%%%%%%%%%%%%%%%
%%%%%%%%%%%%%%%%%%%%%%%%%%%%%%%%%%%%%%%%%%%%%%%%%%%%%%%%%%%%%%%%%%%%%%%%%%%%%%%%
%%%%%%%%%%%%%%%%%%%%%%%%%%%%%%%%%%%%%%%%%%%%%%%%%%%%%%%%%%%%%%%%%%%%%%%%%%%%%%%%
\begin{mdframed}[style=darkQuesion]
  
\end{mdframed}
%%%%%%%%%%%%%%%%%%%%%%%%%%%%%%%%%%%%%%%%%%%%%%%%%%%%%%%%%%%%%%%%%%%%%%%%%%%%%%%%
\begin{mdframed}[style=darkAnswer,frametitle={Joe Starr}]
%TODO Question not started
\end{mdframed}
\newpage
\clearpage
\subsection{Factors}
%%%%%%%%%%%%%%%%%%%%%%%%%%%%%%%%%%%%%%%%%%%%%%%%%%%%%%%%%%%%%%%%%%%%%%%%%%%%%%%%
%%%%%%%%%%%%%%%%%%%%%%%%%%%%%%%%%%%%%%%%%%%%%%%%%%%%%%%%%%%%%%%%%%%%%%%%%%%%%%%%
%%%%%%%%%%%%%%%%%%%%%%%%%%%%%%%%%%%%%%%%%%%%%%%%%%%%%%%%%%%%%%%%%%%%%%%%%%%%%%%%
%%%%%%%%%%%%%%%%%%%%%%%%%%%%%%%%%%%%%%%%%%%%%%%%%%%%%%%%%%%%%%%%%%%%%%%%%%%%%%%%
\begin{mdframed}[style=darkQuesion]
1. Use the division algorithm to find the quotient and remainder when $f(x)$ is divided by $g(x)$ over the field of rational numbers $\mathbf{Q}$.
\begin{enumerate}[(a)]
\item{$f(x)=2 x^{4}+5 x^{3}-7 x^{2}+4 x+8 \quad g(x)=2 x-1$}
\item{$f(x)=2 x^{7}-5 x^{6}+5 x^{5}-x^{3}-x^{2}+4 x-5 \quad g(x)=x^{2}-x+1$}
\item{$f(x)=x^{5}+1 \quad g(x)=x+1$}
\item{$f(x)=2 x^{4}+x^{3}-6 x^{2}-x+2 \quad g(x)=2 x^{2}-5$}
\end{enumerate}
\end{mdframed}
%%%%%%%%%%%%%%%%%%%%%%%%%%%%%%%%%%%%%%%%%%%%%%%%%%%%%%%%%%%%%%%%%%%%%%%%%%%%%%%%
\begin{mdframed}[style=darkAnswer,frametitle={Joe Starr}]
%TODO Question not started
\end{mdframed}
\newpage
%%%%%%%%%%%%%%%%%%%%%%%%%%%%%%%%%%%%%%%%%%%%%%%%%%%%%%%%%%%%%%%%%%%%%%%%%%%%%%%%
%%%%%%%%%%%%%%%%%%%%%%%%%%%%%%%%%%%%%%%%%%%%%%%%%%%%%%%%%%%%%%%%%%%%%%%%%%%%%%%%
%%%%%%%%%%%%%%%%%%%%%%%%%%%%%%%%%%%%%%%%%%%%%%%%%%%%%%%%%%%%%%%%%%%%%%%%%%%%%%%%
%%%%%%%%%%%%%%%%%%%%%%%%%%%%%%%%%%%%%%%%%%%%%%%%%%%%%%%%%%%%%%%%%%%%%%%%%%%%%%%%
\begin{mdframed}[style=darkQuesion]
  3. Find the greatest common divisor of $f(x)$ and $f^{\prime}(x),$ over $\mathbf{Q}$
  \begin{enumerate}[(a)]
  \item{$f(a) f(x)=x^{4}-x^{3}-x+1$}
  \item{$f(x)=x^{3}-3 x-2$}
  \item{$f(x)=x^{3}+2 x^{2}-x-2$}
  \item{$f(x)=x^{4}+2 x^{3}+3 x^{2}+2 x+1$}
  \end{enumerate} 
\end{mdframed}
%%%%%%%%%%%%%%%%%%%%%%%%%%%%%%%%%%%%%%%%%%%%%%%%%%%%%%%%%%%%%%%%%%%%%%%%%%%%%%%%
\begin{mdframed}[style=darkAnswer,frametitle={Joe Starr}]
%TODO Question not started
\end{mdframed}
\newpage
%%%%%%%%%%%%%%%%%%%%%%%%%%%%%%%%%%%%%%%%%%%%%%%%%%%%%%%%%%%%%%%%%%%%%%%%%%%%%%%%
%%%%%%%%%%%%%%%%%%%%%%%%%%%%%%%%%%%%%%%%%%%%%%%%%%%%%%%%%%%%%%%%%%%%%%%%%%%%%%%%
%%%%%%%%%%%%%%%%%%%%%%%%%%%%%%%%%%%%%%%%%%%%%%%%%%%%%%%%%%%%%%%%%%%%%%%%%%%%%%%%
%%%%%%%%%%%%%%%%%%%%%%%%%%%%%%%%%%%%%%%%%%%%%%%%%%%%%%%%%%%%%%%%%%%%%%%%%%%%%%%%
\begin{mdframed}[style=darkQuesion]
  9. Let $a \in \mathbf{R},$ and let $f(x) \in \mathbf{R}[x],$ with derivative $f^{\prime}(x) .$ Show that the remainder when $f(x)$ is divided by $(x-a)^{2}$ is $f^{\prime}(a)(x-a)+f(a)$
\end{mdframed}
%%%%%%%%%%%%%%%%%%%%%%%%%%%%%%%%%%%%%%%%%%%%%%%%%%%%%%%%%%%%%%%%%%%%%%%%%%%%%%%%
\begin{mdframed}[style=darkAnswer,frametitle={Joe Starr}]
%TODO Question not started
\end{mdframed}
\newpage
%%%%%%%%%%%%%%%%%%%%%%%%%%%%%%%%%%%%%%%%%%%%%%%%%%%%%%%%%%%%%%%%%%%%%%%%%%%%%%%%
%%%%%%%%%%%%%%%%%%%%%%%%%%%%%%%%%%%%%%%%%%%%%%%%%%%%%%%%%%%%%%%%%%%%%%%%%%%%%%%%
%%%%%%%%%%%%%%%%%%%%%%%%%%%%%%%%%%%%%%%%%%%%%%%%%%%%%%%%%%%%%%%%%%%%%%%%%%%%%%%%
%%%%%%%%%%%%%%%%%%%%%%%%%%%%%%%%%%%%%%%%%%%%%%%%%%%%%%%%%%%%%%%%%%%%%%%%%%%%%%%%
\begin{mdframed}[style=darkQuesion]
  17. Show that for any real number $a \neq 0,$ the polynomial $x^{n}-a$ has no multiple roots
  in R.
\end{mdframed}
%%%%%%%%%%%%%%%%%%%%%%%%%%%%%%%%%%%%%%%%%%%%%%%%%%%%%%%%%%%%%%%%%%%%%%%%%%%%%%%%
\begin{mdframed}[style=darkAnswer,frametitle={Joe Starr}]
%TODO Question not started
\end{mdframed}
\newpage
\clearpage
\subsection{Existence of Roots}
%%%%%%%%%%%%%%%%%%%%%%%%%%%%%%%%%%%%%%%%%%%%%%%%%%%%%%%%%%%%%%%%%%%%%%%%%%%%%%%%
%%%%%%%%%%%%%%%%%%%%%%%%%%%%%%%%%%%%%%%%%%%%%%%%%%%%%%%%%%%%%%%%%%%%%%%%%%%%%%%%
%%%%%%%%%%%%%%%%%%%%%%%%%%%%%%%%%%%%%%%%%%%%%%%%%%%%%%%%%%%%%%%%%%%%%%%%%%%%%%%%
%%%%%%%%%%%%%%%%%%%%%%%%%%%%%%%%%%%%%%%%%%%%%%%%%%%%%%%%%%%%%%%%%%%%%%%%%%%%%%%%
\begin{mdframed}[style=darkQuesion]
  1. Let $F$ be a field. Given $p(x) \in F[x],$ prove that congruence modulo $p(x)$ defines
an equivalence relation on $F[x] .$
\end{mdframed}
%%%%%%%%%%%%%%%%%%%%%%%%%%%%%%%%%%%%%%%%%%%%%%%%%%%%%%%%%%%%%%%%%%%%%%%%%%%%%%%%
\begin{mdframed}[style=darkAnswer,frametitle={Joe Starr}]
%TODO Question not started
\end{mdframed}
\newpage
%%%%%%%%%%%%%%%%%%%%%%%%%%%%%%%%%%%%%%%%%%%%%%%%%%%%%%%%%%%%%%%%%%%%%%%%%%%%%%%%
%%%%%%%%%%%%%%%%%%%%%%%%%%%%%%%%%%%%%%%%%%%%%%%%%%%%%%%%%%%%%%%%%%%%%%%%%%%%%%%%
%%%%%%%%%%%%%%%%%%%%%%%%%%%%%%%%%%%%%%%%%%%%%%%%%%%%%%%%%%%%%%%%%%%%%%%%%%%%%%%%
%%%%%%%%%%%%%%%%%%%%%%%%%%%%%%%%%%%%%%%%%%%%%%%%%%%%%%%%%%%%%%%%%%%%%%%%%%%%%%%%
\begin{mdframed}[style=darkQuesion]
  2. F Prove Proposition 4.3.4.
\end{mdframed}
%%%%%%%%%%%%%%%%%%%%%%%%%%%%%%%%%%%%%%%%%%%%%%%%%%%%%%%%%%%%%%%%%%%%%%%%%%%%%%%%
\begin{mdframed}[style=darkAnswer,frametitle={Joe Starr}]
%TODO Question not started
\end{mdframed}
\newpage
%%%%%%%%%%%%%%%%%%%%%%%%%%%%%%%%%%%%%%%%%%%%%%%%%%%%%%%%%%%%%%%%%%%%%%%%%%%%%%%%
%%%%%%%%%%%%%%%%%%%%%%%%%%%%%%%%%%%%%%%%%%%%%%%%%%%%%%%%%%%%%%%%%%%%%%%%%%%%%%%%
%%%%%%%%%%%%%%%%%%%%%%%%%%%%%%%%%%%%%%%%%%%%%%%%%%%%%%%%%%%%%%%%%%%%%%%%%%%%%%%%
%%%%%%%%%%%%%%%%%%%%%%%%%%%%%%%%%%%%%%%%%%%%%%%%%%%%%%%%%%%%%%%%%%%%%%%%%%%%%%%%
\begin{mdframed}[style=darkQuesion]
  3. Let $E$ be a field, and let $F$ be a subfield of $E .$ Prove that the multiplicative identity
  of $F$ must be the same as that of $E$
\end{mdframed}
%%%%%%%%%%%%%%%%%%%%%%%%%%%%%%%%%%%%%%%%%%%%%%%%%%%%%%%%%%%%%%%%%%%%%%%%%%%%%%%%
\begin{mdframed}[style=darkAnswer,frametitle={Joe Starr}]
%TODO Question not started
\end{mdframed}
\newpage
%%%%%%%%%%%%%%%%%%%%%%%%%%%%%%%%%%%%%%%%%%%%%%%%%%%%%%%%%%%%%%%%%%%%%%%%%%%%%%%%
%%%%%%%%%%%%%%%%%%%%%%%%%%%%%%%%%%%%%%%%%%%%%%%%%%%%%%%%%%%%%%%%%%%%%%%%%%%%%%%%
%%%%%%%%%%%%%%%%%%%%%%%%%%%%%%%%%%%%%%%%%%%%%%%%%%%%%%%%%%%%%%%%%%%%%%%%%%%%%%%%
%%%%%%%%%%%%%%%%%%%%%%%%%%%%%%%%%%%%%%%%%%%%%%%%%%%%%%%%%%%%%%%%%%%%%%%%%%%%%%%%
\begin{mdframed}[style=darkQuesion]
  12. Prove that $\mathbf{Q}[x] /\left\langle x^{2}-2\right\rangle$ is isomorphic to $\mathbf{Q}(\sqrt{2})=\{a+b \sqrt{2} | a, b \in \mathbf{Q}\},$ which
was shown to be a field in Example 4.1 .1
\end{mdframed}
%%%%%%%%%%%%%%%%%%%%%%%%%%%%%%%%%%%%%%%%%%%%%%%%%%%%%%%%%%%%%%%%%%%%%%%%%%%%%%%%
\begin{mdframed}[style=darkAnswer,frametitle={Joe Starr}]
%TODO Question not started
\end{mdframed}
\newpage
%%%%%%%%%%%%%%%%%%%%%%%%%%%%%%%%%%%%%%%%%%%%%%%%%%%%%%%%%%%%%%%%%%%%%%%%%%%%%%%%
%%%%%%%%%%%%%%%%%%%%%%%%%%%%%%%%%%%%%%%%%%%%%%%%%%%%%%%%%%%%%%%%%%%%%%%%%%%%%%%%
%%%%%%%%%%%%%%%%%%%%%%%%%%%%%%%%%%%%%%%%%%%%%%%%%%%%%%%%%%%%%%%%%%%%%%%%%%%%%%%%
%%%%%%%%%%%%%%%%%%%%%%%%%%%%%%%%%%%%%%%%%%%%%%%%%%%%%%%%%%%%%%%%%%%%%%%%%%%%%%%%
\begin{mdframed}[style=darkQuesion]
  25. For which values of $k=2,3,5,7,11$ is $\mathbf{Z}_{k}[x] /\left\langle x^{2}+1\right\rangle$ a field? Show your work.
\end{mdframed}
%%%%%%%%%%%%%%%%%%%%%%%%%%%%%%%%%%%%%%%%%%%%%%%%%%%%%%%%%%%%%%%%%%%%%%%%%%%%%%%%
\begin{mdframed}[style=darkAnswer,frametitle={Joe Starr}]
%TODO Question not started
\end{mdframed}
\newpage
\clearpage
\subsection{Polynomials over $\Z,\ \Q,\ \R,\ \text{and} \C$}
%%%%%%%%%%%%%%%%%%%%%%%%%%%%%%%%%%%%%%%%%%%%%%%%%%%%%%%%%%%%%%%%%%%%%%%%%%%%%%%%
%%%%%%%%%%%%%%%%%%%%%%%%%%%%%%%%%%%%%%%%%%%%%%%%%%%%%%%%%%%%%%%%%%%%%%%%%%%%%%%%
%%%%%%%%%%%%%%%%%%%%%%%%%%%%%%%%%%%%%%%%%%%%%%%%%%%%%%%%%%%%%%%%%%%%%%%%%%%%%%%%
%%%%%%%%%%%%%%%%%%%%%%%%%%%%%%%%%%%%%%%%%%%%%%%%%%%%%%%%%%%%%%%%%%%%%%%%%%%%%%%%
\begin{mdframed}[style=darkQuesion]
  1. let $f(x), g(x) \in \mathbf{Z}[x],$ and suppose that $g(x)$ is monic. Show that there exist unique polynomials $q(x), r(x) \in \mathbf{Z}[x]$ with $f(x)=q(x) g(x)+r(x),$ where either $\operatorname{deg}(r(x))<\operatorname{deg}(g(x))$ or $r(x)=0$
\end{mdframed}
%%%%%%%%%%%%%%%%%%%%%%%%%%%%%%%%%%%%%%%%%%%%%%%%%%%%%%%%%%%%%%%%%%%%%%%%%%%%%%%%
\begin{mdframed}[style=darkAnswer,frametitle={Joe Starr}]
%TODO Question not started
\end{mdframed}
\newpage
%%%%%%%%%%%%%%%%%%%%%%%%%%%%%%%%%%%%%%%%%%%%%%%%%%%%%%%%%%%%%%%%%%%%%%%%%%%%%%%%
%%%%%%%%%%%%%%%%%%%%%%%%%%%%%%%%%%%%%%%%%%%%%%%%%%%%%%%%%%%%%%%%%%%%%%%%%%%%%%%%
%%%%%%%%%%%%%%%%%%%%%%%%%%%%%%%%%%%%%%%%%%%%%%%%%%%%%%%%%%%%%%%%%%%%%%%%%%%%%%%%
%%%%%%%%%%%%%%%%%%%%%%%%%%%%%%%%%%%%%%%%%%%%%%%%%%%%%%%%%%%%%%%%%%%%%%%%%%%%%%%%
\begin{mdframed}[style=darkQuesion]
  6. Use Eisenstein's criterion to show that each of these polynomials is irreducible over the field of rational numbers. (You may need to make a substitution.)
  \begin{enumerate}[(a)]
\item{$\left.x^{4}+1 \quad \text { (substitute } x+1\right)$}
\item{$x^{6}+x^{3}+1 \quad$ (substitute $x+1$ )}
\item{$ x^{3}+3 x^{2}+5 x+5$}
\item{$x^{3}-3 x^{2}+9 x-10$}
\end{enumerate} 
\end{mdframed}
%%%%%%%%%%%%%%%%%%%%%%%%%%%%%%%%%%%%%%%%%%%%%%%%%%%%%%%%%%%%%%%%%%%%%%%%%%%%%%%%
\begin{mdframed}[style=darkAnswer,frametitle={Joe Starr}]
%TODO Question not started
\end{mdframed}
\newpage
%%%%%%%%%%%%%%%%%%%%%%%%%%%%%%%%%%%%%%%%%%%%%%%%%%%%%%%%%%%%%%%%%%%%%%%%%%%%%%%%
%%%%%%%%%%%%%%%%%%%%%%%%%%%%%%%%%%%%%%%%%%%%%%%%%%%%%%%%%%%%%%%%%%%%%%%%%%%%%%%%
%%%%%%%%%%%%%%%%%%%%%%%%%%%%%%%%%%%%%%%%%%%%%%%%%%%%%%%%%%%%%%%%%%%%%%%%%%%%%%%%
%%%%%%%%%%%%%%%%%%%%%%%%%%%%%%%%%%%%%%%%%%%%%%%%%%%%%%%%%%%%%%%%%%%%%%%%%%%%%%%%
\begin{mdframed}[style=darkQuesion]
  8. Let $f(x)=x^{2}+100 x+n$
  \begin{enumerate}[(a)]
\item{Give an infinite set of integers $n$ such that $f(x)$ is reducible over $\mathbf{Q}$}
\item{Give an infinite set of integers $n$ such that $f(x)$ is irreducible over $\mathbf{Q}$}
\end{enumerate} 
\end{mdframed}
%%%%%%%%%%%%%%%%%%%%%%%%%%%%%%%%%%%%%%%%%%%%%%%%%%%%%%%%%%%%%%%%%%%%%%%%%%%%%%%%
\begin{mdframed}[style=darkAnswer,frametitle={Joe Starr}]
%TODO Question not started
\end{mdframed}
\newpage
\clearpage
%%%%%%%%%%%%%%%%%%%%%%%%%%%%%%%%%%%%%%%%%%%%%%%%%%%%%%%%%%%%%%%%%%%%%%%%%%%%%%%%
%Compile Chapter 5
\section{Commutative Rings}
\subsection{Commutative Rings; Integral Domains}
%%%%%%%%%%%%%%%%%%%%%%%%%%%%%%%%%%%%%%%%%%%%%%%%%%%%%%%%%%%%%%%%%%%%%%%%%%%%%%%%
%%%%%%%%%%%%%%%%%%%%%%%%%%%%%%%%%%%%%%%%%%%%%%%%%%%%%%%%%%%%%%%%%%%%%%%%%%%%%%%%
%%%%%%%%%%%%%%%%%%%%%%%%%%%%%%%%%%%%%%%%%%%%%%%%%%%%%%%%%%%%%%%%%%%%%%%%%%%%%%%%
%%%%%%%%%%%%%%%%%%%%%%%%%%%%%%%%%%%%%%%%%%%%%%%%%%%%%%%%%%%%%%%%%%%%%%%%%%%%%%%%
\begin{mdframed}[style=darkQuesion]
  6. Show that no proper nontrivial subset of Z can form a ring under the usual operations of addition and multiplication.
\end{mdframed}
%%%%%%%%%%%%%%%%%%%%%%%%%%%%%%%%%%%%%%%%%%%%%%%%%%%%%%%%%%%%%%%%%%%%%%%%%%%%%%%%
\begin{mdframed}[style=darkAnswer,frametitle={Joe Starr}]
%TODO Question not started
\end{mdframed}
\newpage
%%%%%%%%%%%%%%%%%%%%%%%%%%%%%%%%%%%%%%%%%%%%%%%%%%%%%%%%%%%%%%%%%%%%%%%%%%%%%%%%
%%%%%%%%%%%%%%%%%%%%%%%%%%%%%%%%%%%%%%%%%%%%%%%%%%%%%%%%%%%%%%%%%%%%%%%%%%%%%%%%
%%%%%%%%%%%%%%%%%%%%%%%%%%%%%%%%%%%%%%%%%%%%%%%%%%%%%%%%%%%%%%%%%%%%%%%%%%%%%%%%
%%%%%%%%%%%%%%%%%%%%%%%%%%%%%%%%%%%%%%%%%%%%%%%%%%%%%%%%%%%%%%%%%%%%%%%%%%%%%%%%
\begin{mdframed}[style=darkQuesion]
  11. Let $R$ be a commutative ring such that $a^{2}=a$ for all $a \in R .$ Show that $a+a=0$ for all $a \in R$
\end{mdframed}
%%%%%%%%%%%%%%%%%%%%%%%%%%%%%%%%%%%%%%%%%%%%%%%%%%%%%%%%%%%%%%%%%%%%%%%%%%%%%%%%
\begin{mdframed}[style=darkAnswer,frametitle={Joe Starr}]
%TODO Question not started
\end{mdframed}
\newpage
%%%%%%%%%%%%%%%%%%%%%%%%%%%%%%%%%%%%%%%%%%%%%%%%%%%%%%%%%%%%%%%%%%%%%%%%%%%%%%%%
%%%%%%%%%%%%%%%%%%%%%%%%%%%%%%%%%%%%%%%%%%%%%%%%%%%%%%%%%%%%%%%%%%%%%%%%%%%%%%%%
%%%%%%%%%%%%%%%%%%%%%%%%%%%%%%%%%%%%%%%%%%%%%%%%%%%%%%%%%%%%%%%%%%%%%%%%%%%%%%%%
%%%%%%%%%%%%%%%%%%%%%%%%%%%%%%%%%%%%%%%%%%%%%%%%%%%%%%%%%%%%%%%%%%%%%%%%%%%%%%%%
\begin{mdframed}[style=darkQuesion]
  15. Let $I$ be any set and let $R$ be the collection of all subsets of $I$. Define addition and multiplication of subsets $A, B \subseteq I$ as follows:
$$
A+B=A \cup B \quad \text { and } \quad A \cdot B=A \cap B
$$
Is $R$ a commutative ring under this addition and multiplication?
\end{mdframed}
%%%%%%%%%%%%%%%%%%%%%%%%%%%%%%%%%%%%%%%%%%%%%%%%%%%%%%%%%%%%%%%%%%%%%%%%%%%%%%%%
\begin{mdframed}[style=darkAnswer,frametitle={Joe Starr}]
%TODO Question not started
\end{mdframed}
\newpage
%%%%%%%%%%%%%%%%%%%%%%%%%%%%%%%%%%%%%%%%%%%%%%%%%%%%%%%%%%%%%%%%%%%%%%%%%%%%%%%%
%%%%%%%%%%%%%%%%%%%%%%%%%%%%%%%%%%%%%%%%%%%%%%%%%%%%%%%%%%%%%%%%%%%%%%%%%%%%%%%%
%%%%%%%%%%%%%%%%%%%%%%%%%%%%%%%%%%%%%%%%%%%%%%%%%%%%%%%%%%%%%%%%%%%%%%%%%%%%%%%%
%%%%%%%%%%%%%%%%%%%%%%%%%%%%%%%%%%%%%%%%%%%%%%%%%%%%%%%%%%%%%%%%%%%%%%%%%%%%%%%%
\begin{mdframed}[style=darkQuesion]
  20. Give addition and multiplication tables for $\mathbf{Z}_{2} \oplus \mathbf{Z}_{2}$
\end{mdframed}
%%%%%%%%%%%%%%%%%%%%%%%%%%%%%%%%%%%%%%%%%%%%%%%%%%%%%%%%%%%%%%%%%%%%%%%%%%%%%%%%
\begin{mdframed}[style=darkAnswer,frametitle={Joe Starr}]
  %TODO Question not started
\end{mdframed}
\newpage
%%%%%%%%%%%%%%%%%%%%%%%%%%%%%%%%%%%%%%%%%%%%%%%%%%%%%%%%%%%%%%%%%%%%%%%%%%%%%%%%
%%%%%%%%%%%%%%%%%%%%%%%%%%%%%%%%%%%%%%%%%%%%%%%%%%%%%%%%%%%%%%%%%%%%%%%%%%%%%%%%
%%%%%%%%%%%%%%%%%%%%%%%%%%%%%%%%%%%%%%%%%%%%%%%%%%%%%%%%%%%%%%%%%%%%%%%%%%%%%%%%
%%%%%%%%%%%%%%%%%%%%%%%%%%%%%%%%%%%%%%%%%%%%%%%%%%%%%%%%%%%%%%%%%%%%%%%%%%%%%%%%
\begin{mdframed}[style=darkQuesion]
  22. Find all units of the following rings.
   \begin{enumerate}[(a)]
  \item{$\mathbf{Z} \oplus \mathbf{Z}$}
  \item{$\mathbf{Z}_{4} \oplus \mathbf{Z}_{9}$}
  \end{enumerate} 
  
\end{mdframed}
%%%%%%%%%%%%%%%%%%%%%%%%%%%%%%%%%%%%%%%%%%%%%%%%%%%%%%%%%%%%%%%%%%%%%%%%%%%%%%%%
\begin{mdframed}[style=darkAnswer,frametitle={Joe Starr}]
%TODO Question not started
\end{mdframed}
\newpage
\clearpage
\subsection{Ring Homomorphisms}
%%%%%%%%%%%%%%%%%%%%%%%%%%%%%%%%%%%%%%%%%%%%%%%%%%%%%%%%%%%%%%%%%%%%%%%%%%%%%%%%
%%%%%%%%%%%%%%%%%%%%%%%%%%%%%%%%%%%%%%%%%%%%%%%%%%%%%%%%%%%%%%%%%%%%%%%%%%%%%%%%
%%%%%%%%%%%%%%%%%%%%%%%%%%%%%%%%%%%%%%%%%%%%%%%%%%%%%%%%%%%%%%%%%%%%%%%%%%%%%%%%
%%%%%%%%%%%%%%%%%%%%%%%%%%%%%%%%%%%%%%%%%%%%%%%%%%%%%%%%%%%%%%%%%%%%%%%%%%%%%%%%
\begin{mdframed}[style=darkQuesion]
  1. Let $R$ be a commutative ring, and let $D$ be an integral domain. Let $\phi: R \rightarrow D$ be a nonzero function such that $\phi(a+b)=\phi(a)+\phi(b)$ and $\phi(a b)=\phi(a) \phi(b),$ for
all $a, b \in R .$ Show that $\phi$ is a ring homomorphism.
\end{mdframed}
%%%%%%%%%%%%%%%%%%%%%%%%%%%%%%%%%%%%%%%%%%%%%%%%%%%%%%%%%%%%%%%%%%%%%%%%%%%%%%%%
\begin{mdframed}[style=darkAnswer,frametitle={Joe Starr}]
%TODO Question not started
\end{mdframed}
\newpage
%%%%%%%%%%%%%%%%%%%%%%%%%%%%%%%%%%%%%%%%%%%%%%%%%%%%%%%%%%%%%%%%%%%%%%%%%%%%%%%%
%%%%%%%%%%%%%%%%%%%%%%%%%%%%%%%%%%%%%%%%%%%%%%%%%%%%%%%%%%%%%%%%%%%%%%%%%%%%%%%%
%%%%%%%%%%%%%%%%%%%%%%%%%%%%%%%%%%%%%%%%%%%%%%%%%%%%%%%%%%%%%%%%%%%%%%%%%%%%%%%%
%%%%%%%%%%%%%%%%%%%%%%%%%%%%%%%%%%%%%%%%%%%%%%%%%%%%%%%%%%%%%%%%%%%%%%%%%%%%%%%%
\begin{mdframed}[style=darkQuesion]
  4. Show that taking complex conjugates defines an automorphism of $\mathbf{C .}$ That is, for $z \in \mathbf{C},$ define $\phi(z)=\bar{z},$ and show that $\phi$ is an automorphism of $\mathbf{C}$
\end{mdframed}
%%%%%%%%%%%%%%%%%%%%%%%%%%%%%%%%%%%%%%%%%%%%%%%%%%%%%%%%%%%%%%%%%%%%%%%%%%%%%%%%
\begin{mdframed}[style=darkAnswer,frametitle={Joe Starr}]
%TODO Question not started
\end{mdframed}
\newpage
%%%%%%%%%%%%%%%%%%%%%%%%%%%%%%%%%%%%%%%%%%%%%%%%%%%%%%%%%%%%%%%%%%%%%%%%%%%%%%%%
%%%%%%%%%%%%%%%%%%%%%%%%%%%%%%%%%%%%%%%%%%%%%%%%%%%%%%%%%%%%%%%%%%%%%%%%%%%%%%%%
%%%%%%%%%%%%%%%%%%%%%%%%%%%%%%%%%%%%%%%%%%%%%%%%%%%%%%%%%%%%%%%%%%%%%%%%%%%%%%%%
%%%%%%%%%%%%%%%%%%%%%%%%%%%%%%%%%%%%%%%%%%%%%%%%%%%%%%%%%%%%%%%%%%%%%%%%%%%%%%%%
\begin{mdframed}[style=darkQuesion]
  9. Define $\phi: \mathbf{Z}[\sqrt{2}] \rightarrow \mathbf{Z}[\sqrt{2}]$ by $\phi(m+n \sqrt{2})=m-n \sqrt{2},$ for all $m, n \in \mathbf{Z} .$ Show
that $\phi$ is an automorphism of $\mathbf{Z}[\sqrt{2}]$
\end{mdframed}
%%%%%%%%%%%%%%%%%%%%%%%%%%%%%%%%%%%%%%%%%%%%%%%%%%%%%%%%%%%%%%%%%%%%%%%%%%%%%%%%
\begin{mdframed}[style=darkAnswer,frametitle={Joe Starr}]
%TODO Question not started
\end{mdframed}
\newpage
%%%%%%%%%%%%%%%%%%%%%%%%%%%%%%%%%%%%%%%%%%%%%%%%%%%%%%%%%%%%%%%%%%%%%%%%%%%%%%%%
%%%%%%%%%%%%%%%%%%%%%%%%%%%%%%%%%%%%%%%%%%%%%%%%%%%%%%%%%%%%%%%%%%%%%%%%%%%%%%%%
%%%%%%%%%%%%%%%%%%%%%%%%%%%%%%%%%%%%%%%%%%%%%%%%%%%%%%%%%%%%%%%%%%%%%%%%%%%%%%%%
%%%%%%%%%%%%%%%%%%%%%%%%%%%%%%%%%%%%%%%%%%%%%%%%%%%%%%%%%%%%%%%%%%%%%%%%%%%%%%%%
\begin{mdframed}[style=darkQuesion]
  21. Are $\Zm{9}$ and $\Zm{3}\bigoplus\Zm{3}$ isomorphic as rings?
\end{mdframed}
%%%%%%%%%%%%%%%%%%%%%%%%%%%%%%%%%%%%%%%%%%%%%%%%%%%%%%%%%%%%%%%%%%%%%%%%%%%%%%%%
\begin{mdframed}[style=darkAnswer,frametitle={Joe Starr}]
%TODO Question not started
\end{mdframed}
\newpage
\clearpage
\subsection{Ideals and Factor Rings}
%%%%%%%%%%%%%%%%%%%%%%%%%%%%%%%%%%%%%%%%%%%%%%%%%%%%%%%%%%%%%%%%%%%%%%%%%%%%%%%%
%%%%%%%%%%%%%%%%%%%%%%%%%%%%%%%%%%%%%%%%%%%%%%%%%%%%%%%%%%%%%%%%%%%%%%%%%%%%%%%%
%%%%%%%%%%%%%%%%%%%%%%%%%%%%%%%%%%%%%%%%%%%%%%%%%%%%%%%%%%%%%%%%%%%%%%%%%%%%%%%%
%%%%%%%%%%%%%%%%%%%%%%%%%%%%%%%%%%%%%%%%%%%%%%%%%%%%%%%%%%%%%%%%%%%%%%%%%%%%%%%%
\begin{mdframed}[style=darkQuesion]
  1. Give a multiplication table for the ring $\bigslant{\Zm{2}\lrb{x}}{\lra{x^2+1}}$
\end{mdframed}
%%%%%%%%%%%%%%%%%%%%%%%%%%%%%%%%%%%%%%%%%%%%%%%%%%%%%%%%%%%%%%%%%%%%%%%%%%%%%%%%
\begin{mdframed}[style=darkAnswer,frametitle={Joe Starr}]
%TODO Question not started
\end{mdframed}
\newpage
%%%%%%%%%%%%%%%%%%%%%%%%%%%%%%%%%%%%%%%%%%%%%%%%%%%%%%%%%%%%%%%%%%%%%%%%%%%%%%%%
%%%%%%%%%%%%%%%%%%%%%%%%%%%%%%%%%%%%%%%%%%%%%%%%%%%%%%%%%%%%%%%%%%%%%%%%%%%%%%%%
%%%%%%%%%%%%%%%%%%%%%%%%%%%%%%%%%%%%%%%%%%%%%%%%%%%%%%%%%%%%%%%%%%%%%%%%%%%%%%%%
%%%%%%%%%%%%%%%%%%%%%%%%%%%%%%%%%%%%%%%%%%%%%%%%%%%%%%%%%%%%%%%%%%%%%%%%%%%%%%%%
\begin{mdframed}[style=darkQuesion]
 4. Give the multiplication table for the ring $\bigslant{\Zm{3}\lrb{x}}{\lra{x^2-1}}$
\end{mdframed}
%%%%%%%%%%%%%%%%%%%%%%%%%%%%%%%%%%%%%%%%%%%%%%%%%%%%%%%%%%%%%%%%%%%%%%%%%%%%%%%%
\begin{mdframed}[style=darkAnswer,frametitle={Joe Starr}]
%TODO Question not started
\end{mdframed}
\newpage
%%%%%%%%%%%%%%%%%%%%%%%%%%%%%%%%%%%%%%%%%%%%%%%%%%%%%%%%%%%%%%%%%%%%%%%%%%%%%%%%
%%%%%%%%%%%%%%%%%%%%%%%%%%%%%%%%%%%%%%%%%%%%%%%%%%%%%%%%%%%%%%%%%%%%%%%%%%%%%%%%
%%%%%%%%%%%%%%%%%%%%%%%%%%%%%%%%%%%%%%%%%%%%%%%%%%%%%%%%%%%%%%%%%%%%%%%%%%%%%%%%
%%%%%%%%%%%%%%%%%%%%%%%%%%%%%%%%%%%%%%%%%%%%%%%%%%%%%%%%%%%%%%%%%%%%%%%%%%%%%%%%
\begin{mdframed}[style=darkQuesion]
 10. Show that if $R$ is a finite ring, then every prime ideal of $R$ is maximal.
\end{mdframed}
%%%%%%%%%%%%%%%%%%%%%%%%%%%%%%%%%%%%%%%%%%%%%%%%%%%%%%%%%%%%%%%%%%%%%%%%%%%%%%%%
\begin{mdframed}[style=darkAnswer,frametitle={Joe Starr}]
%TODO Question not started
\end{mdframed}
\newpage
%%%%%%%%%%%%%%%%%%%%%%%%%%%%%%%%%%%%%%%%%%%%%%%%%%%%%%%%%%%%%%%%%%%%%%%%%%%%%%%%
%%%%%%%%%%%%%%%%%%%%%%%%%%%%%%%%%%%%%%%%%%%%%%%%%%%%%%%%%%%%%%%%%%%%%%%%%%%%%%%%
%%%%%%%%%%%%%%%%%%%%%%%%%%%%%%%%%%%%%%%%%%%%%%%%%%%%%%%%%%%%%%%%%%%%%%%%%%%%%%%%
%%%%%%%%%%%%%%%%%%%%%%%%%%%%%%%%%%%%%%%%%%%%%%%%%%%%%%%%%%%%%%%%%%%%%%%%%%%%%%%%
\begin{mdframed}[style=darkQuesion]
  16. Let $R$ be a commutative ring with ideals $I, J .$ Let
$$
I+J=\{x \in R | x=a+b \text { for some } a \in I, b \in J\}
$$
  \begin{enumerate}[(a)]
  \item{Show that $I+J$ is an ideal.}
\item{Determine $n \mathbf{Z}+m \mathbf{Z}$ in the ring of integers.}
\end{enumerate} 
\end{mdframed}
%%%%%%%%%%%%%%%%%%%%%%%%%%%%%%%%%%%%%%%%%%%%%%%%%%%%%%%%%%%%%%%%%%%%%%%%%%%%%%%%
\begin{mdframed}[style=darkAnswer,frametitle={Joe Starr}]
%TODO Question not started
\end{mdframed}
\newpage
\clearpage
\subsection{Quotient Fields}
%%%%%%%%%%%%%%%%%%%%%%%%%%%%%%%%%%%%%%%%%%%%%%%%%%%%%%%%%%%%%%%%%%%%%%%%%%%%%%%%
%%%%%%%%%%%%%%%%%%%%%%%%%%%%%%%%%%%%%%%%%%%%%%%%%%%%%%%%%%%%%%%%%%%%%%%%%%%%%%%%
%%%%%%%%%%%%%%%%%%%%%%%%%%%%%%%%%%%%%%%%%%%%%%%%%%%%%%%%%%%%%%%%%%%%%%%%%%%%%%%%
%%%%%%%%%%%%%%%%%%%%%%%%%%%%%%%%%%%%%%%%%%%%%%%%%%%%%%%%%%%%%%%%%%%%%%%%%%%%%%%%
\begin{mdframed}[style=darkQuesion]
  
\end{mdframed}
%%%%%%%%%%%%%%%%%%%%%%%%%%%%%%%%%%%%%%%%%%%%%%%%%%%%%%%%%%%%%%%%%%%%%%%%%%%%%%%%
\begin{mdframed}[style=darkAnswer,frametitle={Joe Starr}]
%TODO Question not started
\end{mdframed}
\newpage
%%%%%%%%%%%%%%%%%%%%%%%%%%%%%%%%%%%%%%%%%%%%%%%%%%%%%%%%%%%%%%%%%%%%%%%%%%%%%%%%
%%%%%%%%%%%%%%%%%%%%%%%%%%%%%%%%%%%%%%%%%%%%%%%%%%%%%%%%%%%%%%%%%%%%%%%%%%%%%%%%
%%%%%%%%%%%%%%%%%%%%%%%%%%%%%%%%%%%%%%%%%%%%%%%%%%%%%%%%%%%%%%%%%%%%%%%%%%%%%%%%
%%%%%%%%%%%%%%%%%%%%%%%%%%%%%%%%%%%%%%%%%%%%%%%%%%%%%%%%%%%%%%%%%%%%%%%%%%%%%%%%
\begin{mdframed}[style=darkQuesion]
  
\end{mdframed}
%%%%%%%%%%%%%%%%%%%%%%%%%%%%%%%%%%%%%%%%%%%%%%%%%%%%%%%%%%%%%%%%%%%%%%%%%%%%%%%%
\begin{mdframed}[style=darkAnswer,frametitle={Joe Starr}]
%TODO Question not started
\end{mdframed}
\newpage
%%%%%%%%%%%%%%%%%%%%%%%%%%%%%%%%%%%%%%%%%%%%%%%%%%%%%%%%%%%%%%%%%%%%%%%%%%%%%%%%
%%%%%%%%%%%%%%%%%%%%%%%%%%%%%%%%%%%%%%%%%%%%%%%%%%%%%%%%%%%%%%%%%%%%%%%%%%%%%%%%
%%%%%%%%%%%%%%%%%%%%%%%%%%%%%%%%%%%%%%%%%%%%%%%%%%%%%%%%%%%%%%%%%%%%%%%%%%%%%%%%
%%%%%%%%%%%%%%%%%%%%%%%%%%%%%%%%%%%%%%%%%%%%%%%%%%%%%%%%%%%%%%%%%%%%%%%%%%%%%%%%
\begin{mdframed}[style=darkQuesion]
  
\end{mdframed}
%%%%%%%%%%%%%%%%%%%%%%%%%%%%%%%%%%%%%%%%%%%%%%%%%%%%%%%%%%%%%%%%%%%%%%%%%%%%%%%%
\begin{mdframed}[style=darkAnswer,frametitle={Joe Starr}]
%TODO Question not started
\end{mdframed}
\newpage
\clearpage
%%%%%%%%%%%%%%%%%%%%%%%%%%%%%%%%%%%%%%%%%%%%%%%%%%%%%%%%%%%%%%%%%%%%%%%%%%%%%%%%
%Compile Chapter 6
\section{Section}  

I <3 my Wayne State Libraries! Do you? 

\subsection{A Subsection}

\clearpage
%%%%%%%%%%%%%%%%%%%%%%%%%%%%%%%%%%%%%%%%%%%%%%%%%%%%%%%%%%%%%%%%%%%%%%%%%%%%%%%%
%Compile Chapter 7
\section{Section}  

I <3 my Wayne State Libraries! Do you? 

\subsection{A Subsection}

\clearpage
%%%%%%%%%%%%%%%%%%%%%%%%%%%%%%%%%%%%%%%%%%%%%%%%%%%%%%%%%%%%%%%%%%%%%%%%%%%%%%%%
%Compile Chapter 8
\section{Section}  

I <3 my Wayne State Libraries! Do you? 

\subsection{A Subsection}

\clearpage
%%%%%%%%%%%%%%%%%%%%%%%%%%%%%%%%%%%%%%%%%%%%%%%%%%%%%%%%%%%%%%%%%%%%%%%%%%%%%%%%
%Compile Chapter 9
\section{Section}  

I <3 my Wayne State Libraries! Do you? 

\subsection{A Subsection}

\clearpage
%%%%%%%%%%%%%%%%%%%%%%%%%%%%%%%%%%%%%%%%%%%%%%%%%%%%%%%%%%%%%%%%%%%%%%%%%%%%%%%%
%Compile Chapter 10
\section{Section}  

I <3 my Wayne State Libraries! Do you? 

\subsection{A Subsection}

\clearpage
%%%%%%%%%%%%%%%%%%%%%%%%%%%%%%%%%%%%%%%%%%%%%%%%%%%%%%%%%%%%%%%%%%%%%%%%%%%%%%%%
%Compile Appendix
\section{General Proofs}  




\clearpage
\end{document}