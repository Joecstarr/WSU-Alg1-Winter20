%Select document class: 12 point, article
\documentclass[12pt]{article}

% This template is a combination of work done by Mike Catanzaro and Gabe Angelini-Knoll, both formerly of the WSU math dept, with some additions and synthesizations by Clayton Hayes (clayton.hayes@wayne.edu) for broader use
% Last updated 2018-03-10 by Aaron Willcock (ez9213@wayne.edu)

%AMS = American Mathematical Society
%math, symbols, theorems, fonts
\usepackage{amsmath, amssymb, amsthm, amsfonts}
\usepackage{times, flexisym, mdframed, xcolor}
\usepackage{ulem,multicol}


%%%%%%%%%%%%%%%%%%%%%%%%%%%%%%%%%%%%%%%%%%%%%%%%%%%%%%%%%%%%%%%%%%%%%%%%%%%%%%%%
%Commands definitions
\definecolor{backgroundcolorDef}{rgb}{0.1,0.1,0.12}
\definecolor{textcolorDef}{rgb}{0.77,0.77,0.77}
\definecolor{waynegreen}{RGB}{0,89,79}
\newcommand{\setbackgroundcolour}{\pagecolor{backgroundcolorDef}}  
\newcommand{\settextcolour}{\color{textcolorDef}}   
\newcommand{\invertbackgroundtext}{\setbackgroundcolour\settextcolour}
\mdfdefinestyle{darkAnswer}{%
    fontcolor=textcolorDef,
    backgroundcolor=backgroundcolorDef,
    linecolor=textcolorDef,
    % ----------------------------------
    frametitlebackgroundcolor=backgroundcolorDef,
    frametitlefontcolor=textcolorDef,
    frametitlerulecolor=waynegreen,
    frametitlerule=true, 
    frametitlerulewidth=2pt
    }
\mdfdefinestyle{darkQuesion}{%
    fontcolor=textcolorDef,
    backgroundcolor=backgroundcolorDef,
    linecolor=textcolorDef,
    linewidth=1pt
    }

%If this line is commented, then the appearance remains as usual.
\invertbackgroundtext
\newcommand{\lrb}[1]{\left[#1\right]}
\newcommand{\lrp}[1]{\left(#1\right)}
\newcommand{\lrs}[1]{\left\{#1\right\}}
\newcommand{\gof}[1]{g\lrp{#1}}
\newcommand{\fof}[1]{f\lrp{#1}}
\newcommand{\kof}[1]{k\lrp{#1}}
\newcommand{\hof}[1]{h\lrp{#1}}
\newcommand{\pof}[1]{\phi\lrp{#1}}
\newcommand{\Zof}[1]{Z\lrp{#1}}
\newcommand{\gofinv}[1]{g^{-1}\lrp{#1}}
\newcommand{\fofinv}[1]{f^{-1}\lrp{#1}}
\newcommand{\Zofinv}[1]{Z^{-1}\lrp{#1}}
\newcommand{\nmod}[2]{#1\,\left(\text{mod}\,#2\right)}
\newcommand{\ngcd}[2]{\text{gcd}\left(#1\,,\,#2\right)}
\newcommand{\grp}[2]{\left(#1\,,\,#2\right)}
\newcommand{\GL}[2]{GL_{#1}\left(#2\right)}
\newcommand{\nord}[2]{\text{ord}_{#1}\left(#2\right)}
\newcommand{\dom}[1]{\text{dom}\left(#1\right)}
\newcommand{\ran}[1]{\text{ran}\left(#1\right)}
\newcommand{\degr}[1]{\text{deg}\left(#1\right)}
\newcommand{\N}{\mathbb{N}}
\newcommand{\Z}{\mathbb{Z}}
\newcommand{\Q}{\mathbb{Q}}
\newcommand{\R}{\mathbb{R}}
\newcommand{\C}{\mathbb{C}}
\newcommand{\abs}[1]{\left\vert#1\right\vert}
\newcommand{\Zm}[1]{\mathbb{Z}_{#1}}
\newcommand{\Zp}{\mathbb{Z}_p}
\newcommand{\numb}[1]{\noindent{\bf #1)}}
\newcommand{\inv}[1]{#1^{-1}}
\newcommand{\totient}[1]{\phi\left(#1\right)} 
\newcommand{\vhalfpg}{\vspace{5in}}
\newcommand{\vthirdpg}{\vspace{3in}}
\newcommand{\vquartpg}{\vspace{2in}}
\newcommand{\Assoc}[1]{\item[Associativity:]{#1}}
\newcommand{\Invs}[1]{\item[Inverses:]{#1}}
\newcommand{\Clos}[1]{\item[Closure:]{#1}}
\newcommand{\Ident}[1]{\item[Identity:]{#1}}
\newcommand{\twocase}[2]{\begin{enumerate}
        \item[$ \implies$]{
            #1
        }
        \bigskip
        \item[$ \impliedby$]{
            #2
        }
    \end{enumerate}}
\newcommand{\msout}[1]{\text{\sout{\ensuremath{#1}}}}

%%%%%%%%%%%%%%%%%%%%%%%%%%%%%%%%%%%%%%%%%%%%%%%%%%%%%%%%%%%%%%%%%%%%%%%%%%%%%%%%
%Enhanced graphics support (https://ctan.org/pkg/graphicx?lang=en)
\usepackage{graphicx}

%Set default graphics path (replace 'figures/' with whatever directory your images are in)
%\graphicspath{{figures/}}


%Control layout of itemize, enumerate, description (https://ctan.org/pkg/enumitem?lang=en)
\usepackage{enumitem}

% Header and footer formatting options
\usepackage{fancyhdr}

%Control float placement. [section] = "stop floats at section boundaries is to change the definition of “\section” to include “\FloatBarrier”"
\usepackage[section]{placeins}

%Hypertext (links) in LaTeX. [option] = remove color and border on links.
\usepackage[hidelinks]{hyperref}

%\usepackage[all]{xy}
%\usepackage{mathtools}

%Pro­gram­ming fa­cil­i­ties for LaTeX class and pack­age authors
\usepackage{etoolbox}

%Indent first paragraph of every 'chapter' aka section
\usepackage{indentfirst}

%Title formatting option. [explicit] = make titles all caps
\usepackage[explicit]{titlesec}

%Standard package for selecting font encodings. [T1] = Support for accented characters
%(https://texfaq.org/FAQ-why-inp-font) - 
%(https://tex.stackexchange.com/questions/664/why-should-i-use-usepackaget1fontenc)
\usepackage[T1]{fontenc}

%Charter fonts
\usepackage{charter}

%\usepackage[expert]{mathdesign}

%Control table of contents, figures, etc
\usepackage{tocloft}

%Set space between lines. [option] = Double/single spacing necessary for properly formatting ToC/LoFT and titles.
\usepackage[singlespacing]{setspace}

%Create normal/logarithmic plots in two and three dimensions
\usepackage{pgfplots}


% Remove extra space above and below theorems, lemmas, props, etc.
%The important point in the following is: 0pt preskip and 0pt postskip
\makeatletter
\def\thm@space@setup{\thm@preskip=0pt
\thm@postskip=0pt}
\makeatother
\newtheoremstyle{newstyle}      
{} % Aboveskip 
{} % Below skip
{\mdseries} % Body font e.g.\mdseries,\bfseries,\scshape,\itshape
{} % Indent
{\bfseries}  % Head font e.g.\bfseries,\scshape,\itshape
{.} % Punctuation afer theorem header
{ } % Space after theorem header
{} % Heading

%The above does not fix the spacing around proof environments.
%Use the following to fix.
%The crucial point is "topsep0\p@", i.e., topsep = 0 pt.
%The rest is essentially copied from the standard AMS environment.
\makeatletter
\renewenvironment{proof}[1][\proofname]{\par
  \pushQED{\qed}%
  \normalfont \topsep0\p@\relax
  \trivlist
  \item[\hskip\labelsep\itshape
  #1\@addpunct{.}]\ignorespaces 
}{
  \popQED\endtrivlist\@endpefalse
}
\makeatother


%Formatting
%   Margins
\usepackage[left=1in,right=1in, top=1in, bottom=1in]{geometry}
%
\newcommand{\nd}{\noindent}
\def\para#1{\vskip 0.4\baselineskip\noindent{\bf #1}}
\newcommand{\Vspc}{\vspace*{0.1in}}



\usepackage{etoolbox}
\newcommand{\zerodisplayskips}{%
  \setlength{\abovedisplayskip}{1pt}%
  \setlength{\belowdisplayskip}{1pt}%
  \setlength{\abovedisplayshortskip}{1pt}%
  \setlength{\belowdisplayshortskip}{1pt}}
\appto{\footnotesize}{\zerodisplayskips}
\appto{\tiny}{\zerodisplayskips}
\appto{\scriptsize}{\zerodisplayskips}
\appto{\footnotesize}{\zerodisplayskips}
\appto{\small}{\zerodisplayskips}
\appto{\large}{\zerodisplayskips}
\appto{\Large}{\zerodisplayskips}
\appto{\LARGE}{\zerodisplayskips}
\appto{\huge}{\zerodisplayskips}
\appto{\Huge}{\zerodisplayskips}
\setlength{\columnseprule}{0.2pt}
%Enter single spacing environment for toc, lot, and lof (see below)
\begin{document}

%Compile the title and jump to new page
%Create title (centered and bold font)
\centerline{\bf Algebra I}
\vspace{-0.4cm}
\centerline{\bf Winter 2020}

\begin{figure}[!htbp]
    \centering
    \includegraphics[width=0.25\linewidth]{fig/wsu_primary_stacked_color.pdf}
\end{figure}

\thispagestyle{empty}
\newpage
\tableofcontents
\newpage
\setcounter{page}{3}
%Compile Chapter 1
\section{Integers}  
\subsection{Divisors}
%%%%%%%%%%%%%%%%%%%%%%%%%%%%%%%%%%%%%%%%%%%%%%%%%%%%%%%%%%%%%%%%%%%%%%%%%%%%%%%%
%%%%%%%%%%%%%%%%%%%%%%%%%%%%%%%%%%%%%%%%%%%%%%%%%%%%%%%%%%%%%%%%%%%%%%%%%%%%%%%%
%%%%%%%%%%%%%%%%%%%%%%%%%%%%%%%%%%%%%%%%%%%%%%%%%%%%%%%%%%%%%%%%%%%%%%%%%%%%%%%%
%%%%%%%%%%%%%%%%%%%%%%%%%%%%%%%%%%%%%%%%%%%%%%%%%%%%%%%%%%%%%%%%%%%%%%%%%%%%%%%%
\begin{mdframed}[style=darkQuesion]
1.    Let $m,n.r.s\in \Z$. If $m^2+n^2=r^2+s^2=mr+ns$, prove that $m=r$ and 
$n=s$.
\end{mdframed}

%%%%%%%%%%%%%%%%%%%%%%%%%%%%%%%%%%%%%%%%%%%%%%%%%%%%%%%%%%%%%%%%%%%%%%%%%%%%%%%%
\begin{mdframed}[style=darkAnswer,frametitle={Joe Starr}]
We select $m,n.r.s\in \Z$, given $m^2+n^2=r^2+s^2=mr+ns$ which can write
as $m^2+n^2-mr-ns=r^2+s^2-mr-ns$. From here we can simplify:
\begin{align*}
m^2+n^2-mr-ns=r^2+s^2-mr-ns &\Rightarrow 
m\lrp{m-r}+n\lrp{n-s}=r\lrp{r-m}+s\lrp{s-n} \\
&\Rightarrow m\lrp{m-r}+n\lrp{n-s}-r\lrp{r-m}-s\lrp{s-n}=0 \\
&\Rightarrow m\lrp{m-r}+r\lrp{m-r}+n\lrp{n-s}+s\lrp{n-s}=0 \\
&\Rightarrow \lrp{m-r}\lrp{m+r}+\lrp{n-s}\lrp{n+s}=0 \\
\end{align*}
from here we can see that in order for $\lrp{m-r}\lrp{m+r}+\lrp{n-s}\lrp{n+s}=0$
to be true $m=r$ and $n=s$.
\end{mdframed}
\newpage
%%%%%%%%%%%%%%%%%%%%%%%%%%%%%%%%%%%%%%%%%%%%%%%%%%%%%%%%%%%%%%%%%%%%%%%%%%%%%%%%
%%%%%%%%%%%%%%%%%%%%%%%%%%%%%%%%%%%%%%%%%%%%%%%%%%%%%%%%%%%%%%%%%%%%%%%%%%%%%%%%
%%%%%%%%%%%%%%%%%%%%%%%%%%%%%%%%%%%%%%%%%%%%%%%%%%%%%%%%%%%%%%%%%%%%%%%%%%%%%%%%
%%%%%%%%%%%%%%%%%%%%%%%%%%%%%%%%%%%%%%%%%%%%%%%%%%%%%%%%%%%%%%%%%%%%%%%%%%%%%%%%
\begin{mdframed}[style=darkQuesion]
3.    Find the quotient and reminder when $a$ id divided by $b$. 
\begin{itemize}
    \item [a] {$a=99$, $b=17$}
    \item [b] {$a=-99$, $b=17$}
    \item [c] {$a=17$, $b=99$}
    \item [d] {$a=-1017$, $b=99$}
\end{itemize}

\end{mdframed}

%%%%%%%%%%%%%%%%%%%%%%%%%%%%%%%%%%%%%%%%%%%%%%%%%%%%%%%%%%%%%%%%%%%%%%%%%%%%%%%%
\begin{mdframed}[style=darkAnswer,frametitle={Joe Starr}]
\begin{itemize}
    \item [a] {$99=17q+r\Rightarrow q=5, r=14$}
    \item [b] {$-99=17q+r\Rightarrow q=-6, r=3$}
    \item [c] {$17=99q+r\Rightarrow q=0, r=17$}
    \item [d] {$-1017=99q+r\Rightarrow q=-11, r=72$}
\end{itemize}
\end{mdframed}

\newpage
%%%%%%%%%%%%%%%%%%%%%%%%%%%%%%%%%%%%%%%%%%%%%%%%%%%%%%%%%%%%%%%%%%%%%%%%%%%%%%%%
%%%%%%%%%%%%%%%%%%%%%%%%%%%%%%%%%%%%%%%%%%%%%%%%%%%%%%%%%%%%%%%%%%%%%%%%%%%%%%%%
%%%%%%%%%%%%%%%%%%%%%%%%%%%%%%%%%%%%%%%%%%%%%%%%%%%%%%%%%%%%%%%%%%%%%%%%%%%%%%%%
%%%%%%%%%%%%%%%%%%%%%%%%%%%%%%%%%%%%%%%%%%%%%%%%%%%%%%%%%%%%%%%%%%%%%%%%%%%%%%%%
\begin{mdframed}[style=darkQuesion]
5.    Use the Euclidean algorithm to find the following greatest common divisors
\begin{multicols}{2}
    \begin{itemize}
        \item [a] {$\lrp{6643,2873}$}
        \item [b] {$\lrp{7684,4148}$}
        \item [c] {$\lrp{26460,12600}$}
        \item [d] {$\lrp{6540,1206}$}
        \item [e] {$\lrp{12091,8439}$}
    \end{itemize}
\end{multicols}
\end{mdframed}

%%%%%%%%%%%%%%%%%%%%%%%%%%%%%%%%%%%%%%%%%%%%%%%%%%%%%%%%%%%%%%%%%%%%%%%%%%%%%%%%
\begin{mdframed}[style=darkAnswer,frametitle={Joe Starr}]
\begin{multicols}{2}
\begin{itemize}
    \item [(a)] {$\lrp{6643,2873}$
    \begin{align*}
        6643&=2873*2+897\\
        2873&=897*3+182\\
        897&=182*4+169\\
        182&=169*1+13\\
        169&=13*13\\
    \end{align*}
    }
    \item [(b)] {$\lrp{7684,4148}$
    \begin{align*}
        7684&=4148*1+3536\\
        4148&=3536*1+612\\
        3536&=612*5+476\\
        612&=476*1+136\\
        476&=136*3+68\\
        136&=68*68\\
    \end{align*}
    }
    \item [(c)] {$\lrp{26460,12600}$
    \begin{align*}
        26460&=12600*2+1260\\
        12600&=1260*10\\
    \end{align*}
    }
\end{itemize}
\end{multicols}
    \begin{multicols}{2}
        \begin{itemize}
    \item [(d)] {$\lrp{6540,1206}$
    \begin{align*}
        6540&=1206*5+510\\
        1206&=510*2+186\\
        510&=186*2+138\\
        186&=138*1+48\\
        138&=48*2+42\\
        48&=42*1+6\\
        42&=6*7\\
    \end{align*}
    }
    \item [(e)] {$\lrp{12091,8439}$
    \begin{align*}
        12091&=8439*1+3652\\
        8439&=3652*2+1135\\
        3652&=1135*3+247\\
        1135&=247*4+147\\
        247&=147*1+100\\
        147&=100*1+47\\
        100&=47*2+6\\
        47&=6*7+5\\
        6&=5*1+1\\
        5&=1*5\\
    \end{align*}
    }
\end{itemize}
\end{multicols}
\end{mdframed}
\newpage
%%%%%%%%%%%%%%%%%%%%%%%%%%%%%%%%%%%%%%%%%%%%%%%%%%%%%%%%%%%%%%%%%%%%%%%%%%%%%%%%
%%%%%%%%%%%%%%%%%%%%%%%%%%%%%%%%%%%%%%%%%%%%%%%%%%%%%%%%%%%%%%%%%%%%%%%%%%%%%%%%
%%%%%%%%%%%%%%%%%%%%%%%%%%%%%%%%%%%%%%%%%%%%%%%%%%%%%%%%%%%%%%%%%%%%%%%%%%%%%%%%
%%%%%%%%%%%%%%%%%%%%%%%%%%%%%%%%%%%%%%%%%%%%%%%%%%%%%%%%%%%%%%%%%%%%%%%%%%%%%%%%
\begin{mdframed}[style=darkQuesion]
    7.    For each part of Exercise 5, find integers $m$ and $n$ such that 
$\lrp{a,b}$ is expressed in the form $ma+nb$. 

\end{mdframed}

%%%%%%%%%%%%%%%%%%%%%%%%%%%%%%%%%%%%%%%%%%%%%%%%%%%%%%%%%%%%%%%%%%%%%%%%%%%%%%%%
\begin{mdframed}[style=darkAnswer,frametitle={Joe Starr}]
\begin{itemize}
    \item [(a)] {$\lrp{6643,2873}$\\
        $\lrp{6643}-16+\lrp{2873}37=13$
    }
    \item [(b)] {$\lrp{7684,4148}$\\
    $\lrp{7684}27+\lrp{4148}-50=68$
    }
    \item [(c)] {$\lrp{26460,12600}$
    $\lrp{26460}1+\lrp{12600}-2=1260$
    }
    \item [(d)] {$\lrp{6540,1206}$
   $\lrp{6540}-26+\lrp{1206}141=6$
    }
    \item [(e)] {$\lrp{12091,8439}$
    $\lrp{12091}1435+\lrp{8439}-2056=1$
    }
\end{itemize}
\end{mdframed}
\newpage
%%%%%%%%%%%%%%%%%%%%%%%%%%%%%%%%%%%%%%%%%%%%%%%%%%%%%%%%%%%%%%%%%%%%%%%%%%%%%%%%
%%%%%%%%%%%%%%%%%%%%%%%%%%%%%%%%%%%%%%%%%%%%%%%%%%%%%%%%%%%%%%%%%%%%%%%%%%%%%%%%
%%%%%%%%%%%%%%%%%%%%%%%%%%%%%%%%%%%%%%%%%%%%%%%%%%%%%%%%%%%%%%%%%%%%%%%%%%%%%%%%
%%%%%%%%%%%%%%%%%%%%%%%%%%%%%%%%%%%%%%%%%%%%%%%%%%%%%%%%%%%%%%%%%%%%%%%%%%%%%%%%
\begin{mdframed}[style=darkQuesion]
9.  let $a,b,c$ be integers such that $a+b+c=0$. Show that if $n$ is an integer
which is a divisor of two of the three integers, then it is also a divisor of 
the third. 

\end{mdframed}

%%%%%%%%%%%%%%%%%%%%%%%%%%%%%%%%%%%%%%%%%%%%%%%%%%%%%%%%%%%%%%%%%%%%%%%%%%%%%%%%
\begin{mdframed}[style=darkAnswer,frametitle={Joe Starr}]
Select $a,b,c\in \Z$ to satisfy $a+b+c=0$, WLOG let $n\in \Z$ such that 
$n\vert a$ and $n\vert b$. Since $\lrp{a+b}+c=0$ it must be that $\lrp{a+b}=-c$.
From here we must show $n\vert\lrp{a+b}$, or $a+b=nq$. Since $n\vert a$ and 
$n\vert b$ we may write $a=nq_1$ and $b=nq_2$, yielding, 
$nq_1+nq_2=n\lrp{q_1+q_2}=nq$ thus $n\vert c$, as desired.$\square$ 
\end{mdframed}
\newpage
%%%%%%%%%%%%%%%%%%%%%%%%%%%%%%%%%%%%%%%%%%%%%%%%%%%%%%%%%%%%%%%%%%%%%%%%%%%%%%%%
%%%%%%%%%%%%%%%%%%%%%%%%%%%%%%%%%%%%%%%%%%%%%%%%%%%%%%%%%%%%%%%%%%%%%%%%%%%%%%%%
%%%%%%%%%%%%%%%%%%%%%%%%%%%%%%%%%%%%%%%%%%%%%%%%%%%%%%%%%%%%%%%%%%%%%%%%%%%%%%%%
%%%%%%%%%%%%%%%%%%%%%%%%%%%%%%%%%%%%%%%%%%%%%%%%%%%%%%%%%%%%%%%%%%%%%%%%%%%%%%%%
\begin{mdframed}[style=darkQuesion]
13.  Show that if $n$ is any integer, then $\lrp{10n_3,5n+2}=1$
\end{mdframed}

%%%%%%%%%%%%%%%%%%%%%%%%%%%%%%%%%%%%%%%%%%%%%%%%%%%%%%%%%%%%%%%%%%%%%%%%%%%%%%%%
\begin{mdframed}[style=darkAnswer,frametitle={Joe Starr}]
We begin with the Euclidean algorithm, 
\begin{align*}
    10n+3&=\lrp{5n+2}1+\lrp{5n+1}\\
    5n+2&=\lrp{5n+1}1+1\\
\end{align*}
from here we have $\lrp{10n+3,5n+2}=\lrp{5n+2,5n+1}=1$, as desired.
\end{mdframed}
\newpage
%%%%%%%%%%%%%%%%%%%%%%%%%%%%%%%%%%%%%%%%%%%%%%%%%%%%%%%%%%%%%%%%%%%%%%%%%%%%%%%%
%%%%%%%%%%%%%%%%%%%%%%%%%%%%%%%%%%%%%%%%%%%%%%%%%%%%%%%%%%%%%%%%%%%%%%%%%%%%%%%%
%%%%%%%%%%%%%%%%%%%%%%%%%%%%%%%%%%%%%%%%%%%%%%%%%%%%%%%%%%%%%%%%%%%%%%%%%%%%%%%%
%%%%%%%%%%%%%%%%%%%%%%%%%%%%%%%%%%%%%%%%%%%%%%%%%%%%%%%%%%%%%%%%%%%%%%%%%%%%%%%%
\begin{mdframed}[style=darkQuesion]
15.  For what positive integers $n$ is it true that $\lrp{n,n+2}=2$? Prove your
claim.
\end{mdframed}

%%%%%%%%%%%%%%%%%%%%%%%%%%%%%%%%%%%%%%%%%%%%%%%%%%%%%%%%%%%%%%%%%%%%%%%%%%%%%%%%
\begin{mdframed}[style=darkAnswer,frametitle={Joe Starr}]
The conjecture is that the statement is true for even values of $n$.
We begin with rewriting $n$ in terms of $k$, $n=2k$the Euclidean algorithm, 
\begin{align*}
    \lrp{2k}+2&=\lrp{2k}1+\lrp{2}\\
    2k&=\lrp{2}{k}\\
\end{align*}
from here we have $\lrp{n+2,n}=\lrp{2k+2,2k}=2$, as desired.
\end{mdframed}
\newpage
%%%%%%%%%%%%%%%%%%%%%%%%%%%%%%%%%%%%%%%%%%%%%%%%%%%%%%%%%%%%%%%%%%%%%%%%%%%%%%%%
%%%%%%%%%%%%%%%%%%%%%%%%%%%%%%%%%%%%%%%%%%%%%%%%%%%%%%%%%%%%%%%%%%%%%%%%%%%%%%%%
%%%%%%%%%%%%%%%%%%%%%%%%%%%%%%%%%%%%%%%%%%%%%%%%%%%%%%%%%%%%%%%%%%%%%%%%%%%%%%%%
%%%%%%%%%%%%%%%%%%%%%%%%%%%%%%%%%%%%%%%%%%%%%%%%%%%%%%%%%%%%%%%%%%%%%%%%%%%%%%%%
\begin{mdframed}[style=darkQuesion]
17.  Show that the positive integer $k$ is the difference of two odd squares if 
and only if $k$ is divisible by $8$.
\end{mdframed}

%%%%%%%%%%%%%%%%%%%%%%%%%%%%%%%%%%%%%%%%%%%%%%%%%%%%%%%%%%%%%%%%%%%%%%%%%%%%%%%%
\begin{mdframed}[style=darkAnswer,frametitle={Joe Starr}]

We begin by writing $k=a^2-b^2$, since $a$ and $b$ are odd we can write, 
\begin{align*}
    a&=2r+1\\
    b&=2s+1
\end{align*} 
from here we have $q^2-b^2=4\lrp{r+s+1}\lrp{r-s}$. Since $k>0$ we must consider 
two cases $r-s=2m+1$ and $r-s=2m$. 
\begin{itemize}[align=left]
    \item [$r-s=2m$:]{\hspace{.5in}\newline
        In this case we have $q^2-b^2=4\lrp{r+s+1}2m=8\lrp{r+s+1}m$ and we are 
        done. 
    }
    \item [$r-s=2m+1$:]{\hspace{.5in}\newline
        In this case we have $r-s=2m+1$ and $r+s=r-s+2s=2m+1+2s$
        \begin{align*}
            q^2-b^2&=4\lrp{r+s+1}\lrp{2m+1}\\
            &=4\lrp{2m\lrp{r+s+1}+\lrp{r+s+1}}\\            
            &=4\lrp{\lrp{2mr+2ms+2m}+\lrp{r+s+1}}\\            
            &=4\lrp{2mr+2ms+2m+r+s+1}\\         
            &=4\lrp{2mr+2ms+2m+2m+1+2s+1}\\         
            &=4\lrp{2mr+2ms+2m+2m+2s+2}\\         
            &=8\lrp{mr+ms+m+m+s+1}\\         
        \end{align*}
        as desired.
    }
\end{itemize}

\end{mdframed}
\newpage
%%%%%%%%%%%%%%%%%%%%%%%%%%%%%%%%%%%%%%%%%%%%%%%%%%%%%%%%%%%%%%%%%%%%%%%%%%%%%%%%
%%%%%%%%%%%%%%%%%%%%%%%%%%%%%%%%%%%%%%%%%%%%%%%%%%%%%%%%%%%%%%%%%%%%%%%%%%%%%%%%
%%%%%%%%%%%%%%%%%%%%%%%%%%%%%%%%%%%%%%%%%%%%%%%%%%%%%%%%%%%%%%%%%%%%%%%%%%%%%%%%
%%%%%%%%%%%%%%%%%%%%%%%%%%%%%%%%%%%%%%%%%%%%%%%%%%%%%%%%%%%%%%%%%%%%%%%%%%%%%%%%
\subsection{Primes}
%%%%%%%%%%%%%%%%%%%%%%%%%%%%%%%%%%%%%%%%%%%%%%%%%%%%%%%%%%%%%%%%%%%%%%%%%%%%%%%%
%%%%%%%%%%%%%%%%%%%%%%%%%%%%%%%%%%%%%%%%%%%%%%%%%%%%%%%%%%%%%%%%%%%%%%%%%%%%%%%%
%%%%%%%%%%%%%%%%%%%%%%%%%%%%%%%%%%%%%%%%%%%%%%%%%%%%%%%%%%%%%%%%%%%%%%%%%%%%%%%%
%%%%%%%%%%%%%%%%%%%%%%%%%%%%%%%%%%%%%%%%%%%%%%%%%%%%%%%%%%%%%%%%%%%%%%%%%%%%%%%%
\begin{mdframed}[style=darkQuesion]
1. Find the prime factorizations of each of the following numbers, and use the 
them to compute the greatest common divisor and least common multiple of the 
given pairs of numbers. 
\begin{multicols}{3}
\begin{itemize}
    \item [(a)] {$\lrp{35,14}$
    }
    \item [(b)] {$\lrp{15,11}$
    }
    \item [(c)] {$\lrp{252,11}$
    }
    \item [(d)] {$\lrp{7684,4148}$
    }
    \item [(e)] {$\lrp{6643,2873}$
    }
\end{itemize}
\end{multicols}
\end{mdframed}

%%%%%%%%%%%%%%%%%%%%%%%%%%%%%%%%%%%%%%%%%%%%%%%%%%%%%%%%%%%%%%%%%%%%%%%%%%%%%%%%
\begin{mdframed}[style=darkAnswer,frametitle={Joe Starr}]
    \begin{multicols}{2}
    \begin{itemize}
        \item [(a)] {
        \begin{multicols}{2}
        $\lrp{35,14}$\\
        $35: 5,7$ \\
        $14: 2,7$ \\
        gcd: $7$ \\
        lcm: $70$ 
        \end{multicols}
        }
        \item [(b)] {
            \begin{multicols}{2}
        $\lrp{15,11}$ \\
        $15: 3,5$ \\
        $11: 11$ \\
        gcd: $1$ \\
        lcm: $165$ 
    \end{multicols}
        }
        \item [(c)] {
            \begin{multicols}{2}
        $\lrp{252,180}$ \\
        $252: 2,2,3,3,7$ \\
        $180: 2,2,3,3,5$ \\
        gcd: $36$ \\
        lcm: $1260$ 
    \end{multicols}
        }
        \item [(d)] {
            \begin{multicols}{2}
        $\lrp{7684,4148}$ \\
        $7684: 2,2,17,113$ \\
        $4148: 2,2,17,61$ \\
        gcd: $68$ \\
        lcm: $468724$ 
    \end{multicols}
        }
        \item [(e)] {
            \begin{multicols}{2}
        $\lrp{6643,2873}$ \\
        $6643: 7,13,73$ \\
        $2873: 13,13,17$ \\
        gcd: $13$ \\
        lcm: $1468103$ 
    \end{multicols}
        }
    \end{itemize}
\end{multicols}
\end{mdframed}
\newpage
%%%%%%%%%%%%%%%%%%%%%%%%%%%%%%%%%%%%%%%%%%%%%%%%%%%%%%%%%%%%%%%%%%%%%%%%%%%%%%%%
%%%%%%%%%%%%%%%%%%%%%%%%%%%%%%%%%%%%%%%%%%%%%%%%%%%%%%%%%%%%%%%%%%%%%%%%%%%%%%%%
%%%%%%%%%%%%%%%%%%%%%%%%%%%%%%%%%%%%%%%%%%%%%%%%%%%%%%%%%%%%%%%%%%%%%%%%%%%%%%%%
%%%%%%%%%%%%%%%%%%%%%%%%%%%%%%%%%%%%%%%%%%%%%%%%%%%%%%%%%%%%%%%%%%%%%%%%%%%%%%%%
\begin{mdframed}[style=darkQuesion]
2. US the sieve of Eratosthenes to find all prime numbers less than 200.
\end{mdframed}

%%%%%%%%%%%%%%%%%%%%%%%%%%%%%%%%%%%%%%%%%%%%%%%%%%%%%%%%%%%%%%%%%%%%%%%%%%%%%%%%
\begin{mdframed}[style=darkAnswer,frametitle={Joe Starr}]
    \begin{center}
        \begin{tabular}{| c | c | c | c | c | c | c | c | c | c |}
        \hline
        $ $ & $2$ & $3$ & $\msout{4}$ & $5$ & $\msout{6}$ & $7$ & $\msout{8}$ & $\msout{9}$ & $\msout{10}$ \\
        \hline
        $11$ & $\msout{12}$ & $13$ & $\msout{14}$ & $\msout{15}$ & $\msout{16}$ & $17$ & $\msout{18}$ & $19$ & $\msout{20}$ \\
        \hline
        $\msout{21}$ & $\msout{22}$ & $23$ & $\msout{24}$ & $\msout{25}$ & $\msout{26}$ & $\msout{27}$ & $\msout{28}$ & $29$ & $\msout{30}$ \\
        \hline
        $31$ & $\msout{32}$ & $\msout{33}$ & $\msout{34}$ & $\msout{35}$ & $\msout{36}$ & $37$ & $\msout{38}$ & $\msout{39}$ & $\msout{40}$ \\
        \hline
        $41$ & $\msout{42}$ & $43$ & $\msout{44}$ & $\msout{45}$ & $\msout{46}$ & $47$ & $\msout{48}$ & $\msout{49}$ & $\msout{50}$ \\
        \hline
        $\msout{51}$ & $\msout{52}$ & $53$ & $\msout{54}$ & $\msout{55}$ & $\msout{56}$ & $\msout{57}$ & $\msout{58}$ & $59$ & $\msout{60}$ \\
        \hline
        $61$ & $\msout{62}$ & $\msout{63}$ & $\msout{64}$ & $\msout{65}$ & $\msout{66}$ & $67$ & $\msout{68}$ & $\msout{69}$ & $\msout{70}$ \\
        \hline
        $71$ & $\msout{72}$ & $73$ & $\msout{74}$ & $\msout{75}$ & $\msout{76}$ & $\msout{77}$ & $\msout{78}$ & $79$ & $\msout{80}$ \\
        \hline
        $\msout{81}$ & $\msout{82}$ & $83$ & $\msout{84}$ & $\msout{85}$ & $\msout{86}$ & $\msout{87}$ & $\msout{88}$ & $89$ & $\msout{90}$ \\
        \hline
        $\msout{91}$ & $\msout{92}$ & $\msout{93}$ & $\msout{94}$ & $\msout{95}$ & $\msout{96}$ & $97$ & $\msout{98}$ & $\msout{99}$ & $\msout{100}$ \\
        \hline
        $101$ & $\msout{102}$ & $103$ & $\msout{104}$ & $\msout{105}$ & $\msout{106}$ & $107$ & $\msout{108}$ & $109$ & $\msout{110}$ \\
        \hline
        $\msout{111}$ & $\msout{112}$ & $113$ & $\msout{114}$ & $\msout{115}$ & $\msout{116}$ & $\msout{117}$ & $\msout{118}$ & $\msout{119}$ & $\msout{120}$ \\
        \hline
        $\msout{121}$ & $\msout{122}$ & $\msout{123}$ & $\msout{124}$ & $\msout{125}$ & $\msout{126}$ & $127$ & $\msout{128}$ & $\msout{129}$ & $\msout{130}$ \\
        \hline
        $131$ & $\msout{132}$ & $\msout{133}$ & $\msout{134}$ & $\msout{135}$ & $\msout{136}$ & $137$ & $\msout{138}$ & $139$ & $\msout{140}$ \\
        \hline
        $\msout{141}$ & $\msout{142}$ & $\msout{143}$ & $\msout{144}$ & $\msout{145}$ & $\msout{146}$ & $\msout{147}$ & $\msout{148}$ & $149$ & $\msout{150}$ \\
        \hline
        $151$ & $\msout{152}$ & $\msout{153}$ & $\msout{154}$ & $\msout{155}$ & $\msout{156}$ & $157$ & $\msout{158}$ & $\msout{159}$ & $\msout{160}$ \\
        \hline
        $\msout{161}$ & $\msout{162}$ & $163$ & $\msout{164}$ & $\msout{165}$ & $\msout{166}$ & $167$ & $\msout{168}$ & $\msout{169}$ & $\msout{170}$ \\
        \hline
        $\msout{171}$ & $\msout{172}$ & $173$ & $\msout{174}$ & $\msout{175}$ & $\msout{176}$ & $\msout{177}$ & $\msout{178}$ & $179$ & $\msout{180}$ \\
        \hline
        $181$ & $\msout{182}$ & $\msout{183}$ & $\msout{184}$ & $\msout{185}$ & $\msout{186}$ & $\msout{187}$ & $\msout{188}$ & $\msout{189}$ & $\msout{190}$ \\
        \hline
        $191$ & $\msout{192}$ & $193$ & $\msout{194}$ & $\msout{195}$ & $\msout{196}$ & $197$ & $\msout{198}$ & $199$ & $\msout{200}$\\
        \hline
        \end{tabular}
        \end{center}    
\end{mdframed}
\newpage
%%%%%%%%%%%%%%%%%%%%%%%%%%%%%%%%%%%%%%%%%%%%%%%%%%%%%%%%%%%%%%%%%%%%%%%%%%%%%%%%
%%%%%%%%%%%%%%%%%%%%%%%%%%%%%%%%%%%%%%%%%%%%%%%%%%%%%%%%%%%%%%%%%%%%%%%%%%%%%%%%
%%%%%%%%%%%%%%%%%%%%%%%%%%%%%%%%%%%%%%%%%%%%%%%%%%%%%%%%%%%%%%%%%%%%%%%%%%%%%%%%
%%%%%%%%%%%%%%%%%%%%%%%%%%%%%%%%%%%%%%%%%%%%%%%%%%%%%%%%%%%%%%%%%%%%%%%%%%%%%%%%
\begin{mdframed}[style=darkQuesion]
3. For each composite number $a$. with $4\leq a\leq 20$, find all positive 
numbers less than $a$ that are relatively prime to $a$.
\end{mdframed}

%%%%%%%%%%%%%%%%%%%%%%%%%%%%%%%%%%%%%%%%%%%%%%%%%%%%%%%%%%%%%%%%%%%%%%%%%%%%%%%%
\begin{mdframed}[style=darkAnswer,frametitle={Joe Starr}]
\begin{multicols}{1}
\begin{itemize}
    \item[$4:$]  {$2, 3$}
    \item[$6:$]  {$2, 3, 5$}
    \item[$8:$]  {$2, 3, 5, 7$}
    \item[$9:$]  {$2, 3, 4, 5, 7, 8$}
    \item[$10:$] {$2, 3, 5, 7, 9$}
    \item[$12:$] {$2, 3, 5, 7, 11$}
    \item[$14:$] {$2, 3, 5, 7, 9, 11, 13$}
    \item[$15:$] {$2, 3, 4, 5, 7, 8, 11, 13, 14$}
    \item[$16:$] {$2, 3, 5, 7, 9, 11, 13, 15$}
    \item[$18:$] {$2, 3, 5, 7, 11, 13, 17$}
    \item[$20:$] {$2, 3, 5, 7, 9, 11, 13, 17, 19$}
\end{itemize}
\end{multicols}
\end{mdframed}
\newpage
%%%%%%%%%%%%%%%%%%%%%%%%%%%%%%%%%%%%%%%%%%%%%%%%%%%%%%%%%%%%%%%%%%%%%%%%%%%%%%%%
%%%%%%%%%%%%%%%%%%%%%%%%%%%%%%%%%%%%%%%%%%%%%%%%%%%%%%%%%%%%%%%%%%%%%%%%%%%%%%%%
%%%%%%%%%%%%%%%%%%%%%%%%%%%%%%%%%%%%%%%%%%%%%%%%%%%%%%%%%%%%%%%%%%%%%%%%%%%%%%%%
%%%%%%%%%%%%%%%%%%%%%%%%%%%%%%%%%%%%%%%%%%%%%%%%%%%%%%%%%%%%%%%%%%%%%%%%%%%%%%%%
\begin{mdframed}[style=darkQuesion]
4. Find all positive integers less than 60 and relatively prime to $60$.
\end{mdframed}

%%%%%%%%%%%%%%%%%%%%%%%%%%%%%%%%%%%%%%%%%%%%%%%%%%%%%%%%%%%%%%%%%%%%%%%%%%%%%%%%
\begin{mdframed}[style=darkAnswer,frametitle={Joe Starr}]
\begin{itemize}
    \item[$60:$] {$2, 3, 5, 7, 11, 13, 17, 19, 23, 29, 31, 37, 41, 43, 47, 49, 53, 59 $}
\end{itemize}
\end{mdframed}
\newpage
%%%%%%%%%%%%%%%%%%%%%%%%%%%%%%%%%%%%%%%%%%%%%%%%%%%%%%%%%%%%%%%%%%%%%%%%%%%%%%%%
%%%%%%%%%%%%%%%%%%%%%%%%%%%%%%%%%%%%%%%%%%%%%%%%%%%%%%%%%%%%%%%%%%%%%%%%%%%%%%%%
%%%%%%%%%%%%%%%%%%%%%%%%%%%%%%%%%%%%%%%%%%%%%%%%%%%%%%%%%%%%%%%%%%%%%%%%%%%%%%%%
%%%%%%%%%%%%%%%%%%%%%%%%%%%%%%%%%%%%%%%%%%%%%%%%%%%%%%%%%%%%%%%%%%%%%%%%%%%%%%%%
\begin{mdframed}[style=darkQuesion]
\begin{itemize}
    \item [9. (a)] {For which $n\in \Z^{\text{+}}$ is $n^{3}-1$ a prime number?}
    \item [(b)] {For which $n\in \Z^{\text{+}}$ is $n^{3}+1$ a prime number?}
    \item [(c)] {For which $n\in \Z^{\text{+}}$ is $n^{2}-1$ a prime number?}
    \item [(d)] {For which $n\in \Z^{\text{+}}$ is $n^{2}+1$ a prime number?}
    \end{itemize}
\end{mdframed}

%%%%%%%%%%%%%%%%%%%%%%%%%%%%%%%%%%%%%%%%%%%%%%%%%%%%%%%%%%%%%%%%%%%%%%%%%%%%%%%%
\begin{mdframed}[style=darkAnswer,frametitle={Joe Starr}]
    \begin{itemize}
        \item [(a)] {We can factor $n^{3}-1$ into $(n - 1) (n^2 + n + 1)$. We 
        have then $n-1\vert n^{3}-1$, for $n^{3}-1$ to be prime $n-1$ must be 
        $1$. This happens only for $n=2$.
        }
        \item [(b)] {We can factor $n^{3}+1$ into $(n + 1) (n^2 - n + 1)$. We 
        have then $(n^2 - n + 1)\vert n^{3}+1$, 
        for $n^{3}+1$ to be prime $(n^2 - n + 1)$ must be $1$. 
        This happens only for $n=1$. }
        \item [(c)] {We can factor $n^{2}-1$ into $(n - 1) (n + 1)$. We 
        have then $(n 1 1)\vert n^{2}-1$, 
        for $n^{2}-1$ to be prime $(n - 1)$ must be $1$. 
        This happens only for $n=2$.
        For which $n\in \Z^{\text{+}}$ is $n^{2}-1$ a prime number?}
        \item [(d)] {????
        }
    \end{itemize}
\end{mdframed}

\newpage
%%%%%%%%%%%%%%%%%%%%%%%%%%%%%%%%%%%%%%%%%%%%%%%%%%%%%%%%%%%%%%%%%%%%%%%%%%%%%%%%
%%%%%%%%%%%%%%%%%%%%%%%%%%%%%%%%%%%%%%%%%%%%%%%%%%%%%%%%%%%%%%%%%%%%%%%%%%%%%%%%
%%%%%%%%%%%%%%%%%%%%%%%%%%%%%%%%%%%%%%%%%%%%%%%%%%%%%%%%%%%%%%%%%%%%%%%%%%%%%%%%
%%%%%%%%%%%%%%%%%%%%%%%%%%%%%%%%%%%%%%%%%%%%%%%%%%%%%%%%%%%%%%%%%%%%%%%%%%%%%%%%
\begin{mdframed}[style=darkQuesion]
11. Prove that $n^4+4^n$ is composite if $n>1$.
\end{mdframed}

%%%%%%%%%%%%%%%%%%%%%%%%%%%%%%%%%%%%%%%%%%%%%%%%%%%%%%%%%%%%%%%%%%%%%%%%%%%%%%%%
\begin{mdframed}[style=darkAnswer,frametitle={Joe Starr}]
We are presented with two potability's, $n$ is even or $n$ is odd.
\begin{itemize}[align=left]
    \item [$n$ even]{\hspace{.5in}\newline
        It's obvious that $n^4+4^n$ is an even not $2$ and can't be prime. 
    }
    \item [$n$ odd]{\hspace{.5in}\newline
    We begin by completing the square 
    \begin{align*}
        n^4+4^n&= n^4+4^n\\
        &=\lrp{n^2}^2+\lrp{2^n}^2\\
        &=\lrp{n^2+2^n}^2-2n^2 2^n
    \end{align*}
    We from here we observe that $2^n2=2^{n+1}$, since $n$ is odd $n+1$ is even 
    yielding $2^{n+1}=2^{2k}$. We can see we have a difference of squares
    \begin{align*}
        \lrp{n^2+2^n}^2-2n^2 2^n&=\lrp{n^2+2^n}^2-\lrp{2^nn}^2\\
        &=\lrp{n^2+2^n+2^nn}\lrp{n^2+2^n-2^nn}
    \end{align*}
    since we are restricted to $n>1$ we can see that both $\lrp{n^2+2^n+2^nn}>1$
    and $\lrp{n^2+2^n-2^nn}>1$ for all $n$. Making $n^4+4^n$ composite as 
    desired. 
    }
\end{itemize}
\end{mdframed}
\newpage
%%%%%%%%%%%%%%%%%%%%%%%%%%%%%%%%%%%%%%%%%%%%%%%%%%%%%%%%%%%%%%%%%%%%%%%%%%%%%%%%
%%%%%%%%%%%%%%%%%%%%%%%%%%%%%%%%%%%%%%%%%%%%%%%%%%%%%%%%%%%%%%%%%%%%%%%%%%%%%%%%
%%%%%%%%%%%%%%%%%%%%%%%%%%%%%%%%%%%%%%%%%%%%%%%%%%%%%%%%%%%%%%%%%%%%%%%%%%%%%%%%
%%%%%%%%%%%%%%%%%%%%%%%%%%%%%%%%%%%%%%%%%%%%%%%%%%%%%%%%%%%%%%%%%%%%%%%%%%%%%%%%
\begin{mdframed}[style=darkQuesion]
13. Let $a,b,c$  be positive integers, and let $d=\lrp{a,b}$. Since $d\vert a$, 
there exists an integer $h$ with $a=dh$. Show that $a\vert bc$, then $h\vert c$. 
\end{mdframed}

%%%%%%%%%%%%%%%%%%%%%%%%%%%%%%%%%%%%%%%%%%%%%%%%%%%%%%%%%%%%%%%%%%%%%%%%%%%%%%%%
\begin{mdframed}[style=darkAnswer,frametitle={Joe Starr}]
    
\end{mdframed}
\newpage
%%%%%%%%%%%%%%%%%%%%%%%%%%%%%%%%%%%%%%%%%%%%%%%%%%%%%%%%%%%%%%%%%%%%%%%%%%%%%%%%
%%%%%%%%%%%%%%%%%%%%%%%%%%%%%%%%%%%%%%%%%%%%%%%%%%%%%%%%%%%%%%%%%%%%%%%%%%%%%%%%
%%%%%%%%%%%%%%%%%%%%%%%%%%%%%%%%%%%%%%%%%%%%%%%%%%%%%%%%%%%%%%%%%%%%%%%%%%%%%%%%
%%%%%%%%%%%%%%%%%%%%%%%%%%%%%%%%%%%%%%%%%%%%%%%%%%%%%%%%%%%%%%%%%%%%%%%%%%%%%%%%
\begin{mdframed}[style=darkQuesion]
14. Show that $a\Z \cap b\Z=\lrb{a,b}$.
\end{mdframed}

%%%%%%%%%%%%%%%%%%%%%%%%%%%%%%%%%%%%%%%%%%%%%%%%%%%%%%%%%%%%%%%%%%%%%%%%%%%%%%%%
\begin{mdframed}[style=darkAnswer,frametitle={Joe Starr}]
    Let $x\in \lrp{a\Z \cap b\Z}$, since $x\in a\Z$ we have $x=aq_1$, similarly 
    since $x\in b\Z$ we have$x=bq_2$. We can see that $x=abq$, this means $x$ 
    is a multiple of $\lrb{a,b}$ putting $x\in \lrb{a,b}$. Next, we let 
    $x\in \lrb{a,b}\Z$, this means $x$ is of the form $x=\lrb{a,b}q$. We can see
    that $a\vert x$ and $b\vert x$ since $a\vert \lrb{a,b}$, This makes 
    $x\in a\Z$ and $x\in b\Z$, as desired. 
\end{mdframed}
\newpage
%%%%%%%%%%%%%%%%%%%%%%%%%%%%%%%%%%%%%%%%%%%%%%%%%%%%%%%%%%%%%%%%%%%%%%%%%%%%%%%%
%%%%%%%%%%%%%%%%%%%%%%%%%%%%%%%%%%%%%%%%%%%%%%%%%%%%%%%%%%%%%%%%%%%%%%%%%%%%%%%%
%%%%%%%%%%%%%%%%%%%%%%%%%%%%%%%%%%%%%%%%%%%%%%%%%%%%%%%%%%%%%%%%%%%%%%%%%%%%%%%%
%%%%%%%%%%%%%%%%%%%%%%%%%%%%%%%%%%%%%%%%%%%%%%%%%%%%%%%%%%%%%%%%%%%%%%%%%%%%%%%%
\begin{mdframed}[style=darkQuesion]
17. Let $a,b$ be nonzero integers. Prove $\lrp{a,b}=1$ if and only if 
$\lrp{a+b,ab}=1$. 
\end{mdframed}

%%%%%%%%%%%%%%%%%%%%%%%%%%%%%%%%%%%%%%%%%%%%%%%%%%%%%%%%%%%%%%%%%%%%%%%%%%%%%%%%
\begin{mdframed}[style=darkAnswer,frametitle={Joe Starr}]
    
\end{mdframed}
\newpage
%%%%%%%%%%%%%%%%%%%%%%%%%%%%%%%%%%%%%%%%%%%%%%%%%%%%%%%%%%%%%%%%%%%%%%%%%%%%%%%%
%%%%%%%%%%%%%%%%%%%%%%%%%%%%%%%%%%%%%%%%%%%%%%%%%%%%%%%%%%%%%%%%%%%%%%%%%%%%%%%%
%%%%%%%%%%%%%%%%%%%%%%%%%%%%%%%%%%%%%%%%%%%%%%%%%%%%%%%%%%%%%%%%%%%%%%%%%%%%%%%%
%%%%%%%%%%%%%%%%%%%%%%%%%%%%%%%%%%%%%%%%%%%%%%%%%%%%%%%%%%%%%%%%%%%%%%%%%%%%%%%%
\begin{mdframed}[style=darkQuesion]
18. Let $a,b$ be nonzero integers with $\lrp{a,b}=1$. Compute $\lrp{a+b,a-b}$. 
\end{mdframed}

%%%%%%%%%%%%%%%%%%%%%%%%%%%%%%%%%%%%%%%%%%%%%%%%%%%%%%%%%%%%%%%%%%%%%%%%%%%%%%%%
\begin{mdframed}[style=darkAnswer,frametitle={Joe Starr}]
    
\end{mdframed}
\newpage
%%%%%%%%%%%%%%%%%%%%%%%%%%%%%%%%%%%%%%%%%%%%%%%%%%%%%%%%%%%%%%%%%%%%%%%%%%%%%%%%
%%%%%%%%%%%%%%%%%%%%%%%%%%%%%%%%%%%%%%%%%%%%%%%%%%%%%%%%%%%%%%%%%%%%%%%%%%%%%%%%
%%%%%%%%%%%%%%%%%%%%%%%%%%%%%%%%%%%%%%%%%%%%%%%%%%%%%%%%%%%%%%%%%%%%%%%%%%%%%%%%
%%%%%%%%%%%%%%%%%%%%%%%%%%%%%%%%%%%%%%%%%%%%%%%%%%%%%%%%%%%%%%%%%%%%%%%%%%%%%%%%
\begin{mdframed}[style=darkQuesion]
19. Let $a$ and $b$ be positive integers, and let $m$ be an integer such that 
$ab=m\lrp{a,b}$. Without using the prime factorization theorem, prove that 
$\lrp{a,b}\lrb{a,b}=ab$ by verifying that $m$ satisfies the necessary properties
of $\lrb{a,b}$.
\end{mdframed}

%%%%%%%%%%%%%%%%%%%%%%%%%%%%%%%%%%%%%%%%%%%%%%%%%%%%%%%%%%%%%%%%%%%%%%%%%%%%%%%%
\begin{mdframed}[style=darkAnswer,frametitle={Joe Starr}]
    
\end{mdframed}
\newpage
%%%%%%%%%%%%%%%%%%%%%%%%%%%%%%%%%%%%%%%%%%%%%%%%%%%%%%%%%%%%%%%%%%%%%%%%%%%%%%%%
%%%%%%%%%%%%%%%%%%%%%%%%%%%%%%%%%%%%%%%%%%%%%%%%%%%%%%%%%%%%%%%%%%%%%%%%%%%%%%%%
%%%%%%%%%%%%%%%%%%%%%%%%%%%%%%%%%%%%%%%%%%%%%%%%%%%%%%%%%%%%%%%%%%%%%%%%%%%%%%%%
%%%%%%%%%%%%%%%%%%%%%%%%%%%%%%%%%%%%%%%%%%%%%%%%%%%%%%%%%%%%%%%%%%%%%%%%%%%%%%%%
\begin{mdframed}[style=darkQuesion]
20. A positive integer $a$ is called a square if $a=n^2$ for some $n\in \Z$. 
Show that the integer $a>1$ is a square if and only if every exponent in its 
prime factorization is even. 
\end{mdframed}

%%%%%%%%%%%%%%%%%%%%%%%%%%%%%%%%%%%%%%%%%%%%%%%%%%%%%%%%%%%%%%%%%%%%%%%%%%%%%%%%
\begin{mdframed}[style=darkAnswer,frametitle={Joe Starr}]
    
\end{mdframed}
\newpage
%%%%%%%%%%%%%%%%%%%%%%%%%%%%%%%%%%%%%%%%%%%%%%%%%%%%%%%%%%%%%%%%%%%%%%%%%%%%%%%%
%%%%%%%%%%%%%%%%%%%%%%%%%%%%%%%%%%%%%%%%%%%%%%%%%%%%%%%%%%%%%%%%%%%%%%%%%%%%%%%%
%%%%%%%%%%%%%%%%%%%%%%%%%%%%%%%%%%%%%%%%%%%%%%%%%%%%%%%%%%%%%%%%%%%%%%%%%%%%%%%%
%%%%%%%%%%%%%%%%%%%%%%%%%%%%%%%%%%%%%%%%%%%%%%%%%%%%%%%%%%%%%%%%%%%%%%%%%%%%%%%%
\begin{mdframed}[style=darkQuesion]
23. Let $p$ and $q$ be prime numbers. Prove that $pq+1$ is a square if and only 
if $p$ and $q$ are twin primes.
\end{mdframed}

%%%%%%%%%%%%%%%%%%%%%%%%%%%%%%%%%%%%%%%%%%%%%%%%%%%%%%%%%%%%%%%%%%%%%%%%%%%%%%%%
\begin{mdframed}[style=darkAnswer,frametitle={Joe Starr}]
    
\end{mdframed}
\newpage
%%%%%%%%%%%%%%%%%%%%%%%%%%%%%%%%%%%%%%%%%%%%%%%%%%%%%%%%%%%%%%%%%%%%%%%%%%%%%%%%
%%%%%%%%%%%%%%%%%%%%%%%%%%%%%%%%%%%%%%%%%%%%%%%%%%%%%%%%%%%%%%%%%%%%%%%%%%%%%%%%
%%%%%%%%%%%%%%%%%%%%%%%%%%%%%%%%%%%%%%%%%%%%%%%%%%%%%%%%%%%%%%%%%%%%%%%%%%%%%%%%
%%%%%%%%%%%%%%%%%%%%%%%%%%%%%%%%%%%%%%%%%%%%%%%%%%%%%%%%%%%%%%%%%%%%%%%%%%%%%%%%
\begin{mdframed}[style=darkQuesion]
26. Prove that if $a>1$, then there is a prime $p$ with $a<p\leq a!+1$.
\end{mdframed}

%%%%%%%%%%%%%%%%%%%%%%%%%%%%%%%%%%%%%%%%%%%%%%%%%%%%%%%%%%%%%%%%%%%%%%%%%%%%%%%%
\begin{mdframed}[style=darkAnswer,frametitle={Joe Starr}]
    
\end{mdframed}
\newpage
%%%%%%%%%%%%%%%%%%%%%%%%%%%%%%%%%%%%%%%%%%%%%%%%%%%%%%%%%%%%%%%%%%%%%%%%%%%%%%%%
%%%%%%%%%%%%%%%%%%%%%%%%%%%%%%%%%%%%%%%%%%%%%%%%%%%%%%%%%%%%%%%%%%%%%%%%%%%%%%%%
%%%%%%%%%%%%%%%%%%%%%%%%%%%%%%%%%%%%%%%%%%%%%%%%%%%%%%%%%%%%%%%%%%%%%%%%%%%%%%%%
%%%%%%%%%%%%%%%%%%%%%%%%%%%%%%%%%%%%%%%%%%%%%%%%%%%%%%%%%%%%%%%%%%%%%%%%%%%%%%%%
\begin{mdframed}[style=darkQuesion]
29. Show that $\log{2}/\log{3}$ is not a rational number. 
\end{mdframed}

%%%%%%%%%%%%%%%%%%%%%%%%%%%%%%%%%%%%%%%%%%%%%%%%%%%%%%%%%%%%%%%%%%%%%%%%%%%%%%%%
\begin{mdframed}[style=darkAnswer,frametitle={Joe Starr}]
    
\end{mdframed}



\clearpage
%Compile Chapter 2

\section{Functions}
\subsection{Functions}
%%%%%%%%%%%%%%%%%%%%%%%%%%%%%%%%%%%%%%%%%%%%%%%%%%%%%%%%%%%%%%%%%%%%%%%%%%%%%%%%
%%%%%%%%%%%%%%%%%%%%%%%%%%%%%%%%%%%%%%%%%%%%%%%%%%%%%%%%%%%%%%%%%%%%%%%%%%%%%%%%
%%%%%%%%%%%%%%%%%%%%%%%%%%%%%%%%%%%%%%%%%%%%%%%%%%%%%%%%%%%%%%%%%%%%%%%%%%%%%%%%
%%%%%%%%%%%%%%%%%%%%%%%%%%%%%%%%%%%%%%%%%%%%%%%%%%%%%%%%%%%%%%%%%%%%%%%%%%%%%%%%
\begin{mdframed}[style=darkQuesion]
  1.   In each of the following parts, determine whether the given function is
  1:1 and whether it is onto.
  \begin{itemize}
    \item [(a)]{
          $f:\R\to \R; \fof{x}=x+3$
          }
    \item [(b)]{
          $f:\C\to \C; \fof{x}=x^2+2x+1$
          }
    \item [(c]{
          $f:\Z_n\to \Z_n; \fof{\lrb{x}_n}=\lrb{mx+b}_n, \text{where} m,b\in \Z$
          }
    \item [(d)]{
          $f:\R^+\to \R; \fof{x}=\ln x$
          }
  \end{itemize}
\end{mdframed}

%%%%%%%%%%%%%%%%%%%%%%%%%%%%%%%%%%%%%%%%%%%%%%%%%%%%%%%%%%%%%%%%%%%%%%%%%%%%%%%%
\begin{mdframed}[style=darkAnswer,frametitle={Joe Starr}]
  \begin{itemize}
    \item [(a)]{ We can see that $\fof{x}=x+3$ then $\fofinv{x}=x-3$,
          $\fof{\fofinv{x}}=\lrp{x-3}+3=x$. Showing $f$ is a bijection.
          }
    \item [(b)]{
          \begin{itemize}
            \item[1:1:]{\hspace{.5in}\newline
                  Assume $\fof{x}=25=\fof{y}$, we can see that if $x=4$, $\fof{x}=25$,
                  and $y=-6$, $\fof{y}=25$, showing $f$ not injective.
                  }
            \item[onto:]{\hspace{.5in}\newline
                  Let $y\in \C$ we must now show there exists a $x \in \C$ such that
                  $\fof{x}=y$. Consider $x=\sqrt{y}-1$, we can then take:
                  \begin{align*}
                    \fof{x} & = x^2+2x+1                               \\
                            & = \lrp{\sqrt{y}-1}^2+2\lrp{\sqrt{y}-1}+1 \\
                            & = \lrp{\sqrt{y}-1}^2+2\sqrt{y}-2+1       \\
                            & = 1 - 2 \sqrt{y} + y+2\sqrt{y}-1         \\
                            & =  y                                     \\
                  \end{align*}
                  showing $f$ surjective.
                  }
          \end{itemize}
          }
    \item [(c)]{
          Consider $\fofinv{x}=\lrb{\lrp{y-b}m^{-1}}_n$, now taking
          $\fof{\fofinv{x}}$
          \begin{align*}
            \fof{\fofinv{x}} & = \lrb{m\lrp{x-b}m^{-1}+b}_n \\
                             & = \lrb{\lrp{x-b}+b}_n        \\
                             & = \lrb{x}_n                  \\
          \end{align*}
          showing $f$ a bijection.
          }
    \item [(d)]{
          $f:\R^+\to \R; \fof{x}=\ln x$
          If we take $\fofinv{x}=e^x$, $\fof{\fofinv{x}}=\ln e^x = x$, showing $f$
          a bijection.
          }
  \end{itemize}
\end{mdframed}
\newpage
%%%%%%%%%%%%%%%%%%%%%%%%%%%%%%%%%%%%%%%%%%%%%%%%%%%%%%%%%%%%%%%%%%%%%%%%%%%%%%%%
%%%%%%%%%%%%%%%%%%%%%%%%%%%%%%%%%%%%%%%%%%%%%%%%%%%%%%%%%%%%%%%%%%%%%%%%%%%%%%%%
%%%%%%%%%%%%%%%%%%%%%%%%%%%%%%%%%%%%%%%%%%%%%%%%%%%%%%%%%%%%%%%%%%%%%%%%%%%%%%%%
%%%%%%%%%%%%%%%%%%%%%%%%%%%%%%%%%%%%%%%%%%%%%%%%%%%%%%%%%%%%%%%%%%%%%%%%%%%%%%%%
\begin{mdframed}[style=darkQuesion]
  3.   For each 1:1 and onto function in Exercise 1, find the inverse of the
  function
  \begin{itemize}
    \item [(a)]{
          $f:\R\to \R; \fof{x}=x+3$
          }
    \item [(b)]{
          $f:\C\to \C; \fof{x}=x^2+2x+1$
          }
    \item [(c]{
          $f:\Z_n\to \Z_n; \fof{\lrb{x}_n}=\lrb{mx+b}_n, \text{where} m,b\in \Z$
          }
    \item [(d)]{
          $f:\R^+\to \R; \fof{x}=\ln x$
          }
  \end{itemize}
\end{mdframed}

%%%%%%%%%%%%%%%%%%%%%%%%%%%%%%%%%%%%%%%%%%%%%%%%%%%%%%%%%%%%%%%%%%%%%%%%%%%%%%%%
\begin{mdframed}[style=darkAnswer,frametitle={Joe Starr}]
  \begin{multicols}{2}
    \begin{itemize}
      \item [(a)]{
            see question 1
            }
      \item [(b)]{
            not a bijection
            }
      \item [(c]{
            see question 1
            }
      \item [(d)]{
            see question 1
            }
    \end{itemize}
  \end{multicols}
\end{mdframed}
\newpage
%%%%%%%%%%%%%%%%%%%%%%%%%%%%%%%%%%%%%%%%%%%%%%%%%%%%%%%%%%%%%%%%%%%%%%%%%%%%%%%%
%%%%%%%%%%%%%%%%%%%%%%%%%%%%%%%%%%%%%%%%%%%%%%%%%%%%%%%%%%%%%%%%%%%%%%%%%%%%%%%%
%%%%%%%%%%%%%%%%%%%%%%%%%%%%%%%%%%%%%%%%%%%%%%%%%%%%%%%%%%%%%%%%%%%%%%%%%%%%%%%%
%%%%%%%%%%%%%%%%%%%%%%%%%%%%%%%%%%%%%%%%%%%%%%%%%%%%%%%%%%%%%%%%%%%%%%%%%%%%%%%%
\begin{mdframed}[style=darkQuesion]
  4.   For each 1:1 and onto function in Exercise 2, find the inverse of the
  function
  \begin{itemize}
    \item [(a)]{
          $f:\R\to \R; \fof{x}=x^2$
          }
    \item [(b)]{
          $f:\C\to \C; \fof{x}=x^2$
          }
    \item [(c]{
          $f:\R^+\to \R^+; \fof{x}=x^2$
          }
    \item [(d)]{
          $f:\R^+\to \R^+; \fof{x}=
            \begin{cases}
              x   & \text{if }x\text{ is rational}   \\
              x^2 & \text{if }x\text{ is irrational} \\
            \end{cases}$
          }
  \end{itemize}
\end{mdframed}

%%%%%%%%%%%%%%%%%%%%%%%%%%%%%%%%%%%%%%%%%%%%%%%%%%%%%%%%%%%%%%%%%%%%%%%%%%%%%%%%
\begin{mdframed}[style=darkAnswer,frametitle={Joe Starr}]
  \begin{itemize}
    \item [(a)]{
          Not a bijection
          }
    \item [(b)]{
          Not a bijection
          }
    \item [(c]{
          $\fofinv{x}=+\sqrt{x}$
          }
    \item [(d)]{
          $f:\R^+\to \R^+; \fofinv{x}=
            \begin{cases}
              x         & \text{if }x\text{ is rational}   \\
              +\sqrt{x} & \text{if }x\text{ is irrational} \\
            \end{cases}$
          }
  \end{itemize}
\end{mdframed}
\newpage
%%%%%%%%%%%%%%%%%%%%%%%%%%%%%%%%%%%%%%%%%%%%%%%%%%%%%%%%%%%%%%%%%%%%%%%%%%%%%%%%
%%%%%%%%%%%%%%%%%%%%%%%%%%%%%%%%%%%%%%%%%%%%%%%%%%%%%%%%%%%%%%%%%%%%%%%%%%%%%%%%
%%%%%%%%%%%%%%%%%%%%%%%%%%%%%%%%%%%%%%%%%%%%%%%%%%%%%%%%%%%%%%%%%%%%%%%%%%%%%%%%
%%%%%%%%%%%%%%%%%%%%%%%%%%%%%%%%%%%%%%%%%%%%%%%%%%%%%%%%%%%%%%%%%%%%%%%%%%%%%%%%
\begin{mdframed}[style=darkQuesion]
  13. Let $f:A\to B$ be a function, and let $\fof{A}=\lrs{\fof{a}\vert a\in A}$
  be the image of $f$. Show that $f$is onto if and only if $\fof{A}=B$.
\end{mdframed}

%%%%%%%%%%%%%%%%%%%%%%%%%%%%%%%%%%%%%%%%%%%%%%%%%%%%%%%%%%%%%%%%%%%%%%%%%%%%%%%%
\begin{mdframed}[style=darkAnswer,frametitle={Joe Starr}]
  Let $\fof{A}=B$, select $y\in B$, since $y\in B$ we have $y\in \fof{A}$, that
  means there exists $a\in A$ such that $\fof{a}=y$. Showing $f$ surjective.
  Let $\fof{A}\neq B$, let $y\in B$, such that $y\notin B \cap \fof{A}$. Since
  we
  have $y\notin \fof{A}$ we have no $a\in A$ that maps to $y$ showing $f$ not
  surjective, as desired.
\end{mdframed}
\newpage
%%%%%%%%%%%%%%%%%%%%%%%%%%%%%%%%%%%%%%%%%%%%%%%%%%%%%%%%%%%%%%%%%%%%%%%%%%%%%%%%
%%%%%%%%%%%%%%%%%%%%%%%%%%%%%%%%%%%%%%%%%%%%%%%%%%%%%%%%%%%%%%%%%%%%%%%%%%%%%%%%
%%%%%%%%%%%%%%%%%%%%%%%%%%%%%%%%%%%%%%%%%%%%%%%%%%%%%%%%%%%%%%%%%%%%%%%%%%%%%%%%
%%%%%%%%%%%%%%%%%%%%%%%%%%%%%%%%%%%%%%%%%%%%%%%%%%%%%%%%%%%%%%%%%%%%%%%%%%%%%%%%
\begin{mdframed}[style=darkQuesion]
  15. Let $f:A\to B$ and $g:B\to C$ be functions. Prove that if $g\circ f$ is
  1:1, then $f$ is 1:1, and that if $g\circ f$ is onto $g$ is onto.
\end{mdframed}

%%%%%%%%%%%%%%%%%%%%%%%%%%%%%%%%%%%%%%%%%%%%%%%%%%%%%%%%%%%%%%%%%%%%%%%%%%%%%%%%
\begin{mdframed}[style=darkAnswer,frametitle={Joe Starr}]
  Let $g\circ f$ be injective, but $f$ not injective. Since $f$ is not injective
  $\exists a,x\in A$ such tha $a\neq x$ but $\fof{x}=\fof{a}$. We consider
  $g\circ f\lrp{a}$ and $g\circ f\lrp{x}$, since $\fof{x}=\fof{a}$ it must be
  that $\gof{\fof{x}}=\gof{\fof{a}}$. This means that with $a\neq b$,
  $\gof{\fof{x}}=\gof{\fof{a}}$, making $g\circ f$ not injective a contradiction
  so $f$ injective.

  Let $g\circ f$ be surjective, but $g$ not surjective. Since $g$ not surjective
  there exists some $c\in C$ such that $\gof{b}\neq c$ for all $b\in B$. However
  since $g\circ f$ surjective there exists $g\circ f\lrp{a}=c$ a contradiction,
  making $g$ surjective.
\end{mdframed}
\newpage
%%%%%%%%%%%%%%%%%%%%%%%%%%%%%%%%%%%%%%%%%%%%%%%%%%%%%%%%%%%%%%%%%%%%%%%%%%%%%%%%
%%%%%%%%%%%%%%%%%%%%%%%%%%%%%%%%%%%%%%%%%%%%%%%%%%%%%%%%%%%%%%%%%%%%%%%%%%%%%%%%
%%%%%%%%%%%%%%%%%%%%%%%%%%%%%%%%%%%%%%%%%%%%%%%%%%%%%%%%%%%%%%%%%%%%%%%%%%%%%%%%
%%%%%%%%%%%%%%%%%%%%%%%%%%%%%%%%%%%%%%%%%%%%%%%%%%%%%%%%%%%%%%%%%%%%%%%%%%%%%%%%
\begin{mdframed}[style=darkQuesion]
  17. Let $f:A\to B$ be a function. Prove that $f$ is onto if and only if
  $h\circ f = k \circ f$ implies $h=k$, for every set $C$ and all choices of
  functions $h:B\to C$ and $k:B\to C$.
\end{mdframed}

%%%%%%%%%%%%%%%%%%%%%%%%%%%%%%%%%%%%%%%%%%%%%%%%%%%%%%%%%%%%%%%%%%%%%%%%%%%%%%%%
\begin{mdframed}[style=darkAnswer,frametitle={Joe Starr}]
  Assume $f$ is surjective, that is for all $y\in B$ there exists $x\in A$ such
  that $\fof{x}=y$. We know $\hof{\fof{x}}=\kof{\fof{x}}$, so
  $\hof{y}=\kof{y}$ for all $y$ in the domain of $f$ as desired.

  Next assume $f$ is not surjective, then for some $y\in B$ there exists no $x$
  such that $\fof{x}=y$. We can select $C=\lrs{a,b}$. We say that $\hof{c}=a$
  now we construct $k$,
  $$\kof{x}=\begin{cases}
      b & \text{if }x=y     \\
      a & \text{if }x\neq y \\
    \end{cases}$$
  from here we have that when the input of $h$ and $k$ are in the domain of $f$
  $h\neq k$, as desired.
\end{mdframed}
\newpage
%%%%%%%%%%%%%%%%%%%%%%%%%%%%%%%%%%%%%%%%%%%%%%%%%%%%%%%%%%%%%%%%%%%%%%%%%%%%%%%%
%%%%%%%%%%%%%%%%%%%%%%%%%%%%%%%%%%%%%%%%%%%%%%%%%%%%%%%%%%%%%%%%%%%%%%%%%%%%%%%%
%%%%%%%%%%%%%%%%%%%%%%%%%%%%%%%%%%%%%%%%%%%%%%%%%%%%%%%%%%%%%%%%%%%%%%%%%%%%%%%%
%%%%%%%%%%%%%%%%%%%%%%%%%%%%%%%%%%%%%%%%%%%%%%%%%%%%%%%%%%%%%%%%%%%%%%%%%%%%%%%%
\begin{mdframed}[style=darkQuesion]
  19. Let $f:A\to B$ be a function. Prove that $f$ is 1:1 if and only if
  $f\circ h=f\circ k$ implies $h=k$, for every set $C$ and all choices of
  functions $h:C\to A$ and $k:C\to A$.
\end{mdframed}

%%%%%%%%%%%%%%%%%%%%%%%%%%%%%%%%%%%%%%%%%%%%%%%%%%%%%%%%%%%%%%%%%%%%%%%%%%%%%%%%
\begin{mdframed}[style=darkAnswer,frametitle={Joe Starr}]
  Assume that $f$ is injective, this means that $\fof{x}=\fof{y}$ implies $x=y$.
  Also assume $f\circ h=f\circ k$ for some $h,k$. We assume $h\neq k$ for some
  $c\in C$. Since we know $f$ is injective we have
  $\fof{\hof{c}}\neq \fof{\kof{c}}$, a contradiction from our assumption that
  $f\circ h=f\circ k$ meaning $h=k$ as desired.

  Assume $f\circ h=f\circ k$ implies $h=k$. suppose $f$ be not injective, this means
  that there exists some $x,y$ such that $\fof{x}=\fof{y}$ but $x\neq y$.
  We can select $C=\lrs{a,b}$, and define $h$, $k$:
  \begin{align*}
    \kof{a}=x & \hspace{.5in}  \hof{a}=y \\
    \kof{b}=y & \hspace{.5in} \hof{b}=x
  \end{align*}
  We can see from here we have that for $\fof{\hof{a}}=\fof{\kof{a}}$ but
  $h\neq k$ a contradiction making $f$ injective as desired.
\end{mdframed}
\newpage
%%%%%%%%%%%%%%%%%%%%%%%%%%%%%%%%%%%%%%%%%%%%%%%%%%%%%%%%%%%%%%%%%%%%%%%%%%%%%%%%
%%%%%%%%%%%%%%%%%%%%%%%%%%%%%%%%%%%%%%%%%%%%%%%%%%%%%%%%%%%%%%%%%%%%%%%%%%%%%%%%
%%%%%%%%%%%%%%%%%%%%%%%%%%%%%%%%%%%%%%%%%%%%%%%%%%%%%%%%%%%%%%%%%%%%%%%%%%%%%%%%
%%%%%%%%%%%%%%%%%%%%%%%%%%%%%%%%%%%%%%%%%%%%%%%%%%%%%%%%%%%%%%%%%%%%%%%%%%%%%%%%
\subsection{Equivalence Relations}
%%%%%%%%%%%%%%%%%%%%%%%%%%%%%%%%%%%%%%%%%%%%%%%%%%%%%%%%%%%%%%%%%%%%%%%%%%%%%%%%
%%%%%%%%%%%%%%%%%%%%%%%%%%%%%%%%%%%%%%%%%%%%%%%%%%%%%%%%%%%%%%%%%%%%%%%%%%%%%%%%
%%%%%%%%%%%%%%%%%%%%%%%%%%%%%%%%%%%%%%%%%%%%%%%%%%%%%%%%%%%%%%%%%%%%%%%%%%%%%%%%
%%%%%%%%%%%%%%%%%%%%%%%%%%%%%%%%%%%%%%%%%%%%%%%%%%%%%%%%%%%%%%%%%%%%%%%%%%%%%%%%
\newpage
%%%%%%%%%%%%%%%%%%%%%%%%%%%%%%%%%%%%%%%%%%%%%%%%%%%%%%%%%%%%%%%%%%%%%%%%%%%%%%%%
%%%%%%%%%%%%%%%%%%%%%%%%%%%%%%%%%%%%%%%%%%%%%%%%%%%%%%%%%%%%%%%%%%%%%%%%%%%%%%%%
%%%%%%%%%%%%%%%%%%%%%%%%%%%%%%%%%%%%%%%%%%%%%%%%%%%%%%%%%%%%%%%%%%%%%%%%%%%%%%%%
%%%%%%%%%%%%%%%%%%%%%%%%%%%%%%%%%%%%%%%%%%%%%%%%%%%%%%%%%%%%%%%%%%%%%%%%%%%%%%%%
\subsection{Permutations} %1,2,3,4,5
%%%%%%%%%%%%%%%%%%%%%%%%%%%%%%%%%%%%%%%%%%%%%%%%%%%%%%%%%%%%%%%%%%%%%%%%%%%%%%%%
%%%%%%%%%%%%%%%%%%%%%%%%%%%%%%%%%%%%%%%%%%%%%%%%%%%%%%%%%%%%%%%%%%%%%%%%%%%%%%%%
%%%%%%%%%%%%%%%%%%%%%%%%%%%%%%%%%%%%%%%%%%%%%%%%%%%%%%%%%%%%%%%%%%%%%%%%%%%%%%%%
%%%%%%%%%%%%%%%%%%%%%%%%%%%%%%%%%%%%%%%%%%%%%%%%%%%%%%%%%%%%%%%%%%%%%%%%%%%%%%%%
\begin{mdframed}[style=darkQuesion]
  1. Consider the following Permutations in $S_7$.
  \begin{multicols}{2}
    $\sigma=
      \begin{pmatrix}
        1 & 2 & 3 & 4 & 5 & 6 & 7 \\
        3 & 2 & 5 & 4 & 6 & 1 & 7 \\
      \end{pmatrix}$
    $\tau=
      \begin{pmatrix}
        1 & 2 & 3 & 4 & 5 & 6 & 7 \\
        2 & 1 & 5 & 7 & 4 & 6 & 3 \\
      \end{pmatrix}$
  \end{multicols}
  \vspace{.25in}
  \begin{multicols}{4}
    \begin{itemize}
      \item [(a)]{$\sigma\tau$

            }
      \item [(b)]{$\tau\sigma$

            }
      \item [(c)]{$\tau^2\sigma$

            }
      \item [(d)]{$\sigma^{-1}$

            }
      \item [(e)]{$\sigma\tau\sigma^{-1}$

            }
      \item [(f)]{$\tau^{-1}\sigma\tau$

            }
    \end{itemize}
  \end{multicols}
\end{mdframed}

%%%%%%%%%%%%%%%%%%%%%%%%%%%%%%%%%%%%%%%%%%%%%%%%%%%%%%%%%%%%%%%%%%%%%%%%%%%%%%%%
\begin{mdframed}[style=darkAnswer,frametitle={Joe Starr}]
  \begin{multicols}{2}
    \begin{itemize}
      \item [(a)]{
            $\sigma\tau=
              \begin{pmatrix}
                1 & 2 & 3 & 4 & 5 & 6 & 7 \\
                2 & 3 & 6 & 7 & 4 & 1 & 5 \\
              \end{pmatrix}$
            }
      \item [(b)]{
            $\tau\sigma=
              \begin{pmatrix}
                1 & 2 & 3 & 4 & 5 & 6 & 7 \\
                5 & 1 & 4 & 7 & 6 & 2 & 3 \\
              \end{pmatrix}$
            }
      \item [(c)]{
            $\tau^2\sigma=
              \begin{pmatrix}
                1 & 2 & 3 & 4 & 5 & 6 & 7 \\
                4 & 2 & 7 & 3 & 6 & 1 & 5 \\
              \end{pmatrix}$
            }
      \item [(d)]{
            $\sigma^{-1}=
              \begin{pmatrix}
                1 & 2 & 3 & 4 & 5 & 6 & 7 \\
                6 & 2 & 1 & 4 & 3 & 5 & 7 \\
              \end{pmatrix}$
            }
      \item [(e)]{
            $\sigma\tau\sigma^{-1}=
              \begin{pmatrix}
                1 & 2 & 3 & 4 & 5 & 6 & 7 \\
                1 & 3 & 2 & 7 & 6 & 4 & 5 \\
              \end{pmatrix}$
            }
      \item [(f)]{
            $\tau^{-1}\sigma\tau=
              \begin{pmatrix}
                1 & 2 & 3 & 4 & 5 & 6 & 7 \\
                3 & 2 & 5 & 4 & 6 & 1 & 7 \\

                %tau inv 1 & 2 & 3 & 4 & 5 & 6 & 7 \\
                %        2 & 1 & 7 & 5 & 3 & 6 & 4 \\
              \end{pmatrix}$
            }
    \end{itemize}
  \end{multicols}
\end{mdframed}
\newpage
%%%%%%%%%%%%%%%%%%%%%%%%%%%%%%%%%%%%%%%%%%%%%%%%%%%%%%%%%%%%%%%%%%%%%%%%%%%%%%%%
%%%%%%%%%%%%%%%%%%%%%%%%%%%%%%%%%%%%%%%%%%%%%%%%%%%%%%%%%%%%%%%%%%%%%%%%%%%%%%%%
%%%%%%%%%%%%%%%%%%%%%%%%%%%%%%%%%%%%%%%%%%%%%%%%%%%%%%%%%%%%%%%%%%%%%%%%%%%%%%%%
%%%%%%%%%%%%%%%%%%%%%%%%%%%%%%%%%%%%%%%%%%%%%%%%%%%%%%%%%%%%%%%%%%%%%%%%%%%%%%%%
\begin{mdframed}[style=darkQuesion]
  2. Write each of the permutations $\sigma\tau, \tau\sigma, \tau^2\sigma,
    \sigma^{-1}, \sigma\tau\sigma^{-1}, \text{ and } \tau^{-1}\sigma\tau$ in
  Exercise 1 as a product of disjoint cycles. Write $\sigma$ and $\tau$ as
  products of transpositions.
\end{mdframed}

%%%%%%%%%%%%%%%%%%%%%%%%%%%%%%%%%%%%%%%%%%%%%%%%%%%%%%%%%%%%%%%%%%%%%%%%%%%%%%%%
\begin{mdframed}[style=darkAnswer,frametitle={Joe Starr}]
  \begin{multicols}{2}
    \begin{itemize}
      \item [(a)]{
            $\sigma\tau=\lrp{1236}\lrp{475}$
            }
      \item [(b)]{
            $\tau\sigma=\lrp{1562}\lrp{347}$

            }
      \item [(c)]{
            $\tau^2\sigma=\lrp{143756}$

            }
      \item [(d)]{
            $\sigma^{-1}=\lrp{1653}$

            }
      \item [(e)]{
            $\sigma\tau\sigma^{-1}=\lrp{23}\lrp{4756}$

            }
      \item [(f)]{
            $\tau^{-1}\sigma\tau=\lrp{1356}$
            }
      \item [($\sigma$)]{
            $\sigma=\lrp{13}\lrp{35}\lrp{56}$
            }
      \item [($\tau$)]{
            $\tau=\lrp{12}\lrp{35}\lrp{54}\lrp{47}\lrp{73}$
            }
    \end{itemize}
  \end{multicols}
\end{mdframed}
\newpage
%%%%%%%%%%%%%%%%%%%%%%%%%%%%%%%%%%%%%%%%%%%%%%%%%%%%%%%%%%%%%%%%%%%%%%%%%%%%%%%%
%%%%%%%%%%%%%%%%%%%%%%%%%%%%%%%%%%%%%%%%%%%%%%%%%%%%%%%%%%%%%%%%%%%%%%%%%%%%%%%%
%%%%%%%%%%%%%%%%%%%%%%%%%%%%%%%%%%%%%%%%%%%%%%%%%%%%%%%%%%%%%%%%%%%%%%%%%%%%%%%%
%%%%%%%%%%%%%%%%%%%%%%%%%%%%%%%%%%%%%%%%%%%%%%%%%%%%%%%%%%%%%%%%%%%%%%%%%%%%%%%%
\begin{mdframed}[style=darkQuesion]
  3. Write
  $\begin{pmatrix}
      1 & 2 & 3  & 4 & 5 & 6 & 7 & 8 & 9 & 10 \\
      3 & 4 & 10 & 5 & 7 & 8 & 2 & 6 & 9 & 1  \\
    \end{pmatrix}$ as the product of disjoint cycles and as a product of
  transpositions. Construct its associated diagram, find its inverse, and find
  it's order.
\end{mdframed}

%%%%%%%%%%%%%%%%%%%%%%%%%%%%%%%%%%%%%%%%%%%%%%%%%%%%%%%%%%%%%%%%%%%%%%%%%%%%%%%%
\begin{mdframed}[style=darkAnswer,frametitle={Joe Starr}]
  \begin{itemize}[align=left]
    \item [Disjoint cycles:]{\hspace{.5in}\newline
          $\lrp{1,3,10}\lrp{2,4,5,7}\lrp{6,8}$
          }
    \item [Transpositions:]{\hspace{.5in}\newline
          $\lrp{1,3}\lrp{3,10}\lrp{2,4}\lrp{4,5}\lrp{5,7}\lrp{6,8}$
          }
    \item [Diagrams:]{\hspace{.5in}\newline
          \begin{tikzpicture}[node distance=2cm]
            % nodes
            \node (A1) at (0, 0) {1};
            \node (B1) at (-1, -2) {3};
            \node (C1) at (1, -2) {10};

            \node (A2) at (3, 0) {2};
            \node (B2) at (5, 0) {4};
            \node (C2) at (5, -2) {5};
            \node (D2) at (3, -2) {7};

            \node (A3) at (7, 0) {6};
            \node (B3) at (9, -2) {8};


            \draw[->]
            (A1) edge (B1) (B1) edge (C1) (C1) edge (A1);
            \draw[->]
            (A2) edge (B2) (B2) edge (C2) (C2) edge (D2) (D2) edge (A2);
            \draw[->, to path={-| (\tikztotarget)}]
            (A3) edge (B3) (B3) edge (A3);
          \end{tikzpicture}
          }
    \item [Inverse:]{\hspace{.5in}\newline\hspace{.5in}\newline
          $\begin{pmatrix}
              1  & 2 & 3 & 4 & 5 & 6 & 7 & 8 & 9 & 10 \\
              10 & 7 & 1 & 2 & 4 & 8 & 5 & 6 & 9 & 3  \\
            \end{pmatrix}$
          }
    \item [Order:]{\hspace{.5in}\newline
          $\lrp{1,3,10}=3, \lrp{2,4,5,7}=4 \lrp{6,8}=2$
          order is $12$
          }
  \end{itemize}
  \begin{multicols}{3}
  \end{multicols}
\end{mdframed}
\newpage
%%%%%%%%%%%%%%%%%%%%%%%%%%%%%%%%%%%%%%%%%%%%%%%%%%%%%%%%%%%%%%%%%%%%%%%%%%%%%%%%
%%%%%%%%%%%%%%%%%%%%%%%%%%%%%%%%%%%%%%%%%%%%%%%%%%%%%%%%%%%%%%%%%%%%%%%%%%%%%%%%
%%%%%%%%%%%%%%%%%%%%%%%%%%%%%%%%%%%%%%%%%%%%%%%%%%%%%%%%%%%%%%%%%%%%%%%%%%%%%%%%
%%%%%%%%%%%%%%%%%%%%%%%%%%%%%%%%%%%%%%%%%%%%%%%%%%%%%%%%%%%%%%%%%%%%%%%%%%%%%%%%
\begin{mdframed}[style=darkQuesion]
  4. Find the oder of each of the following permutations.
  \begin{itemize}
    \item [(a)] {
          $\begin{pmatrix}
              1 & 2 & 3 & 4 & 5 & 6 \\
              6 & 4 & 5 & 3 & 2 & 1 \\
            \end{pmatrix}$
          }
    \item [(b)] {
          $\begin{pmatrix}
              1 & 2 & 3 & 4 & 5 & 6 & 7 & 8 \\
              4 & 6 & 7 & 5 & 1 & 8 & 2 & 3 \\
            \end{pmatrix}$
          }
    \item [(c)] {
          $\begin{pmatrix}
              1 & 2 & 3 & 4 & 5 & 6 & 7 & 8 & 9 \\
              5 & 9 & 8 & 7 & 3 & 4 & 6 & 1 & 2 \\
            \end{pmatrix}$
          }
    \item [(d)] {
          $\begin{pmatrix}
              1 & 2 & 3 & 4 & 5 & 6 & 7 & 8 & 9 \\
              8 & 4 & 9 & 6 & 5 & 2 & 3 & 1 & 7 \\
            \end{pmatrix}$
          }
  \end{itemize}
\end{mdframed}

%%%%%%%%%%%%%%%%%%%%%%%%%%%%%%%%%%%%%%%%%%%%%%%%%%%%%%%%%%%%%%%%%%%%%%%%%%%%%%%%
\begin{mdframed}[style=darkAnswer,frametitle={Joe Starr}]
  \begin{itemize}
    \item [(a)] {
          $\lrp{1,6}\lrp{2,4,2,5}$
          order is 4
          }
    \item [(b)] {
          $\lrp{1,4,5}\lrp{2,6,8,3,7}$
          order 15
          }
    \item [(c)] {
          $\lrp{1,5,3,8}\lrp{2,9}\lrp{4,7,6}$
          order 12
          }
    \item [(d)] {
          $\lrp{1,8}\lrp{2,4,6}\lrp{3,9,7}$
          order 6
          }
  \end{itemize}
\end{mdframed}
\newpage
%%%%%%%%%%%%%%%%%%%%%%%%%%%%%%%%%%%%%%%%%%%%%%%%%%%%%%%%%%%%%%%%%%%%%%%%%%%%%%%%
%%%%%%%%%%%%%%%%%%%%%%%%%%%%%%%%%%%%%%%%%%%%%%%%%%%%%%%%%%%%%%%%%%%%%%%%%%%%%%%%
%%%%%%%%%%%%%%%%%%%%%%%%%%%%%%%%%%%%%%%%%%%%%%%%%%%%%%%%%%%%%%%%%%%%%%%%%%%%%%%%
%%%%%%%%%%%%%%%%%%%%%%%%%%%%%%%%%%%%%%%%%%%%%%%%%%%%%%%%%%%%%%%%%%%%%%%%%%%%%%%%
\begin{mdframed}[style=darkQuesion]
  5. Let $3\leq m\leq n$. Calculate $\sigma\tau^{-1}$ for cycles
  $\sigma= \lrp{1,2,\dots , m-1}$ and \\ $\tau= \lrp{1,2,\dots,m-1,m}$ in $S_n$.
\end{mdframed}

%%%%%%%%%%%%%%%%%%%%%%%%%%%%%%%%%%%%%%%%%%%%%%%%%%%%%%%%%%%%%%%%%%%%%%%%%%%%%%%%
\begin{mdframed}[style=darkAnswer,frametitle={Joe Starr}]
  We begin with finding $\tau^{-1}$. We take $\tau$ of, $\lrb{1,2,\dots,m-1,m}$,
  we get $\lrb{2,3,\dots,m,1}$. If we now apply
  $\tau^{-1}=\lrp{m,1,2,\dots, m-1}$, we get $\lrb{1,2,\dots,m-1,m}$.

  We can now compose $\tau^{-1}$ and $\sigma$ yielding
  $\lrp{m,m-1,1,2,\dots,m-3,m-2}$.
\end{mdframed}
\newpage

\clearpage
%Compile Chapter 3
\section{Groups}
\subsection{Definition of a Group}
%3.1: 1 2 3 5 8-18 22 24 25 26
%3.2: 1-4 6-20 22 23 26 27 28 29
%%%%%%%%%%%%%%%%%%%%%%%%%%%%%%%%%%%%%%%%%%%%%%%%%%%%%%%%%%%%%%%%%%%%%%%%%%%%%%%%
%%%%%%%%%%%%%%%%%%%%%%%%%%%%%%%%%%%%%%%%%%%%%%%%%%%%%%%%%%%%%%%%%%%%%%%%%%%%%%%%
%%%%%%%%%%%%%%%%%%%%%%%%%%%%%%%%%%%%%%%%%%%%%%%%%%%%%%%%%%%%%%%%%%%%%%%%%%%%%%%%
%%%%%%%%%%%%%%%%%%%%%%%%%%%%%%%%%%%%%%%%%%%%%%%%%%%%%%%%%%%%%%%%%%%%%%%%%%%%%%%%
\begin{mdframed}[style=darkQuesion]
  1. Using ordinary addition of integers as the operation, show that the set of
  even integers is a group but the set of odd integers is not.
\end{mdframed}

%%%%%%%%%%%%%%%%%%%%%%%%%%%%%%%%%%%%%%%%%%%%%%%%%%%%%%%%%%%%%%%%%%%%%%%%%%%%%%%%
\begin{mdframed}[style=darkAnswer,frametitle={Joe Starr}]
  We begin considering the even integers, that is integers of the form $2k$. We
  must also include 0 in the even integers. We get the identity element as well
  as Associativity and inverses for free from integer addition on $\Z$.
  We then consider closure. Let $n$ and $m$ be even integers, if we take $m+n$
  we can see we have, $m+n=2k_m+2k_n=2(k_m+k_n)$ an even integer. Making the
  even integers a group under addition.

  Next we consider the odd integers, take $3+3=2(3)$ an even integer, showing
  odds are not closed under addition and not a group.
\end{mdframed}
\newpage
%%%%%%%%%%%%%%%%%%%%%%%%%%%%%%%%%%%%%%%%%%%%%%%%%%%%%%%%%%%%%%%%%%%%%%%%%%%%%%%%
%%%%%%%%%%%%%%%%%%%%%%%%%%%%%%%%%%%%%%%%%%%%%%%%%%%%%%%%%%%%%%%%%%%%%%%%%%%%%%%%
%%%%%%%%%%%%%%%%%%%%%%%%%%%%%%%%%%%%%%%%%%%%%%%%%%%%%%%%%%%%%%%%%%%%%%%%%%%%%%%%
%%%%%%%%%%%%%%%%%%%%%%%%%%%%%%%%%%%%%%%%%%%%%%%%%%%%%%%%%%%%%%%%%%%%%%%%%%%%%%%%
\begin{mdframed}[style=darkQuesion]
  2. For each binary operation $\ast$ defined on a set below,
  determine whether or not $\ast$ gives a group structure on the set.
  If it is not a group, say which axioms fail to hold.
  \begin{multicols}{2}
    \begin{itemize}
      \item[(a)]{Define $\ast$ on $\Z$ by $a\ast b=ab$.}
      \item[(b)]{Define $\ast$ on $\Z$ by $a\ast b=\max{a,b}$.}
      \item[(c)]{Define $\ast$ on $\Z$ by $a\ast b=a-b$.}
      \item[(d)]{Define $\ast$ on $\Z$ by $a\ast b=\abs{ab}$.}
      \item[(e)]{Define $\ast$ on $\R^{+}$ by $a\ast b=ab$.}
      \item[(f)]{Define $\ast$ on $\Q$ by $a\ast b=ab$.}
    \end{itemize}
  \end{multicols}
\end{mdframed}

%%%%%%%%%%%%%%%%%%%%%%%%%%%%%%%%%%%%%%%%%%%%%%%%%%%%%%%%%%%%%%%%%%%%%%%%%%%%%%%%
\begin{mdframed}[style=darkAnswer,frametitle={Joe Starr}]
  \begin{multicols}{2}
    \begin{itemize}[align=left]
      \item[(a)]{
            \begin{itemize}[align=left]
              \Assoc{$a\ast \lrp{b\ast c}=\lrp{a\ast b}\ast c$}
              \Invs{}
              \Clos{}
              \Ident{}
            \end{itemize}
            }
      \item[(b)]{
            \begin{itemize}[align=left]
              \Assoc{}
              \Invs{}
              \Clos{}
              \Ident{}
            \end{itemize}
            }
      \item[(c)]{
            \begin{itemize}[align=left]
              \Assoc{}
              \Invs{}
              \Clos{}
              \Ident{}
            \end{itemize}
            }
      \item[(d)]{
            \begin{itemize}[align=left]
              \Assoc{}
              \Invs{}
              \Clos{}
              \Ident{}
            \end{itemize}
            }
      \item[(e)]{
            \begin{itemize}[align=left]
              \Assoc{}
              \Invs{}
              \Clos{}
              \Ident{}
            \end{itemize}
            }
      \item[(f)]{
            \begin{itemize}[align=left]
              \Assoc{}
              \Invs{}
              \Clos{}
              \Ident{}
            \end{itemize}
            }
    \end{itemize}
  \end{multicols}
\end{mdframed}
\newpage
%%%%%%%%%%%%%%%%%%%%%%%%%%%%%%%%%%%%%%%%%%%%%%%%%%%%%%%%%%%%%%%%%%%%%%%%%%%%%%%%
%%%%%%%%%%%%%%%%%%%%%%%%%%%%%%%%%%%%%%%%%%%%%%%%%%%%%%%%%%%%%%%%%%%%%%%%%%%%%%%%
%%%%%%%%%%%%%%%%%%%%%%%%%%%%%%%%%%%%%%%%%%%%%%%%%%%%%%%%%%%%%%%%%%%%%%%%%%%%%%%%
%%%%%%%%%%%%%%%%%%%%%%%%%%%%%%%%%%%%%%%%%%%%%%%%%%%%%%%%%%%%%%%%%%%%%%%%%%%%%%%%
\begin{mdframed}[style=darkQuesion]
  2.
\end{mdframed}

%%%%%%%%%%%%%%%%%%%%%%%%%%%%%%%%%%%%%%%%%%%%%%%%%%%%%%%%%%%%%%%%%%%%%%%%%%%%%%%%
\begin{mdframed}[style=darkAnswer,frametitle={Joe Starr}]

\end{mdframed}
\newpage
%%%%%%%%%%%%%%%%%%%%%%%%%%%%%%%%%%%%%%%%%%%%%%%%%%%%%%%%%%%%%%%%%%%%%%%%%%%%%%%%
%%%%%%%%%%%%%%%%%%%%%%%%%%%%%%%%%%%%%%%%%%%%%%%%%%%%%%%%%%%%%%%%%%%%%%%%%%%%%%%%
%%%%%%%%%%%%%%%%%%%%%%%%%%%%%%%%%%%%%%%%%%%%%%%%%%%%%%%%%%%%%%%%%%%%%%%%%%%%%%%%
%%%%%%%%%%%%%%%%%%%%%%%%%%%%%%%%%%%%%%%%%%%%%%%%%%%%%%%%%%%%%%%%%%%%%%%%%%%%%%%%
\begin{mdframed}[style=darkQuesion]
  2.
\end{mdframed}

%%%%%%%%%%%%%%%%%%%%%%%%%%%%%%%%%%%%%%%%%%%%%%%%%%%%%%%%%%%%%%%%%%%%%%%%%%%%%%%%
\begin{mdframed}[style=darkAnswer,frametitle={Joe Starr}]

\end{mdframed}
\newpage
%%%%%%%%%%%%%%%%%%%%%%%%%%%%%%%%%%%%%%%%%%%%%%%%%%%%%%%%%%%%%%%%%%%%%%%%%%%%%%%%
%%%%%%%%%%%%%%%%%%%%%%%%%%%%%%%%%%%%%%%%%%%%%%%%%%%%%%%%%%%%%%%%%%%%%%%%%%%%%%%%
%%%%%%%%%%%%%%%%%%%%%%%%%%%%%%%%%%%%%%%%%%%%%%%%%%%%%%%%%%%%%%%%%%%%%%%%%%%%%%%%
%%%%%%%%%%%%%%%%%%%%%%%%%%%%%%%%%%%%%%%%%%%%%%%%%%%%%%%%%%%%%%%%%%%%%%%%%%%%%%%%
\begin{mdframed}[style=darkQuesion]
  2.
\end{mdframed}

%%%%%%%%%%%%%%%%%%%%%%%%%%%%%%%%%%%%%%%%%%%%%%%%%%%%%%%%%%%%%%%%%%%%%%%%%%%%%%%%
\begin{mdframed}[style=darkAnswer,frametitle={Joe Starr}]

\end{mdframed}
\newpage
%%%%%%%%%%%%%%%%%%%%%%%%%%%%%%%%%%%%%%%%%%%%%%%%%%%%%%%%%%%%%%%%%%%%%%%%%%%%%%%%
%%%%%%%%%%%%%%%%%%%%%%%%%%%%%%%%%%%%%%%%%%%%%%%%%%%%%%%%%%%%%%%%%%%%%%%%%%%%%%%%
%%%%%%%%%%%%%%%%%%%%%%%%%%%%%%%%%%%%%%%%%%%%%%%%%%%%%%%%%%%%%%%%%%%%%%%%%%%%%%%%
%%%%%%%%%%%%%%%%%%%%%%%%%%%%%%%%%%%%%%%%%%%%%%%%%%%%%%%%%%%%%%%%%%%%%%%%%%%%%%%%
\begin{mdframed}[style=darkQuesion]
  2.
\end{mdframed}

%%%%%%%%%%%%%%%%%%%%%%%%%%%%%%%%%%%%%%%%%%%%%%%%%%%%%%%%%%%%%%%%%%%%%%%%%%%%%%%%
\begin{mdframed}[style=darkAnswer,frametitle={Joe Starr}]

\end{mdframed}
\newpage
%%%%%%%%%%%%%%%%%%%%%%%%%%%%%%%%%%%%%%%%%%%%%%%%%%%%%%%%%%%%%%%%%%%%%%%%%%%%%%%%
%%%%%%%%%%%%%%%%%%%%%%%%%%%%%%%%%%%%%%%%%%%%%%%%%%%%%%%%%%%%%%%%%%%%%%%%%%%%%%%%
%%%%%%%%%%%%%%%%%%%%%%%%%%%%%%%%%%%%%%%%%%%%%%%%%%%%%%%%%%%%%%%%%%%%%%%%%%%%%%%%
%%%%%%%%%%%%%%%%%%%%%%%%%%%%%%%%%%%%%%%%%%%%%%%%%%%%%%%%%%%%%%%%%%%%%%%%%%%%%%%%
\begin{mdframed}[style=darkQuesion]
  2.
\end{mdframed}

%%%%%%%%%%%%%%%%%%%%%%%%%%%%%%%%%%%%%%%%%%%%%%%%%%%%%%%%%%%%%%%%%%%%%%%%%%%%%%%%
\begin{mdframed}[style=darkAnswer,frametitle={Joe Starr}]

\end{mdframed}
\newpage
%%%%%%%%%%%%%%%%%%%%%%%%%%%%%%%%%%%%%%%%%%%%%%%%%%%%%%%%%%%%%%%%%%%%%%%%%%%%%%%%
%%%%%%%%%%%%%%%%%%%%%%%%%%%%%%%%%%%%%%%%%%%%%%%%%%%%%%%%%%%%%%%%%%%%%%%%%%%%%%%%
%%%%%%%%%%%%%%%%%%%%%%%%%%%%%%%%%%%%%%%%%%%%%%%%%%%%%%%%%%%%%%%%%%%%%%%%%%%%%%%%
%%%%%%%%%%%%%%%%%%%%%%%%%%%%%%%%%%%%%%%%%%%%%%%%%%%%%%%%%%%%%%%%%%%%%%%%%%%%%%%%
\begin{mdframed}[style=darkQuesion]
  2.
\end{mdframed}

%%%%%%%%%%%%%%%%%%%%%%%%%%%%%%%%%%%%%%%%%%%%%%%%%%%%%%%%%%%%%%%%%%%%%%%%%%%%%%%%
\begin{mdframed}[style=darkAnswer,frametitle={Joe Starr}]

\end{mdframed}
\newpage
%%%%%%%%%%%%%%%%%%%%%%%%%%%%%%%%%%%%%%%%%%%%%%%%%%%%%%%%%%%%%%%%%%%%%%%%%%%%%%%%
%%%%%%%%%%%%%%%%%%%%%%%%%%%%%%%%%%%%%%%%%%%%%%%%%%%%%%%%%%%%%%%%%%%%%%%%%%%%%%%%
%%%%%%%%%%%%%%%%%%%%%%%%%%%%%%%%%%%%%%%%%%%%%%%%%%%%%%%%%%%%%%%%%%%%%%%%%%%%%%%%
%%%%%%%%%%%%%%%%%%%%%%%%%%%%%%%%%%%%%%%%%%%%%%%%%%%%%%%%%%%%%%%%%%%%%%%%%%%%%%%%
\begin{mdframed}[style=darkQuesion]
  2.
\end{mdframed}

%%%%%%%%%%%%%%%%%%%%%%%%%%%%%%%%%%%%%%%%%%%%%%%%%%%%%%%%%%%%%%%%%%%%%%%%%%%%%%%%
\begin{mdframed}[style=darkAnswer,frametitle={Joe Starr}]

\end{mdframed}
\newpage
%%%%%%%%%%%%%%%%%%%%%%%%%%%%%%%%%%%%%%%%%%%%%%%%%%%%%%%%%%%%%%%%%%%%%%%%%%%%%%%%
%%%%%%%%%%%%%%%%%%%%%%%%%%%%%%%%%%%%%%%%%%%%%%%%%%%%%%%%%%%%%%%%%%%%%%%%%%%%%%%%
%%%%%%%%%%%%%%%%%%%%%%%%%%%%%%%%%%%%%%%%%%%%%%%%%%%%%%%%%%%%%%%%%%%%%%%%%%%%%%%%
%%%%%%%%%%%%%%%%%%%%%%%%%%%%%%%%%%%%%%%%%%%%%%%%%%%%%%%%%%%%%%%%%%%%%%%%%%%%%%%%
\begin{mdframed}[style=darkQuesion]
  2.
\end{mdframed}

%%%%%%%%%%%%%%%%%%%%%%%%%%%%%%%%%%%%%%%%%%%%%%%%%%%%%%%%%%%%%%%%%%%%%%%%%%%%%%%%
\begin{mdframed}[style=darkAnswer,frametitle={Joe Starr}]

\end{mdframed}
\newpage
%%%%%%%%%%%%%%%%%%%%%%%%%%%%%%%%%%%%%%%%%%%%%%%%%%%%%%%%%%%%%%%%%%%%%%%%%%%%%%%%
%%%%%%%%%%%%%%%%%%%%%%%%%%%%%%%%%%%%%%%%%%%%%%%%%%%%%%%%%%%%%%%%%%%%%%%%%%%%%%%%
%%%%%%%%%%%%%%%%%%%%%%%%%%%%%%%%%%%%%%%%%%%%%%%%%%%%%%%%%%%%%%%%%%%%%%%%%%%%%%%%
%%%%%%%%%%%%%%%%%%%%%%%%%%%%%%%%%%%%%%%%%%%%%%%%%%%%%%%%%%%%%%%%%%%%%%%%%%%%%%%%
\begin{mdframed}[style=darkQuesion]
  2.
\end{mdframed}

%%%%%%%%%%%%%%%%%%%%%%%%%%%%%%%%%%%%%%%%%%%%%%%%%%%%%%%%%%%%%%%%%%%%%%%%%%%%%%%%
\begin{mdframed}[style=darkAnswer,frametitle={Joe Starr}]

\end{mdframed}
\newpage
%%%%%%%%%%%%%%%%%%%%%%%%%%%%%%%%%%%%%%%%%%%%%%%%%%%%%%%%%%%%%%%%%%%%%%%%%%%%%%%%
%%%%%%%%%%%%%%%%%%%%%%%%%%%%%%%%%%%%%%%%%%%%%%%%%%%%%%%%%%%%%%%%%%%%%%%%%%%%%%%%
%%%%%%%%%%%%%%%%%%%%%%%%%%%%%%%%%%%%%%%%%%%%%%%%%%%%%%%%%%%%%%%%%%%%%%%%%%%%%%%%
%%%%%%%%%%%%%%%%%%%%%%%%%%%%%%%%%%%%%%%%%%%%%%%%%%%%%%%%%%%%%%%%%%%%%%%%%%%%%%%%
\begin{mdframed}[style=darkQuesion]
  2.
\end{mdframed}

%%%%%%%%%%%%%%%%%%%%%%%%%%%%%%%%%%%%%%%%%%%%%%%%%%%%%%%%%%%%%%%%%%%%%%%%%%%%%%%%
\begin{mdframed}[style=darkAnswer,frametitle={Joe Starr}]

\end{mdframed}
\newpage
%%%%%%%%%%%%%%%%%%%%%%%%%%%%%%%%%%%%%%%%%%%%%%%%%%%%%%%%%%%%%%%%%%%%%%%%%%%%%%%%
%%%%%%%%%%%%%%%%%%%%%%%%%%%%%%%%%%%%%%%%%%%%%%%%%%%%%%%%%%%%%%%%%%%%%%%%%%%%%%%%
%%%%%%%%%%%%%%%%%%%%%%%%%%%%%%%%%%%%%%%%%%%%%%%%%%%%%%%%%%%%%%%%%%%%%%%%%%%%%%%%
%%%%%%%%%%%%%%%%%%%%%%%%%%%%%%%%%%%%%%%%%%%%%%%%%%%%%%%%%%%%%%%%%%%%%%%%%%%%%%%%
\begin{mdframed}[style=darkQuesion]
  2.
\end{mdframed}

%%%%%%%%%%%%%%%%%%%%%%%%%%%%%%%%%%%%%%%%%%%%%%%%%%%%%%%%%%%%%%%%%%%%%%%%%%%%%%%%
\begin{mdframed}[style=darkAnswer,frametitle={Joe Starr}]

\end{mdframed}
\newpage
%%%%%%%%%%%%%%%%%%%%%%%%%%%%%%%%%%%%%%%%%%%%%%%%%%%%%%%%%%%%%%%%%%%%%%%%%%%%%%%%
%%%%%%%%%%%%%%%%%%%%%%%%%%%%%%%%%%%%%%%%%%%%%%%%%%%%%%%%%%%%%%%%%%%%%%%%%%%%%%%%
%%%%%%%%%%%%%%%%%%%%%%%%%%%%%%%%%%%%%%%%%%%%%%%%%%%%%%%%%%%%%%%%%%%%%%%%%%%%%%%%
%%%%%%%%%%%%%%%%%%%%%%%%%%%%%%%%%%%%%%%%%%%%%%%%%%%%%%%%%%%%%%%%%%%%%%%%%%%%%%%%
\begin{mdframed}[style=darkQuesion]
  2.
\end{mdframed}

%%%%%%%%%%%%%%%%%%%%%%%%%%%%%%%%%%%%%%%%%%%%%%%%%%%%%%%%%%%%%%%%%%%%%%%%%%%%%%%%
\begin{mdframed}[style=darkAnswer,frametitle={Joe Starr}]

\end{mdframed}
\newpage
%%%%%%%%%%%%%%%%%%%%%%%%%%%%%%%%%%%%%%%%%%%%%%%%%%%%%%%%%%%%%%%%%%%%%%%%%%%%%%%%
%%%%%%%%%%%%%%%%%%%%%%%%%%%%%%%%%%%%%%%%%%%%%%%%%%%%%%%%%%%%%%%%%%%%%%%%%%%%%%%%
%%%%%%%%%%%%%%%%%%%%%%%%%%%%%%%%%%%%%%%%%%%%%%%%%%%%%%%%%%%%%%%%%%%%%%%%%%%%%%%%
%%%%%%%%%%%%%%%%%%%%%%%%%%%%%%%%%%%%%%%%%%%%%%%%%%%%%%%%%%%%%%%%%%%%%%%%%%%%%%%%
\begin{mdframed}[style=darkQuesion]
  2.
\end{mdframed}

%%%%%%%%%%%%%%%%%%%%%%%%%%%%%%%%%%%%%%%%%%%%%%%%%%%%%%%%%%%%%%%%%%%%%%%%%%%%%%%%
\begin{mdframed}[style=darkAnswer,frametitle={Joe Starr}]

\end{mdframed}
\newpage
%%%%%%%%%%%%%%%%%%%%%%%%%%%%%%%%%%%%%%%%%%%%%%%%%%%%%%%%%%%%%%%%%%%%%%%%%%%%%%%%
%%%%%%%%%%%%%%%%%%%%%%%%%%%%%%%%%%%%%%%%%%%%%%%%%%%%%%%%%%%%%%%%%%%%%%%%%%%%%%%%
%%%%%%%%%%%%%%%%%%%%%%%%%%%%%%%%%%%%%%%%%%%%%%%%%%%%%%%%%%%%%%%%%%%%%%%%%%%%%%%%
%%%%%%%%%%%%%%%%%%%%%%%%%%%%%%%%%%%%%%%%%%%%%%%%%%%%%%%%%%%%%%%%%%%%%%%%%%%%%%%%
\begin{mdframed}[style=darkQuesion]
  2.
\end{mdframed}

%%%%%%%%%%%%%%%%%%%%%%%%%%%%%%%%%%%%%%%%%%%%%%%%%%%%%%%%%%%%%%%%%%%%%%%%%%%%%%%%
\begin{mdframed}[style=darkAnswer,frametitle={Joe Starr}]

\end{mdframed}
\newpage
%%%%%%%%%%%%%%%%%%%%%%%%%%%%%%%%%%%%%%%%%%%%%%%%%%%%%%%%%%%%%%%%%%%%%%%%%%%%%%%%
%%%%%%%%%%%%%%%%%%%%%%%%%%%%%%%%%%%%%%%%%%%%%%%%%%%%%%%%%%%%%%%%%%%%%%%%%%%%%%%%
%%%%%%%%%%%%%%%%%%%%%%%%%%%%%%%%%%%%%%%%%%%%%%%%%%%%%%%%%%%%%%%%%%%%%%%%%%%%%%%%
%%%%%%%%%%%%%%%%%%%%%%%%%%%%%%%%%%%%%%%%%%%%%%%%%%%%%%%%%%%%%%%%%%%%%%%%%%%%%%%%
\begin{mdframed}[style=darkQuesion]
  2.
\end{mdframed}

%%%%%%%%%%%%%%%%%%%%%%%%%%%%%%%%%%%%%%%%%%%%%%%%%%%%%%%%%%%%%%%%%%%%%%%%%%%%%%%%
\begin{mdframed}[style=darkAnswer,frametitle={Joe Starr}]

\end{mdframed}
\newpage
%%%%%%%%%%%%%%%%%%%%%%%%%%%%%%%%%%%%%%%%%%%%%%%%%%%%%%%%%%%%%%%%%%%%%%%%%%%%%%%%
%%%%%%%%%%%%%%%%%%%%%%%%%%%%%%%%%%%%%%%%%%%%%%%%%%%%%%%%%%%%%%%%%%%%%%%%%%%%%%%%
%%%%%%%%%%%%%%%%%%%%%%%%%%%%%%%%%%%%%%%%%%%%%%%%%%%%%%%%%%%%%%%%%%%%%%%%%%%%%%%%
%%%%%%%%%%%%%%%%%%%%%%%%%%%%%%%%%%%%%%%%%%%%%%%%%%%%%%%%%%%%%%%%%%%%%%%%%%%%%%%%
\begin{mdframed}[style=darkQuesion]
  2.
\end{mdframed}

%%%%%%%%%%%%%%%%%%%%%%%%%%%%%%%%%%%%%%%%%%%%%%%%%%%%%%%%%%%%%%%%%%%%%%%%%%%%%%%%
\begin{mdframed}[style=darkAnswer,frametitle={Joe Starr}]

\end{mdframed}
\newpage
%%%%%%%%%%%%%%%%%%%%%%%%%%%%%%%%%%%%%%%%%%%%%%%%%%%%%%%%%%%%%%%%%%%%%%%%%%%%%%%%
%%%%%%%%%%%%%%%%%%%%%%%%%%%%%%%%%%%%%%%%%%%%%%%%%%%%%%%%%%%%%%%%%%%%%%%%%%%%%%%%
%%%%%%%%%%%%%%%%%%%%%%%%%%%%%%%%%%%%%%%%%%%%%%%%%%%%%%%%%%%%%%%%%%%%%%%%%%%%%%%%
%%%%%%%%%%%%%%%%%%%%%%%%%%%%%%%%%%%%%%%%%%%%%%%%%%%%%%%%%%%%%%%%%%%%%%%%%%%%%%%%
\begin{mdframed}[style=darkQuesion]
  2.
\end{mdframed}

%%%%%%%%%%%%%%%%%%%%%%%%%%%%%%%%%%%%%%%%%%%%%%%%%%%%%%%%%%%%%%%%%%%%%%%%%%%%%%%%
\begin{mdframed}[style=darkAnswer,frametitle={Joe Starr}]

\end{mdframed}
\newpage
\subsection{Subgroups}
%%%%%%%%%%%%%%%%%%%%%%%%%%%%%%%%%%%%%%%%%%%%%%%%%%%%%%%%%%%%%%%%%%%%%%%%%%%%%%%%
%%%%%%%%%%%%%%%%%%%%%%%%%%%%%%%%%%%%%%%%%%%%%%%%%%%%%%%%%%%%%%%%%%%%%%%%%%%%%%%%
%%%%%%%%%%%%%%%%%%%%%%%%%%%%%%%%%%%%%%%%%%%%%%%%%%%%%%%%%%%%%%%%%%%%%%%%%%%%%%%%
%%%%%%%%%%%%%%%%%%%%%%%%%%%%%%%%%%%%%%%%%%%%%%%%%%%%%%%%%%%%%%%%%%%%%%%%%%%%%%%%
\begin{mdframed}[style=darkQuesion]
  2.
\end{mdframed}

%%%%%%%%%%%%%%%%%%%%%%%%%%%%%%%%%%%%%%%%%%%%%%%%%%%%%%%%%%%%%%%%%%%%%%%%%%%%%%%%
\begin{mdframed}[style=darkAnswer,frametitle={Joe Starr}]

\end{mdframed}
\newpage
%%%%%%%%%%%%%%%%%%%%%%%%%%%%%%%%%%%%%%%%%%%%%%%%%%%%%%%%%%%%%%%%%%%%%%%%%%%%%%%%
%%%%%%%%%%%%%%%%%%%%%%%%%%%%%%%%%%%%%%%%%%%%%%%%%%%%%%%%%%%%%%%%%%%%%%%%%%%%%%%%
%%%%%%%%%%%%%%%%%%%%%%%%%%%%%%%%%%%%%%%%%%%%%%%%%%%%%%%%%%%%%%%%%%%%%%%%%%%%%%%%
%%%%%%%%%%%%%%%%%%%%%%%%%%%%%%%%%%%%%%%%%%%%%%%%%%%%%%%%%%%%%%%%%%%%%%%%%%%%%%%%
\begin{mdframed}[style=darkQuesion]
  2.
\end{mdframed}

%%%%%%%%%%%%%%%%%%%%%%%%%%%%%%%%%%%%%%%%%%%%%%%%%%%%%%%%%%%%%%%%%%%%%%%%%%%%%%%%
\begin{mdframed}[style=darkAnswer,frametitle={Joe Starr}]

\end{mdframed}
\newpage
%%%%%%%%%%%%%%%%%%%%%%%%%%%%%%%%%%%%%%%%%%%%%%%%%%%%%%%%%%%%%%%%%%%%%%%%%%%%%%%%
%%%%%%%%%%%%%%%%%%%%%%%%%%%%%%%%%%%%%%%%%%%%%%%%%%%%%%%%%%%%%%%%%%%%%%%%%%%%%%%%
%%%%%%%%%%%%%%%%%%%%%%%%%%%%%%%%%%%%%%%%%%%%%%%%%%%%%%%%%%%%%%%%%%%%%%%%%%%%%%%%
%%%%%%%%%%%%%%%%%%%%%%%%%%%%%%%%%%%%%%%%%%%%%%%%%%%%%%%%%%%%%%%%%%%%%%%%%%%%%%%%
\begin{mdframed}[style=darkQuesion]
  2.
\end{mdframed}

%%%%%%%%%%%%%%%%%%%%%%%%%%%%%%%%%%%%%%%%%%%%%%%%%%%%%%%%%%%%%%%%%%%%%%%%%%%%%%%%
\begin{mdframed}[style=darkAnswer,frametitle={Joe Starr}]

\end{mdframed}
\newpage
%%%%%%%%%%%%%%%%%%%%%%%%%%%%%%%%%%%%%%%%%%%%%%%%%%%%%%%%%%%%%%%%%%%%%%%%%%%%%%%%
%%%%%%%%%%%%%%%%%%%%%%%%%%%%%%%%%%%%%%%%%%%%%%%%%%%%%%%%%%%%%%%%%%%%%%%%%%%%%%%%
%%%%%%%%%%%%%%%%%%%%%%%%%%%%%%%%%%%%%%%%%%%%%%%%%%%%%%%%%%%%%%%%%%%%%%%%%%%%%%%%
%%%%%%%%%%%%%%%%%%%%%%%%%%%%%%%%%%%%%%%%%%%%%%%%%%%%%%%%%%%%%%%%%%%%%%%%%%%%%%%%
\begin{mdframed}[style=darkQuesion]
  2.
\end{mdframed}

%%%%%%%%%%%%%%%%%%%%%%%%%%%%%%%%%%%%%%%%%%%%%%%%%%%%%%%%%%%%%%%%%%%%%%%%%%%%%%%%
\begin{mdframed}[style=darkAnswer,frametitle={Joe Starr}]

\end{mdframed}
\newpage
%%%%%%%%%%%%%%%%%%%%%%%%%%%%%%%%%%%%%%%%%%%%%%%%%%%%%%%%%%%%%%%%%%%%%%%%%%%%%%%%
%%%%%%%%%%%%%%%%%%%%%%%%%%%%%%%%%%%%%%%%%%%%%%%%%%%%%%%%%%%%%%%%%%%%%%%%%%%%%%%%
%%%%%%%%%%%%%%%%%%%%%%%%%%%%%%%%%%%%%%%%%%%%%%%%%%%%%%%%%%%%%%%%%%%%%%%%%%%%%%%%
%%%%%%%%%%%%%%%%%%%%%%%%%%%%%%%%%%%%%%%%%%%%%%%%%%%%%%%%%%%%%%%%%%%%%%%%%%%%%%%%
\begin{mdframed}[style=darkQuesion]
  2.
\end{mdframed}

%%%%%%%%%%%%%%%%%%%%%%%%%%%%%%%%%%%%%%%%%%%%%%%%%%%%%%%%%%%%%%%%%%%%%%%%%%%%%%%%
\begin{mdframed}[style=darkAnswer,frametitle={Joe Starr}]

\end{mdframed}
\newpage
%%%%%%%%%%%%%%%%%%%%%%%%%%%%%%%%%%%%%%%%%%%%%%%%%%%%%%%%%%%%%%%%%%%%%%%%%%%%%%%%
%%%%%%%%%%%%%%%%%%%%%%%%%%%%%%%%%%%%%%%%%%%%%%%%%%%%%%%%%%%%%%%%%%%%%%%%%%%%%%%%
%%%%%%%%%%%%%%%%%%%%%%%%%%%%%%%%%%%%%%%%%%%%%%%%%%%%%%%%%%%%%%%%%%%%%%%%%%%%%%%%
%%%%%%%%%%%%%%%%%%%%%%%%%%%%%%%%%%%%%%%%%%%%%%%%%%%%%%%%%%%%%%%%%%%%%%%%%%%%%%%%
\begin{mdframed}[style=darkQuesion]
  2.
\end{mdframed}

%%%%%%%%%%%%%%%%%%%%%%%%%%%%%%%%%%%%%%%%%%%%%%%%%%%%%%%%%%%%%%%%%%%%%%%%%%%%%%%%
\begin{mdframed}[style=darkAnswer,frametitle={Joe Starr}]

\end{mdframed}
\newpage
%%%%%%%%%%%%%%%%%%%%%%%%%%%%%%%%%%%%%%%%%%%%%%%%%%%%%%%%%%%%%%%%%%%%%%%%%%%%%%%%
%%%%%%%%%%%%%%%%%%%%%%%%%%%%%%%%%%%%%%%%%%%%%%%%%%%%%%%%%%%%%%%%%%%%%%%%%%%%%%%%
%%%%%%%%%%%%%%%%%%%%%%%%%%%%%%%%%%%%%%%%%%%%%%%%%%%%%%%%%%%%%%%%%%%%%%%%%%%%%%%%
%%%%%%%%%%%%%%%%%%%%%%%%%%%%%%%%%%%%%%%%%%%%%%%%%%%%%%%%%%%%%%%%%%%%%%%%%%%%%%%%
\begin{mdframed}[style=darkQuesion]
  2.
\end{mdframed}

%%%%%%%%%%%%%%%%%%%%%%%%%%%%%%%%%%%%%%%%%%%%%%%%%%%%%%%%%%%%%%%%%%%%%%%%%%%%%%%%
\begin{mdframed}[style=darkAnswer,frametitle={Joe Starr}]

\end{mdframed}
\newpage
%%%%%%%%%%%%%%%%%%%%%%%%%%%%%%%%%%%%%%%%%%%%%%%%%%%%%%%%%%%%%%%%%%%%%%%%%%%%%%%%
%%%%%%%%%%%%%%%%%%%%%%%%%%%%%%%%%%%%%%%%%%%%%%%%%%%%%%%%%%%%%%%%%%%%%%%%%%%%%%%%
%%%%%%%%%%%%%%%%%%%%%%%%%%%%%%%%%%%%%%%%%%%%%%%%%%%%%%%%%%%%%%%%%%%%%%%%%%%%%%%%
%%%%%%%%%%%%%%%%%%%%%%%%%%%%%%%%%%%%%%%%%%%%%%%%%%%%%%%%%%%%%%%%%%%%%%%%%%%%%%%%
\begin{mdframed}[style=darkQuesion]
  2.
\end{mdframed}

%%%%%%%%%%%%%%%%%%%%%%%%%%%%%%%%%%%%%%%%%%%%%%%%%%%%%%%%%%%%%%%%%%%%%%%%%%%%%%%%
\begin{mdframed}[style=darkAnswer,frametitle={Joe Starr}]

\end{mdframed}
\newpage
%%%%%%%%%%%%%%%%%%%%%%%%%%%%%%%%%%%%%%%%%%%%%%%%%%%%%%%%%%%%%%%%%%%%%%%%%%%%%%%%
%%%%%%%%%%%%%%%%%%%%%%%%%%%%%%%%%%%%%%%%%%%%%%%%%%%%%%%%%%%%%%%%%%%%%%%%%%%%%%%%
%%%%%%%%%%%%%%%%%%%%%%%%%%%%%%%%%%%%%%%%%%%%%%%%%%%%%%%%%%%%%%%%%%%%%%%%%%%%%%%%
%%%%%%%%%%%%%%%%%%%%%%%%%%%%%%%%%%%%%%%%%%%%%%%%%%%%%%%%%%%%%%%%%%%%%%%%%%%%%%%%
\begin{mdframed}[style=darkQuesion]
  2.
\end{mdframed}

%%%%%%%%%%%%%%%%%%%%%%%%%%%%%%%%%%%%%%%%%%%%%%%%%%%%%%%%%%%%%%%%%%%%%%%%%%%%%%%%
\begin{mdframed}[style=darkAnswer,frametitle={Joe Starr}]

\end{mdframed}
\newpage
%%%%%%%%%%%%%%%%%%%%%%%%%%%%%%%%%%%%%%%%%%%%%%%%%%%%%%%%%%%%%%%%%%%%%%%%%%%%%%%%
%%%%%%%%%%%%%%%%%%%%%%%%%%%%%%%%%%%%%%%%%%%%%%%%%%%%%%%%%%%%%%%%%%%%%%%%%%%%%%%%
%%%%%%%%%%%%%%%%%%%%%%%%%%%%%%%%%%%%%%%%%%%%%%%%%%%%%%%%%%%%%%%%%%%%%%%%%%%%%%%%
%%%%%%%%%%%%%%%%%%%%%%%%%%%%%%%%%%%%%%%%%%%%%%%%%%%%%%%%%%%%%%%%%%%%%%%%%%%%%%%%
\begin{mdframed}[style=darkQuesion]
  2.
\end{mdframed}

%%%%%%%%%%%%%%%%%%%%%%%%%%%%%%%%%%%%%%%%%%%%%%%%%%%%%%%%%%%%%%%%%%%%%%%%%%%%%%%%
\begin{mdframed}[style=darkAnswer,frametitle={Joe Starr}]

\end{mdframed}
\newpage
%%%%%%%%%%%%%%%%%%%%%%%%%%%%%%%%%%%%%%%%%%%%%%%%%%%%%%%%%%%%%%%%%%%%%%%%%%%%%%%%
%%%%%%%%%%%%%%%%%%%%%%%%%%%%%%%%%%%%%%%%%%%%%%%%%%%%%%%%%%%%%%%%%%%%%%%%%%%%%%%%
%%%%%%%%%%%%%%%%%%%%%%%%%%%%%%%%%%%%%%%%%%%%%%%%%%%%%%%%%%%%%%%%%%%%%%%%%%%%%%%%
%%%%%%%%%%%%%%%%%%%%%%%%%%%%%%%%%%%%%%%%%%%%%%%%%%%%%%%%%%%%%%%%%%%%%%%%%%%%%%%%
\begin{mdframed}[style=darkQuesion]
  2.
\end{mdframed}

%%%%%%%%%%%%%%%%%%%%%%%%%%%%%%%%%%%%%%%%%%%%%%%%%%%%%%%%%%%%%%%%%%%%%%%%%%%%%%%%
\begin{mdframed}[style=darkAnswer,frametitle={Joe Starr}]

\end{mdframed}
\newpage
%%%%%%%%%%%%%%%%%%%%%%%%%%%%%%%%%%%%%%%%%%%%%%%%%%%%%%%%%%%%%%%%%%%%%%%%%%%%%%%%
%%%%%%%%%%%%%%%%%%%%%%%%%%%%%%%%%%%%%%%%%%%%%%%%%%%%%%%%%%%%%%%%%%%%%%%%%%%%%%%%
%%%%%%%%%%%%%%%%%%%%%%%%%%%%%%%%%%%%%%%%%%%%%%%%%%%%%%%%%%%%%%%%%%%%%%%%%%%%%%%%
%%%%%%%%%%%%%%%%%%%%%%%%%%%%%%%%%%%%%%%%%%%%%%%%%%%%%%%%%%%%%%%%%%%%%%%%%%%%%%%%
\begin{mdframed}[style=darkQuesion]
  2.
\end{mdframed}

%%%%%%%%%%%%%%%%%%%%%%%%%%%%%%%%%%%%%%%%%%%%%%%%%%%%%%%%%%%%%%%%%%%%%%%%%%%%%%%%
\begin{mdframed}[style=darkAnswer,frametitle={Joe Starr}]

\end{mdframed}
\newpage
%%%%%%%%%%%%%%%%%%%%%%%%%%%%%%%%%%%%%%%%%%%%%%%%%%%%%%%%%%%%%%%%%%%%%%%%%%%%%%%%
%%%%%%%%%%%%%%%%%%%%%%%%%%%%%%%%%%%%%%%%%%%%%%%%%%%%%%%%%%%%%%%%%%%%%%%%%%%%%%%%
%%%%%%%%%%%%%%%%%%%%%%%%%%%%%%%%%%%%%%%%%%%%%%%%%%%%%%%%%%%%%%%%%%%%%%%%%%%%%%%%
%%%%%%%%%%%%%%%%%%%%%%%%%%%%%%%%%%%%%%%%%%%%%%%%%%%%%%%%%%%%%%%%%%%%%%%%%%%%%%%%
\begin{mdframed}[style=darkQuesion]
  2.
\end{mdframed}

%%%%%%%%%%%%%%%%%%%%%%%%%%%%%%%%%%%%%%%%%%%%%%%%%%%%%%%%%%%%%%%%%%%%%%%%%%%%%%%%
\begin{mdframed}[style=darkAnswer,frametitle={Joe Starr}]

\end{mdframed}
\newpage
%%%%%%%%%%%%%%%%%%%%%%%%%%%%%%%%%%%%%%%%%%%%%%%%%%%%%%%%%%%%%%%%%%%%%%%%%%%%%%%%
%%%%%%%%%%%%%%%%%%%%%%%%%%%%%%%%%%%%%%%%%%%%%%%%%%%%%%%%%%%%%%%%%%%%%%%%%%%%%%%%
%%%%%%%%%%%%%%%%%%%%%%%%%%%%%%%%%%%%%%%%%%%%%%%%%%%%%%%%%%%%%%%%%%%%%%%%%%%%%%%%
%%%%%%%%%%%%%%%%%%%%%%%%%%%%%%%%%%%%%%%%%%%%%%%%%%%%%%%%%%%%%%%%%%%%%%%%%%%%%%%%
\begin{mdframed}[style=darkQuesion]
  2.
\end{mdframed}

%%%%%%%%%%%%%%%%%%%%%%%%%%%%%%%%%%%%%%%%%%%%%%%%%%%%%%%%%%%%%%%%%%%%%%%%%%%%%%%%
\begin{mdframed}[style=darkAnswer,frametitle={Joe Starr}]

\end{mdframed}
\newpage
%%%%%%%%%%%%%%%%%%%%%%%%%%%%%%%%%%%%%%%%%%%%%%%%%%%%%%%%%%%%%%%%%%%%%%%%%%%%%%%%
%%%%%%%%%%%%%%%%%%%%%%%%%%%%%%%%%%%%%%%%%%%%%%%%%%%%%%%%%%%%%%%%%%%%%%%%%%%%%%%%
%%%%%%%%%%%%%%%%%%%%%%%%%%%%%%%%%%%%%%%%%%%%%%%%%%%%%%%%%%%%%%%%%%%%%%%%%%%%%%%%
%%%%%%%%%%%%%%%%%%%%%%%%%%%%%%%%%%%%%%%%%%%%%%%%%%%%%%%%%%%%%%%%%%%%%%%%%%%%%%%%
\begin{mdframed}[style=darkQuesion]
  2.
\end{mdframed}

%%%%%%%%%%%%%%%%%%%%%%%%%%%%%%%%%%%%%%%%%%%%%%%%%%%%%%%%%%%%%%%%%%%%%%%%%%%%%%%%
\begin{mdframed}[style=darkAnswer,frametitle={Joe Starr}]

\end{mdframed}
\newpage
%%%%%%%%%%%%%%%%%%%%%%%%%%%%%%%%%%%%%%%%%%%%%%%%%%%%%%%%%%%%%%%%%%%%%%%%%%%%%%%%
%%%%%%%%%%%%%%%%%%%%%%%%%%%%%%%%%%%%%%%%%%%%%%%%%%%%%%%%%%%%%%%%%%%%%%%%%%%%%%%%
%%%%%%%%%%%%%%%%%%%%%%%%%%%%%%%%%%%%%%%%%%%%%%%%%%%%%%%%%%%%%%%%%%%%%%%%%%%%%%%%
%%%%%%%%%%%%%%%%%%%%%%%%%%%%%%%%%%%%%%%%%%%%%%%%%%%%%%%%%%%%%%%%%%%%%%%%%%%%%%%%
\begin{mdframed}[style=darkQuesion]
  2.
\end{mdframed}

%%%%%%%%%%%%%%%%%%%%%%%%%%%%%%%%%%%%%%%%%%%%%%%%%%%%%%%%%%%%%%%%%%%%%%%%%%%%%%%%
\begin{mdframed}[style=darkAnswer,frametitle={Joe Starr}]

\end{mdframed}
\newpage
%%%%%%%%%%%%%%%%%%%%%%%%%%%%%%%%%%%%%%%%%%%%%%%%%%%%%%%%%%%%%%%%%%%%%%%%%%%%%%%%
%%%%%%%%%%%%%%%%%%%%%%%%%%%%%%%%%%%%%%%%%%%%%%%%%%%%%%%%%%%%%%%%%%%%%%%%%%%%%%%%
%%%%%%%%%%%%%%%%%%%%%%%%%%%%%%%%%%%%%%%%%%%%%%%%%%%%%%%%%%%%%%%%%%%%%%%%%%%%%%%%
%%%%%%%%%%%%%%%%%%%%%%%%%%%%%%%%%%%%%%%%%%%%%%%%%%%%%%%%%%%%%%%%%%%%%%%%%%%%%%%%
\begin{mdframed}[style=darkQuesion]
  2.
\end{mdframed}

%%%%%%%%%%%%%%%%%%%%%%%%%%%%%%%%%%%%%%%%%%%%%%%%%%%%%%%%%%%%%%%%%%%%%%%%%%%%%%%%
\begin{mdframed}[style=darkAnswer,frametitle={Joe Starr}]

\end{mdframed}
\newpage
%%%%%%%%%%%%%%%%%%%%%%%%%%%%%%%%%%%%%%%%%%%%%%%%%%%%%%%%%%%%%%%%%%%%%%%%%%%%%%%%
%%%%%%%%%%%%%%%%%%%%%%%%%%%%%%%%%%%%%%%%%%%%%%%%%%%%%%%%%%%%%%%%%%%%%%%%%%%%%%%%
%%%%%%%%%%%%%%%%%%%%%%%%%%%%%%%%%%%%%%%%%%%%%%%%%%%%%%%%%%%%%%%%%%%%%%%%%%%%%%%%
%%%%%%%%%%%%%%%%%%%%%%%%%%%%%%%%%%%%%%%%%%%%%%%%%%%%%%%%%%%%%%%%%%%%%%%%%%%%%%%%
\begin{mdframed}[style=darkQuesion]
  2.
\end{mdframed}

%%%%%%%%%%%%%%%%%%%%%%%%%%%%%%%%%%%%%%%%%%%%%%%%%%%%%%%%%%%%%%%%%%%%%%%%%%%%%%%%
\begin{mdframed}[style=darkAnswer,frametitle={Joe Starr}]

\end{mdframed}
\newpage
%%%%%%%%%%%%%%%%%%%%%%%%%%%%%%%%%%%%%%%%%%%%%%%%%%%%%%%%%%%%%%%%%%%%%%%%%%%%%%%%
%%%%%%%%%%%%%%%%%%%%%%%%%%%%%%%%%%%%%%%%%%%%%%%%%%%%%%%%%%%%%%%%%%%%%%%%%%%%%%%%
%%%%%%%%%%%%%%%%%%%%%%%%%%%%%%%%%%%%%%%%%%%%%%%%%%%%%%%%%%%%%%%%%%%%%%%%%%%%%%%%
%%%%%%%%%%%%%%%%%%%%%%%%%%%%%%%%%%%%%%%%%%%%%%%%%%%%%%%%%%%%%%%%%%%%%%%%%%%%%%%%
\begin{mdframed}[style=darkQuesion]
  2.
\end{mdframed}

%%%%%%%%%%%%%%%%%%%%%%%%%%%%%%%%%%%%%%%%%%%%%%%%%%%%%%%%%%%%%%%%%%%%%%%%%%%%%%%%
\begin{mdframed}[style=darkAnswer,frametitle={Joe Starr}]

\end{mdframed}
\newpage
%%%%%%%%%%%%%%%%%%%%%%%%%%%%%%%%%%%%%%%%%%%%%%%%%%%%%%%%%%%%%%%%%%%%%%%%%%%%%%%%
%%%%%%%%%%%%%%%%%%%%%%%%%%%%%%%%%%%%%%%%%%%%%%%%%%%%%%%%%%%%%%%%%%%%%%%%%%%%%%%%
%%%%%%%%%%%%%%%%%%%%%%%%%%%%%%%%%%%%%%%%%%%%%%%%%%%%%%%%%%%%%%%%%%%%%%%%%%%%%%%%
%%%%%%%%%%%%%%%%%%%%%%%%%%%%%%%%%%%%%%%%%%%%%%%%%%%%%%%%%%%%%%%%%%%%%%%%%%%%%%%%
\begin{mdframed}[style=darkQuesion]
  2.
\end{mdframed}

%%%%%%%%%%%%%%%%%%%%%%%%%%%%%%%%%%%%%%%%%%%%%%%%%%%%%%%%%%%%%%%%%%%%%%%%%%%%%%%%
\begin{mdframed}[style=darkAnswer,frametitle={Joe Starr}]

\end{mdframed}
\newpage
%%%%%%%%%%%%%%%%%%%%%%%%%%%%%%%%%%%%%%%%%%%%%%%%%%%%%%%%%%%%%%%%%%%%%%%%%%%%%%%%
%%%%%%%%%%%%%%%%%%%%%%%%%%%%%%%%%%%%%%%%%%%%%%%%%%%%%%%%%%%%%%%%%%%%%%%%%%%%%%%%
%%%%%%%%%%%%%%%%%%%%%%%%%%%%%%%%%%%%%%%%%%%%%%%%%%%%%%%%%%%%%%%%%%%%%%%%%%%%%%%%
%%%%%%%%%%%%%%%%%%%%%%%%%%%%%%%%%%%%%%%%%%%%%%%%%%%%%%%%%%%%%%%%%%%%%%%%%%%%%%%%
\begin{mdframed}[style=darkQuesion]
  2.
\end{mdframed}

%%%%%%%%%%%%%%%%%%%%%%%%%%%%%%%%%%%%%%%%%%%%%%%%%%%%%%%%%%%%%%%%%%%%%%%%%%%%%%%%
\begin{mdframed}[style=darkAnswer,frametitle={Joe Starr}]

\end{mdframed}
\newpage
%%%%%%%%%%%%%%%%%%%%%%%%%%%%%%%%%%%%%%%%%%%%%%%%%%%%%%%%%%%%%%%%%%%%%%%%%%%%%%%%
%%%%%%%%%%%%%%%%%%%%%%%%%%%%%%%%%%%%%%%%%%%%%%%%%%%%%%%%%%%%%%%%%%%%%%%%%%%%%%%%
%%%%%%%%%%%%%%%%%%%%%%%%%%%%%%%%%%%%%%%%%%%%%%%%%%%%%%%%%%%%%%%%%%%%%%%%%%%%%%%%
%%%%%%%%%%%%%%%%%%%%%%%%%%%%%%%%%%%%%%%%%%%%%%%%%%%%%%%%%%%%%%%%%%%%%%%%%%%%%%%%
\begin{mdframed}[style=darkQuesion]
  2.
\end{mdframed}

%%%%%%%%%%%%%%%%%%%%%%%%%%%%%%%%%%%%%%%%%%%%%%%%%%%%%%%%%%%%%%%%%%%%%%%%%%%%%%%%
\begin{mdframed}[style=darkAnswer,frametitle={Joe Starr}]

\end{mdframed}
\newpage
%%%%%%%%%%%%%%%%%%%%%%%%%%%%%%%%%%%%%%%%%%%%%%%%%%%%%%%%%%%%%%%%%%%%%%%%%%%%%%%%
%%%%%%%%%%%%%%%%%%%%%%%%%%%%%%%%%%%%%%%%%%%%%%%%%%%%%%%%%%%%%%%%%%%%%%%%%%%%%%%%
%%%%%%%%%%%%%%%%%%%%%%%%%%%%%%%%%%%%%%%%%%%%%%%%%%%%%%%%%%%%%%%%%%%%%%%%%%%%%%%%
%%%%%%%%%%%%%%%%%%%%%%%%%%%%%%%%%%%%%%%%%%%%%%%%%%%%%%%%%%%%%%%%%%%%%%%%%%%%%%%%
\begin{mdframed}[style=darkQuesion]
  2.
\end{mdframed}

%%%%%%%%%%%%%%%%%%%%%%%%%%%%%%%%%%%%%%%%%%%%%%%%%%%%%%%%%%%%%%%%%%%%%%%%%%%%%%%%
\begin{mdframed}[style=darkAnswer,frametitle={Joe Starr}]

\end{mdframed}
\newpage
%%%%%%%%%%%%%%%%%%%%%%%%%%%%%%%%%%%%%%%%%%%%%%%%%%%%%%%%%%%%%%%%%%%%%%%%%%%%%%%%
%%%%%%%%%%%%%%%%%%%%%%%%%%%%%%%%%%%%%%%%%%%%%%%%%%%%%%%%%%%%%%%%%%%%%%%%%%%%%%%%
%%%%%%%%%%%%%%%%%%%%%%%%%%%%%%%%%%%%%%%%%%%%%%%%%%%%%%%%%%%%%%%%%%%%%%%%%%%%%%%%
%%%%%%%%%%%%%%%%%%%%%%%%%%%%%%%%%%%%%%%%%%%%%%%%%%%%%%%%%%%%%%%%%%%%%%%%%%%%%%%%
\begin{mdframed}[style=darkQuesion]
  2.
\end{mdframed}

%%%%%%%%%%%%%%%%%%%%%%%%%%%%%%%%%%%%%%%%%%%%%%%%%%%%%%%%%%%%%%%%%%%%%%%%%%%%%%%%
\begin{mdframed}[style=darkAnswer,frametitle={Joe Starr}]

\end{mdframed}
\newpage
%%%%%%%%%%%%%%%%%%%%%%%%%%%%%%%%%%%%%%%%%%%%%%%%%%%%%%%%%%%%%%%%%%%%%%%%%%%%%%%%
%%%%%%%%%%%%%%%%%%%%%%%%%%%%%%%%%%%%%%%%%%%%%%%%%%%%%%%%%%%%%%%%%%%%%%%%%%%%%%%%
%%%%%%%%%%%%%%%%%%%%%%%%%%%%%%%%%%%%%%%%%%%%%%%%%%%%%%%%%%%%%%%%%%%%%%%%%%%%%%%%
%%%%%%%%%%%%%%%%%%%%%%%%%%%%%%%%%%%%%%%%%%%%%%%%%%%%%%%%%%%%%%%%%%%%%%%%%%%%%%%%
\begin{mdframed}[style=darkQuesion]
  2.
\end{mdframed}

%%%%%%%%%%%%%%%%%%%%%%%%%%%%%%%%%%%%%%%%%%%%%%%%%%%%%%%%%%%%%%%%%%%%%%%%%%%%%%%%
\begin{mdframed}[style=darkAnswer,frametitle={Joe Starr}]

\end{mdframed}
\newpage
\clearpage
%Compile Chapter 4
\input{chapter5}
\clearpage
%Compile Chapter 5
\input{chapter5}
\clearpage
%Compile Chapter 6
\section{Section}  

I <3 my Wayne State Libraries! Do you? 

\subsection{A Subsection}

\clearpage
%Compile Chapter 7
\input{chapter7}
\clearpage
%Compile Chapter 8
\input{chapter8}
\clearpage
%Compile Chapter 9
\input{chapter9}
\clearpage
%Compile Chapter 10
\input{chapter10}
\clearpage

\end{document}